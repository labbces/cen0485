\documentclass[letter,11pt]{book}
\usepackage[in]{fullpage}
\usepackage[utf8]{inputenc}
\usepackage[portuges]{babel}
\usepackage{graphicx}
\usepackage[]{natbib}
\usepackage{setspace}
\usepackage{pdfpages}
\usepackage[toc,page,title,titletoc]{appendix}
\usepackage[colorlinks={false},a4paper={true}]{hyperref}
\usepackage{url}
%\usepackage{listings}
\usepackage{fancyvrb}
\usepackage[small,bf,up]{caption}
\usepackage{subfigure}
\usepackage{natbib}
\usepackage{multirow}
\usepackage{courier}
\usepackage{rotating}
\usepackage{dirtytalk}

\onehalfspacing

\renewcommand{\appendixtocname}{Apendices}
\renewcommand{\appendixname}{Apendices}

\newcommand{\Backslash}[1]{\texttt{\symbol{92}}#1}

%\lstset{numbers=left,numberstyle=\tiny}
%\lstset{morecomment=[s][\color{blue}]{[u}{]\$}}
%\lstset{stringstyle=\ttfamily}
%\lstset{firstnumber=last}
%\lstset{basicstyle=\small}

\author{Diego Mauricio Ria\~{n}o Pach\'{o}n}
\title{Introdução a Bioinformática - CEN0485}


\begin{document}

\maketitle
\tableofcontents
\listoffigures

\chapter{Bases de bioinformática}


A bioinformática é uma disciplina que surge da interação entre biologia, estatística e ciência da computação. (Figura~\ref{bioinf}). Seus principais objetivos são a gestão e análise de grandes volumes de dados, principalmente o produto de novas tecnologias em biologia molecular, como genômica, proteômica e metabolômica, especialmente hoje com o advento de novas tecnologias de sequenciamento de ácidos nucleicos que estão revolucionando a forma como estudamos os genomas. Outro aspecto importante inclui o desenvolvimento de novos métodos computacionais, algoritmos e/ou softwares para a análise desses dados. 

De acordo com Philip Bourne (UCSD), \say{A bioinformática tornou-se a intérprete da linguagem genômica do DNA e está tentando decifrar linguagens mais complexas em que as proteínas são os substantivos, as interações são a sintaxe, as vias metabólicas são frases e os sistemas vivos são o volume completo} \citep{Bourne2004}.

Portanto, semelhante à biologia molecular, a bioinformática hoje constitui uma caixa de ferramentas que todo pesquisador de biologia tem apreender a usar \citetext{\citealp{Stein2008} apresenta um ponto de vista muito interessante}.

\begin{figure}[h]
\centering
   \includegraphics[height=5cm]{Figs/Bioinformatica.png}
  \caption[O que é bioinformática?]{\label{bioinf} A bioinformática é a disciplina que surge da interação de três ciências básicas: Biologia, Matemática/Estatística e Ciências da Computação. Quando alguna delas dominam o resto, outra disciplina diferente da bioinformática é obtida, por exemplo, se matemática e biologia são mais importantes, obtemos biomatemáticos. O importante é que as três ciências-base sejam equilibradas para realizar projetos de bioinformática.}
\end{figure}

Neste curso nos concentraremos na análise de dados biológicos, utilizando, na maioria dos casos, ferramentas de livre acesso, a maioria das quais têm melhor desempenho em sistemas operacionais Unix\footnote{Linux, MacOSX, BSD, etc. Se você quiser tentar ter uma cópia em sua home ou escritório de qualquer um desses sistemas operacionais, recomendo que você use o VirtualBox (ou outra tecnologia de virtualização), para instalar, por exemplo, o Linux dentro do sistema operacional existente, por exemplo, o Windows XP; Claro que é se você tem um camputador com pelo menos dois Núcleos e 2GB de RAM, caso contrário é mais conveniente ter um sistema dual boot.}, é por isso que o próximo capítulo está planejado para desenvolver habilidades básicas nesse sistema operacional.

\chapter{Ferramentas Unix úteis em bioinformática}
\section{Introdução ao sistema Unix}

O sistema operativo\footnote{Mais informações em \url{http://en.wikipedia.org/wiki/Operating_system}} é o conjunto de programas (``software'') que serve como uma interface entre a máquina ('hardware') e o usuário, e que permite que este último execute aplicativos. Os sistemas operacionais mais comuns são: Windows (XP, Vista), Unix e MacOS X. Sistemas operacionais semelhantes ao Unix (por exemplo, Linux) são usados principalmente em servidores, mas seu uso em estações de trabalho e desktops está em ascensão.  As principais características do Unix são: multitarefas, multi-usuário e portabilidade\footnote{Refere-se a quais programas criados em diferentes Unixes podem ser executados em um ou outro geralmente sem problemas.}. A maioria dos Unixes hoje tem uma interface gráfica fácil de usar, a partir da qual você pode realizar quase todas as tarefas de uso diário, como criar documentos, imprimir e navegar na Internet. Além dessa interface gráfica, há uma interface de linha de comando que permite ao usuário executar tarefas muito mais complexas e poderosas. Em seguida, aprenderemos como usar a linha de comando e alguns comandos que facilitam o manuseio de arquivos grandes, usando o Linux como sistema operacional. orientação sobre o uso de vários desses comandos está disponível no apêndice~\ref{unixguide}\footnote{Guias para outros programas comumente usados em bioinformática estão disponíveis em \url{http://www.embnet.org/en/QuickGuides}}.

\subsection{A linha de comando}

A linha de comando é acessada através de um programa de interpretação chamado ``shell''\footnote{\url{http://en.wikipedia.org/wiki/Unix_shell}}.  Existem vários tipos de ``shell'' em Unix. Na maioria das distribuições Linux o ``shell'' bash é instalado por padrão. Para usar o ``shell'' ou linha de comando do seu computador, inicie o programa \textbf{Terminal}, que tem um ícone semelhante ao mostrado na Figura~\ref{terminalico}.

\begin{figure}[ht]
\centering
   \includegraphics[width=3cm]{Figs/terminalico.png}
  \caption{\label{terminalico}Ícone do programa da terminal}
\end{figure}

Clicando (uma ou duas vezes, dependendo da configuração) iniciará o programa \textbf{Terminal}, semelhante ao mostrado na Figura~\ref{terminal}. Este aplicativo da acesso à linha de comando Linux através de um \textit{prompt}, que informa que o sistema está esperando suas instruções. Na Figura~\ref{terminal}, o \textit{prompt} consiste na string \Verb+[user@server]$+, que consiste no nome do usuário que está usando o programa \textbf{Terminal}, seguido pelo nome da máquina e pelo símbolo do dólar, imediatamente após tem um cursor piscando esperando por seus comandos.

\begin{figure}[ht]
\centering
   \includegraphics[width=10cm]{Figs/terminal.png}
  \caption{\label{terminal}Terminal no Linux}
\end{figure}

O \textit{prompt} pode ser modificado alterando a variável do sistema \Verb+PS1+\footnote{\url{http://tldp.org/HOWTO/Bash-Prompt-HOWTO/c141.html}}. Vamos alterar o \textit{prompt} para ter certeza de que todos temos o mesmo.

Na sessão  \textbf{Terminal} execute os comandos conforme mostrado na lista \textit{Alterando prompt}. Na linha~\ref{saveprompt} salvamos O prompt na nova variável \Verb+SAVE+, caso precisemos recuperá-lo. Na linha \ref{changeprompt} modificamos o \textit{prompt} atual, \Verb+\u+\footnote{Lista de modificadores de \textit{prompt} no bash: \url{http://tldp.org/HOWTO/Bash-Prompt-HOWTO/bash-prompt-escape-sequences.html}}, indica a nossa ``shell'' mostrar o usuario atual, \Verb+\h+, mostra o nome da  máquina e \Verb+\w+, mostra o diretório atual, o resto de caracteres são exibidos sem qualquer modificação\footnote{Exercício opcional: {\textquestiondown}Como tornar permanente a alteração de \textit{prompt}?}. Compare seu novo \textit{prompt} (línea~\ref{describeprompt}) com o antiguo (línea~\ref{promptini}), o símbolo \Verb+~+ refere-se ao diretório da sua home ou ao diretório do usuário, no sistema  (vea Sección~\ref{homedir})

\begin{Verbatim}[commandchars=!\{\},numbers=left,label=Alterando o prompt,frame=topline,fontsize=\scriptsize]
!textcolor{red}{[user@server]$} !label{promptini}
!textcolor{red}{[user@server]$} echo $PS1 !label{printprompt}
[\u@\h]$
!textcolor{red}{[user@server]$} SAVE=$PS1 !label{saveprompt}
!textcolor{red}{[user@server]$} PS1="[\u@\h:\w]$ " !label{changeprompt}
!textcolor{red}{[user@server:~]$} !label{describeprompt}
\end{Verbatim} 

Vamos começar interagir com o sistema através de comandos. para começar a executar o comando mostrado na linha~\ref{wgetini}, \Verb+wget+ é um programa para baixar arquivos da rede. A linha \ref{wgetinistart} ate \ref{wgetinistop} mostram a saida tipica deste comando, pode mudar levemente do que se amostra no seu \textbf{Terminal}. quando este comando termina executar o mostrado na linha~\ref{tarini}, que abre o arquivo que você acabou de baixar.

\begin{Verbatim}[commandchars=!\{\},numbers=left,firstnumber=last,label=Baixando arquivos, frame=topline,fontsize=\scriptsize]
!textcolor{red}{[user@server:~]$} wget https://github.com/labbces/cen0485/raw/main/linux/practicas/file1.tar.gz !label{wgetini}
--2022-03-25 16:26:12--  https://github.com/labbces/cen0485/raw/main/linux/practicas/file1.tar.gz !label{wgetinistart}
Resolving github.com (github.com)... 20.201.28.151
Connecting to github.com (github.com)|20.201.28.151|:443... connected.
HTTP request sent, awaiting response... 302 Found
Location: https://raw.githubusercontent.com/labbces/cen0485/main/linux/practicas/file1.tar.gz [following]
--2022-03-25 16:26:18--  https://raw.githubusercontent.com/labbces/cen0485/main/linux/practicas/file1.tar.gz
Resolving raw.githubusercontent.com (raw.githubusercontent.com)... 185.199.108.133, 185.199.109.133, 185.199.111.133, ...
Connecting to raw.githubusercontent.com (raw.githubusercontent.com)|185.199.108.133|:443... connected.
HTTP request sent, awaiting response... 200 OK
Length: 73501 (72K) [application/octet-stream]
Saving to: ‘file1.tar.gz’

file1.tar.gz                                      100%[==========================================================================================================>]  71,78K   141KB/s    in 0,5s    

2022-03-25 16:26:19 (141 KB/s) - ‘file1.tar.gz’ saved [73501/73501] !label{wgetinistop}

!textcolor{red}{[user@server:~]$} tar xzf file1.tgz !label{tarini}
\end{Verbatim} 

\subsection{Sua home e árvore diretórios\label{homedir}}

Cada usuário em um sistema Unix tem um espaço reservado, geralmente dentro do diretório ``\Verb+/home+'', em um subdiretório que tem o mesmo nome do usuário, e.g., para o usuário ''diriano'' seu diretório pessoal é ``\Verb+/home/diriano+'', e é chamado de diretório ''home'' ou diretório de usuário.  A primeira vez que você faz login no Linux ou \textbf{Terminal}, está localizado em seu diretório home. se a qualquer momento você não sabe onde você está, você pode usar o comando mostrado na linha \ref{pwd} para localizar o caminho dentro da árvore do diretório em que está localizada. é importante que você note que diretórios usam o caracter ``\Verb+/+'' para se referir a um caminho subdiretório aninhado como mostrado na linha~\ref{currdir} no listado \textit{Navegando pela árvore de diretórios}.

A árvore diretório refere-se à organização aninhada de diretórios no sistema de arquivos (Figura~\ref{arboldir}), semelhante à organização de diretórios no Microsoft Windows\texttrademark que pode ser visto com o \textbf{Windows Explorer}.

\begin{figure}[ht]
\centering
   \includegraphics[width=4cm]{Figs/arboldir.png}
  \caption{\label{arboldir}Árvore de diretórios no Linux}
\end{figure}

Com o comando ``listar'' (Línea~\ref{ls}) exibe os diretórios e arquivos que estão no diretório atual. Este comando recebe argumentos/opções que permitem obter mais informações sobre arquivos e diretórios. Uma das opções mais utilizadas é `\Verb+-l+' (``menos ele''; Linha\ref{lsl}), cuja saída é exibida nas linhas \ref{lslout11} ate \ref{lslout22}, onde a lista de diretórios no local atual é exibida, juntamente com permissões nesses diretórios, o número de subdiretórios, tamanho, data da última modificação e nome.

\begin{Verbatim}[commandchars=!\{\},numbers=left,firstnumber=last,label=Navegando pela árvore de diretórios,frame=topline,fontsize=\scriptsize]
!textcolor{red}{[user@server:~]$} pwd !label{pwd}
/home/user !label{currdir}
!textcolor{red}{[user@server:~]$} ls  !label{ls}
dia1 dia2
!textcolor{red}{[user@server:~]$} ls -l  !label{lsl}
total 0  !label{lslout11}
drwxr-xr-x  2 user  group  68 Aug  5 09:01 dia1/
drwxr-xr-x  2 user  group  68 Aug  5 09:02 dia2/ !label{lslout22}
!textcolor{red}{[user@server:~/dia1]$} cd dia1 !label{cddia}
!textcolor{red}{[user@server:~]$} cd ..
!textcolor{red}{[user@server:~]$} cd /home/user/dia2/ !label{cdabsolutepath}
\end{Verbatim} 

Como mencionado acima, os sistemas Unix são multi-usuários, o que implica que deve haver um sistema de permissão no sistema de arquivos, para evitar perdas acidentais de dados, e.g., que um usuário delete dados de outro. Na linha~\ref{lslout2} as permissões de diretório são exibidas \Verb+dia2+ na primeira corda antes do primeiro espaço. O primeiro caractere indica se estamos em n diretório (\Verb+d+), um arquivo (\Verb+-+), ou um link (\Verb+l+). os 9 caracteres a seguir são divididos em 3 grupos de 3 caracteres cada, como mostrado na figura~\ref{permisos}\footnote{Exercício opcional: {\textquestiondown}Como alterar as permissões de um arquivo ou diretório?}.

\begin{figure}[ht]
\centering
   \includegraphics[width=5cm]{Figs/permisos.png}
  \caption[Sistema de permissão no Linux]{\label{permisos}Sistema de permissão no Linux. r: permissão de leitura; w: permissão de escrita; x: permissão de Execução.}
\end{figure}

Já sabemos como exibir informações sobre diretórios e arquivos na localização atual. Para alterar o diretório usamos o comando `\Verb+cd nome_direitorio+', como se mostra na linha~\ref{cddia}. se você quiser subir um nível na hierarquia do diretório executar o comando \Verb+cd ..+, outra opção é usar o caminho absoluto do diretório que você quer alcançar, como mostrado na linha~\ref{cdabsolutepath}. Retornar ao subdiretório \Verb+/home/usuario/dia1+.

Antes de continuar, eu gostaria de introduzir o comando mais importante de qualquer sistema Unix, é o comando ``manual'', que mostra informações sobre o uso dos diferentes comandos, por favor, use-os sempre que você tiver alguma dúvida sobre as opções ou sintaxe de qualquer comando, e.g., \Verb+man ls+.

\subsection{Organizando arquivos}

As operações mais comuns com arquivos são: copiar, mover e excluir. A sintaxe dos comandos para mover ou copiar é a mesma: ``comando fonte destino''. Por exemplo, suponha que você tem um arquivo chamado ''test1.txt'' em seu diretório home e você quer movê-lo para o diretório``\Verb+~/dia1/+'', você teria que executar o comando mostrado na linha~\ref{movertest}. Você pode criar e remover diretórios (vazios) usando os comandos \Verb+mkdir+ y \Verb+rmdir+, respectivamente.

\begin{Verbatim}[commandchars=!\{\},numbers=left,firstnumber=last,label=Organizando aquivos e direitorios,frame=topline,fontsize=\scriptsize]
!textcolor{red}{[user@server:~]$} cd
!textcolor{red}{[user@server:~]$} ls -l
total 0  !label{lslout1}
drwxr-xr-x  2 user  group  68 Aug  5 09:01 dia1/
drwxr-xr-x  2 user  group  68 Aug  5 09:02 dia2/
!textcolor{red}{[user@server:~]$} touch test1.txt
!textcolor{red}{[user@server:~]$} ls -l
total 0  !label{lslout2}
drwxr-xr-x  2 user  group  68 Aug  5 09:01 dia1/
drwxr-xr-x  2 user  group  68 Aug  5 09:02 dia2/
-rw-r--r--  1 user  group   0 Aug 18 20:42 test1.txt
!textcolor{red}{[user@server:~]$} mv test1.txt dia1/ !label{movertest}
!textcolor{red}{[user@server:~]$} ls -l dia1/
total 0
-rw-r--r--  1 user  group  0 Aug 18 20:42 test1.txt
!textcolor{red}{[user@server:~]$} ls -l
drwxr-xr-x  2 user  group  68 Aug  5 09:01 dia1/
drwxr-xr-x  2 user  group  68 Aug  5 09:02 dia2/
!textcolor{red}{[user@server:~]$}
\end{Verbatim} 

\subsection{Algumas operações básicas com arquivos}
Usando alguns comandos UNIX podemos obter informações sobre arquivos, e as informações que eles contêm, de forma rápida e eficiente, muitas vezes não é necessário abrir o arquivo, que pode ter vários megabytes, para obter essas informações.

No subdiretório``\Verb+~/dia1/+'', encontra o arquivo ``TAIR9\_pep\_20090619'', que corresponde ao banco de dados de sequências proteicas previstas no genoma da planta modelo \textit{Arabidopsis thaliana}. para saber quantas linhas este arquivo tem execute o comando mostrado na linha~\ref{wc}.

Porque as diferenças nas saídas dos comandos executados nas linhas~\ref{wc}~e~\ref{wcl}\footnote{Revise a página do manual: \Verb+man wc+}?

Como mostrado na linha~\ref{tamanoarapep}, o tamanho deste banco de dados é de $18.173.159$ bytes. Para saber o quanto isso corresponde em uma unidade mais amigável use o comando mostrado na linha~\ref{lslh}.

Na maioria dos casos é importante ver como o arquivo é, seja no seu início ou no final, mas devido ao grande tamanho dos arquivos com os quais você normalmente trabalha, não é conveniente abrir o arquivo com qualquer editor de texto, pois isso poderia reduzir o tempo de resposta do computador. Os comandos exibidos nas linhas~\ref{head10}~y~\ref{tail10}, mostram a primeira e últimas 10 linhas no arquivo, respectivamente.

Usando o comando \Verb+grep+, como mostrado na linha~\ref{grepsimple}, você pode obter uma lista das linhas no arquivo de interesse que contêm um determinado padrão, i.e., uma sequência de texto específica.

\begin{Verbatim}[commandchars=!\{\},numbers=left,firstnumber=last,label=Operações básicas com arquivos,frame=topline,fontsize=\scriptsize]
!textcolor{red}{[user@server:~]$} cd dia1/
!textcolor{red}{[user@server:~/dia1]$} ls -l
total 35496
-rw-r--r--  1 user group  18173159 Aug 30 16:14 TAIR9_pep_20090619 !label{tamanoarapep}
-rw-r--r--  1 user group                0 Aug 18 20:42 test1.txt
!textcolor{red}{[user@server:~/dia1]$} wc TAIR9_pep_20090619 !label{wc}
  274243  790613 18173159 TAIR9_pep_20090619
!textcolor{red}{[user@server:~]$} wc -l TAIR9_pep_20090619 !label{wcl}
  274243 TAIR9_pep_20090619
!textcolor{red}{[user@server:~/dia1]$} ls -lh !label{lslh}
total 35496
-rw-r--r--  1 user group    17M Aug 30 16:14 TAIR9_pep_20090619
-rw-r--r--  1 user group       0B Aug 18 20:42 test1.txt
!textcolor{red}{[user@server:~/dia1]$} head TAIR9_pep_20090619 !label{head10}
>AT1G51370.2 | Symbols:  | F-box family protein
MVGGKKKTKICDKVSHEEDRISQLPEPLISEILFHLSTKDSVRTSALSTKWRYLWQSVPG
LDLDPYASSNTNTIVSFVESFFDSHRDSWIRKLRLDLGYHHDKYDLMSWIDAATTRRIQH
LDVHCFHDNKIPLSIYTCTTLVHLRLRWAVLTNPEFVSLPCLKIMHFENVSYPNETTLQK
LISGSPVLEELILFSTMYPKGNVLQLRSDTLKRLDINEFIDVVIYAPLLQCLRAKMYSTK
NFQIISSGFPAKLDIDFVNTGGRYQKKKVIEDILIDISRVRDLVISSNTWKEFFLYSKSR
PLLQFRYISHLNARFYISDLEMLPTLLESCPKLESLILVMSSFNPS*
>AT1G50920.1 | Symbols:  | GTP-binding protein-related
MVQYNFKRITVVPNGKEFVDIILSRTQRQTPTVVHKGYKINRLRQFYMRKVKYTQTNFHA
KLSAIIDEFPRLEQIHPFYGDLLHVLYNKDHYKLALGQVNTARNLISKISKDYVKLLKYG
!textcolor{red}{[user@server:~/dia1]$} tail TAIR9_pep_20090619 !label{tail10}
LLRYLTI*
>ATMG00070.1 | Symbols: NAD9 | NADH dehydrogenase subunit 9
MDNQFIFKYSWETLPKKWVKKMERSEHGNRSDTNTDYLFQLLCFLKLHTYTRVQVSIDIC
GVDHPSRKRRFEVVYNLLSTRYNSRIRVQTSADEVTRISPVVSLFPSAGRWEREVWDMFG
VSFINHPDLRRISTDYGFEGHPLRKDLPLSGYVQVRYDDPEKRVVSEPIEMTQEFRYFDF
ASPWEQRSDG*
>ATMG00130.1 | Symbols: ORF121A | hypothetical protein
MASKIRKVTNQNMRINSSLSKSSTFSTRLRITDSYLSSPSVTELAPLTLTTGDDFTVTLS
VTPTMNSLESQVICPRAYDCKERIPPNQHIVSLELTYHPASIEPTATGSPETRDPDPSAY
A*
!textcolor{red}{[user@server:~/dia1]$} grep ">" TAIR9_pep_20090619 | head -n 4  !label{grepsimple}
>AT1G51370.2 | Symbols:  | F-box family protein
>AT1G50920.1 | Symbols:  | GTP-binding protein-related
>AT1G36960.1 | Symbols:  | unknown protein
>AT1G44020.1 | Symbols:  | DC1 domain-containing protein
\end{Verbatim} 

Nem sempre na bioinformática lidamos com sequências, em muitos casos temos dados em forma tabular, onde os campos são separados por algum caractere definido, por exemplo, Guias ou vírgulas. Na maioria dos casos, isso envolve armazenar e gerenciar os dados usando um sistema de banco de dados, como o MySQL. No entanto, é importante ter uma ideia dos resultados antes de integrá-los ao sistema de banco de dados, uma opção que apareceu recentemente, voltada para biólogos que trabalham com grandes quantidades de dados, é o Scriptome\footnote{\url{http://sysbio.harvard.edu/csb/resources/computational/scriptome/UNIX/}}, em que o autor oferece uma coleção de scripts PERL que podem ser executados na linha de comando. Nas linhas~\ref{scriptomeexample1}~ate~\ref{scriptomeexample2} pode ver um exemplo onde todos os caracteres são alterados para maiúsdia, o comando tem que ser executado em uma única linha, aqui é mostrado em linhas separadas apenas para facilitar sua visualização.

\begin{Verbatim}[commandchars=!\{\},numbers=left,firstnumber=last,label=Exemplo do Scriptome,frame=topline,fontsize=\scriptsize]
!textcolor{red}{[user@server:~/dia1]$} perl -e ' while(<>) !{print lc($_);!} \  !label{scriptomeexample1}
warn "Changed $. lines to lower case\n" ' \
!textcolor{blue}{TAIR9_pep_20090619} > !textcolor{blue}{TAIR9_pep_20090619.lc}    !label{scriptomeexample2}
changed 274243 lines to lower case
!textcolor{red}{[user@server:~/dia1]$} ls -l
total 70992
-rw-r--r--  1 user  group  18173159 Aug 30 16:14 TAIR9_pep_20090619
-rw-r--r--  1 user  group  18173159 Aug 30 19:54 TAIR9_pep_20090619.lc
-rw-r--r--  1 user  group                0 Aug 18 20:42 test1.txt
!textcolor{red}{[user@server:~/dia1]$} head -n 2 TAIR9_pep_20090619.lc
>at1g51370.2 | symbols:  | f-box family protein
mvggkkktkicdkvsheedrisqlpepliseilfhlstkdsvrtsalstkwrylwqsvpg
!textcolor{red}{[user@server:~/dia1]$} 
\end{Verbatim} 

Na linha ~\ref{grepsimple} foi usado o símbolo ``\Verb+|+'' ou ``barra vertical'', ou ``pipe'' no UNIX, permite conectar comandos, de modo que a saída da esquerda da barra vertical sirva de entrada para o comando à direita da barra. Na linha~\ref{scriptomeexample2} foi usado o símbolo ``\Verb+>+'' para redirecionar a saída padrão do comando para um arquivo.

\section{Formatos de sequência\label{sequenceformats}}

Existem diferentes formatos para sequências, geralmente em texto simples. O que significa que eles podem ser vistos e editados com qualquer editor de texto, como \Verb+vi+ o \Verb+pico+. alguns desses formatos são mais comuns do que outros e muitos programas de bioinformática aceitam vários dos formatos mais comuns. \citep{Leonard2007}.

Todos os formatos de sequência têm uma característica (campo) em comum: um identificador para cada sequência. Para que possa ser reconhecido inequivocamente.

\subsection{Fasta}

O formato mais simples é conhecido como Fasta\footnote{\url{http://www.ncbi.nlm.nih.gov/blast/fasta.shtml}}. Em que uma entrada, sequência, pode ser dividida em duas partes: A linha de identificação, que \textbf{deve} começar com símbolo ``\Verb+>+'' e imediatamente seguido pelo identificador de sequência (Ver linha~\ref{fastaid}), qpode ser qualquer sequência de caracteres sem espaços. As linhas imediatamente após o identificador correspondem à sequência em si (Líneas~\ref{fastastartseq}-\ref{fastaendseq}).

Fasta é o formato de sequência mais usado em aplicações na bioinformática.
 
\begin{Verbatim}[commandchars=!\{\},numbers=left,firstnumber=last,label=Secuencia en formato FastA,frame=topline,fontsize=\scriptsize]
>gi|110742030|dbj|BAE98952.1| putative NAC domain protein [Arabidopsis thaliana] !label{fastaid}
MEDQVGFGFRPNDEELVGHYLRNKIEGNTSRDVEVAISEVNICSYDPWNLRFQSKYKSRDAMWYFFSRRE !label{fastastartseq}
NNKGNRQSRTTVSGKWKLTGESVEVKDQWGFCSEGFRGKIGHKRVLAFLDGRYPDKTKSDWVIHEFHYDL
LPEHQRTYVICRLEYKGDDADILSAYAIDPTPAFVPNMTSSAGSVVNQSRQRNSGSYNTYSEYDSANHGQ
QFNENSNIMQQQPLQGSFNPLLEYDFANHGGQWLSDYIDLQQQVPYLAPYENESEMIWKHVIEENFEFLV
DERTSMQQHYSDHRPKKPVSGVLPDDSSDTETGSMIFEDTSSSTDSVGSSDEPGHTRIDDIPSLNIIEPL
HNYKAQEQPKQQSKEKVISSQKSECEWKMAEDSIKIPPSTNTVKQSWIVLENAQWNYLKNMIIGVLLFIS
VISWIILVG !label{fastaendseq}
\end{Verbatim} 

\subsection{GenBank}

O formato GenBank\footnote{\url{http://www.ncbi.nlm.nih.gov/Sitemap/samplerecord.html}}\footnote{\url{ftp://ftp.ncbi.nih.gov/genbank/release.notes/gb172.release.notes}} é usado pelo ``National Center for Biotechnology Information'' (NCBI\footnote{\url{http://www.ncbi.nlm.nih.gov/}}), o maior repositório de sequências, tanto ácidos nucleicos quanto proteínas, em todo o mundo. O NCBI juntamente com o EMBL\footnote{\url{http://www.ebi.ac.uk/embl/}} e o DDBJ\footnote{\url{http://www.ddbj.nig.ac.jp/}},  manter em conjunto ``The International Nucleotide Sequence Database'' \citep{Mizrachi2008}.

Uma entrada neste formato é composta por duas partes. A primeira parte consiste em posições de 1 a 10, e geralmente contém o nome do campo, e.g., \Verb+LOCUS+, \Verb+DEFINITION+, \Verb+ACCESSION+ o \Verb+SOURCE+. A segunda parte de cada entrada contém as informações para o campo correspondente. Cada entrada termina com o símbolo ``\Verb+\\+'' (Linha~\ref{endgenbank}). Você pode encontrar mais informações sobre este tipo de arquivo seguindo o link \url{http://www.ncbi.nlm.nih.gov/Sitemap/samplerecord.html}


\begin{Verbatim}[commandchars=!\{\},numbers=left,firstnumber=last,label=Sequência em formato GenBank,frame=topline,fontsize=\tiny]
LOCUS       BAE98952                 429 aa            linear   PLN 27-JUL-2006
DEFINITION  putative NAC domain protein [Arabidopsis thaliana].
ACCESSION   BAE98952
VERSION     BAE98952.1  GI:110742030
DBSOURCE    accession AK226863.1
KEYWORDS    .
SOURCE      Arabidopsis thaliana (thale cress)
  ORGANISM  Arabidopsis thaliana
            Eukaryota; Viridiplantae; Streptophyta; Embryophyta; Tracheophyta;
            Spermatophyta; Magnoliophyta; eudicotyledons; core eudicotyledons;
            rosids; eurosids II; Brassicales; Brassicaceae; Arabidopsis.
REFERENCE   1
  AUTHORS   Totoki,Y., Seki,M., Ishida,J., Nakajima,M., Enju,A., Morosawa,T.,
            Kamiya,A., Narusaka,M., Shin-i,T., Nakagawa,M., Sakamoto,N.,
            Oishi,K., Kohara,Y., Kobayashi,M., Toyoda,A., Sakaki,Y.,
            Sakurai,T., Iida,K., Akiyama,K., Satou,M., Toyoda,T., Konagaya,A.,
            Carninci,P., Kawai,J., Hayashizaki,Y. and Shinozaki,K.
  TITLE     Large-scale analysis of RIKEN Arabidopsis full-length (RAFL) cDNAs
  JOURNAL   Unpublished
REFERENCE   2  (residues 1 to 429)
  AUTHORS   Totoki,Y., Seki,M., Ishida,J., Nakajima,M., Enju,A., Morosawa,T.,
            Kamiya,A., Narusaka,M., Shin-i,T., Nakagawa,M., Sakamoto,N.,
            Oishi,K., Kohara,Y., Kobayashi,M., Toyoda,A., Sakaki,Y.,
            Sakurai,T., Iida,K., Akiyama,K., Satou,M., Toyoda,T., Konagaya,A.,
            Carninci,P., Kawai,J., Hayashizaki,Y. and Shinozaki,K.
  TITLE     Direct Submission
  JOURNAL   Submitted (26-JUL-2006) Motoaki Seki, RIKEN Plant Science Center;
            1-7-22 Suehiro-cho, Tsurumi-ku, Yokohama, Kanagawa 230-0045, Japan
            (E-mail:mseki@psc.riken.jp, URL:http://rarge.gsc.riken.jp/,
            Tel:81-45-503-9625, Fax:81-45-503-9586)
COMMENT     An Arabidopsis full-length cDNA library was constructed essentially
            as reported previously (Seki et al. (1998) Plant J. 15:707-720;
            Seki et al. (2002) Science 296:141-145).
            This clone is in a modified pBluescript vector.
            Please visit our web site (http://rarge.gsc.riken.jp/) for further
            details.
FEATURES             Location/Qualifiers
     source          1..429
                     /organism="Arabidopsis thaliana"
                     /db_xref="taxon:3702"
                     /chromosome="1"
                     /clone="RAFL08-19-M04"
                     /ecotype="Columbia"
                     /note="common name: thale cress"
     Protein         1..429
                     /product="putative NAC domain protein"
     Region          5..137
                     /region_name="NAM"
                     /note="No apical meristem (NAM) protein; pfam02365"
                     /db_xref="CDD:111274"
     CDS             1..429
                     /gene="At1g01010"
                     /coded_by="AK226863.1:89..1378"
ORIGIN      
        1 medqvgfgfr pndeelvghy lrnkiegnts rdvevaisev nicsydpwnl rfqskyksrd
       61 amwyffsrre nnkgnrqsrt tvsgkwkltg esvevkdqwg fcsegfrgki ghkrvlafld
      121 grypdktksd wvihefhydl lpehqrtyvi crleykgdda dilsayaidp tpafvpnmts
      181 sagsvvnqsr qrnsgsynty seydsanhgq qfnensnimq qqplqgsfnp lleydfanhg
      241 gqwlsdyidl qqqvpylapy enesemiwkh vieenfeflv dertsmqqhy sdhrpkkpvs
      301 gvlpddssdt etgsmifedt ssstdsvgss depghtridd ipslniiepl hnykaqeqpk
      361 qqskekviss qksecewkma edsikippst ntvkqswivl enaqwnylkn miigvllfis
      421 viswiilvg
// !label{endgenbank}
\end{Verbatim} 

\subsection{Algumas operações básicas com sequências no formato Fasta}

Para o restante desta seção, e para a próxima, usaremos apenas sequências no formato Fasta. Por favor, verifique se as sequências de \textit{A. thaliana} no arquivo \Verb+TAIR9_pep_20090619+ estão neste formato. Você pode usar o comando ``\Verb+head nome_arquivo+'', ou o comando ``\Verb+less nome_arquivo+''\footnote{Para sair de \Verb+less+ pressione ``\Verb+q+''}.

Você já teve que contar o número de sequências ou alterar o identificador de sequência no formato Fasta? Se for uma dúzia de sequências, isso poderia facilmente ser feito em qualquer editor de texto, mas quando há milhares de sequências a opção do editor de texto deixa de ser viável. Felizmente, alguns comandos Unix nos permitem executar essas tarefas simples rapidamente.

Como viu na linha~\ref{grepsimple}, o comando ``\Verb+grep+'' poderia nos ajudar a contar o número de sequências em um arquivo Fasta. o interruptor ``\Verb+-c+'' conta o número de linhas contendo um determinado padrão em um arquivo, e podemos tirar proveito do fato de que em um arquivo Fasta o símbolo ``\Verb+>+'' aparece apenas uma vez para cada sequência como mostrado na linha~\ref{contarseqs}.

\begin{Verbatim}[commandchars=!\{\},numbers=left,firstnumber=last,label=Usando comandos Unix com arquivos Fasta,frame=topline,fontsize=\scriptsize]
!textcolor{red}{[user@server:~]$} cd ~/dia1/
!textcolor{red}{[user@server:~/dia1]$} ls -l
total 70992
-rw-r--r--  1 user  group  18173159 Aug 30 16:14 TAIR9_pep_20090619
-rw-r--r--  1 user  group  18173159 Aug 30 19:54 TAIR9_pep_20090619.lc
-rw-r--r--  1 user  group                0 Aug 18 20:42 test1.txt
!textcolor{red}{[user@server:~/dia1]$} grep -c ">" TAIR9_pep_20090619 !label{contarseqs}
33410
!textcolor{red}{[user@server:~/dia1]$} sed 's/>/>ATH_/' TAIR9_pep_20090619 > TAIR9_pep_20090619.mod !label{sedsp}
!textcolor{red}{[user@server:~/dia1]$} head TAIR9_pep_20090619.mod
>ATH_AT1G51370.2 | Symbols: 
MVGGKKKTKICDKVSHEEDRISQLPEPLISEILFHLSTKDSVRTSALSTKWRYLWQSVPG
LDLDPYASSNTNTIVSFVESFFDSHRDSWIRKLRLDLGYHHDKYDLMSWIDAATTRRIQH
!textcolor{red}{[user@server:~/dia1]$}
\end{Verbatim} 

Em outras ocasiões é importante modificar o identificador de cada sequência, de modo que inclua, por exemplo, uma abreviação que represente o nome da espécie a que a sequência pertence. Novamente Unix nos permite fazer essa mudança muito rapidamente usando o comando \Verb+sed+ como mostrado na linha~\ref{sedsp}.

\chapter{Buscas em banco de dados biológicos}

Este capítulo corresponde a uma versão modificada de um guia original da professora Silvia Restrepo

\section{NCBI – Bancos de dados e busca de informações}

O National Center for Biotechnology Information, NCBI pela sigla em inglês, é uma instituição pública dos Estados Unidos da América, que guarda todas as informações sobre os genomas de várias espécies, bem como o maior banco de dados público sobre sequências de DNA e proteínas. Sua página principal da rede está localizada no seguinte link:
\url{http://www.ncbi.nlm.nih.gov/}

Este site conecta todos os dados disponíveis em seus servidores (PubMed, TODOS os bancos de dados (Entrez), Blast, OMIM, Books, TaxBrowser, Structure), conforme mostrado na Figura \ref{screenshotNCBI}. Embora o Entrez esteja listado como um dos serviços, na realidade quase todos eles dependem diretamente do Entrez. Por exemplo, PubMed e Taxonomy estão intimamente ligados ao Entrez. 

\begin{figure}[ht]
\centering
   \includegraphics[width=15cm]{Figs/NCBIStart.png}
  \caption{\label{screenshotNCBI}Página inicial do NCBI}
\end{figure}

\subsection{Vamos começar uma visita aos seus bancos de dados}

Como primeiro passo, vamos entrar no PubMed. Esta base de dados contém informações sobre publicações científicas, e seus registros foram compilados pela NLM (National Library of Medicine), com a colaboração dos editores. Lá você encontrará a maioria das referências necessárias, incluindo o resumo (Abstract) e em alguns casos a publicação gratuita. 

Para obter ajuda sobre como realizar buscas consulte o seguinte link:  \url{http://www.ncbi.nlm.nih.gov/bookshelf/br.fcgi?book=helppubmed}

As páginas possuem um menu de banco de dados em uma barra superior, as pesquisas devem ser colocadas na janela mostrada na Figura \ref{screenshotsearchbox}.

\begin{figure}[ht]
\centering
   \includegraphics[width=15cm]{Figs/NCBISearchBox.png}
  \caption{\label{screenshotsearchbox}Janela de busca do NCBI }
\end{figure}
 
Uma busca deve ter um formato semelhante a este:

\begin{quote}
``\textbf{palavrachave}''[field] \textbf{operador lógico}  ``\textbf{palavrachave}''[field] \ldots
\end{quote}

Onde \textbf{palavra chave} é a palavra utilizada para identificar um registro (record) de acordo com o campo (field) utilizado. Por exemplo, uma palavra chave pode ser "Silva" no campo "authors". \textbf{Operador lógico} é qualquer um destes operadores booleanos: AND, OR, NOT, BUT, etc. Ao substituir por suas próprias palavras-chave no formato acima, lembre-se de que os campos devem estar entre colchetes [ ], mas os operadores são independentes (sem os símbolos, "" ), além disso, as aspas na palavra chave são opcionais, mas cumprem a função de forçar uma busca com a palavra exata ao invés de serem flexíveis. 

Por exemplo, se eu quiser pesquisar todos os artigos de 1999 publicados por Silva et al na revista Science, eu uso o seguinte comando: "Silva"[AU] AND 1999[DP] AND "science"[TA]. Quanto mais informações forem inseridas na busca, mais restrita será a resposta (por exemplo, se eu incluir mais autores). 

Os campos mais comuns que podem ser solicitados no PubMed são os seguintes:

\begin{description}
\item[All Fields [ALL]]  Inclui todos os campos pesquisáveis do PubMed. No entanto, apenas os termos em que não houver correspondência encontrada em uma das tabelas ou índices de tradução por meio do processo de Mapeamento Automático de Termos serão pesquisados em Todos os Campos. PubMed ignora palavras irrelevantes de consultas de pesquisa.  
\item[Author Name [AU]] Vários limites no número de nomes de autores incluídos na citação MEDLINE existiram ao longo dos anos (consulte a política NLM sobre nomes de autores). MEDLINE não lista o nome completo. O formato para pesquisar o nome do autor é: sobrenome seguido de espaço e até as duas primeiras iniciais seguidas de espaço e abreviação do sufixo, se for o caso, tudo sem pontos ou vírgula após o sobrenome (por exemplo, fauci as ou o'brien jc jr). Iniciais e sufixos podem ser omitidos durante a pesquisa. O PubMed trunca automaticamente o nome de um autor para levar em conta as iniciais variadas, por exemplo, o'brien j [au] recuperará o'brien ja, o'brien jb, o'brien jc jr, bem como o'brien j. Para desativar esse truncamento automático, coloque o nome do autor entre aspas duplas e qualifique com [au] entre colchetes, por exemplo, "o'brien j" [au] para recuperar apenas o'brien j. 
\item[EC/RN Number [RN]] Número atribuído pela Enzyme Commission para designar uma enzima específica ou pelo Chemical Abstracts Service (CAS) para números de registro.  
\item[Entrez Date [EDAT]] Data em que a citação foi adicionada ao banco de dados PubMed. As citações são exibidas na ordem Entrez Date, que é o último a entrar, o primeiro a sair. As datas ou intervalos de datas devem ser inseridos usando o formato AAAA/MM/DD [edat], ex. 1998/04/06 [ed.] . O mês e o dia são opcionais (por exemplo, 1998 [edat] ou 1998/03 [edat]). Para inserir um intervalo de datas, insira dois pontos (:) entre cada data (por exemplo, 1996:1997 [edat] ou 1998/01:1998/04 [edat])
\item[Issue [IP]] O número do número da revista em que o artigo é publicado.  
\item[Journal Title [TA]] A abreviatura do título do periódico, nome completo do periódico ou número ISSN .
\item[Language [LA]] 
\item[Publication Date [DP]] A data em que o artigo foi publicado. As datas ou intervalos de datas devem ser pesquisados usando o formato AAAA/MM/DD [dp], ex. 1998/03/06 [dp] . O mês e o dia são opcionais (por exemplo, 1998 [dp] ou 1998/03 [dp]). Para inserir um intervalo de datas, insira dois pontos (:) entre cada data (por exemplo, 1996:1998 [dp] ou 1998/01:1998/04 [dp]). 
O nome de um produto químico discutido no artigo. Sinônimos para o Nome da Substância do Conceito Complementar serão mapeados automaticamente quando qualificados com [nm]. Este campo foi implementado em meados de 1980. Muitos nomes químicos são pesquisáveis como termos MeSH antes dessa data.
\item[Text Words [TW]] Inclui todas as palavras e números no título e resumo, e termos MeSH, subtítulos, nomes de substâncias químicas, nome pessoal como assunto e campo MEDLINE Secondary Source (SI). O campo Nome pessoal do assunto também pode ser pesquisado diretamente usando a tag do campo de pesquisa [ps], por exemplo, rouxinol f [ps]. 
\item[Title Words [TI]] Palavras e números incluídos no título de uma citação. 
\item[Title/Abstract Words [TIAB]] Palavras e números incluídos no título e resumo de uma citação.
\item[Unique Identifiers [UID]]
\item[Volume [VI]] O número do volume da revista em que um artigo é publicado. 
\end{description}

Agora vamos no site onde o ENTREZ está localizado. Para fazer isso, selecione TODOS OS BANCOS DE DADOS na janela do banco de dados na página principal. Entrez é um sistema de busca de sequências armazenadas em bancos de dados.
Consultas sofisticadas podem ser solicitadas para obter um conjunto de sequências de seu interesse, por exemplo, posso pedir para exibir todas as sequências genômicas de Arabidopsis que foram incluídas no banco de dados entre os anos 97' e 99' que também contêm anotação (em a tabela "features") nas regiões promotoras. A Figura \ref{screenshotentrez} mostra a página de login do servidor Entrez.

\begin{figure}[ht]
\centering
   \includegraphics[width=15cm]{Figs/screenshotEntrez.png}
  \caption{\label{screenshotentrez}Página inicial do Entrez}
\end{figure}

Assim, em um único site podemos realizar buscas simultaneamente em todas as bases de dados ou selecionar uma única base de dados e realizar uma busca por base de dados. 

Na caixa de busca, as sequências podem ser consultadas usando seus números identificadores (como o gi-number ou com o número de acesso). Questões mais complicadas também podem ser formuladas usando a sintaxe entrez, semelhante a como vimos o PubMed: 

\begin{quote}
``\textbf{palavrachave}''[field] \textbf{operador lógico} ``\textbf{palavrachave}''[field] \ldots
\end{quote}

Para mais informações sobre o Entrez você pode seguir o link:
\url{http://www.ncbi.nlm.nih.gov/bookshelf/br.fcgi?book=helpentrez&part=EntrezHelp}

{\color{red}
\subsubsection{Exercícios}

\begin{enumerate}
\item Qual é a classificação taxonômica da alga \textit{Chlamydomonas reinhardtii}?, e quais outras plantas estão próximas, para que possam ser usadas como fonte de marcadores?. Quantas sequências de proteínas estão presentes no GenBank para a espécie \textit{Chlamydomonas reinhardtii}? 
\item Acesse a página do PubMed e obtenha as referências que tratam da biologia molecular e/ou genética da mandioca (\textit{Manihot esculenta}). Quantos foram publicados nos últimos dois anos e de quais laboratórios (ou regiões geográficas) são os autores? Explique como você pesquisou. Dica: GoPubMed \url{http://www.gopubmed.org/}
\item Use o Entrez para encontrar todas as sequências EST (Expressed Sequence Tag) de arroz que foram depositadas no banco de dados.
\end{enumerate}
}

Revise a descrição dos principais formatos de sequências na seção
\ref{sequenceformats}.

\subsubsection{Quais bancos de dados encontramos no NCBI?}

O NCBI possui um grande número de bancos de dados. O mais conhecido no GenBank que contém todas as sequências de nucleotídeos. GenPept contém as sequências de proteínas. Outras bases de dados são Genome, Structure, PubMed

No GenBank as sequências estão organizadas em 17 divisões, 11 tradicionais e 6 Bulk
Nas tradicionais, as sequências foram enviadas diretamente pelos pesquisadores, são caracterizadas e as divisões são:

\begin{description}
\item[PRI] primatas
\item[PLN] plantas
\item[BCT] bactérias
\item[INV] invertebrados
\item[ROD] Roedores
\item[VRL] Viral
\item[VRT] outros vertebrados
\item[MAM] Mamíferos (Ej. ROD + PRI)
\item[PHG] Fagos
\item[SYN] Sintético (vetores de clonagem, etc)
\item[UNA] sem anotação 
\end{description}


O Bulk consiste em sequências enviadas em grupos via email ou ftp, imprecisas e mal caracterizadas, são elas: 

\begin{description}
\item[dbEST] Banco de dados EST, tags de sequência expressa
\item[dbSTS] Sequence-tagged sites: são marcos genômicos curtos para os quais há informações de sequência e mapa. 
\item[dbGSS] Genomic survey sequences. Contém: dados de sequência do genoma de etapa única, sequências terminais BAC, YAC e cosmídeos, sequências de éxon
\item[dbHTGS] High-Throughput Genomic Sequences. Ele foi criado para salvar informações de sequenciamento de genoma que não foram finalizadas ou curadas, mas para torná-las conhecidas da comunidade científica assim que estiverem disponíveis.
\end{description}

e também existem bancos de dados para:

\begin{description}
\item[HTC] High Throughput cDNA
\item[PAT] Patent
\end{description}

\subsubsection{RefSeq}

Queremos colocar ênfase especial em um banco de dados NCBI chamado RefSeq. Este banco de dados foi criado para obter uma coleção biologicamente não redundante de sequências de DNA, RNA e proteínas. Cada RefSeq (sequência de referência) representa uma molécula única que ocorre naturalmente em um organismo. Esta base de dados é do tipo com curadoria de pesquisadores. Cada molécula não é um resultado de pesquisa, mas sim uma síntese de informações.

Vamos voltar para a página principal do NCBI e na janela de busca, deixando all databases, digite NC\_001139\footnote{certifique-se de incluir o símbolo de sublinhado}. Vemos que em Nucleotide temos 1 hit, assim como em Genome e em Gene temos 631.

Vamos abrir Nucleotide: obtemos um flatfile de sequência  que corresponde à sequência completa do cromossomo VII da levedura. Vamos dar uma olhada no arquivo flatfile, \textcolor{red}{quais informações ele contém?} 

Observemos que os identificadores desta base de dados mudam e são do tipo 2+6 com duas letras e 6 números, a tabela a seguir nos mostra o que significam essas letras: 

\begin{tabular}{|c|c|}
\hline  \multicolumn{2}{|l|}{\textbf{mRNA and Proteins}} \\
\hline  NM\_123456 & Curated mRNA  \\ 
\hline  NP\_123456 &  Curated Protein\\ 
\hline  NR\_123456 &  Curated non-coding RNA\\ 
\hline  XM\_123456 &  Predicted mRNA\\ 
\hline  XP\_123456 &  Predicted Protein\\ 
\hline  XR\_123456 &  Predicted non-coding RNA\\ 
\hline   \multicolumn{2}{|l|}{\textbf{Gene records}} \\
\hline  NG\_123456 & Genomic Region \\ 
\hline  \multicolumn{2}{|l|}{\textbf{Chromosome}} \\ 
\hline  NC\_123456 & Complete genomic molecule, Microbial replicons, organelle genomes  \\ 
\hline  \multicolumn{2}{|l|}{\textbf{Assemblies}} \\ 
\hline  NT\_123456 & Contig \\ 
\hline  NW\_123456 & WGS supercontig (assembly of WGS) \\ 
\hline 
\end{tabular} 

\subsection{Recuperação de Sequências no NCBI com buscas mais específicas}

\paragraph{CONHECEMOS O ORGANISMO.} As pesquisas do NCBI podem ser mais direcionadas se conhece o organismo sobre o qual estamos procurando informações. Entramos na página inicial do NCBI, vamos para TaxBrowser, colocamos o nome do organismo que estamos procurando. Ao selecioná-lo, uma tabela do número de sequências por tipo de molécula ou projeto aparece à direita. Clicar em uma delas, por exemplo proteínas, nos leva diretamente às proteínas daquele organismo. 

\paragraph{NÓS SABEMOS OS NÚMEROS DE ACESSO.} Se você souber o número de acesso diretamente, pode colocá-lo na janela de pesquisa da página principal do NCBI. Para várias sequências os números são colocados com a palavra OR entre eles, por exemplo AJ487842 ou AJ487843. Por fim, para uma sequência de números de acesso, digite: AJ487842::AJ487851[ACCN] 

\paragraph{DIRECIONAMOS A PESQUISA COM LIMITES.} Por exemplo, se eu quiser pesquisar as sequências de mRNA curadas relacionadas a um tipo de câncer em humanos, posso fazer a seguinte pesquisa: na janela de pesquisa, coloco COLON CANCER AND NONPOLYPOSIS , eu pesquiso o banco de dados de nucleotídeos. Então em LIMITS seleciono a molécula de mRNA e em only from (banco de dados) seleciono RefSeq. Então eu seleciono a outra janela Preview/index acima e lá em organismos eu escrevo humanos e seleciono AND 

\section{Recuperação de sequência usando SRS@EBI\label{srs}}

Existe no entanto uma excelente alternativa para a busca de sequências biológicas, que nos permite controlar quase todos os aspectos da nossa busca, esta alternativa é o Sequence Retrieval System (SRS). Este sistema foi desenvolvido com esta tarefa de recuperar eficazmente sequências biológicas em mente, daí o seu design e capacidades. 

Neste workshop trabalharemos com o SRS oferecido pelo European Bioinformatics Institute (EBI), \url{http://srs.ebi.ac.uk/}. Ou digitando EBI, (\url{http://www.ebi.ac.uk/}) , database~$\rightarrow$~database~browsing você chega ao SRS. 

Uma forma simples de consultar o SRS é através da caixa Quick Text Search. Nesta caixa é possível pesquisar em vários bancos de dados disponíveis no menu suspenso, conforme mostrado na Figura~\ref{screenshotSRS} 

\begin{figure}[ht]
\centering
   \includegraphics[width=15cm]{Figs/screenshorSRS.png}
  \caption{\label{screenshoteSRS}Página inicial do SRS}
\end{figure}

Por exemplo, selecionando a opção “Nucleotide Sequences”, realizaremos nossa busca no banco de dados EMBL DNA (homólogo ao genBank e DDBJ). 

Faça uma pesquisa rápida pelo HIV-1 com diferentes opções no menu suspenso. Até este ponto, o SRS parece ser um pouco menos completo em comparação com o site do NCBI, mas agora começaremos a ver onde está todo o seu potencial.

Agora vamos realizar uma busca avançada. Selecione a guia Library Page localizada na parte superior da tela e mostrada na Figura~\ref{optionsSRS} 

\begin{figure}[ht]
\centering
   \includegraphics[width=15cm]{Figs/SRSMenu.png}
  \caption{\label{opcionesSRS}Opções SRS}
\end{figure}

Você será então levado para a seção SRS onde estão descritos cada um dos bancos de dados que compõem o sistema (Figura~\ref{SRSDBs}). Como você pode ver, o SRS inclui muitos bancos de dados ao mesmo tempo e essa é uma de suas principais virtudes, por isso o SRS às vezes é conhecido como "banco de dados de bancos de dados", pois através deste sistema podemos consultar vários bancos de dados ao mesmo tempo, de acordo com nossas necessidades particulares. 

\begin{figure}[ht]
\centering
   \includegraphics[width=12cm]{Figs/SRSDBs.png}
  \caption{\label{SRSDBs}Opções SRS}
\end{figure}

Como você pode ver, o SRS é semelhante ao sistema NCBI ENTREZ, no sentido de que nos permite consultar muitas bases de dados ao mesmo tempo, mas desta vez não se restringindo apenas àquelas que o NCBI possui, mas a praticamente qualquer base de dados. A quantidade de bancos de dados que o SRS possui depende de cada implementação, ou seja, o administrador do SRS determina quais bancos de dados deseja ou não incluir em seu sistema 

Posicione o cursor do mouse sobre qualquer uma das entradas, após alguns segundos aparecerá uma caixa de texto explicativa. \textcolor{red}{Que tipo de informação os bancos de dados EMBL (Contig Updates), UniprotKB/Swissprot fornecem?} 

Ao seguir o link para qualquer uma dessas bases de dados obteremos mais informações sobre ela, como o número de entradas presentes, data de atualização, etc. No entanto, por enquanto nosso interesse é selecionar algumas bases de dados para realizar nossas buscas. Marque as caixas para os bancos de dados ``UniprotKB/Swissprot'' e ``UniprotKB/TrEMBL''. Certifique-se de que esses sejam os únicos bancos de dados selecionados. 

À esquerda de sua tela você encontrará a caixa ``Search Options'' que nos permitirá selecionar o nível de profundidade de nossa busca. Como esta é a primeira vez que trabalhamos com este sistema, selecionaremos o formulário padrão de busca. 

Pressione o botão ``\textbf{Standard query Form}'' na caixa ``\textbf{Search Options}'' 

Esta ação o levará ao formulário de busca SRS padrão (Figura~\ref{searchform}).

\begin{description}
\item[Fields you can search] Campos de busca, onde podemos inserir nossos termos de busca de acordo com qualquer uma das opções presentes nos respectivos menus suspensos. 
\item[Create View] Criar vista, esta opção funciona em conjunto com a opção 3, e aqui podemos definir o tipo de campos que queremos ver na nossa página de resultados. Para o nosso exemplo, estamos interessados em selecionar todas as proteínas de superfície conhecidas de Plasmodium falciparum com atividade imunogênica, relacionadas ao merozoíto. 
\item[Result Display Options] Opções para exibir os resultados, onde podemos definir o número de resultados que queremos por página, bem como o formato de saída, seja um dos definidos no menu suspenso ou criando uma visualização personalizada (opção ``create view'').
\item[Search Options] Opções de busca, onde podemos definir, entre outras coisas, o tipo de conector lógico (Booleano) a ser usado para os termos definidos em 1. 
\end{description}


\begin{figure}[ht]
\centering
   \includegraphics[width=12cm]{Figs/SearchFormSRS.png}
  \caption{\label{searchform}Formulário de pesquisa SRS}
\end{figure}

Defina estes critérios na seção ``Fields you can search'' de acordo com a  Figura~\ref{busquedaplasmodium}.

\begin{figure}[ht]
\centering
   \includegraphics[width=10cm]{Figs/busquedaSRSplasmodium.png}
  \caption{\label{busquedaplasmodium}Critérios de pesquisa avançados}
\end{figure}

Em seguida, pressione o botão \textbf{``search''} localizado no topo desta seção e aguarde alguns segundos.

Com certeza você já tem uma visão mais exata das possibilidades oferecidas pelo SRS e suas principais diferenças com o sistema Entrez. Primeiro, conseguimos definir exatamente não apenas o banco de dados que queríamos consultar, mas as seções específicas dele. Além disso, também conseguimos definir exatamente os termos de pesquisa em seções específicas dos posts, o que nos dá total controle sobre os resultados que queremos obter. 

Brinque com as diferentes opções de formato que o SRS oferece na seção ``\textbf{Result Display Options}'' do formulário de pesquisa. Tente também criar seu próprio formato de saída com a opção ``\textbf{Create view}''. 

\textcolor{red}{Encontre todas as proteínas nucleares hipotéticas de \textit{Saccharomyces cerevisiae} e exiba a informação em formato fasta}. 

\chapter{Manipulación básica de secuencias}

Este capítulo corresponde a una versión modificada de una guía original de la profesora Silvia Restrepo.

\section{Limpieza de secuencias}

Un Vector, es un agente que lleva fragmentos de ADN de interés a una célula específica. Si éste es utilizado para reproducir un fragmento de ADN, se le conoce como \textit{Vector de Clonación}, si se utiliza para expresar cierto gen, se conoce como \textit{Vector de Expresión}. Los vectores más usados son plásmidos, BACs, YACs, cósmidos y los bacteriófagos Lambda y P1. En cualquier caso que se utilice un vector, cuando se manda a secuenciar el fragmento de interés, se puede identificar las secuencias vector y eliminarlas. Para esto se puede emplear VecScreen siguiendo el enlace \url{http://www.ncbi.nlm.nih.gov/VecScreen/} (Figura~\ref{screenshotVecScreen}).

\begin{figure}[ht]
\centering
   \includegraphics[width=10cm]{Figs/screenshotVecScreen.png}
  \caption{\label{screenshotVecScreen}VecScreen: Herramienta para detectar contaminación de vectores.}
\end{figure}

En el campo de búsqueda que aparece en la página, pegue la Secuencia Problema 1 y ejecútela como ``\textbf{Run VecScreen}''.

En la siguiente página, deje los campos que se encuentran por defecto (para ver los resultados de manera gráfica) y dele click a ``\textbf{View report}''.

Posibles resultados:

\begin{itemize}
\item Si la secuencia NO tiene secuencias de vector contaminantes: ``Non-significant homology''.
\item Si la secuencia SI tiene secuencias de vector contaminantes: 
 \begin{itemize}
 \item Sección gráfica: con diferentes colores muestra, sobre el mapa de la secuencia problema, donde se encuentran las secuencias contaminantes.
  \item Alineamiento: se muestra el alineamiento entre la secuencia problema y las secuencias contaminantes homólogas de vectores que se encontraron.
 \end{itemize}
\end{itemize}

\begin{Verbatim}[commandchars=!\{\},numbers=none,label=Secuencia problema 1,frame=topline,fontsize=\scriptsize]
>Secuencia_Problema_1 
TCTATNGGCGATTGGGTACCGGGCCCCCCCTCGAGGTCGACGGTATCGATAAGCTTGATA
TCGAATTCATGGGATTCTTAACAACAATAGTTGCTTGTTTCATTACCTTTGCAATATTAA
TTCACTCATCCAAAGCTCAAAACTCCCCCCAAGATTATCTTAACCCTCACAATGCAGCTC
GTAGACAAGTTGGTGTTGGCCCCATGACATGGGACAATAGGCTAGCAGCCTATGCCCAAA
ATTATGCCAATCAAAGAATTGGTGACTGCGGGATGATCCACTCTCATGGCCCTTACGGCG
AAAACCTAGCCGCCGCCTTCCCTCAACTTAACGCTGCTGGTGCTGTAAAAATGTGGGTCG
ATGAGAAGCGTTTCTATGATTACAATTCAAATTCTTGTGTAGGAGGAGTATGTGGACACT
ATACTCAGGTGGTGTGGCGTAACTCAGTACGTCTCGGTTGTGCTAGGGTTCGAAGCAACA
ATGGTTGGTTTTTCATAACTTGCAATTATGATCCACCAGGTAATTTTATAGGACAACGTC
CCTTTGGCGATCTTGAGGAGCAACCCTTTGATTCCAAATTGGAACTTCCAACTGATGTCT
AAGAATTCCTGCAGCCCGGGGGATCCACTAGTTCTAGAGCGGCCGCCACCGCGGTGGAGC
TCCAGCTTTTGTTCCCTTTAGTGAGGGTTAATTTCGAGCTTGGCGTAATCATGGTCATAG
CTGTTTCCTGTGTGAAATTGTTATCCGCTCACAATTCCACACAACATACGAGCCGGAAGC
ATAAAGTGTAAAGCCTGGGGTGCCTAATGAGTGAGCTAACTCACATTAATTGCGTTGCGC
TCACTGCCCGCTTTCCAGTCGGGAAACCTGTCGNGCCAGCTGCATTAATGAATCGGCCAA
CGCGCGGGGAAAAGGCGGGTTTGGCGTATTGGGGCGCTCTTCCGCTTCCTCGCTCACTGG
ACTCNGTTGCGCTCGGTCGTTCGGCTGCGGNGAGNGGNAATCAGCCNCCCCCCAAAAGGN
GGNNAATCCGGTTANCCNCGNAATCCGGGGGAAAACNCCNNGAAAAAACNTGGGGANCAA
AAAGGNCCCCAAAAAGGGCCCAGNAACCNNNNAAAAAGGGCCNGNGTTGNNNGGGGGTTT
TNCCAAAGGGNCCCCCCCCCCGNGAANANNNNCCAAAAANTCCCCCCCTCAATCCAANGG
GGNGAAAACCCCCCGGGNANTTTAAAAANANCGGGGGTTNCCCCNGGAAAACCCCCNGGG
NCNNCCNGGTTCCNACCCGGCCCTTAANGGAAAATGNCNCCNTTT
\end{Verbatim} 

{\color{red}
En la Secuencia Problema 1, ¿Encuentra fragmentos similares a algún vector?

En el caso en que encuentre secuencias contaminantes de vectores, ¿Entre qué nucleótidos se encuentra el inserto de interés?

Elimine las secuencias contaminantes y vuelva a VecSreen con esta nueva secuencia. ¿Qué obtiene?
}

Se debe entonces proceder a limpiar la secuencia eliminando los fragmentos correspondientes a vector. 

\section{Mapa de restricción}

Los mapas de restricción sirven para verificar que la secuencia que se recibió del centro de  secuenciación en efecto corresponde a la secuencia que se mando. Igualmente se puede usar esta herramienta para verificar largas secuencias (como genomas bacterianos) que fueron ensambladas a partir de fragmentos mas cortos. El número y tamaño de los fragmentos predichos deben corresponder al mapa de restricción experimental. 

Una herramienta que permite hacer este tipo de análisis se encuentra siguiendo el enlace \url{http://biotools.umassmed.edu}, seleccionando la opción ``Restriction Mapping Tool''. 

En la siguiente página pegue la secuencia ``35\_292648\_.ab1'' y seleccione la opción ‘entire linear map’ y ‘Submit Sequence to wwwtacg’. Note que tiene la opción de escoger que enzimas de restricción desea usar.

\begin{Verbatim}[commandchars=!\{\},numbers=none,label=Secuencia 35\_292648\_.ab1,frame=topline,fontsize=\scriptsize]
>35_292648_.ab1 ABIX Testing -- no comment RESTRICCION
CGGGCGTCACCGCATTTTTTTTTTTTTTTTTTTTTTTTTTTTTAAGGGATAATCTATTTC
NCTTATTCANANAATTAGTAATTACNCATAACNCNCAACTTTGANGCCCNCATTATAANG
ATTAGCAGGNCATTATATAAGNGGGCANCCTTTTATTTCANACATTAATTACTTAATTTN
GGGCAANCCANAAAANGGACAAGTCTAGAGTCNCATTACNGGGNACATATTTGCCTNGGG
TTCATCACTCTCNCCTTCACATACAAACTTTCCATCTTTACCAAAANAANAGCAACCCTT
GNACCCGGGGCAACANGGGGNACATCCGGGGGGANAAATTAACGATTTTCCTTGGGAACG
GGGACNTCTTGAANAGGCAATATTTGGATCNCAATTAANGGGGCAAGCNTTTGGCCTTTT
NGGATCANATTCNCCTTCNCAAATAAATTTTCCGAAAGAATTATAATAATTACNACCCTT
ATAGCCGGAGCAACAATTGANGCATANGGGATTTAGCGGACTTCCTTCTGAACGGGGGCA
TATCCCGAACCCAANATTACCNCATTCNCTAGTACAAGCCTNGGCATCAACATATAGAAA
CNTTCCAAGAACAATTAGTAGGNAAGCGACAAAATTAACTTCCTTGGGAACNGCCNNGAN
GGANAATTGATTACNAGTACCTNGGCTTCTTTTAATTTNGGGGNCGGGGGGGGGGGGGGG
\end{Verbatim}

La página de resultados le muestra la información de su secuencia, las enzimas que no cortan su secuencia, el número de cortes para cada enzima y un mapa de sus secuencias con los sitios de corte.

{\color{red}
¿La secuencia tiene sitios de corte para la enzima cfoI?

¿Si digiriera el fragmento con las enzimas EcoRI y BamH1 y corriera la digestión en un gel de agarosa, qué tamaños de bandas observaría?
}

\section{Análisis de la composición del ADN}

En esta sección vamos a usar algunos programas del paquete EMBOSS (``The European Molecular Biology Open Software Suite''; \url{http://emboss.sourceforge.net/}) para calcular algunas estadísticas sobre secuencias de ADN. Mas adelante nos volveremos a encontrar con EMBOSS para desarrollar tareas mas complicadas.

\subsection{Contenido de G$+$C}

El contenido en G$+$C de la secuencia de ADN es importante por varias razones. El apareamiento entre las bases G y C es más estable que entre las bases A y T. Así, el contenido en bases de la secuencia determinara el comportamiento de la secuencia en experimentos de laboratorio. 

Siguiendo el enlace \url{http://mobyle.pasteur.fr/cgi-bin/portal.py?form=geecee} llegará a una interfaz web del programa geecee del paquete EMBOSS, que le permite calcular el contenido de G$+$C de una secuencias de ADN.

\textcolor{red}{¿Cuál es el contenido de G$+$C de la secuencia 35\_292648\_.ab1?}

\subsection{Composición monomérica y palabras cortas}

También podemos fácilmente calcular las frecuencias de k-meros, i.e., monómeros, dímeros, trímeros, tetrámeros, pentámeros, \ldots

{\color{red}Siga el enlace \url{http://mobyle.pasteur.fr/cgi-bin/portal.py?form=compseq} y calcule la proporción de monómeros, dímeros y trímeros de la secuencia 35\_292648\_.ab1. Presente los resutados en forma tabular.}.

\chapter{Creación de bases de datos relacionales}

En este capítulo vamos a crear bases de datos relacionales usando SQLite\footnote{\url{http://www.sqlite.org/}\label{descargasqlite}} como motor de base de datos y la extensión de Firefox SQLite Manager\footnote{\url{https://addons.mozilla.org/en-US/firefox/addon/5817/}\label{sqllitemanager}} como interfaz a la base de datos.

Primero tenemos que asegurarnos que el programa \Verb+sqlite3+ está instalado en nuestro computador. Para esto iniciemos el programa \textbf{Terminal}\footnote{Como hacer esto depende del sistema operativo. En MacOSX puede usar spotlight, i.e., el ícono de lupa en la parte superior derecha de su escritorio, y escribir Terminal, luego darle click al ícono del programa.}. Una vez en \textbf{Terminal} podemos escribir \Verb+sqlite3+ en la línea de comandos, si se obtiene un salida similar a la mostrada en las líneas \ref{salidasqlite3ini} a \ref{salidasqlite3end}, \Verb+sqlite3+ está instalado y funcionando correctamente, de lo contrario es necesario descargarlo del sitio web referenciado en la nota al pie número \ref{descargasqlite}

\begin{Verbatim}[commandchars=!\{\},numbers=left,firstnumber=last,label=Ejecutando sqlite3,frame=topline,fontsize=\scriptsize]
!textcolor{red}{[user@server:~]$} sqlite3
SQLite version 3.6.12 !label{salidasqlite3ini}
Enter ".help" for instructions
Enter SQL statements terminated with a ";"
sqlite> !label{salidasqlite3end}
!textcolor{red}{[user@server:~]$} 
\end{Verbatim} 

Una vez hemos comprobado que el motor de bases de datos está instalado y funcionando correctamente, tenemos que asegurarnos que el complento \textbf{SQLite Manager} de \textbf{Firefox} esta instalaldo, para esto, en el \textbf{Firefox}, vaya al menú \textbf{Herramientas} $\to$ \textbf{SQLite Manager}; is esta opción de menú no aparece entonces es necesario instalar la interfaz a SQLite desde el sitio referenciado en la nota al pie número \ref{sqllitemanager}.

Una vez henmos comprobado que el \textbf{SQLite Manager} está instalado podemos hacer click en el menú \textbf{Herramientas} $\to$ \textbf{SQLite Manager}, lo que iniciará una ventana como la que se muestra en la figura \ref{sqlitemanagerrunning}

\begin{figure}[ht]
\centering
   \includegraphics[width=15cm]{Figs/sqlitemanager.png}
  \caption{\label{sqlitemanagerrunning}SQLite Manger en Firefox}
\end{figure}

Aquí podemos empezar a manipular bases de datos relacionales usando el motor \Verb+sqlite3+.

Cree la base de datos PlnTFDB, haciendo click en el botón \textbf{New database}. Seleccione el directorio en donde desea guardarla, puede ser en el directorio \textbf{Documentos}.

Importe los archivos\footnote{Los archivos pueden ser descargados desde Sicua Plus} \Verb+tf.csv+, \Verb+Species.csv+, \Verb+Domains.csv+\footnote{Antes de importat abra cada uno de los archivos con un procesador de texto y defina que tipo de columnas aparecen, VARCHAR, NUMERIC, INTEGER, FLOAT} en las tablas TF, Species, y Domains respectivamente, usando la opción \textbf{import}, y asegurandose de seleccionar \Verb+Tab+ como el separador de campos e indicar que la primera fila consiste en los nombres de los campos. En el siguiente cuadro de diálogo indique el tipo de datos de cada columna.

Cree los siguientes indices\footnote{¿Para qué sirven los indices?}:

\begin{description}
\item[Tabla TF:]\textbf{Sp\_pepid} sin duplicados, como una clave primaria, \textbf{Sp\_ID} y \textbf{family\_id}
\item[Tabla Species:] \textbf{Sp\_ID} sin duplicados
\item[Tabla Domains:] \textbf{Sp\_pepid} con duplicados\footnote{¿Por qué es necesario aceptar duplicados?}, \textbf{domainid} con duplicados.
\end{description}

Relaciones entre las tablas: El campo Sp\_ID de la tabla TF está relacionado con el campo Sp\_ID de la tabla Species. El campo Sp\_pepid de la tabla Domains está relacionado con el campo Sp\_pepid de la tabla TF. ¿Que tipo de relación hay entre los campos: uno-a-uno, uno-a-varios, varios-a-varios? En este ejericio particular no los estamos usando, pero ¿qué son las claves externas (foreign keys)?

A manera de ejemplo vamos a realizar algunas conultas sencillas a la base de datos. Haga click en la pestaña \textbf{Execute SQL}, y en el cuadro \textbf{Enter SQL} escriba lo siguiente:

\begin{Verbatim}
 SELECT TF.Sp_pepid, TF.family_id, Species.Species_full_name 
   FROM TF, Species
     WHERE TF.Sp_ID=Species.Sp_ID 
\end{Verbatim}

Identifique las operaciones de \textbf{proyección}, \textbf{selección} y \textbf{conexión (JOIN)} en la anterior declaración SQL.

Vamos a hacer las siguientes consultas a la base de datos: \textcolor{red}{¿Cuáles son las familias de factores de transcripción presentes en las especies estudiadas?}

\begin{Verbatim}
 SELECT DISTINCT TF.family_id
   FROM TF
\end{Verbatim}

\textcolor{red}{¿Cuántas familias son?}

\begin{Verbatim}
 SELECT COUNT(DISTINCT TF.family_id)
   FROM TF
\end{Verbatim}

\textcolor{red}{¿Cuántas familias hay en cada especie?}
\begin{Verbatim}
 SELECT Species.Species_full_name, COUNT(DISTINCT TF.family_id)
   FROM TF, Species
     WHERE TF.Sp_ID=Species.Sp_ID
      GROUP BY Species.Sp_id
\end{Verbatim}

{\color{red}
Responda las siguientes preguntas:

\begin{enumerate}
\item ¿Cuántos genes por familia y por especie hay?
\item ¿Cuántos genes por especie hay?
\item ¿Que dominos están presentes en los genes de la familia MYB de la especie \textit{Arabidopsis thaliana}?
\item ¿Cuál es el número de dominios diferentes presentes en los genes de las diferentes especies?
\item ¿Cuál es la especie con mayor número de dominos diferentes?
\item ¿En que especie y gen se encuentra el dominio mas largo?
\item ¿Para qué sirve la expresión \Verb+limit+ en una declaración SQL en SQLite?
\end{enumerate}
}

\chapter{Búsquedas en base de datos biológicas - Segunda parte}

\section{PubMed}

Esta sección corresponde a una versión modificada del tutorial \url{http://www.nlm.nih.gov/bsd/disted/pubmedtutorial/}. Esta guía consiste en seguir el tutorial disponible en el enlace anterior y resolver las preguntas que aparecen mas abajo en rojo.

\textbf{PubMed} es la base de datos de literatura mantenida por el NCBI, actualmente tiene alrededor de 19 millones de registros.

\subsection{Entendiendo la información en los registros de PubMed}

Una referencia bibliográfica en PubMed está compuesta de campos que ofrecen información específica (Título, autor, lenguaje, etc) sobre el artículo publicado. La siguiente lista es una muestra de los campos que aparecen generalmente:

\begin{itemize}
\item Título del artículo
\item Nombres de los autores
\item Resumen publicado con el artículo
\item Vocabulario controlado de términos de búsqueda (Medical Subject Headings)
\item Información sobre la revista
\item Instituto o universidad a la que está afiliado el primer autor
\item Lenguaje en que el artículo fue publicado
\item Tipo de publicación (revisión, carta, nota pequeña, etc)
\item Identificador único de PubMed (PubMed Unique Identifier, PMID)
\end{itemize}

\textbf{Ejercicios}:

\begin{itemize}
\item \textcolor{red}{Haga una búsqueda en PubMed e identifique los campos que se mencionaron arriba.}
\item \textcolor{red}{Realize una búsqueda en PubMed con el término ``eye''. ¿Cuáles de los siguientes términos serán recuperados?}
 \begin{itemize}
  \item Eye, chin and forehead
  \item Eye, eyelids, cornea, iris, y todos los demás términos que estén subordinados al termino ``eye'' en MeSH.
   \item Eye (únicamente)
 \end{itemize}
 \item \textcolor{red}{¿Cuál fue la búsqueda exacta que realizó en el paso anterior? Pista: Ubique la caja de text ``\textbf{Search details}'' en la página de resultados.}
 \item \textcolor{red}{Haga una búsqueda en la base de datos MeSH usando como palabras clave sus áreas de interés e identifique los términos MeSH asociados.}
\end{itemize}

\textbf{Preguntas}

\begin{itemize}
\item \textcolor{red}{¿En que consiste el ``status'' de una entrada en PubMed?}
\item \textcolor{red}{¿Cuál es la diferencia entre MEDLINE y PubMed?}
\item \textcolor{red}{¿Qué son y para que sirven los términos MeSH?}
\item \textcolor{red}{Enq ue consiste ``Automatic Term Mapping''?}
\end{itemize}

\subsection{Realizando búsquedas}

Empleando la opción de búsqueda avanzada, usando la opción ``\textbf{Search builder}'', recupere todos los artículos científicos publicados por las profesoras Silvia Restrepo y Adriana Bernal desde el 2008 hasta el 2009, responda:

 \begin{itemize}
\item  \textcolor{red}{¿Cuántos artículos encontró?}
\item  \textcolor{red}{¿En que revistas fueron publicados?}
\item \textcolor{red}{¿A qué tipo de publicación corresponden?}
\item \textcolor{red}{¿Qué términos MeSH hay en común?}
\item \textcolor{red}{¿Qué términos MeSH reflejan el tema principal de los artículos?}
\item \textcolor{red}{Nombre tres referencias relacionadas al artículo mas reciente de la lista de resultados. ¿Cómo las identificó?}
\item \textcolor{red}{Envíe los resultados de su búsqueda a su correo electrónico, usando la opción ``\textbf{Send to}''}
\end{itemize}

\section{Descarga por lotes usando Entrez}

En aquellos casos en que se tiene un colección de identificadores de alguna base de datos consultada por Entrez, el sistema cuenta con una aplicación de descarga por lotes: ``Batch Entrez'' (\url{http://www.ncbi.nlm.nih.gov/sites/batchentrez})

Use el archivo \Verb+ID_list.txt+\footnote{Disponible en Sicua Plus} para hacer una consulta Batch Entrez y responda:

{\color{red}
\begin{itemize}
\item ¿Cuántos identificadores pueden ser recuperados por Entrez?
\item ¿En que base de datos se encuentran esos registros?
\item ¿Existe algún aviso importante para cualquiera de los registros? En caso afirmativo, explique en que consiste y por que puede pasar.
\item Enumere los pasos a seguir para cambiar la visualización de los registros y obtener las secuencias en formato Fasta y descargarlas en un archivo de texto.
\end{itemize}
}

\section{Recuperar todas las secuencias de un organismo o taxon}

En algunas ocasiones es necesario recuperar del NCBI todas las secuencias de ácidos nucleicos o de proteínas para una especie particular o para un grupo de organismos que pertenecen al mismo grupo taxonómico. Podemos empleada el ``Taxonomy Browser'' del NCBI para simplificar este proceso.

Haga una búsqueda en la base de datos de taxonomía usando \textit{Ornithorhynchus anatinus} como especie de interés. Los nombres vulgares comunes pueden ser usados, pero siempre es preferible emplear el nombre científico.  Al llegar al registro para la especie identifique su clasificación taxonómica. El número de registros en cada una de las bases de datos para la especie o grupo seleccionado, aparece como un enlace en la parte derecha de la página de resultados. Siguiendo esos enlaces puede descarga el conjunto completo de secuencias de la base de datos correspondiente.

Responda:

{\color{red}
\begin{itemize}
\item ¿Cuantas proteínas se encuentran?
\item ¿Cuántas secuencias de ácidos nucleicos?
\item ¿Qué otro tipo de información podría extraer?
\end{itemize}
}

\section{Recuperar la información publicada sobre un gen}

Haga una búsqueda en alguna de las base de datos usando como palabra clave el nombre del gen de interés y el organismo, Por ejemplo, usando la base de datos ``Gene'':

\begin{verbatim}
tpo[sym] AND human[orgn]
\end{verbatim}

En la página de resultados, siga el enlace al gen deseado. Si no existe un registro para el gen en las bases de datos seleccionadas, haga una nueva búsqueda en todas las bases de datos ``All databases''.

Cuando encuentre el registro para el gen, identifique en la página de resultados el enlace  ``Link''. Haciendo clic en este enlace desplegará una lista con mas enlaces, seleccione PubMed. Allí encontrará los registros de la base de datos de literatura que hacen referencia a su gen de interés.

Haga una búsqueda en la base de datos de ``Gene'' usando el nombre de gen ``ANAC092'' en la especie \textit{Arabidopsis thaliana}.


\begin{itemize}
\item \textcolor{red}{¿Que artículos en pubmed hacen referencia a ese gen?}
\item \textcolor{red}{Describa el gen usando la información encontrada en la base de datos ``Gene''}
\end{itemize}

\section{Bases de datos en el European Bioinformatics Institute (EBI)}

\subsection{SRS}

El ``Sequence Retrieval System'' (SRS) lo vimos en la Sección\ref{srs}. Siga el enlace \url{http://srs.ebi.ac.uk/srs/doc/index.html} y familiarícese con las opciones de búsqueda de este sistema.

\subsection{EB-eye}

Este es otro sistema de búsqueda en el EBI.

Haga una búsqueda en todas las bases de datos usando las palabras clave ``glutathione s-transferase'' en la página del EBI (\url{http://www.ebi.ac.uk/}). 

\textbf{Responda:}

{\color{red}
\begin{itemize}
\item Describa la página de resultados, ¿Cuántas bases de datos fueron consultadas? ¿En que categorías están agrupadas esas bases de datos?
\item ¿Cuántas y cuáles reacciones enzimáticas son mediadas por la enzima glutathione s-transferasa?
\item ¿Qué ontologías tienen registros asociados para la enzima? Descríbalas.
\item ¿Qué es una ontología? De ejemplos de algunas ontologías en biología
\end{itemize}
}

\section{Expasy}

El Expasy (\url{http://expasy.org/}) en el ``\textbf{Ex}pert \textbf{P}rotein \textbf{A}nalysis \textbf{Sy}stem'' mantenido por el Instituto Suizo de Bioinformática. Como su nombre lo indica está enfocado en el análisis de proteínas.

En esta sección vamos a usar algunas de las aplicaciones que se encuentran en el enlace \url{http://expasy.org/tools/}, principalmente aquellas con el logo del Expasy.

Use la secuencia de la proteína ANAC092 de \textit{Arabidopsis thaliana} para desarrollar los ejercicios de esta sección.

\textbf{Responda:}

{\color{red}
\begin{itemize}
\item ¿Cuál es el peso molecular y punto isoeléctrico de la proteína? ¿Qué herramienta usó para calcular esos parámetros?
\item ¿Cuántos y cuáles fragmentos se generan luego de una digestión con tripsina? ¿Qué herramienta uso para hacer la predicción? Calcule el punto isoelétrico y el peso molecular de cada fragmento.
\item Identifique la composición de amino ácidos de ANAC092. ¿Qué aplicaciones empleó?
\end{itemize}
}

\section{Mas ejercicios}

Encuentran mas guías siguiendo los enlaces \url{http://www.ncbi.nlm.nih.gov/guide/all/howto/} y \url{http://www.ebi.ac.uk/inc/help/search_help.html}.

\chapter{Ontologías en bioinformática: Gene Ontology}

Para desarrollar este capítulo tiene que leer la documentación sobre ``Gene Ontology'' siguiendo el enlace: \url{http://www.geneontology.org/GO.doc.shtml} y posiblemente seguir otros enlaces que allí se encuentren.

Parte de esta guía es una versión modificada del tutorial encontrado en el enlace: \url{http://www.geneontology.org/teaching_resources/tutorials/2007-10_GO-resources_jblake.doc}, pueden seguir independiente ese tutorial para desarrollar mas habilidades usando GO.

{\color{red}
\begin{itemize}
\item ¿Cuál es el objetivo del proyecto ``Gene Ontology'' (GO)?
\item Describa las tres ontologías que hacen parte de GO.
\item ¿En que consiste anotar un producto génico con términos GO?
\item Describa en que consisten las versiones ``Slim'' de GO.
\item ¿Cuál es la diferencia entre las ontologías y las anotaciones? Puede revisar los enlaces en la sección de descargas (``Downloads'').
\end{itemize}
}

Siga el enlace: \url{http://www.obofoundry.org/} de ``The Open Biological and Biomedical Ontologies''. {\color{red}Seleccione tres ontologías (diferentes a GO) que puedan ser útiles en su investigación y descríbalas brevemente.}

\section{Consultas en GO}

Vamos a usar ``AmiGO'' para hacer consultas a GO. AmiGO es un navegador basado en HTML que facilita la formulación de consultas tanto de las ontologías como de las asociaciones a los genes.

Haga una búsqueda de término usando ``carbohydrate metabolism''.

El resultado de la consulta muestra todos los términos que incluyen la cadena de caracteres ``carbohydrate metabolism''.  Haga clic en el primer término ``carbohydrate biosynthetic process''.

Lo primero que ve en cada línea es uno de los símbolos: $+$, $-$, o $\bullet$, como se muestra en la Figura~\ref{QueryGO5}. El símbolo $+$ puede ser usado para expandir un node, mostrando todos los hijos del término seleccionado. El símbolo $-$ puede ser usado para cerrar el nodo seleccionado. Finalmente $\bullet$ significa que el término no tiene hijos. Luego de esos símbolos va a encontrar las letras \textbf{P}, \textbf{I} o \textbf{R}, que identifican el tipo de relación: ``parte de'' (``part of''), ``es un'' (``is a''), o ``regula'' (``regulates''), repectivamente. Enseguida encuentra el identificador del término y el término. Al término le sigue un número en paréntesis que le indica el numero de productos génicos que ha sido anotados con ese término o a términos mas específicos (hijos).

\begin{figure}[ht]
\centering
   \includegraphics[width=15cm]{Figs/QueryGO5.png}
  \caption{\label{QueryGO5}Consultas en ``Gene Ontology''}
\end{figure}

Busque la opción ``Graphical View'' para visualizar esta sección de GO como un grafo acíclico dirigido, como el que se muestra en el Figura~\ref{GO_DAG}.

\begin{figure}[ht]
\centering
   \includegraphics[height=8cm]{Figs/GO_DAG.png}
  \caption{\label{GO_DAG}Visualización del grafo acíclico dirigido de una sección de GO}
\end{figure}


Vamos a realizar otra consulta en GO usando como palabra clave el nombre de un gen (``ANAC092''), como se muestra en la Figura~\ref{QueryGO}.

\begin{figure}[ht]
\centering
   \includegraphics[width=15cm]{Figs/QueryGO1.png}
  \caption{\label{QueryGO}Consultas en ``Gene Ontology''}
\end{figure}

Ya que la búsqueda que realizamos fue muy específica, los resultados nos llevan directamente a la página de descripción de este gen en GO (Figura~\ref{QueryGO2}). El nombre del gen que usamos, ANAC092, no se encuentra en ninguna otra especie cubierta por GO. En esta página de resultados identifique la sección ``Term associations'' y siga el enlace, allí encontraremos el conjunto de términos GO que han sido asignados a este gen en particular, ver Figura~\ref{QueryGO3}.

\begin{figure}[ht]
\centering
   \includegraphics[width=15cm]{Figs/QueryGO2.png}
  \caption{\label{QueryGO2}Resultados de la consulta en ``Gene Ontology'', usando el nombre de gen ANAC092}
\end{figure}

\begin{figure}[ht]
\centering
   \includegraphics[width=15cm]{Figs/QueryGO3.png}
  \caption{\label{QueryGO3}Términos GO asociados al gen ANAC092}
\end{figure}

Haga clic sobre el término \textbf{GO:0007275 : multicellular organismal development}. Esto lo conducirá a la página de detalles del término, donde encuentra toda la información disponible sobre el término: nombre e identificador, sinónimos que pueda tener, definición, su posición en la estructura de GO, referencias a bases de datos externas, y los productos génicos asociados a ese término.

{\color{red}
\begin{itemize}
\item Describa cada uno de los códigos de evidencia asociados a las anotaciones del gene ANAC092 que aparecen en la Figura~\ref{QueryGO3}
\item Haga una lista de los términos GO asociados con este gen. ¿Qué está indicando el calificador ``NOT''?
\item Describa brevemente la función de este gen.
\item Muestre el grafo acíclico dirigido para la sección que incluye el término \textbf{GO:0010150 : leaf senescence}
\end{itemize}
}

\chapter{Introducción al análisis de redes usando Cytoscape}

En este capítulo vamos a aprender a trabajar con Cytoscape\footnote{\url{http://www.cytoscape.org/}}. Sigan el tutorial básico que se encuentra en el enlace: \url{http://cytoscape.wodaklab.org/wiki/Presentations/Basic}. Van a encontrar Cytoscape instalado en sus computadores, así que no tienen que usar la opción de ``Java Web Start''.

De la sección ``Defining visual styles'' del Tutorial 1: Getting Started, responda:

{\color{red}
\begin{itemize}
\item En la subred que incluye los vecinos más cercano a TP53 ¿Cuál es el tipo mas común de lado/arista? ¿Cuál el menos común?
\item ¿Cuántos nodos y aristas hay en la red ``DNA replication'' que cargó desde Reactome?
\end{itemize}
}

Despues de seguir el Tutorial 4: Expression Analysis, responda:

{\color{red}
\begin{itemize}
\item ¿Cuáles son los valores de expresión en las condiciones (genes pertubados): Gal1, Gal4, and Gal80 para el gene de levadura: YOL051W?
\item ¿Cuáles son los vecinos mas cercanos a ese gen (First Neighbors)?
\end{itemize}
}

% \chapter{Análisis de enriquecimiento de anotaciones de genes}
% 
% Siga el tutorial que esta disponible en el enlace:\url{http://www.psb.ugent.be/cbd/papers/BiNGO/Tutorial.html}
% 
% El archivo\Verb+SaltArabidopsis.txt+ que está disponible en Sicua Plus, tienen una lista de genes de Arabidopsis que reponden diferencialmente al tratamiento con sal, y que fueron identificados a traves de ensayos usando microarreglos de ADN.
% 
% {\color{red}Use BinGO para identificar los términos GO que aparecen sobre y sub representados para los genes que aparecen en el archivo \Verb+gene_list.txt+. Muestre  el grafo de los términos e interprete los resultados.}

\chapter{Comparação de Sequências I - Matrizes de pontos}

As matrizes de pontos (``Dot Plot'') são ferramentas exploratórias para comparar strings de texto, ou seja, sequências. Entre outros, eles nos permitem encontrar facilmente regiões repetidas em uma sequência comparando-a com ela mesma. Também podemos ter uma boa ideia da estrutura de um gene comparando a sequência de sua região de codificação com a sequência do locus onde se encontra.

Nesta seção, usaremos a implementação de matrizes de pontos do Instituto Suíço de Bioinformática, conhecida como Dot Let\footnote{\url{http://myhits.isb-sib.ch/cgi-bin/dotlet}}, que vemos na Figura~\ref{DotLet1}.

\begin{figure}[ht]
\centering
   \includegraphics[width=10cm]{Figs/DotLet1.png}
  \caption{\label{DotLet1}Dot Let @ SIB}
\end{figure}

Faça uma comparação da sequência encontrada no arquivo \Verb+aqc-MIR399+ com ela mesma. A primeira coisa que você precisa fazer é clicar no botão ``Input'', que abrirá a janela mostrada na Figura~\ref{inputDotLet}, dar um nome à sequência e colá-la na caixa correspondente.

\begin{figure}[ht]
\centering
   \includegraphics[width=7cm]{Figs/inputDotLet.png}
  \caption{\label{inputDotLet}Adicionar Sequências em Dot Let}
\end{figure}

De volta à janela Dot Let vemos que encontramos dois botões habilitados (Figura~\ref{DotLetbuttons}), eles agora aparecem com o nome da sequência que você acabou de adicionar. Uma delas representa a sequência que aparece na direção horizontal, a outra a sequência que aparece na direção vertical.

\begin{figure}[ht]
\centering
   \includegraphics[width=10cm]{Figs/DotLetBotones.png}
  \caption{\label{DotLetbotones}botões de controle}
\end{figure}


À direita dos botões/listas que identificam as sequências, encontramos uma lista suspensa, atualmente desabilitada, que permite selecionar a matriz de substituição. Em seguida, encontramos uma lista suspensa com os tamanhos de janela que serão usados para a comparação das duas sequências. O próximo botão permite ampliar, ou seja, ``Zoom'', e finalmente encontramos o botão ``Calcular'', que preenche a matriz de pontos.

Uma vez que a matriz de pontos foi calculada, encontramos duas seções de resultados, semelhantes ao que aparece na Figura~\ref{DotLetResultado1}. A região da esquerda é a própria matriz, pixels escuros representam pontuações baixas, ou seja, ruins. À esquerda vemos um histograma da frequência de cada pontuação. Manipulando este histograma, com as barras de rolagem horizontais (para cima e para baixo) podemos modificar a exibição da matriz de pontos.

\begin{figure}[ht]
\centering
   \includegraphics[width=15cm]{Figs/DotLetResultado1.png}
  \caption{\label{DotLetResultado1}Resultado}
\end{figure}

{\color{red}
\begin{itemize}
\item Explique como o tamanho da janela afeta a exibição na matriz de pontos.
\item Qual é o significado da linha rosa no histograma de pontuação?
\item Que interpretação você pode fazer das repetições invertidas que podem ser detectadas na matriz de pontos?
\item Compare a sequência de cDNA e sua contraparte genômica de ANAC092\footnote{Disponível em e-disciplinas}. Descreva os resultados.
\end{itemize}
}

\chapter{EMBOSS}

EMBOSS\footnote{\url{http://emboss.sourceforge.net/}},  ``The European Molecular Biology Open Software Suite'', é um pacote de código aberto gratuito composto por centenas de apps\footnote{\url{http://emboss.sourceforge.net/apps/release/6.6/emboss/apps/}} que foram desenvolvidos especificamente para atender às necessidades da comunidade de biologia molecular. O tutorial que você encontra abaixo faz parte dos tutoriais disponíveis em \url{http://emboss.sourceforge.net/docs/emboss_tutorial/emboss_tutorial.html}

Encontre uma descrição de cada um dos aplicativos presentes no EMBOSS seguindo o link: \url{http://emboss.sourceforge.net/apps/release/6.6/emboss/apps/}

Algumas das áreas abrangidas pelas aplicações EMBOSS são:

\begin{itemize}
\item Alinhamento de sequência
\item Pesquisar em bancos de dados usando padrões
\item Identificação de motivos proteicos
\item Análise de uso de códons
\end{itemize}

\section{Recuperando sequências de bancos de dados}

A recuperação de sequências de um banco de dados obviamente depende dos bancos de dados que temos disponíveis.

Vamos recuperar a sequência do gene ANAC092 do banco de dados de proteínas UniProt. Para fazer este exercício, você precisa ir ao site UniProt \footnote{\url{http://www.uniprot.org/}} e encontrar o identificador apropriado para baixar a sequência em formato .txt Figura~\ref{link_seq} usando wget (linha~\ref{getuniprot}).

\begin{Verbatim}[commandchars=!\{\},numbers=left,label=obtendo sequência do uniprot,frame=topline,fontsize=\scriptsize]
!textcolor{red}{[user@server]$} wget https://www.uniprot.org/uniprot/D7MJK1.txt !label{getuniprot}
--2022-04-06 14:23:06--  https://www.uniprot.org/uniprot/D7MJK1.txt
Resolving www.uniprot.org (www.uniprot.org)... 193.62.193.81
Connecting to www.uniprot.org (www.uniprot.org)|193.62.193.81|:443... connected.
HTTP request sent, awaiting response... 200 
Length: 4570 (4,5K) [text/plain]
Saving to: ‘D7MJK1.txt’
D7MJK1.txt          100%[===================>]   4,46K  --.-KB/s    in 0s      
2022-04-06 14:23:08 (275 MB/s) - ‘D7MJK1.txt’ saved [4570/4570]
\end{Verbatim} 

\begin{figure}[ht]
\centering
   \includegraphics[width=12cm]{Figs/uniprot.png}
  \caption{\label{uniprot_site}Recuperando sequências de bancos de dados}
\end{figure}

\begin{figure}[ht]
\centering
   \includegraphics[width=12cm]{Figs/get_seq_link.png}
  \caption{\label{link_seq}Recuperando link de sequências em uniprot}
\end{figure}


O programa que nos interessa chama-se ``\Verb+seqret+''. Explore o comando ``\Verb+wossname+'' para buscar programas por palavras chave. Lembre-se de carregar o ambiente conda do EMBOOS no seu computador, assim:

\begin{Verbatim}[commandchars=!\{\},firstnumber=last,numbers=left,label=carregando ambiente EMBOSS com conda,frame=topline,fontsize=\scriptsize]
!textcolor{red}{[user@server]$} conda activate emboss !label{condaEMBOSS}
!textcolor{red}{(emboss)[user@server]$} !label{condaEMBOSS2}
\end{Verbatim}

Repare que na linha~\ref{condaEMBOSS2}, seu prompt mudou, e agora aparece o nome do ambiente carregado, i.e., emboss.

Quando terminar de usar os programas de EMBOSS lembre de desactivar o ambiente de conda, como mostrado na linha~\ref{condaEMBOSSDeactivate}:

\begin{Verbatim}[commandchars=!\{\},firstnumber=last,numbers=left,label=encerrando ambiente EMBOSS com conda,frame=topline,fontsize=\scriptsize]
!textcolor{red}{(emboss)[user@server]$} conda deactivate!label{condaEMBOSSDeactivate}
!textcolor{red}{[user@server]$} 
\end{Verbatim}


Agora, você pode usar o mesmo programa para converter a entrada recuperada do UniProt .txt para o formato fasta. \textcolor{red}{Como você faria isso?}

Já que você sabe como recuperar sequências do UniProt, vamos fazer alguns cálculos sobre essa sequência, procure o programa \Verb+compseq+, que permite calcular a composição da palavra de uma sequência. \textcolor{red}{Calcular frequências mais fracas para ANAC092 extraídas do UniProt}\footnote{Ao enviar seu guia esses resultados devem ser enviados em um arquivo de texto plano (.txt).}

\section{Seleção de quadros de leitura aberta}

Os programas \Verb+getorf+ e \Verb+plotorf+ buscam quadros de leitura abertos em sequências de nucleotídeos. Sendo um quadro de leitura aberto, uma sequência (subsequência) de um comprimento mínimo especificado flanqueado por dois códons de parada ou por um códon de partida e parada. Apesar da universalidade do código genético alguns grupos de organismos têm códons diferentes, por isso é importante especificar, seja o código genético que está sendo usado para traduzir a sequência, ou o início e parar códons permitidos.

{
\color{red}Use esses dois programas para encontrar o quadro de leitura aberto correto da sequência ''ANAC092\_cDNA.fa''\footnote{Arquivo \Verb+ANAC092\_cDNA.fa'+ disponível no e-disciplinas}. Você encontra alguma diferença nos resultados oferecidos pelos dois programas? 
}

\section{Embaralhar/misturar Sequências}

Ao fazer certos tipos de análise, por exemplo, procurar sitios de ligação a fatores de transcrição em sequências de promotores (``TFBS''), é importante ter um grupo de sequências que servem como um controle negativo. Para que o TFBS não apareça com frequência neste controle negativo. Uma opção amplamente utilizada é gerar sequências aleatórias que contenham a mesma composição monomérica das sequências originais. O programa ``shuffleseq'' faz exatamente isso, pega uma sequência ``real'' e mistura, como se embaralhando um baralho de cartas, os monômeros constituintes, resultando em uma sequência aleatória. Ao usar este tipo de estratégia, ~1000 sequências aleatórias são geradas para cada sequência original.

\textcolor{red}{Use ``shuffleseq'' para gerar duas sequências aleatórias do rRNA encontradas no arquivo \Verb+FN566965.fasta+ \footnote{Arquivo \Verb+FN566965.fasta+ disponível no e-disciplinas}}.

\section{Previsão de regiões hidrofóbicas} O programa \Verb+pepwindow+ prevê segmentos hidrofóbicos em uma proteína, seguindo a estratégia proposta por \citep{Kyte1982}. O uso de janelas de 19 a 21 resíduos regiões transmembranas pode ser claramente detectado, com valores de índice de hidroofobidade de 1,6 na região central.

\textcolor{red}{Pode detectar quaisquer regiões transmembranas no gene NTM1\footnote{Arquivo \Verb+NTM1.fasta+ disponível no e-disciplinas}?}

\section{Alinhamentos}

\textcolor{red}{Descreva a função do programa distmat}.

\chapter{Comparação de Sequências II - Alinhamentos de pares de sequências}

Algumas partes deste capítulo vêm do tutorial que está seguindo o link: \url{http://emboss.sourceforge.net/docs/emboss_tutorial/emboss_tutorial.pdf}

Para fazer os exercícios lembre-se de activar o ambiente conda do EMBOSS como mostra a linha~\ref{activateEMBOSS3}


\begin{Verbatim}[commandchars=!\{\},numbers=left,label=carregando ambiente EMBOSS com conda,frame=topline,fontsize=\scriptsize]
!textcolor{red}{[user@server]$} conda activate emboss !label{activateEMBOSS3}
!textcolor{red}{(emboss)[user@server]$}
\end{Verbatim}


\section{Matrizes de substituição}

no arquivo archivo \Verb+EPAM250.txt+ você vai encontrar a matriz de substituição PAM250.

%Check: http://cnx.org/content/m11062/latest/

{
\color{red}
\begin{itemize}
\item Quem, e como, criou a família PAM de matrizes de substituição?
\item Onde estão as maiores pontuações? Explicar.
\item Qual é a substituição com a maior pontuação?
\item Por que as identidades não têm sempre a mesma pontuação?
\end{itemize}

}


\section{Alinhamento Global}

No alinhamento global, o objetivo é comparar as duas sequências ao longo de toda a sua duração, portanto é apropriado quando esperamos que a semelhança entre as duas sequências se estenda ao longo de toda a sequência.

No pacote EMBOSS você encontrará o aplicativo \Verb+needle+ que implementa rigorosamente o algoritmo de Needleman e Wunsch \citep{Needleman1970} para obter o alinhamento global ideal por programação dinâmica. Esta implementação pode levar algum tempo para obter o alinhamento quando as sequências são longas.

\textcolor{red}{Quais outros aplicativos no EMBOSS permitem que você faça alinhamentos globais? O que os torna diferentes de \Verb+needle+?}

{\color{red}
\begin{itemize}
\item Faça um alinhamento global entre o cDNA e as sequências genômicas do gene ANAC092, que estão disponíveis no e-disciplinas
\item Qual matriz de substituições e penalidades para abrir e estender GAPs que você usou? Explicar.
\item Qual é a pontuação do alinhamento, seu comprimento e as porcentagens de identidade e semelhança?
\item Explique a diferença entre semelhança e identidade.
\item O que significam os símbolos? \Verb+:+, \Verb+.+ y \Verb+|+?
\end{itemize}
}

Nos arquivos \Verb+ANAC092_pep.fasta+ y \Verb+PpNAC_e_gw1.5.134.1.fasta+ encontra as sequências de aminoácidos de dois genes da família NAC de fatores de transcrição em \textit{Arabidopsis thaliana} e em musgo \textit{Physcomitrella patens} respectivamente.

{\color{red}
\begin{itemize}
\item Faça um alinhamento global entre as sequências de aminoácidos das proteínas NAC de \textit{Arabidopsis thaliana} e em musgo \textit{Physcomitrella patens}.
\item Qual ação e a matriz de penalidades para abrir e estender GAPs que você usou? Explicar.
\item Qual é a pontuação de alinhamento, seu comprimento e os percentuais de identidade e semelhança?
\item Você pode melhorar o alinhamento escolhendo outros parâmetros?
\end{itemize}
}

\section{Alinhamentos locais}

Como mencionado na seção anterior, o alinhamento global alinha sequências ao longo de todo o seu comprimento. Você tem que decidir se essa estratégia é a mais apropriada em cada caso. \textcolor{red}{O que você acha que aconteceria se você comparar duas proteínas multi dominio que só compartilham um domínio entre elas?}

O objetivo do alinhamento local é encontrar regiões de similaridade local, e não é necessário incluir as sequências completas. Esse tipo de alinhamento é muito útil para pesquisar bancos de dados, ou quando você não tem uma ideia clara sobre a semelhança da sequência de interesse com sequências no banco de dados.

No pacote EMBOSS você encontrará o aplicativo\Verb+water+ que implementa rigorosamente o algoritmo de  smith y Waterman \citep{Smith1981} para obter o alinhamento local ideal por programação dinâmica. Esta implementação pode levar algum tempo para obter o alinhamento quando as sequências são longas.

\textcolor{red}{Que outros aplicativos no EMBOSS permitem que você faça alinhamentos locais? O que os torna diferentes de \Verb+water+?}

{\color{red}
\begin{itemize}
\item Faça um alinhamento local entre as sequências de aminoácidos das proteínas NAC de \textit{Arabidopsis thaliana} e no musgo \textit{Physcomitrella patens}, que você usou na seção anterior.
\item Qual a matriz de subtituição e penalidades para abrir e estender gaps que você usou? Explicar.
\item Qual é a pontuação de alinhamento, seu comprimento e os percentuais de identidade e semelhança?
\item Você pode melhorar o alinhamento escolhendo outros parâmetros?
\item Quais são as diferenças entre o alinhamento global e local dessas duas sequências?
\end{itemize}
}

\section{Significado dos alinhamentos}

Não importa quais sequências você dá aos programas de alinhamento, eles sempre criarão um alinhamento.

Pegue as sequências de aminoácidos ANAC092 e use o programa \Verb+shuffleseq+, e gere duas sequências aleatórias com a mesma composição monomérica de ANAC092. Faça um alinhamento global e local com as duas sequências.

{\color{red}
\begin{itemize}
\item Qual matriz de substituições e quais penalidades para abrir e estender gaps que você usou? Explicar.
\item Qual é a pontuação de alinhamento, seu comprimento e as porcentagens de identidade e semelhança?
\item Você pode melhorar o alinhamento escolhendo outros parâmetros?
\end{itemize}
}

Agora faça um alinhamento local e global entre a sequência de aminoácidos do ANAC092 e uma das versões aleatórias.

{\color{red}
\begin{itemize}
\item Qual matriz de substituições e quais penalidades para abrir e estender gaps que você usou? Explicar.
\item Qual é a pontuação de alinhamento, seu comprimento e os percentuais de identidade e semelhança?
\item Você pode melhorar o alinhamento escolhendo outros parâmetros?
\end{itemize}
}
%
%\chapter{\sc{Basic Local Alignment Search Tool}: BLAST}
%
%BLAST nos permite encontrar regiones de similitud local entre una secuencia de interés y secuencias de una base de datos.
%
%En este capítulo vamos a seguir una versión modificada del tutorial disponible en el enlace \url{http://www.ncbi.nlm.nih.gov/Education/BLASTinfo/tut1.html}.
%
%BLAST se encuentra en el enlace \url{http://blast.ncbi.nlm.nih.gov/Blast.cgi}
%
%Como ejemplo, considere la proteína MJ0577 (GI:2501594), de la archaea \textit{Methanococcus jannaschii}. La secuencia de amino ácidos se deriva del marco de lectura abierta MJ0577, y será usada como \textit{query} en la búsqueda de secuencias relacionadas en la base de datos de amino ácidos \Verb+nr+ (``non-redundant''). El programa \Verb+blastp+ es el apropiado para todo tipo de búsqueda en que la secuencia ``query'' es una proteína y se compara contra una base de datos de proteínas. La base de datos \Verb+nr+ es una buena opción para realizar una búsqueda inclusiva.
%
%Puede restringir su búsqueda a un organismo o grupo taxonómico usando la casilla de texto ``Organism''. En este caso no vamos a restringir la búsqueda a un organismo particular. Las relaciones que encontremos con proteínas de cualquier dominio de la vida nos darán pista sobre la función de la proteína de interés.
%
%En la sección de ``Algorithm parameters'' podemos seleccionar el valor E. Para este ejercicio lo vamos a cambiar del valor 10 que tiene por defecto a 1. A pesar de que los valores de E mayores que 0.1 no va reflejar verdaderas secuencias relacionadas, es útil para examinar hits con menor significancia de regiones cortas de similitud. En ausencia de similitudes mayores, esas regiones cortas nos pueden permitir hacer una asignación tentativa de las actividades bioquímicas de la proteína de interés. La significancia de esas regiones debe ser evaluada caso por caso.
%
%Cuando se está tratando de encontrar la función de una secuencia que no ha sido anotada, se deben mirar primero las secuencias homólogas en otros organismos que podrían ya estar anotadas. En segundo lugar se pueden mirar secuencias cortas que tienen alguna similitud con regiones de la secuencias de la base de datos que hayan sido caracterizadas bioquímicamente.
%
%La matriz de sustitución BLOSUM62 es una matriz de propósito general y es la opción por defecto en versiones actuales de BLAST. La matriz asigna un puntaje para cada posición en un alineamiento. Este puntaje esta basado en la frecuencia conque se observó dicha sustitución en bloques (alineamientos múltiples cortos) de proteínas relacionadas. BLOSUM62 está dentro de las mejores proteínas para detectar similitudes débiles entre proteínas.
%
%Las opciones de abrir y extender gaps dependen de la matriz de sustitución seleccionada. Por favor observe como cambiando de matriz estas opcioens cambian.
%
%A la derecha de cada opción en el formulario de búsqueda de BLAST encuentra un ícono de ayuda, al pincharlo obtendrá mas información sobre el parámetro o la opción en particular.
%

\chapter{BLAST: \sc{Basic Local Alignment Search Tool}}

Muitos de vocês conhecem a interface web BLAST no NCBI mostrado na Figura~\ref{blastx_input}.  Na primeira parte deste tutorial vamos fazer alguns exercícios usando esta interface.

\begin{figure}[h!]
\centering
   \includegraphics[width=10cm]{Figs/blasthomepage.png}
  \caption{\label{blasthomepage}Tipos de BLAST disponíveis no NCBI}
\end{figure}

No e-disciplinas encontra \Verb+desconocido.nuc.fa+, que contém a sequência nucleotídea de uma transcrição que você descobriu analisando a expressão diferencial de \textit{A. thaliana} em resposta à luz ultravioleta (UV-A), tratamento no qual esta transcrição foi induzida. Copie a sequência da transcrição e abra o site \url{http://blast.ncbi.nlm.nih.gov/} no navegador Firefox. Vamos realizar uma pesquisa blast básica, olhar na página para uma seção como a que aparece na Figura~\ref{blasthomepage} e selecionar o \Verb+blastx+\textcolor{red}{{\textquestiondown} Por que usar \Verb+blastx+?}

\begin{figure}[h!]
\centering
   \includegraphics[width=10cm]{Figs/blastx_input.png}
  \caption{\label{blastx_input}Interface web NCBI BLAST usando o programa blastx}
\end{figure}

No \Verb+blastx+ cole sua sequência desconhecida no campo ``\textbf{Enter query sequence}'', escreva \textit{Viridiplantae} no campo ``\textbf{Organism}'', para estreitar a busca por sequências de plantas verdes (Figura~\ref{blastx_input}). Certifique-se de que o banco de dados selecionado seja o banco de dados não redundante de sequências proteicas.

Em buscas envolvendo a tradução online de uma sequência de DNA você pode selecionar o código genético que será usado para fazer a tradução. Certifique-se de que o código genético selecionado neste caso é ``Standard''.

\begin{figure}[h!]
\centering
   \includegraphics[width=10cm]{Figs/blastx_parameters.png}
  \caption{\label{blastx_parameters}Parámetros de búsqueda en BLAST}
\end{figure}

Un poco más abajo, haga click en el vínculo ``\textbf{Algorithm parameters}'', lo que le mostrará la serie de opciones que se ven en la Figura~\ref{blastx_parameters}. En la sección de ``\textbf{General parameters}'', encuentra el \textbf{Expected threshold} o \textbf{E value}. El E value es el número de alineamientos con un puntaje igual o mayor al obtenido que se espera que aparezcan por azar. En el momento de seleccionar los alineamientos importantes este es el parámetro mas importante; como regla general alineamientos con E value menor que $1\times10^{-5}$ representan secuencias homólogas. Sin embargo si está alineando secuencias muy cortas, e.g., 20 residuos, debe permitir alineamientos con un E value muy alto, alrededor de 100. En la sección ``\textbf{Scoring parameters}'', puede seleccionar la matriz de sustitución (escoja BLOSUM80) y la penalización por introducir gaps en el alineamiento.  Note que hay una diferencia entre el costo de introducir un gap y el de extenderlo \textcolor{red}{{\textquestiondown}A qué se debe esa diferencia?} Las opciones de abrir y extender gaps dependen de la matriz de sustitución seleccionada. Por favor observe como cambiando de matriz estas opciones cambian\footnote{En el enlace \url{http://www.ncbi.nlm.nih.gov/blast/html/sub_matrix.html} encontrará mayor información sobre la matriz de sustitución y la penalización de gaps.}.

Asegúrese que la opción \textbf{Filter} en la sección \textbf{Filters and Masking} esté seleccionada, con el fin de reducir el número de alineamientos con secuencias no relacionadas evolutivamente. \textcolor{red}{¿Qué programas usa BLAST para detectar regiones de baja complejidad?} \textcolor{red}{¿Qué funciones cumplen las opciones ``Mask for lookup table only'' y ``Mask lower case letters''?}

Ahora pinche el botón \textsc{BLAST} y espere sus resultados.

\begin{figure}[h!]
\centering
 \includegraphics[width=9cm]{Figs/blastxres1.png}
  \caption[Resultados blast: gráfica]{\label{fig:blastxres1}Representación gráfica de los mejores alineamientos obtenidos en la búsqueda con \Verb+blastx+}
\end{figure}

\begin{figure}[h!]
\centering
 \includegraphics[width=9cm]{Figs/blastxres2.png}
 \caption[Resultados blast: hits]{\label{fig:blastxres2}Listado de ``Hits''}
\end{figure}

En la parte superior de la página de resultado encuentra una gráfica como la que se ve en la Figura~\ref{fig:blastxres1}. Consiste en una representación de los mejores alineamientos con un código de colores que representa la longitud del alineamiento.

Un poco mas abajo encuentra la tabla con  los mejores hits, donde se muestra el identificador (Accesion number) de la secuencia hit, parte de su descripción, el puntaje del alineamiento entre su secuencia desconocida y la secuencia de la base de datos, el porcentaje de la secuencia ``query'' que está representada en el alineamiento, la identidad y el E value. Puede re-ordenar los datos en esta tabla pinchando en los nombres de las columnas.

La última parte de la sección de resultados esta compuesta por los alineamientos propiamente dichos (Figura~\ref{fig:blastxres3}). Aquí va a encontrar nuevamente el puntaje y el E value del alineamiento. Adicionalmente, además del alineamiento, encuentra el número de posiciones en que las dos secuencias eran idénticas y similares (de acuerdo a la matriz de sustitución) y el número de gaps.

\textcolor{red}{¿Qué indican las regiones de los alineamientos que aparecen en gris y en minúscula?}

\begin{figure}[h!]
\centering
 \includegraphics[width=9cm]{Figs/blastxres3.png}
 \caption[Resultados blast: alineamientos]{\label{fig:blastxres3}Alineamientos resultantes de la búsqueda con \Verb+blastx+}
\end{figure}

\textcolor{red}{{\textquestiondown}Qué puede decir sobre la función de su transcrito?}

La interfaz web de NCBI BLAST es muy amigable, pero tiene un par de problemas cuando trabajamos en genómica y proteómica, (i) no se pueden hace búsquedas contra bases de datos personalizadas o privadas y (ii) el número de secuencias que puede usar como \textbf{query} en cada búsqueda está restringido. La alternativa más poderosa para solucionar ambos problemas es instalar NCBI BLAST en un computador  local y configurar las bases de datos sobre las cuales se quiere realizar búsquedas (ver sección \ref{BlastCLI}).

\section{Encontrando la región genómica de un transcrito.}

Use la secuencia que se encuentra en el archivo \Verb+desconocido.nuc.fa+ para hacer una búsqueda BLAST, usando \Verb+blastn+ contra el genoma completo de \textit{A. thaliana}.  \textcolor{red}{¿Que opciones tiene que seleccionar para restringir su búsqueda a los cromosomas de \textit{Arabidopsis thaliana}?} Ya que BLAST realiza la búsqueda usando alineamientos locales, este resultado solo le dará una idea muy preliminar de la ubicación del transcrito en el genoma. Pero puede usar esta información para refinar la predicción del locus del transcrito usando \Verb+est2genome+ de EMBOSS. 

\textcolor{red}{¿Que opciones seleccionó para hacer la búsqueda en BLAST?¿Por qué?}

\textcolor{red}{Describa los resultados de la búsqueda.}

Los resultados de esta búsqueda nos permiten concluir que el locus del transcrito está en el cromosoma número 5 de \textit{A. thaliana}. \textcolor{red}{{\textquestiondown}Cuáles son las coordenadas aproximadas en el cromosoma? ¿Hay exones? Explique su respuesta.} Vamos a usar este resultado como entrada para \Verb+est2genome+. Primero extraiga de la secuencia del cromosoma 5, la región detectada por BLAST adicionándole $5000 pb$ corriente arriba y corriente abajo. \textcolor{red}{¿Cómo puede hacer esto?} Use \Verb+est2genome+ para refinar la predicción del locus. \textcolor{red}{{\textquestiondown}Qué ventajas ofrece usar \Verb+est2genome+ comparado con un simple BLAST?}

Para finalizar siga el enlace \url{http://www.ncbi.nlm.nih.gov/bookshelf/br.fcgi?book=comgen&part=psibl} y desarrolle el tutorial de PSI-BLAST.

\section{Blast$+$ en la línea de comandos}\label{BlastCLI}

Los ejecutables mas recientes, para diferentes platformas, de la suite Blast$+$ del NCBI los puede encontrar siguiendo el enlace \url{ftp://ftp.ncbi.nih.gov/blast/executables/blast+/LATEST}.

Para saber si la suite Blast$+$ está instalada en su computador ejecute el comando \Verb+blastp+, si la respuesta del sistema operativo es \Verb+comando no encontrado+ tendrá que descargar e instalar la suite Blast$+$. De lo contrario ya está listo para empezar a usar Blast$+$ desde la línea de comandos. 

Hay muchas opciones en los diferentes programas que componen la suite BLAST+, en este ejercicio solo tendremos tiempo de revisar unas pocas. Puede encontrar la documentación sobre estos en los siguientes enlaces:

\begin{itemize}
\item \url{http://blast.ncbi.nlm.nih.gov/Blast.cgi?CMD=Web&PAGE_TYPE=BlastDocs}
\item \url{http://www.ncbi.nlm.nih.gov/books/NBK1763/#CmdLineAppsManual.5_Cookbook}
\item \url{http://www.ncbi.nlm.nih.gov/books/NBK1763/#CmdLineAppsManual.4_User_manual}
\end{itemize}

Para desarrollar el ejercicio de hoy, descargue el archivo \Verb+TAIR10_pep_20101214.gz+ y descomprimalo como se muestra en la línea~\ref{downTAIRBLASTDB}. En este archivo encuentra todas las proteínas anotadas de \textit{Arabidopsis thaliana} correspondientes a la versión 10 de la anotación del genoma. Descargue las secuencias, en formato FastA, con números de acceso $BAK64065$ y $XP\_002889081$. Su objetivo es encontrar el mejor hit en la base de datos de proteínas de \textit{A. thaliana}, usando BLAST desde la línea de comandos.

\begin{Verbatim}[commandchars=!\{\},numbers=left,firstnumber=1,label=Ejecutando Blast+ en CLI,frame=topline,fontsize=\scriptsize]
!textcolor{red}{[user@server:~]$} mkdir ejercicio_blast
!textcolor{red}{[user@server:~]$} cd ejercicio_blast
!textcolor{red}{[user@server:~]$} wget http://biocomp-cms.uniandes.edu.co/exchange/TAIR10_pep_20101214.gz !label{downTAIRBLASTDB}
!textcolor{red}{[user@server:~]$} gunzip TAIR10_pep_20101214.gz
!textcolor{red}{[user@server:~]$} makeblastdb -in TAIR10_pep_20101214 -dbtype prot -parse_seqids -taxid 3702!label{makeblastdb}
!textcolor{red}{[user@server:~]$} blastp -query BAK64065.fasta -task blastp -db TAIR10_pep_20101214 \\ !label{blastpBAK64065}
-out BAK64065.blastp.out.txt -evalue 1e-5 -matrix BLOSUM62 -num_descriptions 1 -num_alignments 1
!textcolor{red}{[user@server:~]$} blastp -query BAK64065.fasta -task blastp -db TAIR10_pep_20101214 \\!label{blastpBAK64065TBL}
-out BAK64065.blastp.out.xml -evalue 1e-5 -matrix BLOSUM62 -num_descriptions 1 -num_alignments 1 \\
-outfmt 7
!textcolor{red}{[user@server:~]$}
!textcolor{red}{[user@server:~]$}
!textcolor{red}{[user@server:~]$}
!textcolor{red}{[user@server:~]$}
!textcolor{red}{[user@server:~]$}
\end{Verbatim} 

Antes de poder hacer búsquedas usando BLAST es necesario reformatear lel archivo que nos va a servir como base de datos. El comando \Verb+makeblastdb+ que  se distrubuye con la suite BLAST+ es el encargado de realizar esta tarea. En general para obtener información sobre como usar diferentes programas de la suite puede ejecutar \Verb+nombre_programa -help+. La línea~\ref{makeblastdb}, muestra el comando que debe ejecutar para crear la base de datos en formato blast. \textcolor{red}{¿Para que sirven cada uno de los argumentos que se pasan al programa \Verb+makeblastdb+?}

Teniendo la base de datos en el formato adecuado podemos hacer nuestra primera búsqueda. Use la secuencia proteínas BAK64065. Usaremos el programa \Verb+blastp+, para buscar el mejor hit de una proteína en una base de datos de proteínas (Línea~\ref{blastpBAK64065}). \textcolor{red}{¿Para que sirven cada uno de los argumentos que se pasan al programa \Verb+blastp+?}. Revise el archivo de salida, lo puede hacer con cualquiera de los siguientes comandos: \Verb+pico+, \Verb+less+. En la línea~\ref{blastpBAK64065TBL}, encuentra básicamente el mismo comando, solo que esta vez pedimos que el formato de salida sea tabular con la opción \Verb+-outfmt 7+. Hay muchos otros formatos de salida que se pueden pedir durante la búsqueda con el parámetro \Verb+-outfmt+. \textcolor{red}{Describa los formatos de salida posibles}.

\textcolor{red}{Haga la búsqueda con \Verb+blastp+ para la secuencia con número de acceso $XP\_002889081$, solo muestre los primeros 3 hits con e-value igual o menor que $10^{-10}$. Asegúrese de solicitar un formato de salida tabular que incluya la longitud de las secuencias ``Query''y ``Subject''}.

\chapter{Alinhamientos múltiplos}

Na teoria, os algoritmos de programação dinâmica descritos acima para o caso de alinhamentos de pares de sequências podem ser estendidos para o caso de um número arbitrário de sequências. Na prática, isso é computacionalmente muito caro, por isso outros algoritmos foram desenvolvidos que implementam atalhos na busca de alinhamentos ideais (heurísticos). O desenvolvimento de algoritmos para alinhamento de múltiplas sequências é uma área muito dinâmica na bioinformática. Atualmente existem dezenas de programas que implementam algoritmos diferentes \citetext{olha\citealp{Notredame2007} e \citealp{Lemey2009} para uma revisão recente do tópico}, um olhar especial é dedicado ao anlálise de milhoes de sequências \citep{Santus2023}.

Nesta sessão vamos desenvolver a prática apresentada no capítulo 3 de \citealp{Lemey2009}.

Vamos alinhar as sequências dos genes TRIM5 de diferentes espécies de primatas. TRIM5 é um fator de restrição viral que protege a maioria dos macacos do velho mundo (Cercopithecidae) da infecção pelo HIV. Esses dados foram originalmente analisados por \citealp{Sawyer2005}. Usaremos métodos de refinamento iterativo ({\sc Muscle}) para criar vários alinhamentos proteicos e, em seguida, comparar os resultados usando {\sc JalView}. Vamos criar os alinhamentos das sequências proteicas, e vamos gerar o alinhamento correspondente no nível nucleotídeo, terminando com a inspeção manual e o refinamento do alinhamento.

\section{Alinhando as sequências de aminoácidos de TRIM5 de primatas}

\subsection{{\sc MUSCLE}}

Para obter o alinhamento das sequencias no arquivo \Verb+primatesAA+.fasta pelo método de refinamento iterativo, usaremos o programa {\sc Muscle} \citep{Edgar2004}. Este programa esta instalado localmente em seu computador.

\begin{Verbatim}[commandchars=!\{\},numbers=left,label=MUSCLE,frame=topline,fontsize=\scriptsize]
	!textcolor{red}{(base)[user@server]}:~$ muscle -align primatesAA.fasta -output primates.afa

muscle 5.1.linux64 []  4.0Gb RAM, 4 cores
Built Feb 24 2022 03:16:15
(C) Copyright 2004-2021 Robert C. Edgar.
https://drive5.com

Input: 22 seqs, avg length 504, max 551

00:00 17Mb   CPU has 4 cores, running 4 threads
00:04 244Mb   100.0% Calc posteriors
00:04 246Mb   100.0% Consistency (1/2)
00:04 246Mb   100.0% Consistency (2/2)
00:04 246Mb   100.0% UPGMA5           
00:04 248Mb   100.0% Refining
\end{Verbatim} 


\subsection{Visualização e edição de Alinhamentos}

Usaremos o aplicativo \Verb+JalView+ instalado em seus computadores para exibir os alinhamentos. Pode utilizar o arquivo \Verb+primates.afa+ gerado pelo \Verb+MUSCLE+.

Siga o link \url{http://www.jalview.org/examples/editing.html}, se desejar, poderá revisar a documentação completa no link \url{http://www.jalview.org/help.html}.(Figura~\ref{fig:JalView})

\begin{figure}[h!]
\centering
 \includegraphics[width=15cm]{Figs/JalView.png}
 \caption{\label{fig:JalView}Tela do JalView}
\end{figure}

Para usar o jalview tem que ativar o ambiente conda correspondente como mostra a linha~\ref{activateJALVIEW}:

\begin{Verbatim}[commandchars=!\{\},numbers=left,label=ativando ambiente jalview,frame=topline,fontsize=\scriptsize]
!textcolor{red}{[user@server]$} conda activate jalview !label{activateJALVIEW}
!textcolor{red}{(jalview)[user@server]$}
\end{Verbatim}


\chapter{PSSMs, Logo de Sequências e HMMs}

\section{PSSM}

As matrizes de pontuação específicas da posição ``Position-specific scoring matrices'' (PSSMs) oferecem uma maneira sensível de representar a variabilidade em um alinhamento. Os PSSMs são construídos com base no alinhamento múltiplo, por exemplo, dos sites de vinculação de fatores de transcrição.

Abaixo está uma matriz que foi obtida a partir da base de promotores de \textit{Saccharomyces cerevisiae}\footnote{\url{http://rulai.cshl.edu/SCPD/}} e construída usando um alinhamento de 12 sites de vinculação do fator de transcrição de levedura Pho4p.

\vskip5pt
\begin{center}
\begin{tabular}{l|r r r r r r r r r r }
A &  3 &  2 &  0 & 12 &  0 &  0 &  0 &  0 &  1 &  3\\
C &  5 &  2 & 12 &  0 & 12 &  0 &  1 &  0 &  2 &  1\\
G &  3 &  7 &  0 &  0 &  0 & 12 &  0 &  7 &  5 &  4\\
T &  1 &  1 &  0 &  0 &  0 &  0 & 11 &  5 &  4 &  4\\
\end{tabular}
\end{center}

Cada linha representa um resíduo (A, C, G ou T) e cada coluna uma posição no conjunto de sequências alinhadas. Algumas posições são perfeitamente preservadas em todas as sequências, enquanto outras apresentam algumas alternativas.


Ao utilizar esses tipos de matrizes para pesquisar, as posições mais conservadas impõem restrições mais fortes do que aquelas em que qualquer resíduo pode ser apresentado.

{\color{red}
Siga o link \url{http://rulai.cshl.edu/cgi-bin/SCPD/getfactor?ABF1,BAF1} e responde:
\begin{itemize}
\item Qual é o tamanho do alfabeto?
\item Qual é a largura da matriz?
\item Quantos sites de vinculação Abf1p estão armazenados no Banco de Dados de Promotores de Levedura (SCPD)?
\item Quais programas EMBOSS eu poderia usar para pesquisar com PSSMs?
\end{itemize}
}

\section{logo de sequências}

Logo de sequência são uma representação gráfica de alinhamentos múltiplos  baseados na teoria das informações\footnote{\url{https://en.wikipedia.org/wiki/Sequence_logo}}

A Figura~\ref{fig:lexALogo} corresponde ao logo da sequência do site de vinculação do fator de transcrição LexA de \textit{Escherichia coli}

\begin{figure}[h!]
\centering
 \includegraphics[width=9cm]{Figs/lexA.png}
 \caption{\label{fig:lexALogo}Lexa junte-se ao logo da sequência do site}
\end{figure}

A altura do resíduo está correlacionada com sua frequência no alinhamento  múltiplo {Para obter mais informações, consulte: \citealp{Schneider1990}}.

{\color{red}
Seguindo o link \url{http://weblogo.threeplusone.com/}, crie o logo das sequências dos sites de vinculação do fator de transcrição Abf1p que estudeu na seção anterior. Para realizar este exercício, você precisa recuperar todos os sites de ação do Abf1p disponíveis no SCPD.

O que o eixo y representa? ou seja, como o conteúdo das informações de cada posição é calculado?
}

\section{Modelos ocultos de Markov: HMMs}

Mesmo que você não encontre proteínas homólogos ao realizar uma busca com blast, você ainda tem outras opções.

A principal razão pela qual ele não encontra homologues triviais é que pesquisas com sequências usando ferramentas como BLAST têm baixa sensibilidade. O BLAST normalmente não encontra proteínas homólogas que tenham menos de 30\% de identidade. No entanto, algumas proteínas podem ter a mesma estrutura tridimensional e ter apenas 10\% de identidade.

Uma estratégia muito útil para encontrar contrapartes distantes é baseada no uso de Modelos Hidden Markov (HMMs). Um HMM nada mais é do que uma maneira de definir motivos ou domínios.

Para criar um HMM tudo o que você precisa é de um alinhamento múltiplo, que será usado para criar uma representação probabilística, que pode então ser usada para procurar sequências relacionadas.

Os bancos de dados Pfam \citep{Finn2010} e {\sc Superfamily} \citep{Wilson2009} são coleções de múltiplos alinhamentos para os quais os HMMs foram criados e são usados para anotar sequências proteicas. A maior parte do trabalho dos curadores dessas bases de dados é criar os alinhamentos múltiplos.

\subsection{Procurando os domínios de uma proteína}

Vamos usar a seguinte proteína para fazer uma pesquisa no Pfam:

\begin{Verbatim}[commandchars=!\{\},label=proteína desconhecida,frame=topline,fontsize=\scriptsize]
>seq
MEYWHYVETTSSGQPLLREGEKDIFIDQSVGLYHGKSKILQRQRGRIFLTSQRIIYIDDAKPTQ
NSLGLELDDLAYVNYSSGFLTRSPRLILFFKDPSSKDELGKSAETASADVVSTWVCPICMVSNETQGEFTKD
TLPTPICINCGVPADYELTKSSINCSNAIDPNANPRNQFGVNSENICPACTFANHPQIGNCEICGHRLPNAS
KVRSKLNRLNFHDSRVHIELEKNSLARNKSSHSALSSSSSTGSSTEFVQLSFRKSDGVLFSQATERALENIL 
TEKNKHIFN
\end{Verbatim} 

Vá para o site de Pfam\footnote{\url{https://pfam.xfam.org/}} e selecione ''Search'' depois "Sequence"  e cole a sequência da proteína de interesse na caixa de texto para a busca e clique no botão ''submmit'' para iniciar a busca. Figura~\ref{PfamResults} exibe a página de resultados.

\begin{figure}[h!]
\centering
 \includegraphics[width=14cm]{Figs/PfamResults.png}
 \caption{\label{PfamResults}Resultados de Pfam}
\end{figure}

{\color{red}
\begin{itemize}
\item ¿Quál é a diferença entre ``Significant Pfam-A matches'' e ``Insignificant Pfam-A matches''?
\item ¿O que é Pfam-A? 
\item ¿O que é Pfam-B?
\end{itemize}
}

A primeira seção de resultados ``Significant Pfam-A matches'', informa-nos que há dois um hits, o modelo ''Vps36\_ESCRT-II'', com uma pontuação de $97.8$ e um e-value de $3.3^{-28}$ e o moduelo  ''Vps36-NZF-N'' com uma pontuação de $66.9$ e um e-value de $8.1^{-19}$. Encontramos também as coordenadas do domínio, em relação à proteína de consulta e em relação ao modelo.

Uma das principais características, e valores adicionados da Pfam, é que cada um dos modelos da Pfam-A tem sido estudado por um especialista que definiu uma série de limiares para definir hits significativos. A pontuação de limiar mais importante corresponde ao ``gathering cutoff''. \textcolor{red}{Qual é o ``gathering cutoff'' modelo ''Vps36\_ESCRT-II''?}

Clique no nome do modelo. Isso o levará a uma página com informações detalhadas sobre esse modelo em particular. Entre outros, você pode descobrir que outras espécies estão presentes nesse modelo (''Species''). Você pode baixar o modelo ('Curation'') ou o alinhamento múltiplo (''Alignments''), entre outras informações.

{\color{red}
Use a sequência proteica ANAC092 e determine quais domínios estão presentes}

\subsection{Visualização de HMMs}

Você também pode exibir HMMs na forma de logo de sequências. Você pode usar o aplicativo \Verb+LogoMat-M+ encontrado no link \url{http://www.sanger.ac.uk/cgi-bin/software/analysis/logomat-m.cgi}. Ou no mesmo site do pfam

\textcolor{red}{Exibir o logo do domínio ``zf-C2H2''}

\chapter{Diseño de primers para PCR}

La parte teórica que aquí se presenta consiste en un resumen del texto que se encuentra siguiendo el enlace: \url{http://www.premierbiosoft.com/tech_notes/PCR_Primer_Design.html}.

La reacción en cadena de la polimerasa (PCR) inventada por Kary Mullis en la década de los 80s del siglo XX \citep{Mullis1987} es considerada uno de los inventos mas importantes en biología molecular. Mediante esta reacción pequeñas cantidades de material genético se pueden amplificar de tal forma que pueden ser identificadas y/o manipuladas.

La PCR involucra los siguientes pasos:

\begin{description}
\item[Denaturación] El objetivo de este paso es convertir las moléculas de ADN de doble cadena en cadenas sencillas.
\item[Anillamiento] Durante este paso los primers hibridan con las hebrahttps://www.overleaf.com/project/624c8ebfae55316607b5f12bs molde de cadena sencilla.
\item[Extensión] La ADN polimerasa extiende los primers.
\end{description}

Esos pasos dependen y son muy sensibles a la temperatura. Las temperaturas usadas comúnmente son 95$^\circ$C, 60$^\circ$C y 72$^\circ$C, respectivamente.

Un buen diseño de primers es esencial para obtener reacciones exitosas. A continuación se describen las principales consideraciones a tener en cuenta durante el diseño.

\textbf{Longitud de los primers}: Normalmente se acepta que la longitud óptima para primers de PCR está entre 18 y 22 pb. Con esta longitud son lo suficientemente largos para asegurar especificidad y lo suficientemente pequeños para que se unan fácilmente al ADN molde a la temperatura de anillamiento.

\textbf{Temperatura de fusión del primer (T$_m$)}: Se define como la temperatura a la cual la mitad de las moléculas de ADN de doble cadena se van a disociar y volverse de cadena sencilla. Es una forma de indicar la estabilidad del duplex. Primers con temperaturas de fusión entre 52$^\circ$C y 58$^\circ$C normalmente producen los mejores resultados. Primers con temperaturas de fusión superiores a 65$^\circ$C tienen tendencia a formar anillamientos secundarios. El contenido de GC de la secuencia da una buena indicación de la temperatura de fusión del primer. Mayor precisión en su cálculo se alcanza empleado la teoría termodinámica de los vecinos mas cercanos, según la cual:

\begin{equation}
 T_m(^\circ C) = \lbrace\Delta H/\Delta S +Rln(C)\rbrace-273.15 \label{eq:tm}
\end{equation}

donde:

\begin{description}
\item[$\Delta H$ (kcal/mol)]: H es la entalpía. La entalpía es la cantidad de energía calórica que poseen las sustancias. $\Delta H$ es el cambio en entalpía. En la formula \ref{eq:tm}, la $\Delta H$ se obtiene de sumar las entalpías de los pares de di-nucleótidos que son vecinos mas cercanos.
\item[$\Delta S$ (kcal/mol)]: S es la cantidad de desornen deun sistema, recibe el nombre de entropía. $\Delta S$ es el cambio en la entropía. Se obtiene sumando las valores de entropía de pares de di-nucleótidos que son vecinos mas cercanos. Normalmente se adiciona una correción a los parámetros de vecinos mas cercanos. Esta corrección representa el contenido de sales.
\item[$\Delta S$ (corrección por sales)]: $ \Delta S (1M NaCl )+ 0.368  N  ln([Na+]) $, donde $N$ es el número de pares de nucleótidos en el primers, y $[Na+]$ son los equivalentes de sal en mM.
\end{description}

\textbf{Temperatura de anillamiento de los primers}: La T$_M$ es un estimador de la estabilidad del híbrido ADN-ADN y es importante para poder estimar la temperatura de anillamiento (T$_a$).  T$_a$ muy altas harán que se formes pocos híbridos primer - molde resultando en una reducción del producto de PCR.  T$_a$ muy bajas podrán causar anillamientos no específicos. La siguiente ecuación permite estimar la T$_a$ a partir de la T$_m$

\begin{equation}
T_a = 0.3  T_m(primer) + 0.7 T_m (product) – 14.9
\end{equation}

donde,

\begin{description}
\item[T$_m$(primer)]  Es la temperatura de fusión de los primers
\item[T$_m$(product)] Es la temperatura de fusión del producto
\end{description}

\textbf{Contenido de GC}: La proporción de G$+$C en el primer debe ser de 40\% a 60\%.

\textbf{Gancho de GC}: La presencia de las bases G o C en las ultimas 5 bases del extremo 3' del primer (GC clamp) ayuda a tener una unión mas especifica en ese extremos debido a la unión mas fuerte entre G  y C. Sin embargo, se deben evitar mas de 3 Gs o Cs consecutivas en las últimas 5 bases del extremo 3'.

\textbf{Estructuras secundarias de los primers}: La presencia de estructuras secundarias producidas por interacciones intra o intermoleculares puede llevar a una disminución en la producción del amplímero o no producción de este. Esas estructuras disminuyen la cantidad de primer disponible para la reacción.

\textbf{Evitar hidridación cruzada}: Los primers diseñados para una secuencias no deben amplificar otro gen en la mezcla. La opción mas común es tomar los primers candidatos y compararlos contra bases de datos de genes usando una herramienta como BLAST.

\section{Diseño de primers usando Quantprime}

{\sc Quantprime}\footnote{Hay un tutorial disponible en el sitio de {\sc Quantprime} que describe con mayor detalle cada paso y opción.} es una herramienta flexible para el diseño de primers a mediana y gran escala \citep{Arvidsson2008}, principalmente para PCR en tiempo real usando SYBR GREEN. {\sc Quantprime} usa \Verb+primer3+ \citep{Rozen2000}  como motor para la creación de primers y agrega diversas capaz de verificación contra distintas bases de datos y anotación de genomas para proponer primers con mayor probabilidad de funcionar en ensayos experimentales.

Una de las principales ventajas de {\sc Quantprime} e sque aprovecha la anotación de genoma y\/o colecciones de ESt que estén disponibles al público. Por ejemplo, la anotación de genomas puede ser explotada para producir primers que anillen sobre border de exones, disminuyendo considerablemente la probabilidad de amplificar ADN genómico en ensayos de evaluación de la expresión de genes.

Vaya a la página de {\sc Quantprime}, \url{http://www.quantprime.de/}. Con el fin de prestar un mejor servicio a los usuarios finales, es necesario registrarse y activar la cuenta siguiendo las instrucciones que llegarán a su correo electrónico luego de registrarse.

El primer paso en el flujo de trabajo de {\sc Quantprime} es crear un nuevo proyecto, encontrará un botón \Verb+New project+ en el menú de la izquierda. La Figura~\ref{fig:QP_createproject}, muestra el formulario de creación de proyectos. Allí tendrá que dar un nombre a su proyecto, este le servirá para almacenar su información en el servidor de  {\sc Quantprime} y mantener varios proyectos en paralelo si así lo desea. A continuación tiene que seleccionar el organismo de interés y la versión de la anotación de su genoma o de disponibilidad de ESTs, según sea el caso. Para este ejercicio seleccione \textit{Arabidopsis thaliana} como organismo y \textbf{TAIR release 9}, como versión de la anotación. La última sección corresponde a la selección del protocolo de cuantificación, i.e., usando SYBR-GREEN, en tiempo real, o al final de la PCR usando geles de agarosa (end-point  PCR). Para cada una de esas opciones puede seleccionar si desea que los primers tenga hibridación cruzada con diferente variantes de splicing o no. En este ejercicio vamos a usar la segunda opción.

\begin{figure}[h!]
\centering
 \includegraphics[width=9cm]{Figs/QP_createproject.png}
 \caption{\label{fig:QP_createproject}Creación un proyecto en {\sc Quantprime}}
\end{figure}

El siguiente paso, consiste en  incluir los transcritos para los cuales se desea diseñar primers. La Figura~\ref{fig:QP_transcripts}, muestra el formulario que nos permite completar este paso. Tiene dos opciones: i) si conoce los identificadores de los genes de interes, solo los tiene que poner en la caja de texto y presionar el botón \Verb+Add to project+, de lo contrario puede hacer búsqueda BLAST dentro de {\sc Quantprime} para encontrar los identificadores a partir de secuencias propias. En este caso vamos a diseñar primers para los genes: AT2G20825 y AT4G28190, que pertenecen a una familia pequeña de factores de transcripción conocida como ULT. Asegurese de adicionar esos identificadores a su proyecto y luego usar el botón \Verb+Select all+, seguido de \Verb+find primers+.

\begin{figure}[h!]
\centering
 \includegraphics[width=9cm]{Figs/QP_transcripts.png}
 \caption{\label{fig:QP_transcripts}Adicionando transcritos al proyecto en  {\sc Quantprime}}
\end{figure}

En este punto se iniciará el proceso de búsqueda de primers y su posterior verificación explotando la anotación del genoma de \textit{Arabidopsis thaliana}. La Figura~\ref{fig:QP_progress} muestra la ventana de progreso de la búsqueda. Si lo desea puede cerrar esta ventana y volver as tarde a recuperar sus resultados, esta es la ventaja de estar registrado en el sitio. En la Figura~\ref{fig:QP_progress} se ve un indicador de progreso y una serie de cuatro casillas coloreadas por gen. La casilla de color verde oscuro indica el número de primers muy buenos que fueron encontrados, que cumplían con todos los criterios de búsqueda, i.e., específicos para el transcrito de interés, no amplifica ADN genómico, primers individuales no anillan con otros cDNAs. La casilla color verde claro indica el número de primers bueno, peor que podrían amplificar ADN genómico o alguno de los dos primers podría anillar con otro cDNA y por lo tanto reducir la eficiencia de la amplificación. En la casilla amarilla aparece el número de primers que se consideran adecuados, estos pueden amplificar ADN genómico, primers individuales pueden anillar a otros cDNAs. La casilla roja indica el número de primers fallidos.

La casilla verde oscura solo va a estar desactivada en aquellos casos en que la especie de interés no tenga información de su genoma en la base de datos de {\sc Quantprime}.

\begin{figure}[h!]
\centering
 \includegraphics[width=9cm]{Figs/QP_progress.png}
 \caption{\label{fig:QP_progress}{\sc Quantprime} buscando primers para los genes solicitados}
\end{figure}

Una vez la búsqueda ha terminado , presione el botón \Verb+List best primers+ para obtener una lista detallada de los primers encontrados.

La lista de pares de primers está ordenada de acuerdo al color, como se explicó anteriormente, y en segundo lugar por el puntaje de rango de Primer3, la columna en el extremo derecho, el cual refleja la desviación de los criterios de diseño optimo y el riesgo de formar estructuras secundarias y dímeros de primers. Los mejores primers son aquellos en que este número es más pequeño.

El botón \Verb+Select best+ selecciona los mejores primers de los genes que se analizaron. 

\begin{figure}[h!]
\centering
 \includegraphics[width=9cm]{Figs/QP_results.png}
 \caption{\label{fig:QP_results}Listado de los mejores primers encontrados por {\sc Quantprime}}
\end{figure}

Si desea ver información mas detallada sobre cada par de primers haga clic sobre el par de interés, esto lo conducirá a la página de información de ese par de primers en particular (Figura~\ref{fig:QP_primerinfo}. En la parte superior de esta página encontrará información sobre el transcrito para el cual se diseñaron los primers.

\begin{figure}[h!]
\centering
 \includegraphics[width=9cm]{Figs/QP_primerinfo.png}
 \caption{\label{fig:QP_primerinfo}Página de información para un par de primers seleccionados}
\end{figure}

En la Figura~\ref{fig:QP_primerinfo}, el amplicón aparece marcado con fondo gris, primers que anillan en límites entre exones aparecen en color verde, y el limite entre los exones se indica con el símbolo \textasciicircum, los primers que aparecen en color azul no anillan sobre límites entre exones. En está página también encuentra la T$_m$ de cada primer, del amplicón y la temperatura óptima de anillamiento, asi como otras características de los primers. Al final de la página encuentra información sobre los resultados de las pruebas de especificidad que se llevaron  a cabo.

Revise la página de resultados del par de primer para un par bueno o muy bueno y para un para adecuado o malo, identifique las diferencias.

Vuelva a la página de resultados. En la parte inferior de la página encontrará el botón \Verb+Export primer pairs+, que le permite enviar los pares de primers seleccionados a un archivo de texto.

\section{Crear primers a partir de alineamientos de proteínas}

En esta sección vamos a usar el programa iCODEHOP\footnote{\url{https://icodehop.cphi.washington.edu/i-codehop-context/Welcome}} para diseñar primers a partir de un alineamiento de proteínas.

Vamos a usar el alineamiento de las 22 secuencias de primates que usó hace alguans semanas. Asegurese que el alineamiento está en formato fasta o clustal.

Siga el enlace \url{https://icodehop.cphi.washington.edu/i-codehop-context/Welcome} que lo llevará al sitio web de iCODEHOP. 

Inicie una sesión, esto lo llevará a una nueva página que luce como aparece en la Figura~\ref{fig:CODEHOP_start}, seleccione la opción \Verb+Design Primers+

\begin{figure}[h!]
\centering
 \includegraphics[width=9cm]{Figs/CODEHOP_start.png}
 \caption{\label{fig:CODEHOP_start}Página de inicio en iCODEHOP}
\end{figure}

\begin{figure}[hb!]
\centering
 \includegraphics[width=9cm]{Figs/CODEHOP_design.png}
 \caption{\label{fig:CODEHOP_design}Diseño de primer en iCODEHOP}
\end{figure}

En la página de diseño de primers puede seleccionar diferentes fuentes de datos de alineamiento de proteínas (Figura~\ref{fig:CODEHOP_design}). En este ejercicio haga clic en el botón \Verb+Select+ que se encuentra en frente de \textbf{\Verb+A CLUSTAL or FASTA formated alignment+}, seleccione el archivo de alineamiento de 22 secuencias de primates. La nueva página aparece como la Figura~\ref{fig:CODEHOP_design2}. Ahora puede procede con el análisis haciendo clicl en el botón \Verb+Proceed with Analysis+.

\begin{figure}[h!]
\centering
 \includegraphics[width=9cm]{Figs/CODEHOP_design2.png}
 \caption{\label{fig:CODEHOP_design2}Diseño de primer en iCODEHOP}
\end{figure}

El siguiente paso en el algoritmo CODEHOP es determinar los BLOCKS\footnote{¿Qué son y como se determinan los BLOCKS?.}, esto es hecho automáticamente por iCODEHOP (figura~\ref{fig:CODEHOP_design3}). En la siguiente página selecciones el código genético y la tabal de uso de codones que serán usada en el diseño de primers. Hay otros parámetors que puede variar antes de iniciar la busqueda de primers. ¿Qué controla cada uno de esos parámetros?

\begin{figure}[h!]
\centering
 \includegraphics[width=9cm]{Figs/CODEHOP_design3.png}
 \caption{\label{fig:CODEHOP_design3}Detección de BLOCKS en el alineamineto de secuencias de proteínas. Se diseñaran primers para cada BLOCK}
\end{figure}

Uno vez esté satisfech@ con su selección de parámetros, puede dar clic en el botón \Verb+Look for primers+ para iniciar la búsqueda de primers en los BLOCKS detectados. Sea paciente la búsqueda de primers puede tomar bastante tiempo. Al finalizar la búsqueda los resultados se mostrarán en forma gráfica como aparece en la Figura~\ref{fig:CODEHOP_resultados}, al hacer clicl sobre los primers, encotrará información detallada.

Cada uno de los rectángulos que aparece en la imagen representa los BLOCKS originales, i.e., alineamientos múltiples sin gaps. El nombre del BLOCK aparece en la esquina superior izquiera del rectángulo.

Debajo del nombre del BLOCK encontrará una fila con con información sobre el número de amino ácidos que constituyen el BLOCK y la distancia en amino ácidos al BLOCK anterior y al siguiente (esto \'{u}ltimo en par\'{e}ntesis).

Enseguida encuentra el rectángulo que representa el BLOCK, aparece la secuencia consenso del alineamiento m\'{u}ltiple. El s\'{i}mbolo \textbf{*} aparece encima de los residuos completamente conservados. Amino \'{a}acidos en may\'{u}scula representan sitios altamente conservados mientras que aquellos en min\'{u}scula representan sitios con un bajo nivel de conservaci\'{o}n.

Debajo del rectángulo encuetrara los primers degenerados representados por flechas. Las flechas que se dirigen a la derecha, corresponden a los primer \textbf{forward}, las uqe se dirigen a la izquierda corresponden a los primers \textbf{reverse}. si una flecha es roja significa que iCODEHOP no pudo extender la región consenso del gancho en su longitud completa. Esto pasa cuando hay poco conservación en el extremo 5' de la región CORE degenerada de un primer.

Puede seleccionar un primer particular haciendo click sobre la flecha que lo representa y obtener información adicional usando el botón \Verb+Complete summary+ en la parte superior de la página.

En la página \Verb+Compete summary+ encuentra información detallada sobre el BLOCK que se usó para diseñar el primer seleccionado, así como ss temeratuas de anillamiento. Mas abajo encuentra una tabla con todos los primer potenciales para usar como compañeros del primer seleccionado, cada uno con infomración del nombre del BLOCK que se usó para su diseño, y sus temperaturas de anillamiento.

\begin{figure}[h!]
\centering
 \includegraphics[width=9cm]{Figs/CODEHOP_resultados.png}
 \caption{\label{fig:CODEHOP_resultados}Detección de BLOCKS en el alineamineto de secuencias de proteínas. Se diseñaran primers para cada BLOCK}
\end{figure}

\chapter{Montagem de genomas}

\begin{figure}[h!]
\centering
 \includegraphics[width=7cm]{Figs/montagem_de_genoma.jpeg}
 \caption{\label{fig:fastqc}Montagem de genomas}
\end{figure}


O processo de montagem de genomas consiste na recuperação da sequência original a partir dos fragmentos de ADN produzidos pelo sequenciamento

Vamos montar o genoma de \emph{Komagataeibacter rhaeticus}\footnote{NCBI accesion: NZ\_CP050139} a partir de leituras produzidas por duas tecnologias Illumina e PacBio que encontrara na pasta PRATICA\_ENSAMBLAGEM\_DE\_GENOMAS onde tera que descomprimir um arquivo que contem as leituras.

\begin{Verbatim}[commandchars=!\{\}, numbers=left,label= indo na pasta de trabalho, frame=topline,fontsize=\scriptsize]
(base)user@server:$ cd PRATICA_ENSAMBLAGEM_DE_GENOMAS
\end{Verbatim}


\subsection{Limpar sequencias}
As tecnologias de segunda geração como Illumina produzem leituras que devem ser filtradas por critérios de qualidade de leitura, cumprimento, presença de adaptadores, barcodes, contaminantes e artefatos.
Por enquanto os sequenciadores de tecnologias de terceira geração como PacBio geralmente fazem esse processo por nós

Temos dois arquivos \Verb+fastq+ com leituras pareadas de illumina\footnote{illumina\_R1.fq}\footnote{illumina\_R2.fq}

\begin{itemize}
\item \textcolor{red}{O Que são leituras pareadas?}
\item \textcolor{red}{Explique a organização do formato FASTQ}
\end{itemize}

Limpamos eles usando BBduk\footnote{\url{https://jgi.doe.gov/data-and-tools/software-tools/bbtools/bb-tools-user-guide/bbduk-guide/}}, BBduk pode fazer quality-trimming, filtrado, adapter-trimming, contaminant-filtering usando kmer matching.

\begin{Verbatim}[commandchars=!\{\}, numbers=left,label= Limpando leituras Illumina com bbduk,frame=topline,fontsize=\scriptsize]
(base)user@server:$ conda activate bbduk_env
(bbduk_env)user@server:$ bbduk.sh in1=illumina_R1.fq in2=illumina_R2.fq out1=./bbduk/bbduk.R1.fq \
out2=./bbduk/bbduk.R2.fq minlength=75 qtrim=w trimq=20
(bbduk_env)user@server:$ conda deactivate
\end{Verbatim}


\begin{itemize}
\item \textcolor{red}{Qual é o significado das opçoes qtrim=w e trimq=20}
\end{itemize}


bbduk vai gerar dois arquivos filtrados bbduk.R1.fq e bbduk.R2.fq na pasta ./bbduk que usaremos na montagem do genoma

Agora vamos usar o programa FASTQC\footnote{\url{https://github.com/s-andrews/FastQC}} para vizualizar o efeito da filtragem do bbduk nas nossas leituras illumina

\begin{Verbatim}[commandchars=!\{\}, numbers=left,label= Abrindo o FASTQC,frame=topline,fontsize=\scriptsize]
(base)user@server:$ conda activate fastqc_env 
(fastqc_env)user@server:$ fastqc
(fastqc_env)user@server:$ conda deactivate
\end{Verbatim}

\begin{itemize}
\item \textcolor{red}{Carregar as leituras de Illumina antes e depois de filtrar e explique as diferenças no item "Per base quality" do reporte do fastqc}
\end{itemize}

\begin{figure}[h!]
\centering
 \includegraphics[width=12cm]{Figs/fastqc.png}
 \caption{\label{fig:fastqc}Tela FASTQC}
\end{figure}

\subsection{Montagem de genoma usando dados Illumina}
Usaremos o software Unicycler\footnote{\url{https://github.com/rrwick/Unicycler}} um pipeline de varios softwares para ensamblegem de genomas bacterianos. Vamos ensamblar o genoma usando os arquivos arquivos bbduk.R1.fq e bbduk.R2.fq filtrados que gerou o bbduk

\begin{Verbatim}[commandchars=!\{\}, numbers=left,label= Montando leituras com Unicycler,frame=topline,fontsize=\scriptsize]
(base)user@server:$ conda activate unicycler_env
(unicycler_env)user@server:$ unicycler  -1 ./bbduk/bbduk.R1.fq -2 ./bbduk/bbduk.R2.fq -o ./unicycler/
(unicycler_env)user@server:$ conda deactivate
\end{Verbatim}

Agora pode dar uma olhada nos aquivos gerados pelo Unicycler na pasta ./unicycler e usar o programa Bandage\footnote{\url{https://rrwick.github.io/Bandage/}}\footnote{\url{https://pubmed.ncbi.nlm.nih.gov/26099265/}} para visualizar o arquivo assembly.gfa


\begin{Verbatim}[commandchars=!\{\}, numbers=left,label= Abrindo o Bandage,frame=topline,fontsize=\scriptsize]
(base)user@server:$ conda activate bandage_env 
(bandage_env)user@server:$ Bandage
(bandage_env)user@server:$ conda deactivate
\end{Verbatim}

\begin{figure}[ht]
\centering
   \includegraphics[width=12cm]{Figs/bandage.png}
  \caption[Visualização da montagem do unicycler]{\label{bandage}Montagem Illumina}
\end{figure}

\begin{itemize}
\item \textcolor{red}{De que maneira estão representadas as regiões repetitivas do genoma na ensamblagem do unicycler?}
\end{itemize}

\subsection{Montagem de genoma usando dados PacBio}
Usaremos o software Flye\footnote{\url{https://github.com/fenderglass/Flye/}}. Flye é uma montadora de novo para leituras de sequenciamento de moléculas únicas, como as produzidas pela PacBio e Oxford Nanopore Technologies.

Usaremos o arquivo com leituras PacBio PacBio.fq

\begin{Verbatim}[commandchars=!\{\}, numbers=left,label= Montando leituras com Flye,frame=topline,fontsize=\scriptsize]
(base)user@server:$ conda activate flye
(flye_env)user@server:$ flye --pacbio-hifi PacBio.fq -o ./flye
(flye_env)user@server:$ conda deactivate
\end{Verbatim}

Agora poder dar uma olhada nos aquivos gerados pelo Flye na pasta ./flye e usar o programa Bandage para visualizar o arquivo assembly\_graph.gfa

\begin{figure}[ht]
\centering
   \includegraphics[width=12cm]{Figs/bandage_pacbio.png}
  \caption[Visualização da montagem do flye]{\label{bandage_pacbio}Montagem PacBio}
\end{figure}

\begin{itemize}
\item \textcolor{red}{Explique porque as diferenças nas gráficas geradas pelas duas ensamblagens: Illumina e PacBio}
\end{itemize}\


\subsection{Avaliando a qualidade das montagens com QUAST}
Agora vamos utilizar QUAST\footnote{\url{http://quast.sourceforge.net/}} para comparar a qualidade das nossas montagenss. No primeiro lugar a montagem de Illumina (./unicycler/assembly.fasta) comparado com a montagem do genoma de referencia que se encontra em complete\_genome.fasta.

\begin{Verbatim}[commandchars=!\{\}, numbers=left,label= QUAST,frame=topline,fontsize=\scriptsize]
(base)user@server:$ conda activate quast_env
(quast_env)user@server:$ quast -o ./quast/illumina -r complete_genome.fasta ./unicycler/assembly.fasta
(quast_env)user@server:$ conda deactivate
\end{Verbatim}

Olha os arquivos gerados pelo quast na pasta ./quast/illumina e analise o arquivo report.html.
O quast também pode funcionar sem genoma de referencia no caso que seja necessário.

\begin{itemize}
\item \textcolor{red}{Rodar o QUAST sem o genoma de referencia para os dados de Illumina. Compare os resultados da tabela no arquivo report.html e explique as diferenças} \textcolor{red} {(lembre que a melhor forma de  facilitar as coisas é organizar os arquivos e pastas de uma maneira logica\footnote{\url{https://zapier.com/blog/organize-files-folders/}}  e por exemplo ter uma pasta para cada analise}
\item \textcolor{red}{Repetir o processo com os dados PacBio utilizando o arquivo assembly.fasta que se encontra na pasta ./flye e compare os resultados de Illumina e PacBio usando a referencia explique a diferenças no report.html}
\end{itemize}

\subsection{Avaliando a qualidade das montagens com BUSCO}.
O BUSCO\footnote{\url{https://github.com/robsyme/busco}} informa da quantidade de genes ortologos de copia única que espera-se encontrar na nossa ensamblagem a comparação com o que se encontra em organismos do mesmo grupo taxonômico.
Vamos avaliar a montagen de Illumina ./unicycler/assembly.fasta

\begin{Verbatim}[commandchars=!\{\}, numbers=left,label= BUSCO para dados Illumina,frame=topline,fontsize=\scriptsize]
(base)user@server:$ conda activate busco_env
(busco_env)user@server:$ busco -i ./unicycler/assembly.fasta -o illumina -m geno \
--lineage rhodospirillales_odb10 --out_path ./busco
(busco_env)user@server:$ conda deactivate
\end{Verbatim}

Olha os arquivos gerados pelo BUSCO na pasta ./busco/illumina

\begin{itemize}
\item \textcolor{red}{Que significa a opção --lineage no comando do BUSCO}
\item \textcolor{red}{Repetir o processo de avaliação de qualidade da montagem com BUSCO utilizando os dados PacBio no arquivo gerado pelo flye assembly.fasta que se encontra na pasta ./flye. Compare os resultados e explique as diferenças entre as montagens de leituras illumina e leituras PacBio segundo BUSCO}
\item \textcolor{red}{Segundo os resultados do BANDAGE, BUSCO e QUAST qual é a melhor montagem e porque}
\end{itemize}

\chapter{Anotação de Genomas}

Depois de ter montado as suas leituras na ensamblagem do genoma, é útil saber quais são as características genômicas dessa ensamblagem. O processo de identificar e rotular essas características é chamado de anotação do genoma.

Neste capitulo vamos anotar a montagem do genoma montado com leituras PacBio feito no capitulo 17

\subsection{Usando prokka para anotar genomas bacterianos}.

Prokka\footnote{\url{https://pubmed.ncbi.nlm.nih.gov/24642063/}}\footnote{\url{https://github.com/tseemann/prokka}}, é uma ferramenta de software de linha de comando para anotação de genomas bacterianos

No primeiro lugar copiamos a ensamblagem feita no capitulo 17 correspondente aos dados de PacBio, esta montagem que você fez, deve-se encontrar em

\emph{$\sim$/PRATICA\_ENSAMBLAGEM\_DE\_GENOMAS/flye/assembly.fasta}

A nossa pasta de trabalho de hoje é

\emph{$\sim$/GENOME\_ANNOTATION}

Copiamos a ensamblagem na nossa pasta de trabalho

\begin{Verbatim}[commandchars=!\{\}, numbers=left,label= Copiando a ensamblagem para sua anotação,frame=topline,fontsize=\scriptsize]
(base)user@server:$ cp  ~/PRATICA_ENSAMBLAGEM_DE_GENOMAS/flye/assembly.fasta ~/GENOME_ANNOTATION
\end{Verbatim}

Nos dirigimos na nossa pasta de trabalho

\begin{Verbatim}[commandchars=!\{\}, numbers=left,label= Indo na pasta de trabalho,frame=topline,fontsize=\scriptsize]
(base)user@server:$ cd ~/GENOME_ANNOTATION
\end{Verbatim}

Agora ativamos o ambiente virtual que cotém o prokka e rodamos ele no arquivo assembly.fasta

\begin{Verbatim}[commandchars=!\{\}, numbers=left,label= Rodando prokka ,frame=topline,fontsize=\scriptsize]
(base)user@server:$ conda activate prokka
(prokka)user@server:$ prokka --outdir prokka --prefix assembly_pacbio assembly.fasta --force
(prokka)user@server:$ conda deactivate
\end{Verbatim}

O prokka gera varios arquivos de saída na pasta

\emph{$\sim$/GENOME\_ANNOTATION/prokka}

Da uma olhada neles.

\subsection{Usando IGV para olhar os resultados de prokka}.
Agora podemos abrir o IGV para vizualizar as anotações no genoma
\begin{Verbatim}[commandchars=!\{\}, numbers=left,label= Abrindo IGV,frame=topline,fontsize=\scriptsize]
(base)user@server:$ igv.sh
\end{Verbatim}


Carregamos no IGV\footnote{\url{https://igv.org/}} o arquivo 



\emph{$\sim$/GENOME\_ANNOTATION/assembly.fasta}

e o arquivo

\emph{$\sim$/GENOME\_ANNOTATION/prokka/assembly\_pacbio.gff}

\begin{figure}[ht]
\centering
   \includegraphics[width=12cm]{Figs/igv_gff.png}
  \caption[Visualização do arquivo de anotações .gff na ensamblagem]{\label{IGV}Tela IGV}
\end{figure}


\subsection{Rondando blast local}

Agora vamos rodar um blast\footnote{\url{https://blast.ncbi.nlm.nih.gov/Blast.cgi
}} customizado para olhar coincidencias nas proteinas geradas pelos CDS detetados pelo prokka na nossa ensamblagem e numa base de dados que vamos gerar com sequencias conhecidas do género \emph{Komagataeibacter} ao qual pertence a nossa bactéria.

o arquivo com as sequencias para construir a nossa base de dados encontra-se em 

\emph{$\sim$/GENOME\_ANNOTATION/sequences\_for\_db.fa}

\begin{Verbatim}[commandchars=!\{\}, numbers=left,label= Criando base de dados customizada para o blast ,frame=topline,fontsize=\scriptsize]
(base)user@server:$ mkdir -p ~/GENOME_ANNOTATION/blastDB/
(base)user@server:$ export BLASTDB=~/GENOME_ANNOTATION/blastDB/
(base)user@server:$ cd ~/GENOME_ANNOTATION/blastDB/
(base)user@server:$ makeblastdb -in ./../sequences_for_db.fa -dbtype 'prot' -out myDB
(base)user@server:$ cd ..
\end{Verbatim}

\begin{itemize}
\item Na linha 1 criamos uma pasta chamada blastDB na nossa pasta de trabalho
\item Na linha 2 fazemos saber ao BLAST que as bases de dados estão na pasta que acabamos de criar 
\item Na linha 3 nos dirigimos na pasta que acabamos de criar
\item Na linha 4 criamos uma base de dados customizada com o nome myDB
\item Na linha 5 voltamos na nossa pasta de trabalho
\end{itemize}

Agora estamos prontos para rodar o BLAST localmente na nossa base de dados!

\begin{Verbatim}[commandchars=!\{\}, numbers=left,label= Rondando blast local,frame=topline,fontsize=\scriptsize]
(base)user@server:$ mkdir -p blast
(base)user@server:$ blastp -db myDB -query ./prokka/assembly_pacbio.faa -out ./blast/out_blast.txt
\end{Verbatim}
\begin{itemize}
\item Na lina 1 criamos uma pasta chamada blast na nossa pasta de trabalho que vai conter a saída do BLAST
\item Na linha 2 rodamos o BLAST das sequencias das proteínas geradas pelos CDS encontrados pelo prokka na nossa ensamblagem, na nossa base de dados com as proteínas do gênero  \emph{Komagataeibacter}
\end{itemize}

O BLAST vai gerar um arquivo de saida na em:


\emph{$\sim$/GENOME\_ANNOTATION/blast/out\_blast.txt}

o arquivo e o suficientemente grande para matar sua maquina se abre ele tudo de só uma vez
da uma olhada nele usando o comando less


\begin{Verbatim}[commandchars=!\{\}, numbers=left,label= Resultados do BLAST local,frame=topline,fontsize=\scriptsize]
(base)user@server:$ less ./blast/out_blast.txt
\end{Verbatim}

\chapter{Transcriptômica}

\begin{figure}[h]
\centering
   \includegraphics[height=12cm]{Figs/bioinformatics_excel.jpg}
  \caption[O que é bioinformática?]{\label{bioinf_excel}}
\end{figure}

RNA-Seq é uma técnica de sequenciamento que usa tecnologia de segunda geração para revelar a presença e quantidade de RNA em uma amostra biológica em um dado momento.

Usaremos os dados do PMID: 24926665\footnote{https://pubmed.ncbi.nlm.nih.gov/24926665/}, especificamente os conjuntos de dados mostrados na Tabela 1. Para gerar que o RNA total de dados foi extraído das células humanas primárias do músculo liso das vias aéreas (ASM) de quatro doadores diferentes. As células de cada doador foram tratadas com 1~{\textmu}M DEX ou com controle veículo por 18 h. Aproximadamente 1~{\textmu}g de RNA de cada amostra foi usado para gerar RNA-Seq cDNA bibliotecas para sequenciamento usando o TruSeq RNA Sample Prep Kit v2. O sequenciamento foi realizado em um instrumento HiSeq2000 gerando leituras pareadas (2x75bp). O principal objetivo do estudo foi obter ideias de como os glicocorticoides suprimem a inflamação no ASM.

\section{Amostras}
\begin{table}[h]
\begin{center}
\begin{tabular}{|c|c|c|}
\hline 
Accession&Sample&Cell\\
\hline
SRR1039508& Untreated& N61311 \\ \hline 
SRR1039509& Dexamethasone& N61311\\ \hline
SRR1039512& Untreated& N052611 \\ \hline 
SRR1039512& Dexamethasone& N052611\\ \hline
SRR1039516& Untreated& N080611 \\ \hline 
SRR1039517& Dexamethasone& N080611\\ \hline
SRR1039520& Untreated& N061011 \\ \hline 
SRR1039521& Dexamethasone& N061011\\ \hline
\end{tabular}
\caption{Desenho experimental} \label{table:1} \end{center}
\end{table}

\subsection{Processamento de dados}.

Agora, você deve primeiro verificar a qualidade das suas leituras com o FastQC e decidir se você precisa fazer algum filtrado de qualidade . O comando a seguir executa fastqc para cada um de seus arquivos (um por um, um após o outro). Pode levar algum tempo, vai la e busca um café.

\begin{Verbatim}[commandchars=!\{\}, numbers=left,label= Verificação de qualidade nas leituras ,frame=topline,fontsize=\scriptsize]
(base)user@server:$ mkdir -p QC
(base)user@server:$ conda activate fastqc_env
(fastq_evn)user@server:$ for file in $(ls -1 data/*.fastq.gz); do fastqc --noextract \
--threads 1 --nogroup -o QC $file; done
(base)user@server:$ conda deactivate
\end{Verbatim}

da uma olhada nos arquivos gerados pelos FASTQC na pasta QC

Agora utilizamos bbduk para filtrar nossas sequencias

\begin{Verbatim}[commandchars=!\{\}, numbers=left,label= Filtrado de sequencias ,frame=topline,fontsize=\scriptsize]
(base)user@server:$ mkdir -p bbduk
(base)user@server:$ conda activate bbduk_env
(bbduk_env)user@server:$ for sample in $(cut -f1 -d"," metadata.csv | grep SRR); do\
file_R1=${sample}_1.fastq ;\
file_R2=${sample}_2.fastq;\
bbduk.sh in1=./data/${file_R1}\
in2=./data/${file_R2}\
out1=./bbduk/${file_R1}\ out2=./bbduk/${file_R2}\
minlength=50\
qtrim=w\
trimq=20;\
done
(bbduk_env):$ conda deactivate
\end{Verbatim}

Qual porcentagem de leituras foi removido no total?

Agora pode rodar o FASTQC de novo e olhar verificar as estatísticas de qualidade dos arquivos gerados pelo bbduk

Vamos quantificar a expressão genética contra a versão genoma humano GRCh38.p13\footnote{https://www.gencodegenes.org/human/}. Usaremos Salmon\footnote{https://combine-lab.github.io/salmon/} para quantificação da expressão genica.

No primeiro para rodar o Salmon lugar temos que criar um index com nosso genoma de referencia

\begin{Verbatim}[commandchars=!\{\}, numbers=left,label= Criando Salmon index ,frame=topline,fontsize=\scriptsize]
(base)user@server:$ cd ~/PRATICA_RNASEQ
(base)user@server:$ salmon index -t ./references/gencode.v38.transcripts.fa.gz -i salmon_index --threads 2
\end{Verbatim}

O comando anterior debe criar uma pasta chamda salmon\_index em:

\subsection{Rodando o Salmon}.

\emph{$\sim$/PRATICA\_RNASEQ/salmon\_index}

Agora podemos quantificar nossas leituras no index gerado com o genoma de referencia, o seguinte comando executa o Salmon para todos arquivos de cada amostra (R1 e R2) para cada amostra sequencialmente.

\begin{Verbatim}[commandchars=!\{\}, numbers=left,label= Criando Salmon index ,frame=topline,fontsize=\scriptsize]
(base)user@server: cd  ~/PRATICA_RNASEQ
(base)user@server:$ mkdir -p Quantification
(base)user@server:$ cd Quantification
(base)user@server:$ for sample in $(cut -f1 -d"," ./../metadata.csv | grep SRR); do \
file_R1=${sample}_1.fastq; \
file_R2=${sample}_2.fastq; \
salmon quant --libType A --threads 4 --index \
./../salmon_index/ \
--validateMappings --seqBias --posBias --softclip \
-1 ./../bbduk/${file_R1} \
-2 ./../bbduk/${file_R2} \
-o ${sample};\
done
\end{Verbatim}

O Salmon vai tomar um tempo.

Os aquivos com a quantificação encontram-se na pasta:

\emph{$\sim$/PRATICA\_RNASEQ/Quantification}

Lá encontramos uma pasta para cada amostra, e dentro de cada pasta vários arquivos, o aquivo com a informação da quantificação é o arquivo quant.sf.

Agora que temos feito a quantificação podemos começar a nos fazer preguntas sobre os dados.

Vamos usar R\footnote{https://www.r-project.org/}para olhar como mudam os níveis de expressão dos genes para cada condição experimental, R é um software livre para computação estatística e gráficos. Para fazer mais amigável nossa interação com o R vamos usar RStudio\footnote{https://www.rstudio.com/} que é um ambiente de desenvolvimento integrado (IDE pelas siglas em ingles) para R.

\begin{Verbatim}[commandchars=!\{\}, numbers=left,label= Criando Salmon index ,frame=topline,fontsize=\scriptsize]
user@server:$ rstudio
\end{Verbatim}


Agora usando o Rstudio continua com a pratica no outro PDF que contem o código em R:

\begin{figure}[h]
\centering
   \includegraphics[height=8cm]{Figs/rstudio_tela.png}
  \caption[Rstudio tela]{\label{logo_rstudio}}
\end{figure}



\chapter{Genes ortólogos}

esta prática é uma versão em português de \url{https://lab.dessimoz.org/tutorials/omapractical/}

Você está interessado em estudar famílias de genes. vamos usar o Navegador de OMA \url{http://omabrowser.org}

\section{Parte 1a: Navegador OMA}

Você fez uma análise de rede e descobriu que o gene humano com UniProt ID OR2L5\_HUMAN está envolvido em um caminho interessante. Procure por este gene na página inicial do OMA.
\begin{itemize}
\item \textcolor{red}{Com base nas anotações do Gene Ontology, em qual função essa proteína provavelmente está envolvida?}
\item \textcolor{red}{Vá para a tabela de ortólogos. Quantos ortólogos 1:1 pares existem?}
\item \textcolor{red}{Quão conservada é a arquitetura de domínio desses ortólogos?}
\end{itemize}
Agora dê uma olhada nos Grupos Ortólogos Hierárquicos associados a este gene.  
\begin{itemize}
\item \textcolor{red}{Qual é o nível de raiz desse HOG, ou seja, em qual nível taxonômico ancestral esse gene se originou?}
\item \textcolor{red}{Quantos genes existem nesta família de genes?}
\item \textcolor{red}{Quais genomas existentes têm mais cópias desse gene?}
\item \textcolor{red}{Há algo incomum no conteúdo de GC de qualquer espécie?}
\item \textcolor{red}{Quantos genes nesta família (ou seja, raiz HOG) são genes humanos?}
\item \textcolor{red}{Em quais linhagens as duplicações provavelmente ocorreram?}
\item \textcolor{red}{Quantos genes nesta família têm um pequeno comprimento de gene?}
\end{itemize}
\begin{itemize}
\item \textcolor{red}{Esse gene compartilha alguma sintenia conservada localizada entre outras espécies? Se sim, quais são?}
\end{itemize}
\section{Parte 1b: Navegador OMA}
Você leu recentemente sobre enzimas ativas de carboidratos (CAZymes) que estão potencialmente envolvidas na degradação de polissacarídeos em \textit{A. bisporus} quando cultivadas em composto. Você quer saber se este gene é conservado em \textit{Penicillium},(sequência disponível no e disciplinas).

\begin{itemize}
\item \textcolor{red}{Procure esta proteína no site da OMA.}
\item \textcolor{red}{Considere agora os ortólogos previstos pelo OMA. Em que filogenia estão presentes?}
\item \textcolor{red}{Este gene é encontrado em alguma espécie de Penicillium?}
\item \textcolor{red}{Quantos in-parálogos deste gene existem em Agaricus bisporus ao nível de Agaricomycetes?}
\end{itemize}



\bibliography{bibliografia}
\bibliographystyle{genetics}

\begin{appendices}

\label{unixguide}
\includepdf[pages=-,fitpaper=true]{pdfs/guideUNIX.pdf}

\label{embossguide}
\includepdf[pages={2,1},fitpaper=true]{pdfs/guideEMBOSS.pdf}

\label{blastguide}
\includepdf[pages={2,1},fitpaper=true]{pdfs/guideBLAST.pdf}
\end{appendices} 

\end{document}

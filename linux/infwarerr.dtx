% \iffalse meta-comment
% 
% File: infwarerr.dtx
% Version: 2010/04/08 v1.3
% 
% Copyright (C) 2007, 2010 by
%    Heiko Oberdiek <heiko.oberdiek at googlemail.com>
% 
% This work may be distributed and/or modified under the
% conditions of the LaTeX Project Public License, either
% version 1.3c of this license or (at your option) any later
% version. This version of this license is in
%    http://www.latex-project.org/lppl/lppl-1-3c.txt
% and the latest version of this license is in
%    http://www.latex-project.org/lppl.txt
% and version 1.3 or later is part of all distributions of
% LaTeX version 2005/12/01 or later.
% 
% This work has the LPPL maintenance status "maintained".
% 
% This Current Maintainer of this work is Heiko Oberdiek.
% 
% The Base Interpreter refers to any `TeX-Format',
% because some files are installed in TDS:tex/generic//.
% 
% This work consists of the main source file infwarerr.dtx
% and the derived files
%    infwarerr.sty, infwarerr.pdf, infwarerr.ins, infwarerr.drv,
%    infwarerr-test1.tex, infwarerr-test2.tex,
%    infwarerr-test3.tex.
% 
% Distribution:
%    CTAN:macros/latex/contrib/oberdiek/infwarerr.dtx
%    CTAN:macros/latex/contrib/oberdiek/infwarerr.pdf
% 
% Unpacking:
%    (a) If infwarerr.ins is present:
%           tex infwarerr.ins
%    (b) Without infwarerr.ins:
%           tex infwarerr.dtx
%    (c) If you insist on using LaTeX
%           latex \let\install=y% \iffalse meta-comment
% 
% File: infwarerr.dtx
% Version: 2010/04/08 v1.3
% 
% Copyright (C) 2007, 2010 by
%    Heiko Oberdiek <heiko.oberdiek at googlemail.com>
% 
% This work may be distributed and/or modified under the
% conditions of the LaTeX Project Public License, either
% version 1.3c of this license or (at your option) any later
% version. This version of this license is in
%    http://www.latex-project.org/lppl/lppl-1-3c.txt
% and the latest version of this license is in
%    http://www.latex-project.org/lppl.txt
% and version 1.3 or later is part of all distributions of
% LaTeX version 2005/12/01 or later.
% 
% This work has the LPPL maintenance status "maintained".
% 
% This Current Maintainer of this work is Heiko Oberdiek.
% 
% The Base Interpreter refers to any `TeX-Format',
% because some files are installed in TDS:tex/generic//.
% 
% This work consists of the main source file infwarerr.dtx
% and the derived files
%    infwarerr.sty, infwarerr.pdf, infwarerr.ins, infwarerr.drv,
%    infwarerr-test1.tex, infwarerr-test2.tex,
%    infwarerr-test3.tex.
% 
% Distribution:
%    CTAN:macros/latex/contrib/oberdiek/infwarerr.dtx
%    CTAN:macros/latex/contrib/oberdiek/infwarerr.pdf
% 
% Unpacking:
%    (a) If infwarerr.ins is present:
%           tex infwarerr.ins
%    (b) Without infwarerr.ins:
%           tex infwarerr.dtx
%    (c) If you insist on using LaTeX
%           latex \let\install=y% \iffalse meta-comment
% 
% File: infwarerr.dtx
% Version: 2010/04/08 v1.3
% 
% Copyright (C) 2007, 2010 by
%    Heiko Oberdiek <heiko.oberdiek at googlemail.com>
% 
% This work may be distributed and/or modified under the
% conditions of the LaTeX Project Public License, either
% version 1.3c of this license or (at your option) any later
% version. This version of this license is in
%    http://www.latex-project.org/lppl/lppl-1-3c.txt
% and the latest version of this license is in
%    http://www.latex-project.org/lppl.txt
% and version 1.3 or later is part of all distributions of
% LaTeX version 2005/12/01 or later.
% 
% This work has the LPPL maintenance status "maintained".
% 
% This Current Maintainer of this work is Heiko Oberdiek.
% 
% The Base Interpreter refers to any `TeX-Format',
% because some files are installed in TDS:tex/generic//.
% 
% This work consists of the main source file infwarerr.dtx
% and the derived files
%    infwarerr.sty, infwarerr.pdf, infwarerr.ins, infwarerr.drv,
%    infwarerr-test1.tex, infwarerr-test2.tex,
%    infwarerr-test3.tex.
% 
% Distribution:
%    CTAN:macros/latex/contrib/oberdiek/infwarerr.dtx
%    CTAN:macros/latex/contrib/oberdiek/infwarerr.pdf
% 
% Unpacking:
%    (a) If infwarerr.ins is present:
%           tex infwarerr.ins
%    (b) Without infwarerr.ins:
%           tex infwarerr.dtx
%    (c) If you insist on using LaTeX
%           latex \let\install=y% \iffalse meta-comment
% 
% File: infwarerr.dtx
% Version: 2010/04/08 v1.3
% 
% Copyright (C) 2007, 2010 by
%    Heiko Oberdiek <heiko.oberdiek at googlemail.com>
% 
% This work may be distributed and/or modified under the
% conditions of the LaTeX Project Public License, either
% version 1.3c of this license or (at your option) any later
% version. This version of this license is in
%    http://www.latex-project.org/lppl/lppl-1-3c.txt
% and the latest version of this license is in
%    http://www.latex-project.org/lppl.txt
% and version 1.3 or later is part of all distributions of
% LaTeX version 2005/12/01 or later.
% 
% This work has the LPPL maintenance status "maintained".
% 
% This Current Maintainer of this work is Heiko Oberdiek.
% 
% The Base Interpreter refers to any `TeX-Format',
% because some files are installed in TDS:tex/generic//.
% 
% This work consists of the main source file infwarerr.dtx
% and the derived files
%    infwarerr.sty, infwarerr.pdf, infwarerr.ins, infwarerr.drv,
%    infwarerr-test1.tex, infwarerr-test2.tex,
%    infwarerr-test3.tex.
% 
% Distribution:
%    CTAN:macros/latex/contrib/oberdiek/infwarerr.dtx
%    CTAN:macros/latex/contrib/oberdiek/infwarerr.pdf
% 
% Unpacking:
%    (a) If infwarerr.ins is present:
%           tex infwarerr.ins
%    (b) Without infwarerr.ins:
%           tex infwarerr.dtx
%    (c) If you insist on using LaTeX
%           latex \let\install=y\input{infwarerr.dtx}
%        (quote the arguments according to the demands of your shell)
% 
% Documentation:
%    (a) If infwarerr.drv is present:
%           latex infwarerr.drv
%    (b) Without infwarerr.drv:
%           latex infwarerr.dtx; ...
%    The class ltxdoc loads the configuration file ltxdoc.cfg
%    if available. Here you can specify further options, e.g.
%    use A4 as paper format:
%       \PassOptionsToClass{a4paper}{article}
% 
%    Programm calls to get the documentation (example):
%       pdflatex infwarerr.dtx
%       makeindex -s gind.ist infwarerr.idx
%       pdflatex infwarerr.dtx
%       makeindex -s gind.ist infwarerr.idx
%       pdflatex infwarerr.dtx
% 
% Installation:
%    TDS:tex/generic/oberdiek/infwarerr.sty
%    TDS:doc/latex/oberdiek/infwarerr.pdf
%    TDS:doc/latex/oberdiek/test/infwarerr-test1.tex
%    TDS:doc/latex/oberdiek/test/infwarerr-test2.tex
%    TDS:doc/latex/oberdiek/test/infwarerr-test3.tex
%    TDS:source/latex/oberdiek/infwarerr.dtx
%
%<*ignore>
\begingroup
  \catcode123=1 %
  \catcode125=2 %
  \def\x{LaTeX2e}%
\expandafter\endgroup
\ifcase 0\ifx\install y1\fi\expandafter
         \ifx\csname processbatchFile\endcsname\relax\else1\fi
         \ifx\fmtname\x\else 1\fi\relax
\else\csname fi\endcsname
%</ignore>
%<*install>
\input docstrip.tex
\Msg{************************************************************************}
\Msg{* Installation}
\Msg{* Package: infwarerr 2010/04/08 v1.3 Providing info/warning/message (HO)}
\Msg{************************************************************************}

\keepsilent
\askforoverwritefalse

\let\MetaPrefix\relax
\preamble

This is a generated file.

Project: infwarerr
Version: 2010/04/08 v1.3

Copyright (C) 2007, 2010 by
   Heiko Oberdiek <heiko.oberdiek at googlemail.com>

This work may be distributed and/or modified under the
conditions of the LaTeX Project Public License, either
version 1.3c of this license or (at your option) any later
version. This version of this license is in
   http://www.latex-project.org/lppl/lppl-1-3c.txt
and the latest version of this license is in
   http://www.latex-project.org/lppl.txt
and version 1.3 or later is part of all distributions of
LaTeX version 2005/12/01 or later.

This work has the LPPL maintenance status "maintained".

This Current Maintainer of this work is Heiko Oberdiek.

The Base Interpreter refers to any `TeX-Format',
because some files are installed in TDS:tex/generic//.

This work consists of the main source file infwarerr.dtx
and the derived files
   infwarerr.sty, infwarerr.pdf, infwarerr.ins, infwarerr.drv,
   infwarerr-test1.tex, infwarerr-test2.tex,
   infwarerr-test3.tex.

\endpreamble
\let\MetaPrefix\DoubleperCent

\generate{%
  \file{infwarerr.ins}{\from{infwarerr.dtx}{install}}%
  \file{infwarerr.drv}{\from{infwarerr.dtx}{driver}}%
  \usedir{tex/generic/oberdiek}%
  \file{infwarerr.sty}{\from{infwarerr.dtx}{package}}%
  \usedir{doc/latex/oberdiek/test}%
  \file{infwarerr-test1.tex}{\from{infwarerr.dtx}{test1}}%
  \file{infwarerr-test2.tex}{\from{infwarerr.dtx}{test2}}%
  \file{infwarerr-test3.tex}{\from{infwarerr.dtx}{test3}}%
}

\catcode32=13\relax% active space
\let =\space%
\Msg{************************************************************************}
\Msg{*}
\Msg{* To finish the installation you have to move the following}
\Msg{* file into a directory searched by TeX:}
\Msg{*}
\Msg{*     infwarerr.sty}
\Msg{*}
\Msg{* To produce the documentation run the file `infwarerr.drv'}
\Msg{* through LaTeX.}
\Msg{*}
\Msg{* Happy TeXing!}
\Msg{*}
\Msg{************************************************************************}

\endbatchfile
%</install>
%<*ignore>
\fi
%</ignore>
%<*driver>
\NeedsTeXFormat{LaTeX2e}
\ProvidesFile{infwarerr.drv}%
  [2010/04/08 v1.3 Providing info/warning/message (HO)]%
\documentclass{ltxdoc}
\usepackage{holtxdoc}[2011/02/04]
\begin{document}
  \DocInput{infwarerr.dtx}%
\end{document}
%</driver>
% \fi
%
% \CheckSum{871}
%
% \CharacterTable
%  {Upper-case    \A\B\C\D\E\F\G\H\I\J\K\L\M\N\O\P\Q\R\S\T\U\V\W\X\Y\Z
%   Lower-case    \a\b\c\d\e\f\g\h\i\j\k\l\m\n\o\p\q\r\s\t\u\v\w\x\y\z
%   Digits        \0\1\2\3\4\5\6\7\8\9
%   Exclamation   \!     Double quote  \"     Hash (number) \#
%   Dollar        \$     Percent       \%     Ampersand     \&
%   Acute accent  \'     Left paren    \(     Right paren   \)
%   Asterisk      \*     Plus          \+     Comma         \,
%   Minus         \-     Point         \.     Solidus       \/
%   Colon         \:     Semicolon     \;     Less than     \<
%   Equals        \=     Greater than  \>     Question mark \?
%   Commercial at \@     Left bracket  \[     Backslash     \\
%   Right bracket \]     Circumflex    \^     Underscore    \_
%   Grave accent  \`     Left brace    \{     Vertical bar  \|
%   Right brace   \}     Tilde         \~}
%
% \GetFileInfo{infwarerr.drv}
%
% \title{The \xpackage{infwarerr} package}
% \date{2010/04/08 v1.3}
% \author{Heiko Oberdiek\\\xemail{heiko.oberdiek at googlemail.com}}
%
% \maketitle
%
% \begin{abstract}
% This package provides a complete set of macros for informations,
% warnings and error messages with support for \plainTeX.
% \end{abstract}
%
% \tableofcontents
%
% \section{Documentation}
%
% Package \xpackage{infwarerr} defines the following set of commands,
% given as pseudo regular expression:
%   \begin{quote}
%     \verb+\@(Package|Class)(Info|InfoNoLine|Warning|WarningNoLine|Error)+^^A
%     \kern-\rightmargin
%   \end{quote}
% The first argument is the package or class in question. The second
% argument contains the message text. \cs{MessageBreak} is supported.
% In case of error messages, the third argument contains the help
% text. Supported are \LaTeX's message texts \cs{@ehc}:
%   \begin{quote}
%     |Try typing  <return>  to proceed|\\
%     |If that doesn't work, type  X <return>  to quit.|
%   \end{quote}
% and \cs{@ehd}:
%   \begin{quote}
%     |You're in trouble here.  Try typing  <return>  to proceed|\\
%     |If that doesn't work, type  X <return>  to quit.|
%   \end{quote}
%
% The package is intended for macro writers that want to write
% code that works for both \LaTeX\ and \plainTeX. If \LaTeX\ is
% detected, then the macros of this package will only be wrappers
% for \LaTeX's macros. Otherwise \LaTeX's macros will be mimicked
% for use with \plainTeX.
%
% \StopEventually{
% }
%
% \section{Implementation}
%
%    \begin{macrocode}
%<*package>
%    \end{macrocode}
%
% \subsection{Reload check and package identification}
%    Reload check, especially if the package is not used with \LaTeX.
%    \begin{macrocode}
\begingroup\catcode61\catcode48\catcode32=10\relax%
  \catcode13=5 % ^^M
  \endlinechar=13 %
  \catcode35=6 % #
  \catcode39=12 % '
  \catcode44=12 % ,
  \catcode45=12 % -
  \catcode46=12 % .
  \catcode58=12 % :
  \catcode64=11 % @
  \catcode123=1 % {
  \catcode125=2 % }
  \expandafter\let\expandafter\x\csname ver@infwarerr.sty\endcsname
  \ifx\x\relax % plain-TeX, first loading
  \else
    \def\empty{}%
    \ifx\x\empty % LaTeX, first loading,
      % variable is initialized, but \ProvidesPackage not yet seen
    \else
      \expandafter\ifx\csname PackageInfo\endcsname\relax
        \def\x#1#2{%
          \immediate\write-1{Package #1 Info: #2.}%
        }%
      \else
        \def\x#1#2{\PackageInfo{#1}{#2, stopped}}%
      \fi
      \x{infwarerr}{The package is already loaded}%
      \aftergroup\endinput
    \fi
  \fi
\endgroup%
%    \end{macrocode}
%    Package identification:
%    \begin{macrocode}
\begingroup\catcode61\catcode48\catcode32=10\relax%
  \catcode13=5 % ^^M
  \endlinechar=13 %
  \catcode35=6 % #
  \catcode39=12 % '
  \catcode40=12 % (
  \catcode41=12 % )
  \catcode44=12 % ,
  \catcode45=12 % -
  \catcode46=12 % .
  \catcode47=12 % /
  \catcode58=12 % :
  \catcode64=11 % @
  \catcode91=12 % [
  \catcode93=12 % ]
  \catcode123=1 % {
  \catcode125=2 % }
  \expandafter\ifx\csname ProvidesPackage\endcsname\relax
    \def\x#1#2#3[#4]{\endgroup
      \immediate\write-1{Package: #3 #4}%
      \xdef#1{#4}%
    }%
  \else
    \def\x#1#2[#3]{\endgroup
      #2[{#3}]%
      \ifx#1\@undefined
        \xdef#1{#3}%
      \fi
      \ifx#1\relax
        \xdef#1{#3}%
      \fi
    }%
  \fi
\expandafter\x\csname ver@infwarerr.sty\endcsname
\ProvidesPackage{infwarerr}%
  [2010/04/08 v1.3 Providing info/warning/message (HO)]%
%    \end{macrocode}
%
% \subsection{Catcodes}
%
%    \begin{macrocode}
\begingroup\catcode61\catcode48\catcode32=10\relax%
  \catcode13=5 % ^^M
  \endlinechar=13 %
  \catcode123=1 % {
  \catcode125=2 % }
  \catcode64=11 % @
  \def\x{\endgroup
    \expandafter\edef\csname InfWarErr@AtEnd\endcsname{%
      \endlinechar=\the\endlinechar\relax
      \catcode13=\the\catcode13\relax
      \catcode32=\the\catcode32\relax
      \catcode35=\the\catcode35\relax
      \catcode61=\the\catcode61\relax
      \catcode64=\the\catcode64\relax
      \catcode123=\the\catcode123\relax
      \catcode125=\the\catcode125\relax
    }%
  }%
\x\catcode61\catcode48\catcode32=10\relax%
\catcode13=5 % ^^M
\endlinechar=13 %
\catcode35=6 % #
\catcode64=11 % @
\catcode123=1 % {
\catcode125=2 % }
\def\TMP@EnsureCode#1#2{%
  \edef\InfWarErr@AtEnd{%
    \InfWarErr@AtEnd
    \catcode#1=\the\catcode#1\relax
  }%
  \catcode#1=#2\relax
}
\TMP@EnsureCode{10}{12}% ^^J
\TMP@EnsureCode{39}{12}% '
\TMP@EnsureCode{40}{12}% (
\TMP@EnsureCode{41}{12}% )
\TMP@EnsureCode{44}{12}% ,
\TMP@EnsureCode{45}{12}% -
\TMP@EnsureCode{46}{12}% .
\TMP@EnsureCode{58}{12}% :
\TMP@EnsureCode{60}{12}% <
\TMP@EnsureCode{62}{12}% >
\TMP@EnsureCode{94}{7}% ^
\edef\InfWarErr@AtEnd{\InfWarErr@AtEnd\noexpand\endinput}
%    \end{macrocode}
%
% \subsection{\LaTeX\ detection}
%
%    \begin{macrocode}
\begingroup
  \let\x=Y%
  \def\check#1{%
    \@check{Generic}{#1}%
    \@check{Package}{#1}%
    \@check{Class}{#1}%
  }%
  \def\@check#1#2{%
    \expandafter\ifx\csname #1#2\endcsname\relax
      \let\x=N%
    \fi
  }%
  \check{Info}%
  \check{Warning}%
  \check{Error}%
  \@check{Package}{WarningNoLine}%
  \@check{Class}{WarningNoLine}%
  \@check{}{@ehc}%
  \@check{}{@ehd}%
\expandafter\endgroup
\ifx\x Y%
%    \end{macrocode}
%
% \subsection{Version for \LaTeX}
%
%    \begin{macro}{\@PackageInfo}
%    \begin{macrocode}
  \let\@PackageInfo\PackageInfo
%    \end{macrocode}
%    \end{macro}
%    \begin{macro}{\@PackageInfoNoLine}
%    \begin{macrocode}
  \def\@PackageInfoNoLine#1#2{%
    \PackageInfo{#1}{#2\@gobble}%
  }%
%    \end{macrocode}
%    \end{macro}
%    \begin{macro}{\@ClassInfo}
%    \begin{macrocode}
  \let\@ClassInfo\ClassInfo
%    \end{macrocode}
%    \end{macro}
%    \begin{macro}{\@ClassInfoNoLine}
%    \begin{macrocode}
  \def\@ClassInfoNoLine#1#2{%
    \ClassInfo{#1}{#2\@gobble}%
  }%
%    \end{macrocode}
%    \end{macro}
%    \begin{macro}{\@PackageWarning}
%    \begin{macrocode}
  \let\@PackageWarning\PackageWarning
%    \end{macrocode}
%    \end{macro}
%    \begin{macro}{\@PackageWarningNoLine}
%    \begin{macrocode}
  \let\@PackageWarningNoLine\PackageWarningNoLine
%    \end{macrocode}
%    \end{macro}
%    \begin{macro}{\@ClassWarning}
%    \begin{macrocode}
  \let\@ClassWarning\ClassWarning
%    \end{macrocode}
%    \end{macro}
%    \begin{macro}{\@ClassWarningNoLine}
%    \begin{macrocode}
  \let\@ClassWarningNoLine\ClassWarningNoLine
%    \end{macrocode}
%    \end{macro}
%    \begin{macro}{\@PackageError}
%    \begin{macrocode}
  \let\@PackageError\PackageError
%    \end{macrocode}
%    \end{macro}
%    \begin{macro}{\@ClassError}
%    \begin{macrocode}
  \let\@ClassError\ClassError
%    \end{macrocode}
%    \end{macro}
%    \begin{macrocode}
  \expandafter\InfWarErr@AtEnd
\fi%
%    \end{macrocode}
%
% \subsection{Version for \plainTeX}
%
%    \begin{macro}{\InfWarErr@online}
%    \begin{macrocode}
\begingroup\expandafter\endgroup
\ifcase\expandafter\ifx\csname inputlineno\endcsname\relax
         0 %
       \else
         \ifnum\inputlineno<0 %
           0 %
         \else
           1 %
         \fi
       \fi
  \def\InfWarErr@online{}%
\else
  \def\InfWarErr@online{ on input line \the\inputlineno}%
\fi
%    \end{macrocode}
%    \end{macro}
%
%    \begin{macro}{\InfWarErr@protected}
%    \begin{macrocode}
\begingroup\expandafter\expandafter\expandafter\endgroup
\expandafter\ifx\csname protected\endcsname\relax
  \def\InfWarErr@protected{}%
\else
  \let\InfWarErr@protected\protected
\fi
%    \end{macrocode}
%    \end{macro}
%
%    \begin{macro}{\InfWarErr@unused}
%    \begin{macrocode}
\chardef\InfWarErr@unused=16 %
%    \end{macrocode}
%    \end{macro}
%
%    \begin{macro}{\InfWarErr@ehc}
%    \begin{macrocode}
\def\InfWarErr@ehc{%
  Try typing \space <return> \space to proceed.\MessageBreak
  If that doesn't work, type \space X <return> \space to quit.%
}
%    \end{macrocode}
%    \end{macro}
%    \begin{macro}{\InfWarErr@ehd}
%    \begin{macrocode}
\def\InfWarErr@ehd{%
  You're in trouble here. \space\@ehc
}
%    \end{macrocode}
%    \end{macro}
%
%    \begin{macro}{\InfWarErr@GenericInfo}
%    \begin{macrocode}
\InfWarErr@protected\def\InfWarErr@GenericInfo#1#2{%
  \begingroup
    \let\on@line\InfWarErr@online
    \long\def\@gobble##1{}%
    \def\space{ }%
    \def\@spaces{\space\space\space\space}%
    \def\MessageBreak{^^J#1}%
    \newlinechar=10 %
    \escapechar=92 %
    \immediate\write-1{#2\on@line.}%
  \endgroup
}
%    \end{macrocode}
%    \end{macro}
%    \begin{macro}{\InfWarErr@GenericWarning}
%    \begin{macrocode}
\InfWarErr@protected\def\InfWarErr@GenericWarning#1#2{%
  \begingroup
    \let\on@line\InfWarErr@online
    \long\def\@gobble##1{}%
    \def\space{ }%
    \def\@spaces{\space\space\space\space}%
    \def\MessageBreak{^^J#1}%
    \newlinechar=10 %
    \escapechar=92 %
    \immediate\write\InfWarErr@unused{^^J#2\on@line.^^J}%
  \endgroup
}
%    \end{macrocode}
%    \end{macro}
%    \begin{macro}{\InfWarErr@GenericError}
%    Old \TeX\ versions below 3.141 where |^^J| is not working
%    as newline in error message and help texts are not supported.
%    \begin{macrocode}
\InfWarErr@protected\def\InfWarErr@GenericError#1#2#3#4{%
  \begingroup
    \immediate\write\InfWarErr@unused{}%
    \def\space{ }%
    \def\@spaces{\space\space\space\space}%
    \let\@ehc\InfWarErr@ehc
    \let\@ehd\InfWarErr@ehd
    \let\on@line\InfWarErr@online
    \def\MessageBreak{^^J}%
    \newlinechar=10 %
    \escapechar=92 %
    \edef\InfWarErr@help{\noexpand\errhelp{#4}}%
    \InfWarErr@help
    \def\MessageBreak{^^J#1}%
    \errmessage{%
      #2.^^J^^J%
      #3^^J%
      Type \space H <return> \space for immediate help.^^J ...^^J%
    }%
  \endgroup
}
%    \end{macrocode}
%    \end{macro}
%
%    \begin{macro}{\@PackageInfo}
%    \begin{macrocode}
\def\@PackageInfo#1#2{%
  \InfWarErr@GenericInfo{%
    (#1) \@spaces\@spaces\@spaces
  }{%
    Package #1 Info: #2%
  }%
}
%    \end{macrocode}
%    \end{macro}
%    \begin{macro}{\@PackageInfoNoLine}
%    \begin{macrocode}
\def\@PackageInfoNoLine#1#2{%
  \@PackageInfo{#1}{#2\@gobble}%
}
%    \end{macrocode}
%    \end{macro}
%    \begin{macro}{\@ClassInfo}
%    \begin{macrocode}
\def\@ClassInfo#1#2{%
  \InfWarErr@GenericInfo{%
    (#1) \space\space\@spaces\@spaces
  }{%
    Class #1 Info: #2%
  }%
}
%    \end{macrocode}
%    \end{macro}
%    \begin{macro}{\@ClassInfoNoLine}
%    \begin{macrocode}
\def\@ClassInfoNoLine#1#2{%
  \@ClassInfo{#1}{#2\@gobble}%
}
%    \end{macrocode}
%    \end{macro}
%    \begin{macro}{\@PackageWarning}
%    \begin{macrocode}
\def\@PackageWarning#1#2{%
  \InfWarErr@GenericWarning{%
    (#1)\@spaces\@spaces\@spaces\@spaces
  }{%
    Package #1 Warning: #2%
  }%
}
%    \end{macrocode}
%    \end{macro}
%    \begin{macro}{\@PackageWarningNoLine}
%    \begin{macrocode}
\def\@PackageWarningNoLine#1#2{%
  \@PackageWarning{#1}{#2\@gobble}%
}
%    \end{macrocode}
%    \end{macro}
%    \begin{macro}{\@ClassWarning}
%    \begin{macrocode}
\def\@ClassWarning#1#2{%
  \InfWarErr@GenericWarning{%
    (#1) \space\@spaces\@spaces\@spaces
  }{%
    Class #1 Warning: #2%
  }%
}
%    \end{macrocode}
%    \end{macro}
%    \begin{macro}{\@ClassWarningNoLine}
%    \begin{macrocode}
\def\@ClassWarningNoLine#1#2{%
  \@ClassWarning{#1}{#2\@gobble}%
}
%    \end{macrocode}
%    \end{macro}
%    \begin{macro}{\@PackageError}
%    \begin{macrocode}
\def\@PackageError#1#2#3{%
  \InfWarErr@GenericError{%
    (#1)\@spaces\@spaces\@spaces\@spaces
  }{%
    Package #1 Error: #2%
  }{%
    See the #1 package documentation for explanation.%
  }{#3}%
}
%    \end{macrocode}
%    \end{macro}
%    \begin{macro}{\@ClassError}
%    \begin{macrocode}
\def\@ClassError#1#2#3{%
  \InfWarErr@GenericError{%
    (#1) \space\@spaces\@spaces\@spaces
  }{%
    Class #1 Error: #2%
  }{%
    See the #1 class documentation for explanation.%
  }{#3}%
}
%    \end{macrocode}
%    \end{macro}
%
%    \begin{macrocode}
\InfWarErr@AtEnd%
%</package>
%    \end{macrocode}
%
% \section{Test}
%
% \subsection{Catcode checks for loading}
%
%    \begin{macrocode}
%<*test1>
%    \end{macrocode}
%    \begin{macrocode}
\catcode`\{=1 %
\catcode`\}=2 %
\catcode`\#=6 %
\catcode`\@=11 %
\expandafter\ifx\csname count@\endcsname\relax
  \countdef\count@=255 %
\fi
\expandafter\ifx\csname @gobble\endcsname\relax
  \long\def\@gobble#1{}%
\fi
\expandafter\ifx\csname @firstofone\endcsname\relax
  \long\def\@firstofone#1{#1}%
\fi
\expandafter\ifx\csname loop\endcsname\relax
  \expandafter\@firstofone
\else
  \expandafter\@gobble
\fi
{%
  \def\loop#1\repeat{%
    \def\body{#1}%
    \iterate
  }%
  \def\iterate{%
    \body
      \let\next\iterate
    \else
      \let\next\relax
    \fi
    \next
  }%
  \let\repeat=\fi
}%
\def\RestoreCatcodes{}
\count@=0 %
\loop
  \edef\RestoreCatcodes{%
    \RestoreCatcodes
    \catcode\the\count@=\the\catcode\count@\relax
  }%
\ifnum\count@<255 %
  \advance\count@ 1 %
\repeat

\def\RangeCatcodeInvalid#1#2{%
  \count@=#1\relax
  \loop
    \catcode\count@=15 %
  \ifnum\count@<#2\relax
    \advance\count@ 1 %
  \repeat
}
\def\RangeCatcodeCheck#1#2#3{%
  \count@=#1\relax
  \loop
    \ifnum#3=\catcode\count@
    \else
      \errmessage{%
        Character \the\count@\space
        with wrong catcode \the\catcode\count@\space
        instead of \number#3%
      }%
    \fi
  \ifnum\count@<#2\relax
    \advance\count@ 1 %
  \repeat
}
\def\space{ }
\expandafter\ifx\csname LoadCommand\endcsname\relax
  \def\LoadCommand{\input infwarerr.sty\relax}%
\fi
\def\Test{%
  \RangeCatcodeInvalid{0}{47}%
  \RangeCatcodeInvalid{58}{64}%
  \RangeCatcodeInvalid{91}{96}%
  \RangeCatcodeInvalid{123}{255}%
  \catcode`\@=12 %
  \catcode`\\=0 %
  \catcode`\%=14 %
  \LoadCommand
  \RangeCatcodeCheck{0}{36}{15}%
  \RangeCatcodeCheck{37}{37}{14}%
  \RangeCatcodeCheck{38}{47}{15}%
  \RangeCatcodeCheck{48}{57}{12}%
  \RangeCatcodeCheck{58}{63}{15}%
  \RangeCatcodeCheck{64}{64}{12}%
  \RangeCatcodeCheck{65}{90}{11}%
  \RangeCatcodeCheck{91}{91}{15}%
  \RangeCatcodeCheck{92}{92}{0}%
  \RangeCatcodeCheck{93}{96}{15}%
  \RangeCatcodeCheck{97}{122}{11}%
  \RangeCatcodeCheck{123}{255}{15}%
  \RestoreCatcodes
}
\Test
\csname @@end\endcsname
\end
%    \end{macrocode}
%    \begin{macrocode}
%</test1>
%    \end{macrocode}
%
% \subsection{Macro tests}
%
%    \begin{macrocode}
%<*test2>
\let\PackageInfo\relax
\input infwarerr.sty\relax
\let\OrgWrite\write
\chardef\WriteUnused=16 %
\def\msg#{\immediate\OrgWrite\WriteUnused}
\msg{File: infwarerr-test2.tex 2010/04/08 v1.3 Test file for plain-TeX}
\catcode64=11 %
%</test2>
%    \end{macrocode}
%
%    \begin{macrocode}
%<*test3>
\NeedsTeXFormat{LaTeX2e}
\ProvidesFile{infwarerr-test3.tex}[2010/04/08 v1.3 Test file for LaTeX]
\RequirePackage{infwarerr}[2010/04/08]
\makeatletter
\let\OrgWrite\write
\chardef\WriteUnused=\@unused
\def\msg#{\immediate\OrgWrite\WriteUnused}
%</test3>
%    \end{macrocode}
%
%    \begin{macrocode}
%<*test2|test3>
\def\pkgname{PACKAGE}
\def\clsname{CLASS}
\def\msgtext{MESSAGE\MessageBreak SECOND LINE}
\def\hlptext{HELP\MessageBreak SECOND LINE}

\let\OrgWrite\write
\let\OrgErrMessage\errmessage

\def\write#1#{%
  \@write{#1}%
}
\def\@write#1#2{%
  \OrgWrite#1{#2}%
  \xdef\WriteNum{\number#1}%
  \xdef\WriteText{#2}%
}
\def\TestFailed#1#2{%
  \OrgErrMessage{Test (\string#1) failed: #2}%
}
\def\TestWrite#1#2#3{%
  \ifnum\WriteNum=#2\relax
  \else
    \TestFailed#1{write stream number: \WriteNum <> #2}%
  \fi
  \begingroup
    \let~\space
    \edef\TestString{#3}%
    \ifx\WriteText\TestString
    \else
      \msg{}%
      \msg{[\WriteText]}%
      \msg{<>}%
      \msg{[#3]}% hash-ok
      \TestFailed#1{write text}%
    \fi
  \endgroup
}
\newtoks\ErrHelpToks
\def\errmessage#1{%
  \global\ErrHelpToks=\expandafter{\the\errhelp}%
  \msg{**************** error message ****************}%
  \msg{! #1}%
  \msg{**************** help message *****************}%
  \msg{\the\errhelp}%
  \msg{***********************************************}%
  \xdef\ErrMessage{! #1}%
}
\def\TestInfo{%
  \@PackageInfo\pkgname\msgtext
  \TestWrite\@PackageInfo{-1}{%
    Package PACKAGE Info: MESSAGE^^J%
    (PACKAGE)~~~~~~~~~~~~~SECOND LINE on input line \the\inputlineno.%
  }%
  \@PackageInfoNoLine\pkgname\msgtext
  \TestWrite\@PackageInfoNoLine{-1}{%
    Package PACKAGE Info: MESSAGE^^J%
    (PACKAGE)~~~~~~~~~~~~~SECOND LINE.%
  }%
  \@ClassInfo\clsname\msgtext
  \TestWrite\@ClassInfo{-1}{%
    Class CLASS Info: MESSAGE^^J%
    (CLASS)~~~~~~~~~~~SECOND LINE on input line \the\inputlineno.%
  }%
  \@ClassInfoNoLine\clsname\msgtext
  \TestWrite\@ClassInfoNoLine{-1}{%
    Class CLASS Info: MESSAGE^^J%
    (CLASS)~~~~~~~~~~~SECOND LINE.%
  }%
}
\def\TestWarning{%
  \@PackageWarning\pkgname\msgtext
  \TestWrite\@PackageWarning\WriteUnused{%
    ^^J%
    Package PACKAGE Warning: MESSAGE^^J%
    (PACKAGE)~~~~~~~~~~~~~~~~SECOND LINE on input line \the\inputlineno.%
    ^^J%
  }%
  \@PackageWarningNoLine\pkgname\msgtext
  \TestWrite\@PackageWarningNoLine\WriteUnused{%
    ^^J%
    Package PACKAGE Warning: MESSAGE^^J%
    (PACKAGE)~~~~~~~~~~~~~~~~SECOND LINE.%
    ^^J%
  }%
  \@ClassWarning\clsname\msgtext
  \TestWrite\@ClassWarning\WriteUnused{%
    ^^J%
    Class CLASS Warning: MESSAGE^^J%
    (CLASS)~~~~~~~~~~~~~~SECOND LINE on input line \the\inputlineno.%
    ^^J%
  }%
  \@ClassWarningNoLine\clsname\msgtext
  \TestWrite\@ClassWarningNoLine\WriteUnused{%
    ^^J%
    Class CLASS Warning: MESSAGE^^J%
    (CLASS)~~~~~~~~~~~~~~SECOND LINE.%
    ^^J%
  }%
}
\def\TestError{%
%<test3>  \let\write\OrgWrite
%<test3>  \def\plaindots{}
%<test2>  \def\plaindots{.^^J ...^^J}
  \@PackageError\pkgname\msgtext\hlptext
  \@TestError\@PackageError{%
    ! Package PACKAGE Error: MESSAGE^^J%
    (PACKAGE)~~~~~~~~~~~~~~~~SECOND LINE.^^J%
    ^^J%
    See the PACKAGE package documentation for explanation.^^J%
    Type~~H <return>~~for immediate help\plaindots
  }{%
    HELP^^J%
    SECOND LINE%
  }%
  \@ClassError\clsname\msgtext\hlptext
  \@TestError\@ClassError{%
    ! Class CLASS Error: MESSAGE^^J%
    (CLASS)~~~~~~~~~~~~~~SECOND LINE.^^J%
    ^^J%
    See the CLASS class documentation for explanation.^^J%
    Type~~H <return>~~for immediate help\plaindots
  }{%
    HELP^^J%
    SECOND LINE%
  }%
  \@PackageError\pkgname\msgtext\@ehc
  \@TestError\@PackageError{%
    ! Package PACKAGE Error: MESSAGE^^J%
    (PACKAGE)~~~~~~~~~~~~~~~~SECOND LINE.^^J%
    ^^J%
    See the PACKAGE package documentation for explanation.^^J%
    Type~~H <return>~~for immediate help\plaindots
  }{%
    Try typing~~<return>~~to proceed.^^J%
    If that doesn't work, type~~X <return>~~to quit.%
  }%
  \@PackageError\pkgname\msgtext\@ehd
  \@TestError\@PackageError{%
    ! Package PACKAGE Error: MESSAGE^^J%
    (PACKAGE)~~~~~~~~~~~~~~~~SECOND LINE.^^J%
    ^^J%
    See the PACKAGE package documentation for explanation.^^J%
    Type~~H <return>~~for immediate help\plaindots
  }{%
    You're in trouble here.~~Try typing~~<return>~~to proceed.^^J%
    If that doesn't work, type~~X <return>~~to quit.%
  }%
}
\def\@TestError#1#2#3{%
  \begingroup
    \let~\space
    \edef\x{#2}%
%<test3>  \@onelevel@sanitize\x
%<test3>  \@onelevel@sanitize\ErrMessage
    \ifx\x\ErrMessage
    \else
      \msg{}%
      \msg{[\ErrMessage]}%
      \msg{<>}%
      \msg{[\x]}%
      \TestFailed#1{errmessage text}%
    \fi
    \edef\x{\the\ErrHelpToks}%
    \edef\y{#3}%
    \ifx\x\y
    \else
      \msg{}%
      \msg{[\the\ErrHelpToks]}%
      \msg{<>}%
      \msg{[\y]}%
      \TestFailed#1{errhelp text}%
    \fi
  \endgroup
}
\TestInfo
\TestWarning
\TestError
%</test2|test3>
%<test2>\end
%<test3>\@@end
%    \end{macrocode}
%
%
% \section{Installation}
%
% \subsection{Download}
%
% \paragraph{Package.} This package is available on
% CTAN\footnote{\url{ftp://ftp.ctan.org/tex-archive/}}:
% \begin{description}
% \item[\CTAN{macros/latex/contrib/oberdiek/infwarerr.dtx}] The source file.
% \item[\CTAN{macros/latex/contrib/oberdiek/infwarerr.pdf}] Documentation.
% \end{description}
%
%
% \paragraph{Bundle.} All the packages of the bundle `oberdiek'
% are also available in a TDS compliant ZIP archive. There
% the packages are already unpacked and the documentation files
% are generated. The files and directories obey the TDS standard.
% \begin{description}
% \item[\CTAN{install/macros/latex/contrib/oberdiek.tds.zip}]
% \end{description}
% \emph{TDS} refers to the standard ``A Directory Structure
% for \TeX\ Files'' (\CTAN{tds/tds.pdf}). Directories
% with \xfile{texmf} in their name are usually organized this way.
%
% \subsection{Bundle installation}
%
% \paragraph{Unpacking.} Unpack the \xfile{oberdiek.tds.zip} in the
% TDS tree (also known as \xfile{texmf} tree) of your choice.
% Example (linux):
% \begin{quote}
%   |unzip oberdiek.tds.zip -d ~/texmf|
% \end{quote}
%
% \paragraph{Script installation.}
% Check the directory \xfile{TDS:scripts/oberdiek/} for
% scripts that need further installation steps.
% Package \xpackage{attachfile2} comes with the Perl script
% \xfile{pdfatfi.pl} that should be installed in such a way
% that it can be called as \texttt{pdfatfi}.
% Example (linux):
% \begin{quote}
%   |chmod +x scripts/oberdiek/pdfatfi.pl|\\
%   |cp scripts/oberdiek/pdfatfi.pl /usr/local/bin/|
% \end{quote}
%
% \subsection{Package installation}
%
% \paragraph{Unpacking.} The \xfile{.dtx} file is a self-extracting
% \docstrip\ archive. The files are extracted by running the
% \xfile{.dtx} through \plainTeX:
% \begin{quote}
%   \verb|tex infwarerr.dtx|
% \end{quote}
%
% \paragraph{TDS.} Now the different files must be moved into
% the different directories in your installation TDS tree
% (also known as \xfile{texmf} tree):
% \begin{quote}
% \def\t{^^A
% \begin{tabular}{@{}>{\ttfamily}l@{ $\rightarrow$ }>{\ttfamily}l@{}}
%   infwarerr.sty & tex/generic/oberdiek/infwarerr.sty\\
%   infwarerr.pdf & doc/latex/oberdiek/infwarerr.pdf\\
%   test/infwarerr-test1.tex & doc/latex/oberdiek/test/infwarerr-test1.tex\\
%   test/infwarerr-test2.tex & doc/latex/oberdiek/test/infwarerr-test2.tex\\
%   test/infwarerr-test3.tex & doc/latex/oberdiek/test/infwarerr-test3.tex\\
%   infwarerr.dtx & source/latex/oberdiek/infwarerr.dtx\\
% \end{tabular}^^A
% }^^A
% \sbox0{\t}^^A
% \ifdim\wd0>\linewidth
%   \begingroup
%     \advance\linewidth by\leftmargin
%     \advance\linewidth by\rightmargin
%   \edef\x{\endgroup
%     \def\noexpand\lw{\the\linewidth}^^A
%   }\x
%   \def\lwbox{^^A
%     \leavevmode
%     \hbox to \linewidth{^^A
%       \kern-\leftmargin\relax
%       \hss
%       \usebox0
%       \hss
%       \kern-\rightmargin\relax
%     }^^A
%   }^^A
%   \ifdim\wd0>\lw
%     \sbox0{\small\t}^^A
%     \ifdim\wd0>\linewidth
%       \ifdim\wd0>\lw
%         \sbox0{\footnotesize\t}^^A
%         \ifdim\wd0>\linewidth
%           \ifdim\wd0>\lw
%             \sbox0{\scriptsize\t}^^A
%             \ifdim\wd0>\linewidth
%               \ifdim\wd0>\lw
%                 \sbox0{\tiny\t}^^A
%                 \ifdim\wd0>\linewidth
%                   \lwbox
%                 \else
%                   \usebox0
%                 \fi
%               \else
%                 \lwbox
%               \fi
%             \else
%               \usebox0
%             \fi
%           \else
%             \lwbox
%           \fi
%         \else
%           \usebox0
%         \fi
%       \else
%         \lwbox
%       \fi
%     \else
%       \usebox0
%     \fi
%   \else
%     \lwbox
%   \fi
% \else
%   \usebox0
% \fi
% \end{quote}
% If you have a \xfile{docstrip.cfg} that configures and enables \docstrip's
% TDS installing feature, then some files can already be in the right
% place, see the documentation of \docstrip.
%
% \subsection{Refresh file name databases}
%
% If your \TeX~distribution
% (\teTeX, \mikTeX, \dots) relies on file name databases, you must refresh
% these. For example, \teTeX\ users run \verb|texhash| or
% \verb|mktexlsr|.
%
% \subsection{Some details for the interested}
%
% \paragraph{Attached source.}
%
% The PDF documentation on CTAN also includes the
% \xfile{.dtx} source file. It can be extracted by
% AcrobatReader 6 or higher. Another option is \textsf{pdftk},
% e.g. unpack the file into the current directory:
% \begin{quote}
%   \verb|pdftk infwarerr.pdf unpack_files output .|
% \end{quote}
%
% \paragraph{Unpacking with \LaTeX.}
% The \xfile{.dtx} chooses its action depending on the format:
% \begin{description}
% \item[\plainTeX:] Run \docstrip\ and extract the files.
% \item[\LaTeX:] Generate the documentation.
% \end{description}
% If you insist on using \LaTeX\ for \docstrip\ (really,
% \docstrip\ does not need \LaTeX), then inform the autodetect routine
% about your intention:
% \begin{quote}
%   \verb|latex \let\install=y\input{infwarerr.dtx}|
% \end{quote}
% Do not forget to quote the argument according to the demands
% of your shell.
%
% \paragraph{Generating the documentation.}
% You can use both the \xfile{.dtx} or the \xfile{.drv} to generate
% the documentation. The process can be configured by the
% configuration file \xfile{ltxdoc.cfg}. For instance, put this
% line into this file, if you want to have A4 as paper format:
% \begin{quote}
%   \verb|\PassOptionsToClass{a4paper}{article}|
% \end{quote}
% An example follows how to generate the
% documentation with pdf\LaTeX:
% \begin{quote}
%\begin{verbatim}
%pdflatex infwarerr.dtx
%makeindex -s gind.ist infwarerr.idx
%pdflatex infwarerr.dtx
%makeindex -s gind.ist infwarerr.idx
%pdflatex infwarerr.dtx
%\end{verbatim}
% \end{quote}
%
% \begin{History}
%   \begin{Version}{2007/04/26 v1.0}
%   \item
%     First version.
%   \end{Version}
%   \begin{Version}{2007/06/14 v1.1}
%   \item
%     Small fixes.
%   \end{Version}
%   \begin{Version}{2007/09/09 v1.2}
%   \item
%     Catcode section extended.
%   \end{Version}
%   \begin{Version}{2010/04/08 v1.3}
%   \item
%     Ensure \cs{escapechar}=92 (backslash) for infos/warnings/errors.
%   \end{Version}
% \end{History}
%
% \PrintIndex
%
% \Finale
\endinput

%        (quote the arguments according to the demands of your shell)
% 
% Documentation:
%    (a) If infwarerr.drv is present:
%           latex infwarerr.drv
%    (b) Without infwarerr.drv:
%           latex infwarerr.dtx; ...
%    The class ltxdoc loads the configuration file ltxdoc.cfg
%    if available. Here you can specify further options, e.g.
%    use A4 as paper format:
%       \PassOptionsToClass{a4paper}{article}
% 
%    Programm calls to get the documentation (example):
%       pdflatex infwarerr.dtx
%       makeindex -s gind.ist infwarerr.idx
%       pdflatex infwarerr.dtx
%       makeindex -s gind.ist infwarerr.idx
%       pdflatex infwarerr.dtx
% 
% Installation:
%    TDS:tex/generic/oberdiek/infwarerr.sty
%    TDS:doc/latex/oberdiek/infwarerr.pdf
%    TDS:doc/latex/oberdiek/test/infwarerr-test1.tex
%    TDS:doc/latex/oberdiek/test/infwarerr-test2.tex
%    TDS:doc/latex/oberdiek/test/infwarerr-test3.tex
%    TDS:source/latex/oberdiek/infwarerr.dtx
%
%<*ignore>
\begingroup
  \catcode123=1 %
  \catcode125=2 %
  \def\x{LaTeX2e}%
\expandafter\endgroup
\ifcase 0\ifx\install y1\fi\expandafter
         \ifx\csname processbatchFile\endcsname\relax\else1\fi
         \ifx\fmtname\x\else 1\fi\relax
\else\csname fi\endcsname
%</ignore>
%<*install>
\input docstrip.tex
\Msg{************************************************************************}
\Msg{* Installation}
\Msg{* Package: infwarerr 2010/04/08 v1.3 Providing info/warning/message (HO)}
\Msg{************************************************************************}

\keepsilent
\askforoverwritefalse

\let\MetaPrefix\relax
\preamble

This is a generated file.

Project: infwarerr
Version: 2010/04/08 v1.3

Copyright (C) 2007, 2010 by
   Heiko Oberdiek <heiko.oberdiek at googlemail.com>

This work may be distributed and/or modified under the
conditions of the LaTeX Project Public License, either
version 1.3c of this license or (at your option) any later
version. This version of this license is in
   http://www.latex-project.org/lppl/lppl-1-3c.txt
and the latest version of this license is in
   http://www.latex-project.org/lppl.txt
and version 1.3 or later is part of all distributions of
LaTeX version 2005/12/01 or later.

This work has the LPPL maintenance status "maintained".

This Current Maintainer of this work is Heiko Oberdiek.

The Base Interpreter refers to any `TeX-Format',
because some files are installed in TDS:tex/generic//.

This work consists of the main source file infwarerr.dtx
and the derived files
   infwarerr.sty, infwarerr.pdf, infwarerr.ins, infwarerr.drv,
   infwarerr-test1.tex, infwarerr-test2.tex,
   infwarerr-test3.tex.

\endpreamble
\let\MetaPrefix\DoubleperCent

\generate{%
  \file{infwarerr.ins}{\from{infwarerr.dtx}{install}}%
  \file{infwarerr.drv}{\from{infwarerr.dtx}{driver}}%
  \usedir{tex/generic/oberdiek}%
  \file{infwarerr.sty}{\from{infwarerr.dtx}{package}}%
  \usedir{doc/latex/oberdiek/test}%
  \file{infwarerr-test1.tex}{\from{infwarerr.dtx}{test1}}%
  \file{infwarerr-test2.tex}{\from{infwarerr.dtx}{test2}}%
  \file{infwarerr-test3.tex}{\from{infwarerr.dtx}{test3}}%
}

\catcode32=13\relax% active space
\let =\space%
\Msg{************************************************************************}
\Msg{*}
\Msg{* To finish the installation you have to move the following}
\Msg{* file into a directory searched by TeX:}
\Msg{*}
\Msg{*     infwarerr.sty}
\Msg{*}
\Msg{* To produce the documentation run the file `infwarerr.drv'}
\Msg{* through LaTeX.}
\Msg{*}
\Msg{* Happy TeXing!}
\Msg{*}
\Msg{************************************************************************}

\endbatchfile
%</install>
%<*ignore>
\fi
%</ignore>
%<*driver>
\NeedsTeXFormat{LaTeX2e}
\ProvidesFile{infwarerr.drv}%
  [2010/04/08 v1.3 Providing info/warning/message (HO)]%
\documentclass{ltxdoc}
\usepackage{holtxdoc}[2011/02/04]
\begin{document}
  \DocInput{infwarerr.dtx}%
\end{document}
%</driver>
% \fi
%
% \CheckSum{871}
%
% \CharacterTable
%  {Upper-case    \A\B\C\D\E\F\G\H\I\J\K\L\M\N\O\P\Q\R\S\T\U\V\W\X\Y\Z
%   Lower-case    \a\b\c\d\e\f\g\h\i\j\k\l\m\n\o\p\q\r\s\t\u\v\w\x\y\z
%   Digits        \0\1\2\3\4\5\6\7\8\9
%   Exclamation   \!     Double quote  \"     Hash (number) \#
%   Dollar        \$     Percent       \%     Ampersand     \&
%   Acute accent  \'     Left paren    \(     Right paren   \)
%   Asterisk      \*     Plus          \+     Comma         \,
%   Minus         \-     Point         \.     Solidus       \/
%   Colon         \:     Semicolon     \;     Less than     \<
%   Equals        \=     Greater than  \>     Question mark \?
%   Commercial at \@     Left bracket  \[     Backslash     \\
%   Right bracket \]     Circumflex    \^     Underscore    \_
%   Grave accent  \`     Left brace    \{     Vertical bar  \|
%   Right brace   \}     Tilde         \~}
%
% \GetFileInfo{infwarerr.drv}
%
% \title{The \xpackage{infwarerr} package}
% \date{2010/04/08 v1.3}
% \author{Heiko Oberdiek\\\xemail{heiko.oberdiek at googlemail.com}}
%
% \maketitle
%
% \begin{abstract}
% This package provides a complete set of macros for informations,
% warnings and error messages with support for \plainTeX.
% \end{abstract}
%
% \tableofcontents
%
% \section{Documentation}
%
% Package \xpackage{infwarerr} defines the following set of commands,
% given as pseudo regular expression:
%   \begin{quote}
%     \verb+\@(Package|Class)(Info|InfoNoLine|Warning|WarningNoLine|Error)+^^A
%     \kern-\rightmargin
%   \end{quote}
% The first argument is the package or class in question. The second
% argument contains the message text. \cs{MessageBreak} is supported.
% In case of error messages, the third argument contains the help
% text. Supported are \LaTeX's message texts \cs{@ehc}:
%   \begin{quote}
%     |Try typing  <return>  to proceed|\\
%     |If that doesn't work, type  X <return>  to quit.|
%   \end{quote}
% and \cs{@ehd}:
%   \begin{quote}
%     |You're in trouble here.  Try typing  <return>  to proceed|\\
%     |If that doesn't work, type  X <return>  to quit.|
%   \end{quote}
%
% The package is intended for macro writers that want to write
% code that works for both \LaTeX\ and \plainTeX. If \LaTeX\ is
% detected, then the macros of this package will only be wrappers
% for \LaTeX's macros. Otherwise \LaTeX's macros will be mimicked
% for use with \plainTeX.
%
% \StopEventually{
% }
%
% \section{Implementation}
%
%    \begin{macrocode}
%<*package>
%    \end{macrocode}
%
% \subsection{Reload check and package identification}
%    Reload check, especially if the package is not used with \LaTeX.
%    \begin{macrocode}
\begingroup\catcode61\catcode48\catcode32=10\relax%
  \catcode13=5 % ^^M
  \endlinechar=13 %
  \catcode35=6 % #
  \catcode39=12 % '
  \catcode44=12 % ,
  \catcode45=12 % -
  \catcode46=12 % .
  \catcode58=12 % :
  \catcode64=11 % @
  \catcode123=1 % {
  \catcode125=2 % }
  \expandafter\let\expandafter\x\csname ver@infwarerr.sty\endcsname
  \ifx\x\relax % plain-TeX, first loading
  \else
    \def\empty{}%
    \ifx\x\empty % LaTeX, first loading,
      % variable is initialized, but \ProvidesPackage not yet seen
    \else
      \expandafter\ifx\csname PackageInfo\endcsname\relax
        \def\x#1#2{%
          \immediate\write-1{Package #1 Info: #2.}%
        }%
      \else
        \def\x#1#2{\PackageInfo{#1}{#2, stopped}}%
      \fi
      \x{infwarerr}{The package is already loaded}%
      \aftergroup\endinput
    \fi
  \fi
\endgroup%
%    \end{macrocode}
%    Package identification:
%    \begin{macrocode}
\begingroup\catcode61\catcode48\catcode32=10\relax%
  \catcode13=5 % ^^M
  \endlinechar=13 %
  \catcode35=6 % #
  \catcode39=12 % '
  \catcode40=12 % (
  \catcode41=12 % )
  \catcode44=12 % ,
  \catcode45=12 % -
  \catcode46=12 % .
  \catcode47=12 % /
  \catcode58=12 % :
  \catcode64=11 % @
  \catcode91=12 % [
  \catcode93=12 % ]
  \catcode123=1 % {
  \catcode125=2 % }
  \expandafter\ifx\csname ProvidesPackage\endcsname\relax
    \def\x#1#2#3[#4]{\endgroup
      \immediate\write-1{Package: #3 #4}%
      \xdef#1{#4}%
    }%
  \else
    \def\x#1#2[#3]{\endgroup
      #2[{#3}]%
      \ifx#1\@undefined
        \xdef#1{#3}%
      \fi
      \ifx#1\relax
        \xdef#1{#3}%
      \fi
    }%
  \fi
\expandafter\x\csname ver@infwarerr.sty\endcsname
\ProvidesPackage{infwarerr}%
  [2010/04/08 v1.3 Providing info/warning/message (HO)]%
%    \end{macrocode}
%
% \subsection{Catcodes}
%
%    \begin{macrocode}
\begingroup\catcode61\catcode48\catcode32=10\relax%
  \catcode13=5 % ^^M
  \endlinechar=13 %
  \catcode123=1 % {
  \catcode125=2 % }
  \catcode64=11 % @
  \def\x{\endgroup
    \expandafter\edef\csname InfWarErr@AtEnd\endcsname{%
      \endlinechar=\the\endlinechar\relax
      \catcode13=\the\catcode13\relax
      \catcode32=\the\catcode32\relax
      \catcode35=\the\catcode35\relax
      \catcode61=\the\catcode61\relax
      \catcode64=\the\catcode64\relax
      \catcode123=\the\catcode123\relax
      \catcode125=\the\catcode125\relax
    }%
  }%
\x\catcode61\catcode48\catcode32=10\relax%
\catcode13=5 % ^^M
\endlinechar=13 %
\catcode35=6 % #
\catcode64=11 % @
\catcode123=1 % {
\catcode125=2 % }
\def\TMP@EnsureCode#1#2{%
  \edef\InfWarErr@AtEnd{%
    \InfWarErr@AtEnd
    \catcode#1=\the\catcode#1\relax
  }%
  \catcode#1=#2\relax
}
\TMP@EnsureCode{10}{12}% ^^J
\TMP@EnsureCode{39}{12}% '
\TMP@EnsureCode{40}{12}% (
\TMP@EnsureCode{41}{12}% )
\TMP@EnsureCode{44}{12}% ,
\TMP@EnsureCode{45}{12}% -
\TMP@EnsureCode{46}{12}% .
\TMP@EnsureCode{58}{12}% :
\TMP@EnsureCode{60}{12}% <
\TMP@EnsureCode{62}{12}% >
\TMP@EnsureCode{94}{7}% ^
\edef\InfWarErr@AtEnd{\InfWarErr@AtEnd\noexpand\endinput}
%    \end{macrocode}
%
% \subsection{\LaTeX\ detection}
%
%    \begin{macrocode}
\begingroup
  \let\x=Y%
  \def\check#1{%
    \@check{Generic}{#1}%
    \@check{Package}{#1}%
    \@check{Class}{#1}%
  }%
  \def\@check#1#2{%
    \expandafter\ifx\csname #1#2\endcsname\relax
      \let\x=N%
    \fi
  }%
  \check{Info}%
  \check{Warning}%
  \check{Error}%
  \@check{Package}{WarningNoLine}%
  \@check{Class}{WarningNoLine}%
  \@check{}{@ehc}%
  \@check{}{@ehd}%
\expandafter\endgroup
\ifx\x Y%
%    \end{macrocode}
%
% \subsection{Version for \LaTeX}
%
%    \begin{macro}{\@PackageInfo}
%    \begin{macrocode}
  \let\@PackageInfo\PackageInfo
%    \end{macrocode}
%    \end{macro}
%    \begin{macro}{\@PackageInfoNoLine}
%    \begin{macrocode}
  \def\@PackageInfoNoLine#1#2{%
    \PackageInfo{#1}{#2\@gobble}%
  }%
%    \end{macrocode}
%    \end{macro}
%    \begin{macro}{\@ClassInfo}
%    \begin{macrocode}
  \let\@ClassInfo\ClassInfo
%    \end{macrocode}
%    \end{macro}
%    \begin{macro}{\@ClassInfoNoLine}
%    \begin{macrocode}
  \def\@ClassInfoNoLine#1#2{%
    \ClassInfo{#1}{#2\@gobble}%
  }%
%    \end{macrocode}
%    \end{macro}
%    \begin{macro}{\@PackageWarning}
%    \begin{macrocode}
  \let\@PackageWarning\PackageWarning
%    \end{macrocode}
%    \end{macro}
%    \begin{macro}{\@PackageWarningNoLine}
%    \begin{macrocode}
  \let\@PackageWarningNoLine\PackageWarningNoLine
%    \end{macrocode}
%    \end{macro}
%    \begin{macro}{\@ClassWarning}
%    \begin{macrocode}
  \let\@ClassWarning\ClassWarning
%    \end{macrocode}
%    \end{macro}
%    \begin{macro}{\@ClassWarningNoLine}
%    \begin{macrocode}
  \let\@ClassWarningNoLine\ClassWarningNoLine
%    \end{macrocode}
%    \end{macro}
%    \begin{macro}{\@PackageError}
%    \begin{macrocode}
  \let\@PackageError\PackageError
%    \end{macrocode}
%    \end{macro}
%    \begin{macro}{\@ClassError}
%    \begin{macrocode}
  \let\@ClassError\ClassError
%    \end{macrocode}
%    \end{macro}
%    \begin{macrocode}
  \expandafter\InfWarErr@AtEnd
\fi%
%    \end{macrocode}
%
% \subsection{Version for \plainTeX}
%
%    \begin{macro}{\InfWarErr@online}
%    \begin{macrocode}
\begingroup\expandafter\endgroup
\ifcase\expandafter\ifx\csname inputlineno\endcsname\relax
         0 %
       \else
         \ifnum\inputlineno<0 %
           0 %
         \else
           1 %
         \fi
       \fi
  \def\InfWarErr@online{}%
\else
  \def\InfWarErr@online{ on input line \the\inputlineno}%
\fi
%    \end{macrocode}
%    \end{macro}
%
%    \begin{macro}{\InfWarErr@protected}
%    \begin{macrocode}
\begingroup\expandafter\expandafter\expandafter\endgroup
\expandafter\ifx\csname protected\endcsname\relax
  \def\InfWarErr@protected{}%
\else
  \let\InfWarErr@protected\protected
\fi
%    \end{macrocode}
%    \end{macro}
%
%    \begin{macro}{\InfWarErr@unused}
%    \begin{macrocode}
\chardef\InfWarErr@unused=16 %
%    \end{macrocode}
%    \end{macro}
%
%    \begin{macro}{\InfWarErr@ehc}
%    \begin{macrocode}
\def\InfWarErr@ehc{%
  Try typing \space <return> \space to proceed.\MessageBreak
  If that doesn't work, type \space X <return> \space to quit.%
}
%    \end{macrocode}
%    \end{macro}
%    \begin{macro}{\InfWarErr@ehd}
%    \begin{macrocode}
\def\InfWarErr@ehd{%
  You're in trouble here. \space\@ehc
}
%    \end{macrocode}
%    \end{macro}
%
%    \begin{macro}{\InfWarErr@GenericInfo}
%    \begin{macrocode}
\InfWarErr@protected\def\InfWarErr@GenericInfo#1#2{%
  \begingroup
    \let\on@line\InfWarErr@online
    \long\def\@gobble##1{}%
    \def\space{ }%
    \def\@spaces{\space\space\space\space}%
    \def\MessageBreak{^^J#1}%
    \newlinechar=10 %
    \escapechar=92 %
    \immediate\write-1{#2\on@line.}%
  \endgroup
}
%    \end{macrocode}
%    \end{macro}
%    \begin{macro}{\InfWarErr@GenericWarning}
%    \begin{macrocode}
\InfWarErr@protected\def\InfWarErr@GenericWarning#1#2{%
  \begingroup
    \let\on@line\InfWarErr@online
    \long\def\@gobble##1{}%
    \def\space{ }%
    \def\@spaces{\space\space\space\space}%
    \def\MessageBreak{^^J#1}%
    \newlinechar=10 %
    \escapechar=92 %
    \immediate\write\InfWarErr@unused{^^J#2\on@line.^^J}%
  \endgroup
}
%    \end{macrocode}
%    \end{macro}
%    \begin{macro}{\InfWarErr@GenericError}
%    Old \TeX\ versions below 3.141 where |^^J| is not working
%    as newline in error message and help texts are not supported.
%    \begin{macrocode}
\InfWarErr@protected\def\InfWarErr@GenericError#1#2#3#4{%
  \begingroup
    \immediate\write\InfWarErr@unused{}%
    \def\space{ }%
    \def\@spaces{\space\space\space\space}%
    \let\@ehc\InfWarErr@ehc
    \let\@ehd\InfWarErr@ehd
    \let\on@line\InfWarErr@online
    \def\MessageBreak{^^J}%
    \newlinechar=10 %
    \escapechar=92 %
    \edef\InfWarErr@help{\noexpand\errhelp{#4}}%
    \InfWarErr@help
    \def\MessageBreak{^^J#1}%
    \errmessage{%
      #2.^^J^^J%
      #3^^J%
      Type \space H <return> \space for immediate help.^^J ...^^J%
    }%
  \endgroup
}
%    \end{macrocode}
%    \end{macro}
%
%    \begin{macro}{\@PackageInfo}
%    \begin{macrocode}
\def\@PackageInfo#1#2{%
  \InfWarErr@GenericInfo{%
    (#1) \@spaces\@spaces\@spaces
  }{%
    Package #1 Info: #2%
  }%
}
%    \end{macrocode}
%    \end{macro}
%    \begin{macro}{\@PackageInfoNoLine}
%    \begin{macrocode}
\def\@PackageInfoNoLine#1#2{%
  \@PackageInfo{#1}{#2\@gobble}%
}
%    \end{macrocode}
%    \end{macro}
%    \begin{macro}{\@ClassInfo}
%    \begin{macrocode}
\def\@ClassInfo#1#2{%
  \InfWarErr@GenericInfo{%
    (#1) \space\space\@spaces\@spaces
  }{%
    Class #1 Info: #2%
  }%
}
%    \end{macrocode}
%    \end{macro}
%    \begin{macro}{\@ClassInfoNoLine}
%    \begin{macrocode}
\def\@ClassInfoNoLine#1#2{%
  \@ClassInfo{#1}{#2\@gobble}%
}
%    \end{macrocode}
%    \end{macro}
%    \begin{macro}{\@PackageWarning}
%    \begin{macrocode}
\def\@PackageWarning#1#2{%
  \InfWarErr@GenericWarning{%
    (#1)\@spaces\@spaces\@spaces\@spaces
  }{%
    Package #1 Warning: #2%
  }%
}
%    \end{macrocode}
%    \end{macro}
%    \begin{macro}{\@PackageWarningNoLine}
%    \begin{macrocode}
\def\@PackageWarningNoLine#1#2{%
  \@PackageWarning{#1}{#2\@gobble}%
}
%    \end{macrocode}
%    \end{macro}
%    \begin{macro}{\@ClassWarning}
%    \begin{macrocode}
\def\@ClassWarning#1#2{%
  \InfWarErr@GenericWarning{%
    (#1) \space\@spaces\@spaces\@spaces
  }{%
    Class #1 Warning: #2%
  }%
}
%    \end{macrocode}
%    \end{macro}
%    \begin{macro}{\@ClassWarningNoLine}
%    \begin{macrocode}
\def\@ClassWarningNoLine#1#2{%
  \@ClassWarning{#1}{#2\@gobble}%
}
%    \end{macrocode}
%    \end{macro}
%    \begin{macro}{\@PackageError}
%    \begin{macrocode}
\def\@PackageError#1#2#3{%
  \InfWarErr@GenericError{%
    (#1)\@spaces\@spaces\@spaces\@spaces
  }{%
    Package #1 Error: #2%
  }{%
    See the #1 package documentation for explanation.%
  }{#3}%
}
%    \end{macrocode}
%    \end{macro}
%    \begin{macro}{\@ClassError}
%    \begin{macrocode}
\def\@ClassError#1#2#3{%
  \InfWarErr@GenericError{%
    (#1) \space\@spaces\@spaces\@spaces
  }{%
    Class #1 Error: #2%
  }{%
    See the #1 class documentation for explanation.%
  }{#3}%
}
%    \end{macrocode}
%    \end{macro}
%
%    \begin{macrocode}
\InfWarErr@AtEnd%
%</package>
%    \end{macrocode}
%
% \section{Test}
%
% \subsection{Catcode checks for loading}
%
%    \begin{macrocode}
%<*test1>
%    \end{macrocode}
%    \begin{macrocode}
\catcode`\{=1 %
\catcode`\}=2 %
\catcode`\#=6 %
\catcode`\@=11 %
\expandafter\ifx\csname count@\endcsname\relax
  \countdef\count@=255 %
\fi
\expandafter\ifx\csname @gobble\endcsname\relax
  \long\def\@gobble#1{}%
\fi
\expandafter\ifx\csname @firstofone\endcsname\relax
  \long\def\@firstofone#1{#1}%
\fi
\expandafter\ifx\csname loop\endcsname\relax
  \expandafter\@firstofone
\else
  \expandafter\@gobble
\fi
{%
  \def\loop#1\repeat{%
    \def\body{#1}%
    \iterate
  }%
  \def\iterate{%
    \body
      \let\next\iterate
    \else
      \let\next\relax
    \fi
    \next
  }%
  \let\repeat=\fi
}%
\def\RestoreCatcodes{}
\count@=0 %
\loop
  \edef\RestoreCatcodes{%
    \RestoreCatcodes
    \catcode\the\count@=\the\catcode\count@\relax
  }%
\ifnum\count@<255 %
  \advance\count@ 1 %
\repeat

\def\RangeCatcodeInvalid#1#2{%
  \count@=#1\relax
  \loop
    \catcode\count@=15 %
  \ifnum\count@<#2\relax
    \advance\count@ 1 %
  \repeat
}
\def\RangeCatcodeCheck#1#2#3{%
  \count@=#1\relax
  \loop
    \ifnum#3=\catcode\count@
    \else
      \errmessage{%
        Character \the\count@\space
        with wrong catcode \the\catcode\count@\space
        instead of \number#3%
      }%
    \fi
  \ifnum\count@<#2\relax
    \advance\count@ 1 %
  \repeat
}
\def\space{ }
\expandafter\ifx\csname LoadCommand\endcsname\relax
  \def\LoadCommand{\input infwarerr.sty\relax}%
\fi
\def\Test{%
  \RangeCatcodeInvalid{0}{47}%
  \RangeCatcodeInvalid{58}{64}%
  \RangeCatcodeInvalid{91}{96}%
  \RangeCatcodeInvalid{123}{255}%
  \catcode`\@=12 %
  \catcode`\\=0 %
  \catcode`\%=14 %
  \LoadCommand
  \RangeCatcodeCheck{0}{36}{15}%
  \RangeCatcodeCheck{37}{37}{14}%
  \RangeCatcodeCheck{38}{47}{15}%
  \RangeCatcodeCheck{48}{57}{12}%
  \RangeCatcodeCheck{58}{63}{15}%
  \RangeCatcodeCheck{64}{64}{12}%
  \RangeCatcodeCheck{65}{90}{11}%
  \RangeCatcodeCheck{91}{91}{15}%
  \RangeCatcodeCheck{92}{92}{0}%
  \RangeCatcodeCheck{93}{96}{15}%
  \RangeCatcodeCheck{97}{122}{11}%
  \RangeCatcodeCheck{123}{255}{15}%
  \RestoreCatcodes
}
\Test
\csname @@end\endcsname
\end
%    \end{macrocode}
%    \begin{macrocode}
%</test1>
%    \end{macrocode}
%
% \subsection{Macro tests}
%
%    \begin{macrocode}
%<*test2>
\let\PackageInfo\relax
\input infwarerr.sty\relax
\let\OrgWrite\write
\chardef\WriteUnused=16 %
\def\msg#{\immediate\OrgWrite\WriteUnused}
\msg{File: infwarerr-test2.tex 2010/04/08 v1.3 Test file for plain-TeX}
\catcode64=11 %
%</test2>
%    \end{macrocode}
%
%    \begin{macrocode}
%<*test3>
\NeedsTeXFormat{LaTeX2e}
\ProvidesFile{infwarerr-test3.tex}[2010/04/08 v1.3 Test file for LaTeX]
\RequirePackage{infwarerr}[2010/04/08]
\makeatletter
\let\OrgWrite\write
\chardef\WriteUnused=\@unused
\def\msg#{\immediate\OrgWrite\WriteUnused}
%</test3>
%    \end{macrocode}
%
%    \begin{macrocode}
%<*test2|test3>
\def\pkgname{PACKAGE}
\def\clsname{CLASS}
\def\msgtext{MESSAGE\MessageBreak SECOND LINE}
\def\hlptext{HELP\MessageBreak SECOND LINE}

\let\OrgWrite\write
\let\OrgErrMessage\errmessage

\def\write#1#{%
  \@write{#1}%
}
\def\@write#1#2{%
  \OrgWrite#1{#2}%
  \xdef\WriteNum{\number#1}%
  \xdef\WriteText{#2}%
}
\def\TestFailed#1#2{%
  \OrgErrMessage{Test (\string#1) failed: #2}%
}
\def\TestWrite#1#2#3{%
  \ifnum\WriteNum=#2\relax
  \else
    \TestFailed#1{write stream number: \WriteNum <> #2}%
  \fi
  \begingroup
    \let~\space
    \edef\TestString{#3}%
    \ifx\WriteText\TestString
    \else
      \msg{}%
      \msg{[\WriteText]}%
      \msg{<>}%
      \msg{[#3]}% hash-ok
      \TestFailed#1{write text}%
    \fi
  \endgroup
}
\newtoks\ErrHelpToks
\def\errmessage#1{%
  \global\ErrHelpToks=\expandafter{\the\errhelp}%
  \msg{**************** error message ****************}%
  \msg{! #1}%
  \msg{**************** help message *****************}%
  \msg{\the\errhelp}%
  \msg{***********************************************}%
  \xdef\ErrMessage{! #1}%
}
\def\TestInfo{%
  \@PackageInfo\pkgname\msgtext
  \TestWrite\@PackageInfo{-1}{%
    Package PACKAGE Info: MESSAGE^^J%
    (PACKAGE)~~~~~~~~~~~~~SECOND LINE on input line \the\inputlineno.%
  }%
  \@PackageInfoNoLine\pkgname\msgtext
  \TestWrite\@PackageInfoNoLine{-1}{%
    Package PACKAGE Info: MESSAGE^^J%
    (PACKAGE)~~~~~~~~~~~~~SECOND LINE.%
  }%
  \@ClassInfo\clsname\msgtext
  \TestWrite\@ClassInfo{-1}{%
    Class CLASS Info: MESSAGE^^J%
    (CLASS)~~~~~~~~~~~SECOND LINE on input line \the\inputlineno.%
  }%
  \@ClassInfoNoLine\clsname\msgtext
  \TestWrite\@ClassInfoNoLine{-1}{%
    Class CLASS Info: MESSAGE^^J%
    (CLASS)~~~~~~~~~~~SECOND LINE.%
  }%
}
\def\TestWarning{%
  \@PackageWarning\pkgname\msgtext
  \TestWrite\@PackageWarning\WriteUnused{%
    ^^J%
    Package PACKAGE Warning: MESSAGE^^J%
    (PACKAGE)~~~~~~~~~~~~~~~~SECOND LINE on input line \the\inputlineno.%
    ^^J%
  }%
  \@PackageWarningNoLine\pkgname\msgtext
  \TestWrite\@PackageWarningNoLine\WriteUnused{%
    ^^J%
    Package PACKAGE Warning: MESSAGE^^J%
    (PACKAGE)~~~~~~~~~~~~~~~~SECOND LINE.%
    ^^J%
  }%
  \@ClassWarning\clsname\msgtext
  \TestWrite\@ClassWarning\WriteUnused{%
    ^^J%
    Class CLASS Warning: MESSAGE^^J%
    (CLASS)~~~~~~~~~~~~~~SECOND LINE on input line \the\inputlineno.%
    ^^J%
  }%
  \@ClassWarningNoLine\clsname\msgtext
  \TestWrite\@ClassWarningNoLine\WriteUnused{%
    ^^J%
    Class CLASS Warning: MESSAGE^^J%
    (CLASS)~~~~~~~~~~~~~~SECOND LINE.%
    ^^J%
  }%
}
\def\TestError{%
%<test3>  \let\write\OrgWrite
%<test3>  \def\plaindots{}
%<test2>  \def\plaindots{.^^J ...^^J}
  \@PackageError\pkgname\msgtext\hlptext
  \@TestError\@PackageError{%
    ! Package PACKAGE Error: MESSAGE^^J%
    (PACKAGE)~~~~~~~~~~~~~~~~SECOND LINE.^^J%
    ^^J%
    See the PACKAGE package documentation for explanation.^^J%
    Type~~H <return>~~for immediate help\plaindots
  }{%
    HELP^^J%
    SECOND LINE%
  }%
  \@ClassError\clsname\msgtext\hlptext
  \@TestError\@ClassError{%
    ! Class CLASS Error: MESSAGE^^J%
    (CLASS)~~~~~~~~~~~~~~SECOND LINE.^^J%
    ^^J%
    See the CLASS class documentation for explanation.^^J%
    Type~~H <return>~~for immediate help\plaindots
  }{%
    HELP^^J%
    SECOND LINE%
  }%
  \@PackageError\pkgname\msgtext\@ehc
  \@TestError\@PackageError{%
    ! Package PACKAGE Error: MESSAGE^^J%
    (PACKAGE)~~~~~~~~~~~~~~~~SECOND LINE.^^J%
    ^^J%
    See the PACKAGE package documentation for explanation.^^J%
    Type~~H <return>~~for immediate help\plaindots
  }{%
    Try typing~~<return>~~to proceed.^^J%
    If that doesn't work, type~~X <return>~~to quit.%
  }%
  \@PackageError\pkgname\msgtext\@ehd
  \@TestError\@PackageError{%
    ! Package PACKAGE Error: MESSAGE^^J%
    (PACKAGE)~~~~~~~~~~~~~~~~SECOND LINE.^^J%
    ^^J%
    See the PACKAGE package documentation for explanation.^^J%
    Type~~H <return>~~for immediate help\plaindots
  }{%
    You're in trouble here.~~Try typing~~<return>~~to proceed.^^J%
    If that doesn't work, type~~X <return>~~to quit.%
  }%
}
\def\@TestError#1#2#3{%
  \begingroup
    \let~\space
    \edef\x{#2}%
%<test3>  \@onelevel@sanitize\x
%<test3>  \@onelevel@sanitize\ErrMessage
    \ifx\x\ErrMessage
    \else
      \msg{}%
      \msg{[\ErrMessage]}%
      \msg{<>}%
      \msg{[\x]}%
      \TestFailed#1{errmessage text}%
    \fi
    \edef\x{\the\ErrHelpToks}%
    \edef\y{#3}%
    \ifx\x\y
    \else
      \msg{}%
      \msg{[\the\ErrHelpToks]}%
      \msg{<>}%
      \msg{[\y]}%
      \TestFailed#1{errhelp text}%
    \fi
  \endgroup
}
\TestInfo
\TestWarning
\TestError
%</test2|test3>
%<test2>\end
%<test3>\@@end
%    \end{macrocode}
%
%
% \section{Installation}
%
% \subsection{Download}
%
% \paragraph{Package.} This package is available on
% CTAN\footnote{\url{ftp://ftp.ctan.org/tex-archive/}}:
% \begin{description}
% \item[\CTAN{macros/latex/contrib/oberdiek/infwarerr.dtx}] The source file.
% \item[\CTAN{macros/latex/contrib/oberdiek/infwarerr.pdf}] Documentation.
% \end{description}
%
%
% \paragraph{Bundle.} All the packages of the bundle `oberdiek'
% are also available in a TDS compliant ZIP archive. There
% the packages are already unpacked and the documentation files
% are generated. The files and directories obey the TDS standard.
% \begin{description}
% \item[\CTAN{install/macros/latex/contrib/oberdiek.tds.zip}]
% \end{description}
% \emph{TDS} refers to the standard ``A Directory Structure
% for \TeX\ Files'' (\CTAN{tds/tds.pdf}). Directories
% with \xfile{texmf} in their name are usually organized this way.
%
% \subsection{Bundle installation}
%
% \paragraph{Unpacking.} Unpack the \xfile{oberdiek.tds.zip} in the
% TDS tree (also known as \xfile{texmf} tree) of your choice.
% Example (linux):
% \begin{quote}
%   |unzip oberdiek.tds.zip -d ~/texmf|
% \end{quote}
%
% \paragraph{Script installation.}
% Check the directory \xfile{TDS:scripts/oberdiek/} for
% scripts that need further installation steps.
% Package \xpackage{attachfile2} comes with the Perl script
% \xfile{pdfatfi.pl} that should be installed in such a way
% that it can be called as \texttt{pdfatfi}.
% Example (linux):
% \begin{quote}
%   |chmod +x scripts/oberdiek/pdfatfi.pl|\\
%   |cp scripts/oberdiek/pdfatfi.pl /usr/local/bin/|
% \end{quote}
%
% \subsection{Package installation}
%
% \paragraph{Unpacking.} The \xfile{.dtx} file is a self-extracting
% \docstrip\ archive. The files are extracted by running the
% \xfile{.dtx} through \plainTeX:
% \begin{quote}
%   \verb|tex infwarerr.dtx|
% \end{quote}
%
% \paragraph{TDS.} Now the different files must be moved into
% the different directories in your installation TDS tree
% (also known as \xfile{texmf} tree):
% \begin{quote}
% \def\t{^^A
% \begin{tabular}{@{}>{\ttfamily}l@{ $\rightarrow$ }>{\ttfamily}l@{}}
%   infwarerr.sty & tex/generic/oberdiek/infwarerr.sty\\
%   infwarerr.pdf & doc/latex/oberdiek/infwarerr.pdf\\
%   test/infwarerr-test1.tex & doc/latex/oberdiek/test/infwarerr-test1.tex\\
%   test/infwarerr-test2.tex & doc/latex/oberdiek/test/infwarerr-test2.tex\\
%   test/infwarerr-test3.tex & doc/latex/oberdiek/test/infwarerr-test3.tex\\
%   infwarerr.dtx & source/latex/oberdiek/infwarerr.dtx\\
% \end{tabular}^^A
% }^^A
% \sbox0{\t}^^A
% \ifdim\wd0>\linewidth
%   \begingroup
%     \advance\linewidth by\leftmargin
%     \advance\linewidth by\rightmargin
%   \edef\x{\endgroup
%     \def\noexpand\lw{\the\linewidth}^^A
%   }\x
%   \def\lwbox{^^A
%     \leavevmode
%     \hbox to \linewidth{^^A
%       \kern-\leftmargin\relax
%       \hss
%       \usebox0
%       \hss
%       \kern-\rightmargin\relax
%     }^^A
%   }^^A
%   \ifdim\wd0>\lw
%     \sbox0{\small\t}^^A
%     \ifdim\wd0>\linewidth
%       \ifdim\wd0>\lw
%         \sbox0{\footnotesize\t}^^A
%         \ifdim\wd0>\linewidth
%           \ifdim\wd0>\lw
%             \sbox0{\scriptsize\t}^^A
%             \ifdim\wd0>\linewidth
%               \ifdim\wd0>\lw
%                 \sbox0{\tiny\t}^^A
%                 \ifdim\wd0>\linewidth
%                   \lwbox
%                 \else
%                   \usebox0
%                 \fi
%               \else
%                 \lwbox
%               \fi
%             \else
%               \usebox0
%             \fi
%           \else
%             \lwbox
%           \fi
%         \else
%           \usebox0
%         \fi
%       \else
%         \lwbox
%       \fi
%     \else
%       \usebox0
%     \fi
%   \else
%     \lwbox
%   \fi
% \else
%   \usebox0
% \fi
% \end{quote}
% If you have a \xfile{docstrip.cfg} that configures and enables \docstrip's
% TDS installing feature, then some files can already be in the right
% place, see the documentation of \docstrip.
%
% \subsection{Refresh file name databases}
%
% If your \TeX~distribution
% (\teTeX, \mikTeX, \dots) relies on file name databases, you must refresh
% these. For example, \teTeX\ users run \verb|texhash| or
% \verb|mktexlsr|.
%
% \subsection{Some details for the interested}
%
% \paragraph{Attached source.}
%
% The PDF documentation on CTAN also includes the
% \xfile{.dtx} source file. It can be extracted by
% AcrobatReader 6 or higher. Another option is \textsf{pdftk},
% e.g. unpack the file into the current directory:
% \begin{quote}
%   \verb|pdftk infwarerr.pdf unpack_files output .|
% \end{quote}
%
% \paragraph{Unpacking with \LaTeX.}
% The \xfile{.dtx} chooses its action depending on the format:
% \begin{description}
% \item[\plainTeX:] Run \docstrip\ and extract the files.
% \item[\LaTeX:] Generate the documentation.
% \end{description}
% If you insist on using \LaTeX\ for \docstrip\ (really,
% \docstrip\ does not need \LaTeX), then inform the autodetect routine
% about your intention:
% \begin{quote}
%   \verb|latex \let\install=y% \iffalse meta-comment
% 
% File: infwarerr.dtx
% Version: 2010/04/08 v1.3
% 
% Copyright (C) 2007, 2010 by
%    Heiko Oberdiek <heiko.oberdiek at googlemail.com>
% 
% This work may be distributed and/or modified under the
% conditions of the LaTeX Project Public License, either
% version 1.3c of this license or (at your option) any later
% version. This version of this license is in
%    http://www.latex-project.org/lppl/lppl-1-3c.txt
% and the latest version of this license is in
%    http://www.latex-project.org/lppl.txt
% and version 1.3 or later is part of all distributions of
% LaTeX version 2005/12/01 or later.
% 
% This work has the LPPL maintenance status "maintained".
% 
% This Current Maintainer of this work is Heiko Oberdiek.
% 
% The Base Interpreter refers to any `TeX-Format',
% because some files are installed in TDS:tex/generic//.
% 
% This work consists of the main source file infwarerr.dtx
% and the derived files
%    infwarerr.sty, infwarerr.pdf, infwarerr.ins, infwarerr.drv,
%    infwarerr-test1.tex, infwarerr-test2.tex,
%    infwarerr-test3.tex.
% 
% Distribution:
%    CTAN:macros/latex/contrib/oberdiek/infwarerr.dtx
%    CTAN:macros/latex/contrib/oberdiek/infwarerr.pdf
% 
% Unpacking:
%    (a) If infwarerr.ins is present:
%           tex infwarerr.ins
%    (b) Without infwarerr.ins:
%           tex infwarerr.dtx
%    (c) If you insist on using LaTeX
%           latex \let\install=y\input{infwarerr.dtx}
%        (quote the arguments according to the demands of your shell)
% 
% Documentation:
%    (a) If infwarerr.drv is present:
%           latex infwarerr.drv
%    (b) Without infwarerr.drv:
%           latex infwarerr.dtx; ...
%    The class ltxdoc loads the configuration file ltxdoc.cfg
%    if available. Here you can specify further options, e.g.
%    use A4 as paper format:
%       \PassOptionsToClass{a4paper}{article}
% 
%    Programm calls to get the documentation (example):
%       pdflatex infwarerr.dtx
%       makeindex -s gind.ist infwarerr.idx
%       pdflatex infwarerr.dtx
%       makeindex -s gind.ist infwarerr.idx
%       pdflatex infwarerr.dtx
% 
% Installation:
%    TDS:tex/generic/oberdiek/infwarerr.sty
%    TDS:doc/latex/oberdiek/infwarerr.pdf
%    TDS:doc/latex/oberdiek/test/infwarerr-test1.tex
%    TDS:doc/latex/oberdiek/test/infwarerr-test2.tex
%    TDS:doc/latex/oberdiek/test/infwarerr-test3.tex
%    TDS:source/latex/oberdiek/infwarerr.dtx
%
%<*ignore>
\begingroup
  \catcode123=1 %
  \catcode125=2 %
  \def\x{LaTeX2e}%
\expandafter\endgroup
\ifcase 0\ifx\install y1\fi\expandafter
         \ifx\csname processbatchFile\endcsname\relax\else1\fi
         \ifx\fmtname\x\else 1\fi\relax
\else\csname fi\endcsname
%</ignore>
%<*install>
\input docstrip.tex
\Msg{************************************************************************}
\Msg{* Installation}
\Msg{* Package: infwarerr 2010/04/08 v1.3 Providing info/warning/message (HO)}
\Msg{************************************************************************}

\keepsilent
\askforoverwritefalse

\let\MetaPrefix\relax
\preamble

This is a generated file.

Project: infwarerr
Version: 2010/04/08 v1.3

Copyright (C) 2007, 2010 by
   Heiko Oberdiek <heiko.oberdiek at googlemail.com>

This work may be distributed and/or modified under the
conditions of the LaTeX Project Public License, either
version 1.3c of this license or (at your option) any later
version. This version of this license is in
   http://www.latex-project.org/lppl/lppl-1-3c.txt
and the latest version of this license is in
   http://www.latex-project.org/lppl.txt
and version 1.3 or later is part of all distributions of
LaTeX version 2005/12/01 or later.

This work has the LPPL maintenance status "maintained".

This Current Maintainer of this work is Heiko Oberdiek.

The Base Interpreter refers to any `TeX-Format',
because some files are installed in TDS:tex/generic//.

This work consists of the main source file infwarerr.dtx
and the derived files
   infwarerr.sty, infwarerr.pdf, infwarerr.ins, infwarerr.drv,
   infwarerr-test1.tex, infwarerr-test2.tex,
   infwarerr-test3.tex.

\endpreamble
\let\MetaPrefix\DoubleperCent

\generate{%
  \file{infwarerr.ins}{\from{infwarerr.dtx}{install}}%
  \file{infwarerr.drv}{\from{infwarerr.dtx}{driver}}%
  \usedir{tex/generic/oberdiek}%
  \file{infwarerr.sty}{\from{infwarerr.dtx}{package}}%
  \usedir{doc/latex/oberdiek/test}%
  \file{infwarerr-test1.tex}{\from{infwarerr.dtx}{test1}}%
  \file{infwarerr-test2.tex}{\from{infwarerr.dtx}{test2}}%
  \file{infwarerr-test3.tex}{\from{infwarerr.dtx}{test3}}%
}

\catcode32=13\relax% active space
\let =\space%
\Msg{************************************************************************}
\Msg{*}
\Msg{* To finish the installation you have to move the following}
\Msg{* file into a directory searched by TeX:}
\Msg{*}
\Msg{*     infwarerr.sty}
\Msg{*}
\Msg{* To produce the documentation run the file `infwarerr.drv'}
\Msg{* through LaTeX.}
\Msg{*}
\Msg{* Happy TeXing!}
\Msg{*}
\Msg{************************************************************************}

\endbatchfile
%</install>
%<*ignore>
\fi
%</ignore>
%<*driver>
\NeedsTeXFormat{LaTeX2e}
\ProvidesFile{infwarerr.drv}%
  [2010/04/08 v1.3 Providing info/warning/message (HO)]%
\documentclass{ltxdoc}
\usepackage{holtxdoc}[2011/02/04]
\begin{document}
  \DocInput{infwarerr.dtx}%
\end{document}
%</driver>
% \fi
%
% \CheckSum{871}
%
% \CharacterTable
%  {Upper-case    \A\B\C\D\E\F\G\H\I\J\K\L\M\N\O\P\Q\R\S\T\U\V\W\X\Y\Z
%   Lower-case    \a\b\c\d\e\f\g\h\i\j\k\l\m\n\o\p\q\r\s\t\u\v\w\x\y\z
%   Digits        \0\1\2\3\4\5\6\7\8\9
%   Exclamation   \!     Double quote  \"     Hash (number) \#
%   Dollar        \$     Percent       \%     Ampersand     \&
%   Acute accent  \'     Left paren    \(     Right paren   \)
%   Asterisk      \*     Plus          \+     Comma         \,
%   Minus         \-     Point         \.     Solidus       \/
%   Colon         \:     Semicolon     \;     Less than     \<
%   Equals        \=     Greater than  \>     Question mark \?
%   Commercial at \@     Left bracket  \[     Backslash     \\
%   Right bracket \]     Circumflex    \^     Underscore    \_
%   Grave accent  \`     Left brace    \{     Vertical bar  \|
%   Right brace   \}     Tilde         \~}
%
% \GetFileInfo{infwarerr.drv}
%
% \title{The \xpackage{infwarerr} package}
% \date{2010/04/08 v1.3}
% \author{Heiko Oberdiek\\\xemail{heiko.oberdiek at googlemail.com}}
%
% \maketitle
%
% \begin{abstract}
% This package provides a complete set of macros for informations,
% warnings and error messages with support for \plainTeX.
% \end{abstract}
%
% \tableofcontents
%
% \section{Documentation}
%
% Package \xpackage{infwarerr} defines the following set of commands,
% given as pseudo regular expression:
%   \begin{quote}
%     \verb+\@(Package|Class)(Info|InfoNoLine|Warning|WarningNoLine|Error)+^^A
%     \kern-\rightmargin
%   \end{quote}
% The first argument is the package or class in question. The second
% argument contains the message text. \cs{MessageBreak} is supported.
% In case of error messages, the third argument contains the help
% text. Supported are \LaTeX's message texts \cs{@ehc}:
%   \begin{quote}
%     |Try typing  <return>  to proceed|\\
%     |If that doesn't work, type  X <return>  to quit.|
%   \end{quote}
% and \cs{@ehd}:
%   \begin{quote}
%     |You're in trouble here.  Try typing  <return>  to proceed|\\
%     |If that doesn't work, type  X <return>  to quit.|
%   \end{quote}
%
% The package is intended for macro writers that want to write
% code that works for both \LaTeX\ and \plainTeX. If \LaTeX\ is
% detected, then the macros of this package will only be wrappers
% for \LaTeX's macros. Otherwise \LaTeX's macros will be mimicked
% for use with \plainTeX.
%
% \StopEventually{
% }
%
% \section{Implementation}
%
%    \begin{macrocode}
%<*package>
%    \end{macrocode}
%
% \subsection{Reload check and package identification}
%    Reload check, especially if the package is not used with \LaTeX.
%    \begin{macrocode}
\begingroup\catcode61\catcode48\catcode32=10\relax%
  \catcode13=5 % ^^M
  \endlinechar=13 %
  \catcode35=6 % #
  \catcode39=12 % '
  \catcode44=12 % ,
  \catcode45=12 % -
  \catcode46=12 % .
  \catcode58=12 % :
  \catcode64=11 % @
  \catcode123=1 % {
  \catcode125=2 % }
  \expandafter\let\expandafter\x\csname ver@infwarerr.sty\endcsname
  \ifx\x\relax % plain-TeX, first loading
  \else
    \def\empty{}%
    \ifx\x\empty % LaTeX, first loading,
      % variable is initialized, but \ProvidesPackage not yet seen
    \else
      \expandafter\ifx\csname PackageInfo\endcsname\relax
        \def\x#1#2{%
          \immediate\write-1{Package #1 Info: #2.}%
        }%
      \else
        \def\x#1#2{\PackageInfo{#1}{#2, stopped}}%
      \fi
      \x{infwarerr}{The package is already loaded}%
      \aftergroup\endinput
    \fi
  \fi
\endgroup%
%    \end{macrocode}
%    Package identification:
%    \begin{macrocode}
\begingroup\catcode61\catcode48\catcode32=10\relax%
  \catcode13=5 % ^^M
  \endlinechar=13 %
  \catcode35=6 % #
  \catcode39=12 % '
  \catcode40=12 % (
  \catcode41=12 % )
  \catcode44=12 % ,
  \catcode45=12 % -
  \catcode46=12 % .
  \catcode47=12 % /
  \catcode58=12 % :
  \catcode64=11 % @
  \catcode91=12 % [
  \catcode93=12 % ]
  \catcode123=1 % {
  \catcode125=2 % }
  \expandafter\ifx\csname ProvidesPackage\endcsname\relax
    \def\x#1#2#3[#4]{\endgroup
      \immediate\write-1{Package: #3 #4}%
      \xdef#1{#4}%
    }%
  \else
    \def\x#1#2[#3]{\endgroup
      #2[{#3}]%
      \ifx#1\@undefined
        \xdef#1{#3}%
      \fi
      \ifx#1\relax
        \xdef#1{#3}%
      \fi
    }%
  \fi
\expandafter\x\csname ver@infwarerr.sty\endcsname
\ProvidesPackage{infwarerr}%
  [2010/04/08 v1.3 Providing info/warning/message (HO)]%
%    \end{macrocode}
%
% \subsection{Catcodes}
%
%    \begin{macrocode}
\begingroup\catcode61\catcode48\catcode32=10\relax%
  \catcode13=5 % ^^M
  \endlinechar=13 %
  \catcode123=1 % {
  \catcode125=2 % }
  \catcode64=11 % @
  \def\x{\endgroup
    \expandafter\edef\csname InfWarErr@AtEnd\endcsname{%
      \endlinechar=\the\endlinechar\relax
      \catcode13=\the\catcode13\relax
      \catcode32=\the\catcode32\relax
      \catcode35=\the\catcode35\relax
      \catcode61=\the\catcode61\relax
      \catcode64=\the\catcode64\relax
      \catcode123=\the\catcode123\relax
      \catcode125=\the\catcode125\relax
    }%
  }%
\x\catcode61\catcode48\catcode32=10\relax%
\catcode13=5 % ^^M
\endlinechar=13 %
\catcode35=6 % #
\catcode64=11 % @
\catcode123=1 % {
\catcode125=2 % }
\def\TMP@EnsureCode#1#2{%
  \edef\InfWarErr@AtEnd{%
    \InfWarErr@AtEnd
    \catcode#1=\the\catcode#1\relax
  }%
  \catcode#1=#2\relax
}
\TMP@EnsureCode{10}{12}% ^^J
\TMP@EnsureCode{39}{12}% '
\TMP@EnsureCode{40}{12}% (
\TMP@EnsureCode{41}{12}% )
\TMP@EnsureCode{44}{12}% ,
\TMP@EnsureCode{45}{12}% -
\TMP@EnsureCode{46}{12}% .
\TMP@EnsureCode{58}{12}% :
\TMP@EnsureCode{60}{12}% <
\TMP@EnsureCode{62}{12}% >
\TMP@EnsureCode{94}{7}% ^
\edef\InfWarErr@AtEnd{\InfWarErr@AtEnd\noexpand\endinput}
%    \end{macrocode}
%
% \subsection{\LaTeX\ detection}
%
%    \begin{macrocode}
\begingroup
  \let\x=Y%
  \def\check#1{%
    \@check{Generic}{#1}%
    \@check{Package}{#1}%
    \@check{Class}{#1}%
  }%
  \def\@check#1#2{%
    \expandafter\ifx\csname #1#2\endcsname\relax
      \let\x=N%
    \fi
  }%
  \check{Info}%
  \check{Warning}%
  \check{Error}%
  \@check{Package}{WarningNoLine}%
  \@check{Class}{WarningNoLine}%
  \@check{}{@ehc}%
  \@check{}{@ehd}%
\expandafter\endgroup
\ifx\x Y%
%    \end{macrocode}
%
% \subsection{Version for \LaTeX}
%
%    \begin{macro}{\@PackageInfo}
%    \begin{macrocode}
  \let\@PackageInfo\PackageInfo
%    \end{macrocode}
%    \end{macro}
%    \begin{macro}{\@PackageInfoNoLine}
%    \begin{macrocode}
  \def\@PackageInfoNoLine#1#2{%
    \PackageInfo{#1}{#2\@gobble}%
  }%
%    \end{macrocode}
%    \end{macro}
%    \begin{macro}{\@ClassInfo}
%    \begin{macrocode}
  \let\@ClassInfo\ClassInfo
%    \end{macrocode}
%    \end{macro}
%    \begin{macro}{\@ClassInfoNoLine}
%    \begin{macrocode}
  \def\@ClassInfoNoLine#1#2{%
    \ClassInfo{#1}{#2\@gobble}%
  }%
%    \end{macrocode}
%    \end{macro}
%    \begin{macro}{\@PackageWarning}
%    \begin{macrocode}
  \let\@PackageWarning\PackageWarning
%    \end{macrocode}
%    \end{macro}
%    \begin{macro}{\@PackageWarningNoLine}
%    \begin{macrocode}
  \let\@PackageWarningNoLine\PackageWarningNoLine
%    \end{macrocode}
%    \end{macro}
%    \begin{macro}{\@ClassWarning}
%    \begin{macrocode}
  \let\@ClassWarning\ClassWarning
%    \end{macrocode}
%    \end{macro}
%    \begin{macro}{\@ClassWarningNoLine}
%    \begin{macrocode}
  \let\@ClassWarningNoLine\ClassWarningNoLine
%    \end{macrocode}
%    \end{macro}
%    \begin{macro}{\@PackageError}
%    \begin{macrocode}
  \let\@PackageError\PackageError
%    \end{macrocode}
%    \end{macro}
%    \begin{macro}{\@ClassError}
%    \begin{macrocode}
  \let\@ClassError\ClassError
%    \end{macrocode}
%    \end{macro}
%    \begin{macrocode}
  \expandafter\InfWarErr@AtEnd
\fi%
%    \end{macrocode}
%
% \subsection{Version for \plainTeX}
%
%    \begin{macro}{\InfWarErr@online}
%    \begin{macrocode}
\begingroup\expandafter\endgroup
\ifcase\expandafter\ifx\csname inputlineno\endcsname\relax
         0 %
       \else
         \ifnum\inputlineno<0 %
           0 %
         \else
           1 %
         \fi
       \fi
  \def\InfWarErr@online{}%
\else
  \def\InfWarErr@online{ on input line \the\inputlineno}%
\fi
%    \end{macrocode}
%    \end{macro}
%
%    \begin{macro}{\InfWarErr@protected}
%    \begin{macrocode}
\begingroup\expandafter\expandafter\expandafter\endgroup
\expandafter\ifx\csname protected\endcsname\relax
  \def\InfWarErr@protected{}%
\else
  \let\InfWarErr@protected\protected
\fi
%    \end{macrocode}
%    \end{macro}
%
%    \begin{macro}{\InfWarErr@unused}
%    \begin{macrocode}
\chardef\InfWarErr@unused=16 %
%    \end{macrocode}
%    \end{macro}
%
%    \begin{macro}{\InfWarErr@ehc}
%    \begin{macrocode}
\def\InfWarErr@ehc{%
  Try typing \space <return> \space to proceed.\MessageBreak
  If that doesn't work, type \space X <return> \space to quit.%
}
%    \end{macrocode}
%    \end{macro}
%    \begin{macro}{\InfWarErr@ehd}
%    \begin{macrocode}
\def\InfWarErr@ehd{%
  You're in trouble here. \space\@ehc
}
%    \end{macrocode}
%    \end{macro}
%
%    \begin{macro}{\InfWarErr@GenericInfo}
%    \begin{macrocode}
\InfWarErr@protected\def\InfWarErr@GenericInfo#1#2{%
  \begingroup
    \let\on@line\InfWarErr@online
    \long\def\@gobble##1{}%
    \def\space{ }%
    \def\@spaces{\space\space\space\space}%
    \def\MessageBreak{^^J#1}%
    \newlinechar=10 %
    \escapechar=92 %
    \immediate\write-1{#2\on@line.}%
  \endgroup
}
%    \end{macrocode}
%    \end{macro}
%    \begin{macro}{\InfWarErr@GenericWarning}
%    \begin{macrocode}
\InfWarErr@protected\def\InfWarErr@GenericWarning#1#2{%
  \begingroup
    \let\on@line\InfWarErr@online
    \long\def\@gobble##1{}%
    \def\space{ }%
    \def\@spaces{\space\space\space\space}%
    \def\MessageBreak{^^J#1}%
    \newlinechar=10 %
    \escapechar=92 %
    \immediate\write\InfWarErr@unused{^^J#2\on@line.^^J}%
  \endgroup
}
%    \end{macrocode}
%    \end{macro}
%    \begin{macro}{\InfWarErr@GenericError}
%    Old \TeX\ versions below 3.141 where |^^J| is not working
%    as newline in error message and help texts are not supported.
%    \begin{macrocode}
\InfWarErr@protected\def\InfWarErr@GenericError#1#2#3#4{%
  \begingroup
    \immediate\write\InfWarErr@unused{}%
    \def\space{ }%
    \def\@spaces{\space\space\space\space}%
    \let\@ehc\InfWarErr@ehc
    \let\@ehd\InfWarErr@ehd
    \let\on@line\InfWarErr@online
    \def\MessageBreak{^^J}%
    \newlinechar=10 %
    \escapechar=92 %
    \edef\InfWarErr@help{\noexpand\errhelp{#4}}%
    \InfWarErr@help
    \def\MessageBreak{^^J#1}%
    \errmessage{%
      #2.^^J^^J%
      #3^^J%
      Type \space H <return> \space for immediate help.^^J ...^^J%
    }%
  \endgroup
}
%    \end{macrocode}
%    \end{macro}
%
%    \begin{macro}{\@PackageInfo}
%    \begin{macrocode}
\def\@PackageInfo#1#2{%
  \InfWarErr@GenericInfo{%
    (#1) \@spaces\@spaces\@spaces
  }{%
    Package #1 Info: #2%
  }%
}
%    \end{macrocode}
%    \end{macro}
%    \begin{macro}{\@PackageInfoNoLine}
%    \begin{macrocode}
\def\@PackageInfoNoLine#1#2{%
  \@PackageInfo{#1}{#2\@gobble}%
}
%    \end{macrocode}
%    \end{macro}
%    \begin{macro}{\@ClassInfo}
%    \begin{macrocode}
\def\@ClassInfo#1#2{%
  \InfWarErr@GenericInfo{%
    (#1) \space\space\@spaces\@spaces
  }{%
    Class #1 Info: #2%
  }%
}
%    \end{macrocode}
%    \end{macro}
%    \begin{macro}{\@ClassInfoNoLine}
%    \begin{macrocode}
\def\@ClassInfoNoLine#1#2{%
  \@ClassInfo{#1}{#2\@gobble}%
}
%    \end{macrocode}
%    \end{macro}
%    \begin{macro}{\@PackageWarning}
%    \begin{macrocode}
\def\@PackageWarning#1#2{%
  \InfWarErr@GenericWarning{%
    (#1)\@spaces\@spaces\@spaces\@spaces
  }{%
    Package #1 Warning: #2%
  }%
}
%    \end{macrocode}
%    \end{macro}
%    \begin{macro}{\@PackageWarningNoLine}
%    \begin{macrocode}
\def\@PackageWarningNoLine#1#2{%
  \@PackageWarning{#1}{#2\@gobble}%
}
%    \end{macrocode}
%    \end{macro}
%    \begin{macro}{\@ClassWarning}
%    \begin{macrocode}
\def\@ClassWarning#1#2{%
  \InfWarErr@GenericWarning{%
    (#1) \space\@spaces\@spaces\@spaces
  }{%
    Class #1 Warning: #2%
  }%
}
%    \end{macrocode}
%    \end{macro}
%    \begin{macro}{\@ClassWarningNoLine}
%    \begin{macrocode}
\def\@ClassWarningNoLine#1#2{%
  \@ClassWarning{#1}{#2\@gobble}%
}
%    \end{macrocode}
%    \end{macro}
%    \begin{macro}{\@PackageError}
%    \begin{macrocode}
\def\@PackageError#1#2#3{%
  \InfWarErr@GenericError{%
    (#1)\@spaces\@spaces\@spaces\@spaces
  }{%
    Package #1 Error: #2%
  }{%
    See the #1 package documentation for explanation.%
  }{#3}%
}
%    \end{macrocode}
%    \end{macro}
%    \begin{macro}{\@ClassError}
%    \begin{macrocode}
\def\@ClassError#1#2#3{%
  \InfWarErr@GenericError{%
    (#1) \space\@spaces\@spaces\@spaces
  }{%
    Class #1 Error: #2%
  }{%
    See the #1 class documentation for explanation.%
  }{#3}%
}
%    \end{macrocode}
%    \end{macro}
%
%    \begin{macrocode}
\InfWarErr@AtEnd%
%</package>
%    \end{macrocode}
%
% \section{Test}
%
% \subsection{Catcode checks for loading}
%
%    \begin{macrocode}
%<*test1>
%    \end{macrocode}
%    \begin{macrocode}
\catcode`\{=1 %
\catcode`\}=2 %
\catcode`\#=6 %
\catcode`\@=11 %
\expandafter\ifx\csname count@\endcsname\relax
  \countdef\count@=255 %
\fi
\expandafter\ifx\csname @gobble\endcsname\relax
  \long\def\@gobble#1{}%
\fi
\expandafter\ifx\csname @firstofone\endcsname\relax
  \long\def\@firstofone#1{#1}%
\fi
\expandafter\ifx\csname loop\endcsname\relax
  \expandafter\@firstofone
\else
  \expandafter\@gobble
\fi
{%
  \def\loop#1\repeat{%
    \def\body{#1}%
    \iterate
  }%
  \def\iterate{%
    \body
      \let\next\iterate
    \else
      \let\next\relax
    \fi
    \next
  }%
  \let\repeat=\fi
}%
\def\RestoreCatcodes{}
\count@=0 %
\loop
  \edef\RestoreCatcodes{%
    \RestoreCatcodes
    \catcode\the\count@=\the\catcode\count@\relax
  }%
\ifnum\count@<255 %
  \advance\count@ 1 %
\repeat

\def\RangeCatcodeInvalid#1#2{%
  \count@=#1\relax
  \loop
    \catcode\count@=15 %
  \ifnum\count@<#2\relax
    \advance\count@ 1 %
  \repeat
}
\def\RangeCatcodeCheck#1#2#3{%
  \count@=#1\relax
  \loop
    \ifnum#3=\catcode\count@
    \else
      \errmessage{%
        Character \the\count@\space
        with wrong catcode \the\catcode\count@\space
        instead of \number#3%
      }%
    \fi
  \ifnum\count@<#2\relax
    \advance\count@ 1 %
  \repeat
}
\def\space{ }
\expandafter\ifx\csname LoadCommand\endcsname\relax
  \def\LoadCommand{\input infwarerr.sty\relax}%
\fi
\def\Test{%
  \RangeCatcodeInvalid{0}{47}%
  \RangeCatcodeInvalid{58}{64}%
  \RangeCatcodeInvalid{91}{96}%
  \RangeCatcodeInvalid{123}{255}%
  \catcode`\@=12 %
  \catcode`\\=0 %
  \catcode`\%=14 %
  \LoadCommand
  \RangeCatcodeCheck{0}{36}{15}%
  \RangeCatcodeCheck{37}{37}{14}%
  \RangeCatcodeCheck{38}{47}{15}%
  \RangeCatcodeCheck{48}{57}{12}%
  \RangeCatcodeCheck{58}{63}{15}%
  \RangeCatcodeCheck{64}{64}{12}%
  \RangeCatcodeCheck{65}{90}{11}%
  \RangeCatcodeCheck{91}{91}{15}%
  \RangeCatcodeCheck{92}{92}{0}%
  \RangeCatcodeCheck{93}{96}{15}%
  \RangeCatcodeCheck{97}{122}{11}%
  \RangeCatcodeCheck{123}{255}{15}%
  \RestoreCatcodes
}
\Test
\csname @@end\endcsname
\end
%    \end{macrocode}
%    \begin{macrocode}
%</test1>
%    \end{macrocode}
%
% \subsection{Macro tests}
%
%    \begin{macrocode}
%<*test2>
\let\PackageInfo\relax
\input infwarerr.sty\relax
\let\OrgWrite\write
\chardef\WriteUnused=16 %
\def\msg#{\immediate\OrgWrite\WriteUnused}
\msg{File: infwarerr-test2.tex 2010/04/08 v1.3 Test file for plain-TeX}
\catcode64=11 %
%</test2>
%    \end{macrocode}
%
%    \begin{macrocode}
%<*test3>
\NeedsTeXFormat{LaTeX2e}
\ProvidesFile{infwarerr-test3.tex}[2010/04/08 v1.3 Test file for LaTeX]
\RequirePackage{infwarerr}[2010/04/08]
\makeatletter
\let\OrgWrite\write
\chardef\WriteUnused=\@unused
\def\msg#{\immediate\OrgWrite\WriteUnused}
%</test3>
%    \end{macrocode}
%
%    \begin{macrocode}
%<*test2|test3>
\def\pkgname{PACKAGE}
\def\clsname{CLASS}
\def\msgtext{MESSAGE\MessageBreak SECOND LINE}
\def\hlptext{HELP\MessageBreak SECOND LINE}

\let\OrgWrite\write
\let\OrgErrMessage\errmessage

\def\write#1#{%
  \@write{#1}%
}
\def\@write#1#2{%
  \OrgWrite#1{#2}%
  \xdef\WriteNum{\number#1}%
  \xdef\WriteText{#2}%
}
\def\TestFailed#1#2{%
  \OrgErrMessage{Test (\string#1) failed: #2}%
}
\def\TestWrite#1#2#3{%
  \ifnum\WriteNum=#2\relax
  \else
    \TestFailed#1{write stream number: \WriteNum <> #2}%
  \fi
  \begingroup
    \let~\space
    \edef\TestString{#3}%
    \ifx\WriteText\TestString
    \else
      \msg{}%
      \msg{[\WriteText]}%
      \msg{<>}%
      \msg{[#3]}% hash-ok
      \TestFailed#1{write text}%
    \fi
  \endgroup
}
\newtoks\ErrHelpToks
\def\errmessage#1{%
  \global\ErrHelpToks=\expandafter{\the\errhelp}%
  \msg{**************** error message ****************}%
  \msg{! #1}%
  \msg{**************** help message *****************}%
  \msg{\the\errhelp}%
  \msg{***********************************************}%
  \xdef\ErrMessage{! #1}%
}
\def\TestInfo{%
  \@PackageInfo\pkgname\msgtext
  \TestWrite\@PackageInfo{-1}{%
    Package PACKAGE Info: MESSAGE^^J%
    (PACKAGE)~~~~~~~~~~~~~SECOND LINE on input line \the\inputlineno.%
  }%
  \@PackageInfoNoLine\pkgname\msgtext
  \TestWrite\@PackageInfoNoLine{-1}{%
    Package PACKAGE Info: MESSAGE^^J%
    (PACKAGE)~~~~~~~~~~~~~SECOND LINE.%
  }%
  \@ClassInfo\clsname\msgtext
  \TestWrite\@ClassInfo{-1}{%
    Class CLASS Info: MESSAGE^^J%
    (CLASS)~~~~~~~~~~~SECOND LINE on input line \the\inputlineno.%
  }%
  \@ClassInfoNoLine\clsname\msgtext
  \TestWrite\@ClassInfoNoLine{-1}{%
    Class CLASS Info: MESSAGE^^J%
    (CLASS)~~~~~~~~~~~SECOND LINE.%
  }%
}
\def\TestWarning{%
  \@PackageWarning\pkgname\msgtext
  \TestWrite\@PackageWarning\WriteUnused{%
    ^^J%
    Package PACKAGE Warning: MESSAGE^^J%
    (PACKAGE)~~~~~~~~~~~~~~~~SECOND LINE on input line \the\inputlineno.%
    ^^J%
  }%
  \@PackageWarningNoLine\pkgname\msgtext
  \TestWrite\@PackageWarningNoLine\WriteUnused{%
    ^^J%
    Package PACKAGE Warning: MESSAGE^^J%
    (PACKAGE)~~~~~~~~~~~~~~~~SECOND LINE.%
    ^^J%
  }%
  \@ClassWarning\clsname\msgtext
  \TestWrite\@ClassWarning\WriteUnused{%
    ^^J%
    Class CLASS Warning: MESSAGE^^J%
    (CLASS)~~~~~~~~~~~~~~SECOND LINE on input line \the\inputlineno.%
    ^^J%
  }%
  \@ClassWarningNoLine\clsname\msgtext
  \TestWrite\@ClassWarningNoLine\WriteUnused{%
    ^^J%
    Class CLASS Warning: MESSAGE^^J%
    (CLASS)~~~~~~~~~~~~~~SECOND LINE.%
    ^^J%
  }%
}
\def\TestError{%
%<test3>  \let\write\OrgWrite
%<test3>  \def\plaindots{}
%<test2>  \def\plaindots{.^^J ...^^J}
  \@PackageError\pkgname\msgtext\hlptext
  \@TestError\@PackageError{%
    ! Package PACKAGE Error: MESSAGE^^J%
    (PACKAGE)~~~~~~~~~~~~~~~~SECOND LINE.^^J%
    ^^J%
    See the PACKAGE package documentation for explanation.^^J%
    Type~~H <return>~~for immediate help\plaindots
  }{%
    HELP^^J%
    SECOND LINE%
  }%
  \@ClassError\clsname\msgtext\hlptext
  \@TestError\@ClassError{%
    ! Class CLASS Error: MESSAGE^^J%
    (CLASS)~~~~~~~~~~~~~~SECOND LINE.^^J%
    ^^J%
    See the CLASS class documentation for explanation.^^J%
    Type~~H <return>~~for immediate help\plaindots
  }{%
    HELP^^J%
    SECOND LINE%
  }%
  \@PackageError\pkgname\msgtext\@ehc
  \@TestError\@PackageError{%
    ! Package PACKAGE Error: MESSAGE^^J%
    (PACKAGE)~~~~~~~~~~~~~~~~SECOND LINE.^^J%
    ^^J%
    See the PACKAGE package documentation for explanation.^^J%
    Type~~H <return>~~for immediate help\plaindots
  }{%
    Try typing~~<return>~~to proceed.^^J%
    If that doesn't work, type~~X <return>~~to quit.%
  }%
  \@PackageError\pkgname\msgtext\@ehd
  \@TestError\@PackageError{%
    ! Package PACKAGE Error: MESSAGE^^J%
    (PACKAGE)~~~~~~~~~~~~~~~~SECOND LINE.^^J%
    ^^J%
    See the PACKAGE package documentation for explanation.^^J%
    Type~~H <return>~~for immediate help\plaindots
  }{%
    You're in trouble here.~~Try typing~~<return>~~to proceed.^^J%
    If that doesn't work, type~~X <return>~~to quit.%
  }%
}
\def\@TestError#1#2#3{%
  \begingroup
    \let~\space
    \edef\x{#2}%
%<test3>  \@onelevel@sanitize\x
%<test3>  \@onelevel@sanitize\ErrMessage
    \ifx\x\ErrMessage
    \else
      \msg{}%
      \msg{[\ErrMessage]}%
      \msg{<>}%
      \msg{[\x]}%
      \TestFailed#1{errmessage text}%
    \fi
    \edef\x{\the\ErrHelpToks}%
    \edef\y{#3}%
    \ifx\x\y
    \else
      \msg{}%
      \msg{[\the\ErrHelpToks]}%
      \msg{<>}%
      \msg{[\y]}%
      \TestFailed#1{errhelp text}%
    \fi
  \endgroup
}
\TestInfo
\TestWarning
\TestError
%</test2|test3>
%<test2>\end
%<test3>\@@end
%    \end{macrocode}
%
%
% \section{Installation}
%
% \subsection{Download}
%
% \paragraph{Package.} This package is available on
% CTAN\footnote{\url{ftp://ftp.ctan.org/tex-archive/}}:
% \begin{description}
% \item[\CTAN{macros/latex/contrib/oberdiek/infwarerr.dtx}] The source file.
% \item[\CTAN{macros/latex/contrib/oberdiek/infwarerr.pdf}] Documentation.
% \end{description}
%
%
% \paragraph{Bundle.} All the packages of the bundle `oberdiek'
% are also available in a TDS compliant ZIP archive. There
% the packages are already unpacked and the documentation files
% are generated. The files and directories obey the TDS standard.
% \begin{description}
% \item[\CTAN{install/macros/latex/contrib/oberdiek.tds.zip}]
% \end{description}
% \emph{TDS} refers to the standard ``A Directory Structure
% for \TeX\ Files'' (\CTAN{tds/tds.pdf}). Directories
% with \xfile{texmf} in their name are usually organized this way.
%
% \subsection{Bundle installation}
%
% \paragraph{Unpacking.} Unpack the \xfile{oberdiek.tds.zip} in the
% TDS tree (also known as \xfile{texmf} tree) of your choice.
% Example (linux):
% \begin{quote}
%   |unzip oberdiek.tds.zip -d ~/texmf|
% \end{quote}
%
% \paragraph{Script installation.}
% Check the directory \xfile{TDS:scripts/oberdiek/} for
% scripts that need further installation steps.
% Package \xpackage{attachfile2} comes with the Perl script
% \xfile{pdfatfi.pl} that should be installed in such a way
% that it can be called as \texttt{pdfatfi}.
% Example (linux):
% \begin{quote}
%   |chmod +x scripts/oberdiek/pdfatfi.pl|\\
%   |cp scripts/oberdiek/pdfatfi.pl /usr/local/bin/|
% \end{quote}
%
% \subsection{Package installation}
%
% \paragraph{Unpacking.} The \xfile{.dtx} file is a self-extracting
% \docstrip\ archive. The files are extracted by running the
% \xfile{.dtx} through \plainTeX:
% \begin{quote}
%   \verb|tex infwarerr.dtx|
% \end{quote}
%
% \paragraph{TDS.} Now the different files must be moved into
% the different directories in your installation TDS tree
% (also known as \xfile{texmf} tree):
% \begin{quote}
% \def\t{^^A
% \begin{tabular}{@{}>{\ttfamily}l@{ $\rightarrow$ }>{\ttfamily}l@{}}
%   infwarerr.sty & tex/generic/oberdiek/infwarerr.sty\\
%   infwarerr.pdf & doc/latex/oberdiek/infwarerr.pdf\\
%   test/infwarerr-test1.tex & doc/latex/oberdiek/test/infwarerr-test1.tex\\
%   test/infwarerr-test2.tex & doc/latex/oberdiek/test/infwarerr-test2.tex\\
%   test/infwarerr-test3.tex & doc/latex/oberdiek/test/infwarerr-test3.tex\\
%   infwarerr.dtx & source/latex/oberdiek/infwarerr.dtx\\
% \end{tabular}^^A
% }^^A
% \sbox0{\t}^^A
% \ifdim\wd0>\linewidth
%   \begingroup
%     \advance\linewidth by\leftmargin
%     \advance\linewidth by\rightmargin
%   \edef\x{\endgroup
%     \def\noexpand\lw{\the\linewidth}^^A
%   }\x
%   \def\lwbox{^^A
%     \leavevmode
%     \hbox to \linewidth{^^A
%       \kern-\leftmargin\relax
%       \hss
%       \usebox0
%       \hss
%       \kern-\rightmargin\relax
%     }^^A
%   }^^A
%   \ifdim\wd0>\lw
%     \sbox0{\small\t}^^A
%     \ifdim\wd0>\linewidth
%       \ifdim\wd0>\lw
%         \sbox0{\footnotesize\t}^^A
%         \ifdim\wd0>\linewidth
%           \ifdim\wd0>\lw
%             \sbox0{\scriptsize\t}^^A
%             \ifdim\wd0>\linewidth
%               \ifdim\wd0>\lw
%                 \sbox0{\tiny\t}^^A
%                 \ifdim\wd0>\linewidth
%                   \lwbox
%                 \else
%                   \usebox0
%                 \fi
%               \else
%                 \lwbox
%               \fi
%             \else
%               \usebox0
%             \fi
%           \else
%             \lwbox
%           \fi
%         \else
%           \usebox0
%         \fi
%       \else
%         \lwbox
%       \fi
%     \else
%       \usebox0
%     \fi
%   \else
%     \lwbox
%   \fi
% \else
%   \usebox0
% \fi
% \end{quote}
% If you have a \xfile{docstrip.cfg} that configures and enables \docstrip's
% TDS installing feature, then some files can already be in the right
% place, see the documentation of \docstrip.
%
% \subsection{Refresh file name databases}
%
% If your \TeX~distribution
% (\teTeX, \mikTeX, \dots) relies on file name databases, you must refresh
% these. For example, \teTeX\ users run \verb|texhash| or
% \verb|mktexlsr|.
%
% \subsection{Some details for the interested}
%
% \paragraph{Attached source.}
%
% The PDF documentation on CTAN also includes the
% \xfile{.dtx} source file. It can be extracted by
% AcrobatReader 6 or higher. Another option is \textsf{pdftk},
% e.g. unpack the file into the current directory:
% \begin{quote}
%   \verb|pdftk infwarerr.pdf unpack_files output .|
% \end{quote}
%
% \paragraph{Unpacking with \LaTeX.}
% The \xfile{.dtx} chooses its action depending on the format:
% \begin{description}
% \item[\plainTeX:] Run \docstrip\ and extract the files.
% \item[\LaTeX:] Generate the documentation.
% \end{description}
% If you insist on using \LaTeX\ for \docstrip\ (really,
% \docstrip\ does not need \LaTeX), then inform the autodetect routine
% about your intention:
% \begin{quote}
%   \verb|latex \let\install=y\input{infwarerr.dtx}|
% \end{quote}
% Do not forget to quote the argument according to the demands
% of your shell.
%
% \paragraph{Generating the documentation.}
% You can use both the \xfile{.dtx} or the \xfile{.drv} to generate
% the documentation. The process can be configured by the
% configuration file \xfile{ltxdoc.cfg}. For instance, put this
% line into this file, if you want to have A4 as paper format:
% \begin{quote}
%   \verb|\PassOptionsToClass{a4paper}{article}|
% \end{quote}
% An example follows how to generate the
% documentation with pdf\LaTeX:
% \begin{quote}
%\begin{verbatim}
%pdflatex infwarerr.dtx
%makeindex -s gind.ist infwarerr.idx
%pdflatex infwarerr.dtx
%makeindex -s gind.ist infwarerr.idx
%pdflatex infwarerr.dtx
%\end{verbatim}
% \end{quote}
%
% \begin{History}
%   \begin{Version}{2007/04/26 v1.0}
%   \item
%     First version.
%   \end{Version}
%   \begin{Version}{2007/06/14 v1.1}
%   \item
%     Small fixes.
%   \end{Version}
%   \begin{Version}{2007/09/09 v1.2}
%   \item
%     Catcode section extended.
%   \end{Version}
%   \begin{Version}{2010/04/08 v1.3}
%   \item
%     Ensure \cs{escapechar}=92 (backslash) for infos/warnings/errors.
%   \end{Version}
% \end{History}
%
% \PrintIndex
%
% \Finale
\endinput
|
% \end{quote}
% Do not forget to quote the argument according to the demands
% of your shell.
%
% \paragraph{Generating the documentation.}
% You can use both the \xfile{.dtx} or the \xfile{.drv} to generate
% the documentation. The process can be configured by the
% configuration file \xfile{ltxdoc.cfg}. For instance, put this
% line into this file, if you want to have A4 as paper format:
% \begin{quote}
%   \verb|\PassOptionsToClass{a4paper}{article}|
% \end{quote}
% An example follows how to generate the
% documentation with pdf\LaTeX:
% \begin{quote}
%\begin{verbatim}
%pdflatex infwarerr.dtx
%makeindex -s gind.ist infwarerr.idx
%pdflatex infwarerr.dtx
%makeindex -s gind.ist infwarerr.idx
%pdflatex infwarerr.dtx
%\end{verbatim}
% \end{quote}
%
% \begin{History}
%   \begin{Version}{2007/04/26 v1.0}
%   \item
%     First version.
%   \end{Version}
%   \begin{Version}{2007/06/14 v1.1}
%   \item
%     Small fixes.
%   \end{Version}
%   \begin{Version}{2007/09/09 v1.2}
%   \item
%     Catcode section extended.
%   \end{Version}
%   \begin{Version}{2010/04/08 v1.3}
%   \item
%     Ensure \cs{escapechar}=92 (backslash) for infos/warnings/errors.
%   \end{Version}
% \end{History}
%
% \PrintIndex
%
% \Finale
\endinput

%        (quote the arguments according to the demands of your shell)
% 
% Documentation:
%    (a) If infwarerr.drv is present:
%           latex infwarerr.drv
%    (b) Without infwarerr.drv:
%           latex infwarerr.dtx; ...
%    The class ltxdoc loads the configuration file ltxdoc.cfg
%    if available. Here you can specify further options, e.g.
%    use A4 as paper format:
%       \PassOptionsToClass{a4paper}{article}
% 
%    Programm calls to get the documentation (example):
%       pdflatex infwarerr.dtx
%       makeindex -s gind.ist infwarerr.idx
%       pdflatex infwarerr.dtx
%       makeindex -s gind.ist infwarerr.idx
%       pdflatex infwarerr.dtx
% 
% Installation:
%    TDS:tex/generic/oberdiek/infwarerr.sty
%    TDS:doc/latex/oberdiek/infwarerr.pdf
%    TDS:doc/latex/oberdiek/test/infwarerr-test1.tex
%    TDS:doc/latex/oberdiek/test/infwarerr-test2.tex
%    TDS:doc/latex/oberdiek/test/infwarerr-test3.tex
%    TDS:source/latex/oberdiek/infwarerr.dtx
%
%<*ignore>
\begingroup
  \catcode123=1 %
  \catcode125=2 %
  \def\x{LaTeX2e}%
\expandafter\endgroup
\ifcase 0\ifx\install y1\fi\expandafter
         \ifx\csname processbatchFile\endcsname\relax\else1\fi
         \ifx\fmtname\x\else 1\fi\relax
\else\csname fi\endcsname
%</ignore>
%<*install>
\input docstrip.tex
\Msg{************************************************************************}
\Msg{* Installation}
\Msg{* Package: infwarerr 2010/04/08 v1.3 Providing info/warning/message (HO)}
\Msg{************************************************************************}

\keepsilent
\askforoverwritefalse

\let\MetaPrefix\relax
\preamble

This is a generated file.

Project: infwarerr
Version: 2010/04/08 v1.3

Copyright (C) 2007, 2010 by
   Heiko Oberdiek <heiko.oberdiek at googlemail.com>

This work may be distributed and/or modified under the
conditions of the LaTeX Project Public License, either
version 1.3c of this license or (at your option) any later
version. This version of this license is in
   http://www.latex-project.org/lppl/lppl-1-3c.txt
and the latest version of this license is in
   http://www.latex-project.org/lppl.txt
and version 1.3 or later is part of all distributions of
LaTeX version 2005/12/01 or later.

This work has the LPPL maintenance status "maintained".

This Current Maintainer of this work is Heiko Oberdiek.

The Base Interpreter refers to any `TeX-Format',
because some files are installed in TDS:tex/generic//.

This work consists of the main source file infwarerr.dtx
and the derived files
   infwarerr.sty, infwarerr.pdf, infwarerr.ins, infwarerr.drv,
   infwarerr-test1.tex, infwarerr-test2.tex,
   infwarerr-test3.tex.

\endpreamble
\let\MetaPrefix\DoubleperCent

\generate{%
  \file{infwarerr.ins}{\from{infwarerr.dtx}{install}}%
  \file{infwarerr.drv}{\from{infwarerr.dtx}{driver}}%
  \usedir{tex/generic/oberdiek}%
  \file{infwarerr.sty}{\from{infwarerr.dtx}{package}}%
  \usedir{doc/latex/oberdiek/test}%
  \file{infwarerr-test1.tex}{\from{infwarerr.dtx}{test1}}%
  \file{infwarerr-test2.tex}{\from{infwarerr.dtx}{test2}}%
  \file{infwarerr-test3.tex}{\from{infwarerr.dtx}{test3}}%
}

\catcode32=13\relax% active space
\let =\space%
\Msg{************************************************************************}
\Msg{*}
\Msg{* To finish the installation you have to move the following}
\Msg{* file into a directory searched by TeX:}
\Msg{*}
\Msg{*     infwarerr.sty}
\Msg{*}
\Msg{* To produce the documentation run the file `infwarerr.drv'}
\Msg{* through LaTeX.}
\Msg{*}
\Msg{* Happy TeXing!}
\Msg{*}
\Msg{************************************************************************}

\endbatchfile
%</install>
%<*ignore>
\fi
%</ignore>
%<*driver>
\NeedsTeXFormat{LaTeX2e}
\ProvidesFile{infwarerr.drv}%
  [2010/04/08 v1.3 Providing info/warning/message (HO)]%
\documentclass{ltxdoc}
\usepackage{holtxdoc}[2011/02/04]
\begin{document}
  \DocInput{infwarerr.dtx}%
\end{document}
%</driver>
% \fi
%
% \CheckSum{871}
%
% \CharacterTable
%  {Upper-case    \A\B\C\D\E\F\G\H\I\J\K\L\M\N\O\P\Q\R\S\T\U\V\W\X\Y\Z
%   Lower-case    \a\b\c\d\e\f\g\h\i\j\k\l\m\n\o\p\q\r\s\t\u\v\w\x\y\z
%   Digits        \0\1\2\3\4\5\6\7\8\9
%   Exclamation   \!     Double quote  \"     Hash (number) \#
%   Dollar        \$     Percent       \%     Ampersand     \&
%   Acute accent  \'     Left paren    \(     Right paren   \)
%   Asterisk      \*     Plus          \+     Comma         \,
%   Minus         \-     Point         \.     Solidus       \/
%   Colon         \:     Semicolon     \;     Less than     \<
%   Equals        \=     Greater than  \>     Question mark \?
%   Commercial at \@     Left bracket  \[     Backslash     \\
%   Right bracket \]     Circumflex    \^     Underscore    \_
%   Grave accent  \`     Left brace    \{     Vertical bar  \|
%   Right brace   \}     Tilde         \~}
%
% \GetFileInfo{infwarerr.drv}
%
% \title{The \xpackage{infwarerr} package}
% \date{2010/04/08 v1.3}
% \author{Heiko Oberdiek\\\xemail{heiko.oberdiek at googlemail.com}}
%
% \maketitle
%
% \begin{abstract}
% This package provides a complete set of macros for informations,
% warnings and error messages with support for \plainTeX.
% \end{abstract}
%
% \tableofcontents
%
% \section{Documentation}
%
% Package \xpackage{infwarerr} defines the following set of commands,
% given as pseudo regular expression:
%   \begin{quote}
%     \verb+\@(Package|Class)(Info|InfoNoLine|Warning|WarningNoLine|Error)+^^A
%     \kern-\rightmargin
%   \end{quote}
% The first argument is the package or class in question. The second
% argument contains the message text. \cs{MessageBreak} is supported.
% In case of error messages, the third argument contains the help
% text. Supported are \LaTeX's message texts \cs{@ehc}:
%   \begin{quote}
%     |Try typing  <return>  to proceed|\\
%     |If that doesn't work, type  X <return>  to quit.|
%   \end{quote}
% and \cs{@ehd}:
%   \begin{quote}
%     |You're in trouble here.  Try typing  <return>  to proceed|\\
%     |If that doesn't work, type  X <return>  to quit.|
%   \end{quote}
%
% The package is intended for macro writers that want to write
% code that works for both \LaTeX\ and \plainTeX. If \LaTeX\ is
% detected, then the macros of this package will only be wrappers
% for \LaTeX's macros. Otherwise \LaTeX's macros will be mimicked
% for use with \plainTeX.
%
% \StopEventually{
% }
%
% \section{Implementation}
%
%    \begin{macrocode}
%<*package>
%    \end{macrocode}
%
% \subsection{Reload check and package identification}
%    Reload check, especially if the package is not used with \LaTeX.
%    \begin{macrocode}
\begingroup\catcode61\catcode48\catcode32=10\relax%
  \catcode13=5 % ^^M
  \endlinechar=13 %
  \catcode35=6 % #
  \catcode39=12 % '
  \catcode44=12 % ,
  \catcode45=12 % -
  \catcode46=12 % .
  \catcode58=12 % :
  \catcode64=11 % @
  \catcode123=1 % {
  \catcode125=2 % }
  \expandafter\let\expandafter\x\csname ver@infwarerr.sty\endcsname
  \ifx\x\relax % plain-TeX, first loading
  \else
    \def\empty{}%
    \ifx\x\empty % LaTeX, first loading,
      % variable is initialized, but \ProvidesPackage not yet seen
    \else
      \expandafter\ifx\csname PackageInfo\endcsname\relax
        \def\x#1#2{%
          \immediate\write-1{Package #1 Info: #2.}%
        }%
      \else
        \def\x#1#2{\PackageInfo{#1}{#2, stopped}}%
      \fi
      \x{infwarerr}{The package is already loaded}%
      \aftergroup\endinput
    \fi
  \fi
\endgroup%
%    \end{macrocode}
%    Package identification:
%    \begin{macrocode}
\begingroup\catcode61\catcode48\catcode32=10\relax%
  \catcode13=5 % ^^M
  \endlinechar=13 %
  \catcode35=6 % #
  \catcode39=12 % '
  \catcode40=12 % (
  \catcode41=12 % )
  \catcode44=12 % ,
  \catcode45=12 % -
  \catcode46=12 % .
  \catcode47=12 % /
  \catcode58=12 % :
  \catcode64=11 % @
  \catcode91=12 % [
  \catcode93=12 % ]
  \catcode123=1 % {
  \catcode125=2 % }
  \expandafter\ifx\csname ProvidesPackage\endcsname\relax
    \def\x#1#2#3[#4]{\endgroup
      \immediate\write-1{Package: #3 #4}%
      \xdef#1{#4}%
    }%
  \else
    \def\x#1#2[#3]{\endgroup
      #2[{#3}]%
      \ifx#1\@undefined
        \xdef#1{#3}%
      \fi
      \ifx#1\relax
        \xdef#1{#3}%
      \fi
    }%
  \fi
\expandafter\x\csname ver@infwarerr.sty\endcsname
\ProvidesPackage{infwarerr}%
  [2010/04/08 v1.3 Providing info/warning/message (HO)]%
%    \end{macrocode}
%
% \subsection{Catcodes}
%
%    \begin{macrocode}
\begingroup\catcode61\catcode48\catcode32=10\relax%
  \catcode13=5 % ^^M
  \endlinechar=13 %
  \catcode123=1 % {
  \catcode125=2 % }
  \catcode64=11 % @
  \def\x{\endgroup
    \expandafter\edef\csname InfWarErr@AtEnd\endcsname{%
      \endlinechar=\the\endlinechar\relax
      \catcode13=\the\catcode13\relax
      \catcode32=\the\catcode32\relax
      \catcode35=\the\catcode35\relax
      \catcode61=\the\catcode61\relax
      \catcode64=\the\catcode64\relax
      \catcode123=\the\catcode123\relax
      \catcode125=\the\catcode125\relax
    }%
  }%
\x\catcode61\catcode48\catcode32=10\relax%
\catcode13=5 % ^^M
\endlinechar=13 %
\catcode35=6 % #
\catcode64=11 % @
\catcode123=1 % {
\catcode125=2 % }
\def\TMP@EnsureCode#1#2{%
  \edef\InfWarErr@AtEnd{%
    \InfWarErr@AtEnd
    \catcode#1=\the\catcode#1\relax
  }%
  \catcode#1=#2\relax
}
\TMP@EnsureCode{10}{12}% ^^J
\TMP@EnsureCode{39}{12}% '
\TMP@EnsureCode{40}{12}% (
\TMP@EnsureCode{41}{12}% )
\TMP@EnsureCode{44}{12}% ,
\TMP@EnsureCode{45}{12}% -
\TMP@EnsureCode{46}{12}% .
\TMP@EnsureCode{58}{12}% :
\TMP@EnsureCode{60}{12}% <
\TMP@EnsureCode{62}{12}% >
\TMP@EnsureCode{94}{7}% ^
\edef\InfWarErr@AtEnd{\InfWarErr@AtEnd\noexpand\endinput}
%    \end{macrocode}
%
% \subsection{\LaTeX\ detection}
%
%    \begin{macrocode}
\begingroup
  \let\x=Y%
  \def\check#1{%
    \@check{Generic}{#1}%
    \@check{Package}{#1}%
    \@check{Class}{#1}%
  }%
  \def\@check#1#2{%
    \expandafter\ifx\csname #1#2\endcsname\relax
      \let\x=N%
    \fi
  }%
  \check{Info}%
  \check{Warning}%
  \check{Error}%
  \@check{Package}{WarningNoLine}%
  \@check{Class}{WarningNoLine}%
  \@check{}{@ehc}%
  \@check{}{@ehd}%
\expandafter\endgroup
\ifx\x Y%
%    \end{macrocode}
%
% \subsection{Version for \LaTeX}
%
%    \begin{macro}{\@PackageInfo}
%    \begin{macrocode}
  \let\@PackageInfo\PackageInfo
%    \end{macrocode}
%    \end{macro}
%    \begin{macro}{\@PackageInfoNoLine}
%    \begin{macrocode}
  \def\@PackageInfoNoLine#1#2{%
    \PackageInfo{#1}{#2\@gobble}%
  }%
%    \end{macrocode}
%    \end{macro}
%    \begin{macro}{\@ClassInfo}
%    \begin{macrocode}
  \let\@ClassInfo\ClassInfo
%    \end{macrocode}
%    \end{macro}
%    \begin{macro}{\@ClassInfoNoLine}
%    \begin{macrocode}
  \def\@ClassInfoNoLine#1#2{%
    \ClassInfo{#1}{#2\@gobble}%
  }%
%    \end{macrocode}
%    \end{macro}
%    \begin{macro}{\@PackageWarning}
%    \begin{macrocode}
  \let\@PackageWarning\PackageWarning
%    \end{macrocode}
%    \end{macro}
%    \begin{macro}{\@PackageWarningNoLine}
%    \begin{macrocode}
  \let\@PackageWarningNoLine\PackageWarningNoLine
%    \end{macrocode}
%    \end{macro}
%    \begin{macro}{\@ClassWarning}
%    \begin{macrocode}
  \let\@ClassWarning\ClassWarning
%    \end{macrocode}
%    \end{macro}
%    \begin{macro}{\@ClassWarningNoLine}
%    \begin{macrocode}
  \let\@ClassWarningNoLine\ClassWarningNoLine
%    \end{macrocode}
%    \end{macro}
%    \begin{macro}{\@PackageError}
%    \begin{macrocode}
  \let\@PackageError\PackageError
%    \end{macrocode}
%    \end{macro}
%    \begin{macro}{\@ClassError}
%    \begin{macrocode}
  \let\@ClassError\ClassError
%    \end{macrocode}
%    \end{macro}
%    \begin{macrocode}
  \expandafter\InfWarErr@AtEnd
\fi%
%    \end{macrocode}
%
% \subsection{Version for \plainTeX}
%
%    \begin{macro}{\InfWarErr@online}
%    \begin{macrocode}
\begingroup\expandafter\endgroup
\ifcase\expandafter\ifx\csname inputlineno\endcsname\relax
         0 %
       \else
         \ifnum\inputlineno<0 %
           0 %
         \else
           1 %
         \fi
       \fi
  \def\InfWarErr@online{}%
\else
  \def\InfWarErr@online{ on input line \the\inputlineno}%
\fi
%    \end{macrocode}
%    \end{macro}
%
%    \begin{macro}{\InfWarErr@protected}
%    \begin{macrocode}
\begingroup\expandafter\expandafter\expandafter\endgroup
\expandafter\ifx\csname protected\endcsname\relax
  \def\InfWarErr@protected{}%
\else
  \let\InfWarErr@protected\protected
\fi
%    \end{macrocode}
%    \end{macro}
%
%    \begin{macro}{\InfWarErr@unused}
%    \begin{macrocode}
\chardef\InfWarErr@unused=16 %
%    \end{macrocode}
%    \end{macro}
%
%    \begin{macro}{\InfWarErr@ehc}
%    \begin{macrocode}
\def\InfWarErr@ehc{%
  Try typing \space <return> \space to proceed.\MessageBreak
  If that doesn't work, type \space X <return> \space to quit.%
}
%    \end{macrocode}
%    \end{macro}
%    \begin{macro}{\InfWarErr@ehd}
%    \begin{macrocode}
\def\InfWarErr@ehd{%
  You're in trouble here. \space\@ehc
}
%    \end{macrocode}
%    \end{macro}
%
%    \begin{macro}{\InfWarErr@GenericInfo}
%    \begin{macrocode}
\InfWarErr@protected\def\InfWarErr@GenericInfo#1#2{%
  \begingroup
    \let\on@line\InfWarErr@online
    \long\def\@gobble##1{}%
    \def\space{ }%
    \def\@spaces{\space\space\space\space}%
    \def\MessageBreak{^^J#1}%
    \newlinechar=10 %
    \escapechar=92 %
    \immediate\write-1{#2\on@line.}%
  \endgroup
}
%    \end{macrocode}
%    \end{macro}
%    \begin{macro}{\InfWarErr@GenericWarning}
%    \begin{macrocode}
\InfWarErr@protected\def\InfWarErr@GenericWarning#1#2{%
  \begingroup
    \let\on@line\InfWarErr@online
    \long\def\@gobble##1{}%
    \def\space{ }%
    \def\@spaces{\space\space\space\space}%
    \def\MessageBreak{^^J#1}%
    \newlinechar=10 %
    \escapechar=92 %
    \immediate\write\InfWarErr@unused{^^J#2\on@line.^^J}%
  \endgroup
}
%    \end{macrocode}
%    \end{macro}
%    \begin{macro}{\InfWarErr@GenericError}
%    Old \TeX\ versions below 3.141 where |^^J| is not working
%    as newline in error message and help texts are not supported.
%    \begin{macrocode}
\InfWarErr@protected\def\InfWarErr@GenericError#1#2#3#4{%
  \begingroup
    \immediate\write\InfWarErr@unused{}%
    \def\space{ }%
    \def\@spaces{\space\space\space\space}%
    \let\@ehc\InfWarErr@ehc
    \let\@ehd\InfWarErr@ehd
    \let\on@line\InfWarErr@online
    \def\MessageBreak{^^J}%
    \newlinechar=10 %
    \escapechar=92 %
    \edef\InfWarErr@help{\noexpand\errhelp{#4}}%
    \InfWarErr@help
    \def\MessageBreak{^^J#1}%
    \errmessage{%
      #2.^^J^^J%
      #3^^J%
      Type \space H <return> \space for immediate help.^^J ...^^J%
    }%
  \endgroup
}
%    \end{macrocode}
%    \end{macro}
%
%    \begin{macro}{\@PackageInfo}
%    \begin{macrocode}
\def\@PackageInfo#1#2{%
  \InfWarErr@GenericInfo{%
    (#1) \@spaces\@spaces\@spaces
  }{%
    Package #1 Info: #2%
  }%
}
%    \end{macrocode}
%    \end{macro}
%    \begin{macro}{\@PackageInfoNoLine}
%    \begin{macrocode}
\def\@PackageInfoNoLine#1#2{%
  \@PackageInfo{#1}{#2\@gobble}%
}
%    \end{macrocode}
%    \end{macro}
%    \begin{macro}{\@ClassInfo}
%    \begin{macrocode}
\def\@ClassInfo#1#2{%
  \InfWarErr@GenericInfo{%
    (#1) \space\space\@spaces\@spaces
  }{%
    Class #1 Info: #2%
  }%
}
%    \end{macrocode}
%    \end{macro}
%    \begin{macro}{\@ClassInfoNoLine}
%    \begin{macrocode}
\def\@ClassInfoNoLine#1#2{%
  \@ClassInfo{#1}{#2\@gobble}%
}
%    \end{macrocode}
%    \end{macro}
%    \begin{macro}{\@PackageWarning}
%    \begin{macrocode}
\def\@PackageWarning#1#2{%
  \InfWarErr@GenericWarning{%
    (#1)\@spaces\@spaces\@spaces\@spaces
  }{%
    Package #1 Warning: #2%
  }%
}
%    \end{macrocode}
%    \end{macro}
%    \begin{macro}{\@PackageWarningNoLine}
%    \begin{macrocode}
\def\@PackageWarningNoLine#1#2{%
  \@PackageWarning{#1}{#2\@gobble}%
}
%    \end{macrocode}
%    \end{macro}
%    \begin{macro}{\@ClassWarning}
%    \begin{macrocode}
\def\@ClassWarning#1#2{%
  \InfWarErr@GenericWarning{%
    (#1) \space\@spaces\@spaces\@spaces
  }{%
    Class #1 Warning: #2%
  }%
}
%    \end{macrocode}
%    \end{macro}
%    \begin{macro}{\@ClassWarningNoLine}
%    \begin{macrocode}
\def\@ClassWarningNoLine#1#2{%
  \@ClassWarning{#1}{#2\@gobble}%
}
%    \end{macrocode}
%    \end{macro}
%    \begin{macro}{\@PackageError}
%    \begin{macrocode}
\def\@PackageError#1#2#3{%
  \InfWarErr@GenericError{%
    (#1)\@spaces\@spaces\@spaces\@spaces
  }{%
    Package #1 Error: #2%
  }{%
    See the #1 package documentation for explanation.%
  }{#3}%
}
%    \end{macrocode}
%    \end{macro}
%    \begin{macro}{\@ClassError}
%    \begin{macrocode}
\def\@ClassError#1#2#3{%
  \InfWarErr@GenericError{%
    (#1) \space\@spaces\@spaces\@spaces
  }{%
    Class #1 Error: #2%
  }{%
    See the #1 class documentation for explanation.%
  }{#3}%
}
%    \end{macrocode}
%    \end{macro}
%
%    \begin{macrocode}
\InfWarErr@AtEnd%
%</package>
%    \end{macrocode}
%
% \section{Test}
%
% \subsection{Catcode checks for loading}
%
%    \begin{macrocode}
%<*test1>
%    \end{macrocode}
%    \begin{macrocode}
\catcode`\{=1 %
\catcode`\}=2 %
\catcode`\#=6 %
\catcode`\@=11 %
\expandafter\ifx\csname count@\endcsname\relax
  \countdef\count@=255 %
\fi
\expandafter\ifx\csname @gobble\endcsname\relax
  \long\def\@gobble#1{}%
\fi
\expandafter\ifx\csname @firstofone\endcsname\relax
  \long\def\@firstofone#1{#1}%
\fi
\expandafter\ifx\csname loop\endcsname\relax
  \expandafter\@firstofone
\else
  \expandafter\@gobble
\fi
{%
  \def\loop#1\repeat{%
    \def\body{#1}%
    \iterate
  }%
  \def\iterate{%
    \body
      \let\next\iterate
    \else
      \let\next\relax
    \fi
    \next
  }%
  \let\repeat=\fi
}%
\def\RestoreCatcodes{}
\count@=0 %
\loop
  \edef\RestoreCatcodes{%
    \RestoreCatcodes
    \catcode\the\count@=\the\catcode\count@\relax
  }%
\ifnum\count@<255 %
  \advance\count@ 1 %
\repeat

\def\RangeCatcodeInvalid#1#2{%
  \count@=#1\relax
  \loop
    \catcode\count@=15 %
  \ifnum\count@<#2\relax
    \advance\count@ 1 %
  \repeat
}
\def\RangeCatcodeCheck#1#2#3{%
  \count@=#1\relax
  \loop
    \ifnum#3=\catcode\count@
    \else
      \errmessage{%
        Character \the\count@\space
        with wrong catcode \the\catcode\count@\space
        instead of \number#3%
      }%
    \fi
  \ifnum\count@<#2\relax
    \advance\count@ 1 %
  \repeat
}
\def\space{ }
\expandafter\ifx\csname LoadCommand\endcsname\relax
  \def\LoadCommand{\input infwarerr.sty\relax}%
\fi
\def\Test{%
  \RangeCatcodeInvalid{0}{47}%
  \RangeCatcodeInvalid{58}{64}%
  \RangeCatcodeInvalid{91}{96}%
  \RangeCatcodeInvalid{123}{255}%
  \catcode`\@=12 %
  \catcode`\\=0 %
  \catcode`\%=14 %
  \LoadCommand
  \RangeCatcodeCheck{0}{36}{15}%
  \RangeCatcodeCheck{37}{37}{14}%
  \RangeCatcodeCheck{38}{47}{15}%
  \RangeCatcodeCheck{48}{57}{12}%
  \RangeCatcodeCheck{58}{63}{15}%
  \RangeCatcodeCheck{64}{64}{12}%
  \RangeCatcodeCheck{65}{90}{11}%
  \RangeCatcodeCheck{91}{91}{15}%
  \RangeCatcodeCheck{92}{92}{0}%
  \RangeCatcodeCheck{93}{96}{15}%
  \RangeCatcodeCheck{97}{122}{11}%
  \RangeCatcodeCheck{123}{255}{15}%
  \RestoreCatcodes
}
\Test
\csname @@end\endcsname
\end
%    \end{macrocode}
%    \begin{macrocode}
%</test1>
%    \end{macrocode}
%
% \subsection{Macro tests}
%
%    \begin{macrocode}
%<*test2>
\let\PackageInfo\relax
\input infwarerr.sty\relax
\let\OrgWrite\write
\chardef\WriteUnused=16 %
\def\msg#{\immediate\OrgWrite\WriteUnused}
\msg{File: infwarerr-test2.tex 2010/04/08 v1.3 Test file for plain-TeX}
\catcode64=11 %
%</test2>
%    \end{macrocode}
%
%    \begin{macrocode}
%<*test3>
\NeedsTeXFormat{LaTeX2e}
\ProvidesFile{infwarerr-test3.tex}[2010/04/08 v1.3 Test file for LaTeX]
\RequirePackage{infwarerr}[2010/04/08]
\makeatletter
\let\OrgWrite\write
\chardef\WriteUnused=\@unused
\def\msg#{\immediate\OrgWrite\WriteUnused}
%</test3>
%    \end{macrocode}
%
%    \begin{macrocode}
%<*test2|test3>
\def\pkgname{PACKAGE}
\def\clsname{CLASS}
\def\msgtext{MESSAGE\MessageBreak SECOND LINE}
\def\hlptext{HELP\MessageBreak SECOND LINE}

\let\OrgWrite\write
\let\OrgErrMessage\errmessage

\def\write#1#{%
  \@write{#1}%
}
\def\@write#1#2{%
  \OrgWrite#1{#2}%
  \xdef\WriteNum{\number#1}%
  \xdef\WriteText{#2}%
}
\def\TestFailed#1#2{%
  \OrgErrMessage{Test (\string#1) failed: #2}%
}
\def\TestWrite#1#2#3{%
  \ifnum\WriteNum=#2\relax
  \else
    \TestFailed#1{write stream number: \WriteNum <> #2}%
  \fi
  \begingroup
    \let~\space
    \edef\TestString{#3}%
    \ifx\WriteText\TestString
    \else
      \msg{}%
      \msg{[\WriteText]}%
      \msg{<>}%
      \msg{[#3]}% hash-ok
      \TestFailed#1{write text}%
    \fi
  \endgroup
}
\newtoks\ErrHelpToks
\def\errmessage#1{%
  \global\ErrHelpToks=\expandafter{\the\errhelp}%
  \msg{**************** error message ****************}%
  \msg{! #1}%
  \msg{**************** help message *****************}%
  \msg{\the\errhelp}%
  \msg{***********************************************}%
  \xdef\ErrMessage{! #1}%
}
\def\TestInfo{%
  \@PackageInfo\pkgname\msgtext
  \TestWrite\@PackageInfo{-1}{%
    Package PACKAGE Info: MESSAGE^^J%
    (PACKAGE)~~~~~~~~~~~~~SECOND LINE on input line \the\inputlineno.%
  }%
  \@PackageInfoNoLine\pkgname\msgtext
  \TestWrite\@PackageInfoNoLine{-1}{%
    Package PACKAGE Info: MESSAGE^^J%
    (PACKAGE)~~~~~~~~~~~~~SECOND LINE.%
  }%
  \@ClassInfo\clsname\msgtext
  \TestWrite\@ClassInfo{-1}{%
    Class CLASS Info: MESSAGE^^J%
    (CLASS)~~~~~~~~~~~SECOND LINE on input line \the\inputlineno.%
  }%
  \@ClassInfoNoLine\clsname\msgtext
  \TestWrite\@ClassInfoNoLine{-1}{%
    Class CLASS Info: MESSAGE^^J%
    (CLASS)~~~~~~~~~~~SECOND LINE.%
  }%
}
\def\TestWarning{%
  \@PackageWarning\pkgname\msgtext
  \TestWrite\@PackageWarning\WriteUnused{%
    ^^J%
    Package PACKAGE Warning: MESSAGE^^J%
    (PACKAGE)~~~~~~~~~~~~~~~~SECOND LINE on input line \the\inputlineno.%
    ^^J%
  }%
  \@PackageWarningNoLine\pkgname\msgtext
  \TestWrite\@PackageWarningNoLine\WriteUnused{%
    ^^J%
    Package PACKAGE Warning: MESSAGE^^J%
    (PACKAGE)~~~~~~~~~~~~~~~~SECOND LINE.%
    ^^J%
  }%
  \@ClassWarning\clsname\msgtext
  \TestWrite\@ClassWarning\WriteUnused{%
    ^^J%
    Class CLASS Warning: MESSAGE^^J%
    (CLASS)~~~~~~~~~~~~~~SECOND LINE on input line \the\inputlineno.%
    ^^J%
  }%
  \@ClassWarningNoLine\clsname\msgtext
  \TestWrite\@ClassWarningNoLine\WriteUnused{%
    ^^J%
    Class CLASS Warning: MESSAGE^^J%
    (CLASS)~~~~~~~~~~~~~~SECOND LINE.%
    ^^J%
  }%
}
\def\TestError{%
%<test3>  \let\write\OrgWrite
%<test3>  \def\plaindots{}
%<test2>  \def\plaindots{.^^J ...^^J}
  \@PackageError\pkgname\msgtext\hlptext
  \@TestError\@PackageError{%
    ! Package PACKAGE Error: MESSAGE^^J%
    (PACKAGE)~~~~~~~~~~~~~~~~SECOND LINE.^^J%
    ^^J%
    See the PACKAGE package documentation for explanation.^^J%
    Type~~H <return>~~for immediate help\plaindots
  }{%
    HELP^^J%
    SECOND LINE%
  }%
  \@ClassError\clsname\msgtext\hlptext
  \@TestError\@ClassError{%
    ! Class CLASS Error: MESSAGE^^J%
    (CLASS)~~~~~~~~~~~~~~SECOND LINE.^^J%
    ^^J%
    See the CLASS class documentation for explanation.^^J%
    Type~~H <return>~~for immediate help\plaindots
  }{%
    HELP^^J%
    SECOND LINE%
  }%
  \@PackageError\pkgname\msgtext\@ehc
  \@TestError\@PackageError{%
    ! Package PACKAGE Error: MESSAGE^^J%
    (PACKAGE)~~~~~~~~~~~~~~~~SECOND LINE.^^J%
    ^^J%
    See the PACKAGE package documentation for explanation.^^J%
    Type~~H <return>~~for immediate help\plaindots
  }{%
    Try typing~~<return>~~to proceed.^^J%
    If that doesn't work, type~~X <return>~~to quit.%
  }%
  \@PackageError\pkgname\msgtext\@ehd
  \@TestError\@PackageError{%
    ! Package PACKAGE Error: MESSAGE^^J%
    (PACKAGE)~~~~~~~~~~~~~~~~SECOND LINE.^^J%
    ^^J%
    See the PACKAGE package documentation for explanation.^^J%
    Type~~H <return>~~for immediate help\plaindots
  }{%
    You're in trouble here.~~Try typing~~<return>~~to proceed.^^J%
    If that doesn't work, type~~X <return>~~to quit.%
  }%
}
\def\@TestError#1#2#3{%
  \begingroup
    \let~\space
    \edef\x{#2}%
%<test3>  \@onelevel@sanitize\x
%<test3>  \@onelevel@sanitize\ErrMessage
    \ifx\x\ErrMessage
    \else
      \msg{}%
      \msg{[\ErrMessage]}%
      \msg{<>}%
      \msg{[\x]}%
      \TestFailed#1{errmessage text}%
    \fi
    \edef\x{\the\ErrHelpToks}%
    \edef\y{#3}%
    \ifx\x\y
    \else
      \msg{}%
      \msg{[\the\ErrHelpToks]}%
      \msg{<>}%
      \msg{[\y]}%
      \TestFailed#1{errhelp text}%
    \fi
  \endgroup
}
\TestInfo
\TestWarning
\TestError
%</test2|test3>
%<test2>\end
%<test3>\@@end
%    \end{macrocode}
%
%
% \section{Installation}
%
% \subsection{Download}
%
% \paragraph{Package.} This package is available on
% CTAN\footnote{\url{ftp://ftp.ctan.org/tex-archive/}}:
% \begin{description}
% \item[\CTAN{macros/latex/contrib/oberdiek/infwarerr.dtx}] The source file.
% \item[\CTAN{macros/latex/contrib/oberdiek/infwarerr.pdf}] Documentation.
% \end{description}
%
%
% \paragraph{Bundle.} All the packages of the bundle `oberdiek'
% are also available in a TDS compliant ZIP archive. There
% the packages are already unpacked and the documentation files
% are generated. The files and directories obey the TDS standard.
% \begin{description}
% \item[\CTAN{install/macros/latex/contrib/oberdiek.tds.zip}]
% \end{description}
% \emph{TDS} refers to the standard ``A Directory Structure
% for \TeX\ Files'' (\CTAN{tds/tds.pdf}). Directories
% with \xfile{texmf} in their name are usually organized this way.
%
% \subsection{Bundle installation}
%
% \paragraph{Unpacking.} Unpack the \xfile{oberdiek.tds.zip} in the
% TDS tree (also known as \xfile{texmf} tree) of your choice.
% Example (linux):
% \begin{quote}
%   |unzip oberdiek.tds.zip -d ~/texmf|
% \end{quote}
%
% \paragraph{Script installation.}
% Check the directory \xfile{TDS:scripts/oberdiek/} for
% scripts that need further installation steps.
% Package \xpackage{attachfile2} comes with the Perl script
% \xfile{pdfatfi.pl} that should be installed in such a way
% that it can be called as \texttt{pdfatfi}.
% Example (linux):
% \begin{quote}
%   |chmod +x scripts/oberdiek/pdfatfi.pl|\\
%   |cp scripts/oberdiek/pdfatfi.pl /usr/local/bin/|
% \end{quote}
%
% \subsection{Package installation}
%
% \paragraph{Unpacking.} The \xfile{.dtx} file is a self-extracting
% \docstrip\ archive. The files are extracted by running the
% \xfile{.dtx} through \plainTeX:
% \begin{quote}
%   \verb|tex infwarerr.dtx|
% \end{quote}
%
% \paragraph{TDS.} Now the different files must be moved into
% the different directories in your installation TDS tree
% (also known as \xfile{texmf} tree):
% \begin{quote}
% \def\t{^^A
% \begin{tabular}{@{}>{\ttfamily}l@{ $\rightarrow$ }>{\ttfamily}l@{}}
%   infwarerr.sty & tex/generic/oberdiek/infwarerr.sty\\
%   infwarerr.pdf & doc/latex/oberdiek/infwarerr.pdf\\
%   test/infwarerr-test1.tex & doc/latex/oberdiek/test/infwarerr-test1.tex\\
%   test/infwarerr-test2.tex & doc/latex/oberdiek/test/infwarerr-test2.tex\\
%   test/infwarerr-test3.tex & doc/latex/oberdiek/test/infwarerr-test3.tex\\
%   infwarerr.dtx & source/latex/oberdiek/infwarerr.dtx\\
% \end{tabular}^^A
% }^^A
% \sbox0{\t}^^A
% \ifdim\wd0>\linewidth
%   \begingroup
%     \advance\linewidth by\leftmargin
%     \advance\linewidth by\rightmargin
%   \edef\x{\endgroup
%     \def\noexpand\lw{\the\linewidth}^^A
%   }\x
%   \def\lwbox{^^A
%     \leavevmode
%     \hbox to \linewidth{^^A
%       \kern-\leftmargin\relax
%       \hss
%       \usebox0
%       \hss
%       \kern-\rightmargin\relax
%     }^^A
%   }^^A
%   \ifdim\wd0>\lw
%     \sbox0{\small\t}^^A
%     \ifdim\wd0>\linewidth
%       \ifdim\wd0>\lw
%         \sbox0{\footnotesize\t}^^A
%         \ifdim\wd0>\linewidth
%           \ifdim\wd0>\lw
%             \sbox0{\scriptsize\t}^^A
%             \ifdim\wd0>\linewidth
%               \ifdim\wd0>\lw
%                 \sbox0{\tiny\t}^^A
%                 \ifdim\wd0>\linewidth
%                   \lwbox
%                 \else
%                   \usebox0
%                 \fi
%               \else
%                 \lwbox
%               \fi
%             \else
%               \usebox0
%             \fi
%           \else
%             \lwbox
%           \fi
%         \else
%           \usebox0
%         \fi
%       \else
%         \lwbox
%       \fi
%     \else
%       \usebox0
%     \fi
%   \else
%     \lwbox
%   \fi
% \else
%   \usebox0
% \fi
% \end{quote}
% If you have a \xfile{docstrip.cfg} that configures and enables \docstrip's
% TDS installing feature, then some files can already be in the right
% place, see the documentation of \docstrip.
%
% \subsection{Refresh file name databases}
%
% If your \TeX~distribution
% (\teTeX, \mikTeX, \dots) relies on file name databases, you must refresh
% these. For example, \teTeX\ users run \verb|texhash| or
% \verb|mktexlsr|.
%
% \subsection{Some details for the interested}
%
% \paragraph{Attached source.}
%
% The PDF documentation on CTAN also includes the
% \xfile{.dtx} source file. It can be extracted by
% AcrobatReader 6 or higher. Another option is \textsf{pdftk},
% e.g. unpack the file into the current directory:
% \begin{quote}
%   \verb|pdftk infwarerr.pdf unpack_files output .|
% \end{quote}
%
% \paragraph{Unpacking with \LaTeX.}
% The \xfile{.dtx} chooses its action depending on the format:
% \begin{description}
% \item[\plainTeX:] Run \docstrip\ and extract the files.
% \item[\LaTeX:] Generate the documentation.
% \end{description}
% If you insist on using \LaTeX\ for \docstrip\ (really,
% \docstrip\ does not need \LaTeX), then inform the autodetect routine
% about your intention:
% \begin{quote}
%   \verb|latex \let\install=y% \iffalse meta-comment
% 
% File: infwarerr.dtx
% Version: 2010/04/08 v1.3
% 
% Copyright (C) 2007, 2010 by
%    Heiko Oberdiek <heiko.oberdiek at googlemail.com>
% 
% This work may be distributed and/or modified under the
% conditions of the LaTeX Project Public License, either
% version 1.3c of this license or (at your option) any later
% version. This version of this license is in
%    http://www.latex-project.org/lppl/lppl-1-3c.txt
% and the latest version of this license is in
%    http://www.latex-project.org/lppl.txt
% and version 1.3 or later is part of all distributions of
% LaTeX version 2005/12/01 or later.
% 
% This work has the LPPL maintenance status "maintained".
% 
% This Current Maintainer of this work is Heiko Oberdiek.
% 
% The Base Interpreter refers to any `TeX-Format',
% because some files are installed in TDS:tex/generic//.
% 
% This work consists of the main source file infwarerr.dtx
% and the derived files
%    infwarerr.sty, infwarerr.pdf, infwarerr.ins, infwarerr.drv,
%    infwarerr-test1.tex, infwarerr-test2.tex,
%    infwarerr-test3.tex.
% 
% Distribution:
%    CTAN:macros/latex/contrib/oberdiek/infwarerr.dtx
%    CTAN:macros/latex/contrib/oberdiek/infwarerr.pdf
% 
% Unpacking:
%    (a) If infwarerr.ins is present:
%           tex infwarerr.ins
%    (b) Without infwarerr.ins:
%           tex infwarerr.dtx
%    (c) If you insist on using LaTeX
%           latex \let\install=y% \iffalse meta-comment
% 
% File: infwarerr.dtx
% Version: 2010/04/08 v1.3
% 
% Copyright (C) 2007, 2010 by
%    Heiko Oberdiek <heiko.oberdiek at googlemail.com>
% 
% This work may be distributed and/or modified under the
% conditions of the LaTeX Project Public License, either
% version 1.3c of this license or (at your option) any later
% version. This version of this license is in
%    http://www.latex-project.org/lppl/lppl-1-3c.txt
% and the latest version of this license is in
%    http://www.latex-project.org/lppl.txt
% and version 1.3 or later is part of all distributions of
% LaTeX version 2005/12/01 or later.
% 
% This work has the LPPL maintenance status "maintained".
% 
% This Current Maintainer of this work is Heiko Oberdiek.
% 
% The Base Interpreter refers to any `TeX-Format',
% because some files are installed in TDS:tex/generic//.
% 
% This work consists of the main source file infwarerr.dtx
% and the derived files
%    infwarerr.sty, infwarerr.pdf, infwarerr.ins, infwarerr.drv,
%    infwarerr-test1.tex, infwarerr-test2.tex,
%    infwarerr-test3.tex.
% 
% Distribution:
%    CTAN:macros/latex/contrib/oberdiek/infwarerr.dtx
%    CTAN:macros/latex/contrib/oberdiek/infwarerr.pdf
% 
% Unpacking:
%    (a) If infwarerr.ins is present:
%           tex infwarerr.ins
%    (b) Without infwarerr.ins:
%           tex infwarerr.dtx
%    (c) If you insist on using LaTeX
%           latex \let\install=y\input{infwarerr.dtx}
%        (quote the arguments according to the demands of your shell)
% 
% Documentation:
%    (a) If infwarerr.drv is present:
%           latex infwarerr.drv
%    (b) Without infwarerr.drv:
%           latex infwarerr.dtx; ...
%    The class ltxdoc loads the configuration file ltxdoc.cfg
%    if available. Here you can specify further options, e.g.
%    use A4 as paper format:
%       \PassOptionsToClass{a4paper}{article}
% 
%    Programm calls to get the documentation (example):
%       pdflatex infwarerr.dtx
%       makeindex -s gind.ist infwarerr.idx
%       pdflatex infwarerr.dtx
%       makeindex -s gind.ist infwarerr.idx
%       pdflatex infwarerr.dtx
% 
% Installation:
%    TDS:tex/generic/oberdiek/infwarerr.sty
%    TDS:doc/latex/oberdiek/infwarerr.pdf
%    TDS:doc/latex/oberdiek/test/infwarerr-test1.tex
%    TDS:doc/latex/oberdiek/test/infwarerr-test2.tex
%    TDS:doc/latex/oberdiek/test/infwarerr-test3.tex
%    TDS:source/latex/oberdiek/infwarerr.dtx
%
%<*ignore>
\begingroup
  \catcode123=1 %
  \catcode125=2 %
  \def\x{LaTeX2e}%
\expandafter\endgroup
\ifcase 0\ifx\install y1\fi\expandafter
         \ifx\csname processbatchFile\endcsname\relax\else1\fi
         \ifx\fmtname\x\else 1\fi\relax
\else\csname fi\endcsname
%</ignore>
%<*install>
\input docstrip.tex
\Msg{************************************************************************}
\Msg{* Installation}
\Msg{* Package: infwarerr 2010/04/08 v1.3 Providing info/warning/message (HO)}
\Msg{************************************************************************}

\keepsilent
\askforoverwritefalse

\let\MetaPrefix\relax
\preamble

This is a generated file.

Project: infwarerr
Version: 2010/04/08 v1.3

Copyright (C) 2007, 2010 by
   Heiko Oberdiek <heiko.oberdiek at googlemail.com>

This work may be distributed and/or modified under the
conditions of the LaTeX Project Public License, either
version 1.3c of this license or (at your option) any later
version. This version of this license is in
   http://www.latex-project.org/lppl/lppl-1-3c.txt
and the latest version of this license is in
   http://www.latex-project.org/lppl.txt
and version 1.3 or later is part of all distributions of
LaTeX version 2005/12/01 or later.

This work has the LPPL maintenance status "maintained".

This Current Maintainer of this work is Heiko Oberdiek.

The Base Interpreter refers to any `TeX-Format',
because some files are installed in TDS:tex/generic//.

This work consists of the main source file infwarerr.dtx
and the derived files
   infwarerr.sty, infwarerr.pdf, infwarerr.ins, infwarerr.drv,
   infwarerr-test1.tex, infwarerr-test2.tex,
   infwarerr-test3.tex.

\endpreamble
\let\MetaPrefix\DoubleperCent

\generate{%
  \file{infwarerr.ins}{\from{infwarerr.dtx}{install}}%
  \file{infwarerr.drv}{\from{infwarerr.dtx}{driver}}%
  \usedir{tex/generic/oberdiek}%
  \file{infwarerr.sty}{\from{infwarerr.dtx}{package}}%
  \usedir{doc/latex/oberdiek/test}%
  \file{infwarerr-test1.tex}{\from{infwarerr.dtx}{test1}}%
  \file{infwarerr-test2.tex}{\from{infwarerr.dtx}{test2}}%
  \file{infwarerr-test3.tex}{\from{infwarerr.dtx}{test3}}%
}

\catcode32=13\relax% active space
\let =\space%
\Msg{************************************************************************}
\Msg{*}
\Msg{* To finish the installation you have to move the following}
\Msg{* file into a directory searched by TeX:}
\Msg{*}
\Msg{*     infwarerr.sty}
\Msg{*}
\Msg{* To produce the documentation run the file `infwarerr.drv'}
\Msg{* through LaTeX.}
\Msg{*}
\Msg{* Happy TeXing!}
\Msg{*}
\Msg{************************************************************************}

\endbatchfile
%</install>
%<*ignore>
\fi
%</ignore>
%<*driver>
\NeedsTeXFormat{LaTeX2e}
\ProvidesFile{infwarerr.drv}%
  [2010/04/08 v1.3 Providing info/warning/message (HO)]%
\documentclass{ltxdoc}
\usepackage{holtxdoc}[2011/02/04]
\begin{document}
  \DocInput{infwarerr.dtx}%
\end{document}
%</driver>
% \fi
%
% \CheckSum{871}
%
% \CharacterTable
%  {Upper-case    \A\B\C\D\E\F\G\H\I\J\K\L\M\N\O\P\Q\R\S\T\U\V\W\X\Y\Z
%   Lower-case    \a\b\c\d\e\f\g\h\i\j\k\l\m\n\o\p\q\r\s\t\u\v\w\x\y\z
%   Digits        \0\1\2\3\4\5\6\7\8\9
%   Exclamation   \!     Double quote  \"     Hash (number) \#
%   Dollar        \$     Percent       \%     Ampersand     \&
%   Acute accent  \'     Left paren    \(     Right paren   \)
%   Asterisk      \*     Plus          \+     Comma         \,
%   Minus         \-     Point         \.     Solidus       \/
%   Colon         \:     Semicolon     \;     Less than     \<
%   Equals        \=     Greater than  \>     Question mark \?
%   Commercial at \@     Left bracket  \[     Backslash     \\
%   Right bracket \]     Circumflex    \^     Underscore    \_
%   Grave accent  \`     Left brace    \{     Vertical bar  \|
%   Right brace   \}     Tilde         \~}
%
% \GetFileInfo{infwarerr.drv}
%
% \title{The \xpackage{infwarerr} package}
% \date{2010/04/08 v1.3}
% \author{Heiko Oberdiek\\\xemail{heiko.oberdiek at googlemail.com}}
%
% \maketitle
%
% \begin{abstract}
% This package provides a complete set of macros for informations,
% warnings and error messages with support for \plainTeX.
% \end{abstract}
%
% \tableofcontents
%
% \section{Documentation}
%
% Package \xpackage{infwarerr} defines the following set of commands,
% given as pseudo regular expression:
%   \begin{quote}
%     \verb+\@(Package|Class)(Info|InfoNoLine|Warning|WarningNoLine|Error)+^^A
%     \kern-\rightmargin
%   \end{quote}
% The first argument is the package or class in question. The second
% argument contains the message text. \cs{MessageBreak} is supported.
% In case of error messages, the third argument contains the help
% text. Supported are \LaTeX's message texts \cs{@ehc}:
%   \begin{quote}
%     |Try typing  <return>  to proceed|\\
%     |If that doesn't work, type  X <return>  to quit.|
%   \end{quote}
% and \cs{@ehd}:
%   \begin{quote}
%     |You're in trouble here.  Try typing  <return>  to proceed|\\
%     |If that doesn't work, type  X <return>  to quit.|
%   \end{quote}
%
% The package is intended for macro writers that want to write
% code that works for both \LaTeX\ and \plainTeX. If \LaTeX\ is
% detected, then the macros of this package will only be wrappers
% for \LaTeX's macros. Otherwise \LaTeX's macros will be mimicked
% for use with \plainTeX.
%
% \StopEventually{
% }
%
% \section{Implementation}
%
%    \begin{macrocode}
%<*package>
%    \end{macrocode}
%
% \subsection{Reload check and package identification}
%    Reload check, especially if the package is not used with \LaTeX.
%    \begin{macrocode}
\begingroup\catcode61\catcode48\catcode32=10\relax%
  \catcode13=5 % ^^M
  \endlinechar=13 %
  \catcode35=6 % #
  \catcode39=12 % '
  \catcode44=12 % ,
  \catcode45=12 % -
  \catcode46=12 % .
  \catcode58=12 % :
  \catcode64=11 % @
  \catcode123=1 % {
  \catcode125=2 % }
  \expandafter\let\expandafter\x\csname ver@infwarerr.sty\endcsname
  \ifx\x\relax % plain-TeX, first loading
  \else
    \def\empty{}%
    \ifx\x\empty % LaTeX, first loading,
      % variable is initialized, but \ProvidesPackage not yet seen
    \else
      \expandafter\ifx\csname PackageInfo\endcsname\relax
        \def\x#1#2{%
          \immediate\write-1{Package #1 Info: #2.}%
        }%
      \else
        \def\x#1#2{\PackageInfo{#1}{#2, stopped}}%
      \fi
      \x{infwarerr}{The package is already loaded}%
      \aftergroup\endinput
    \fi
  \fi
\endgroup%
%    \end{macrocode}
%    Package identification:
%    \begin{macrocode}
\begingroup\catcode61\catcode48\catcode32=10\relax%
  \catcode13=5 % ^^M
  \endlinechar=13 %
  \catcode35=6 % #
  \catcode39=12 % '
  \catcode40=12 % (
  \catcode41=12 % )
  \catcode44=12 % ,
  \catcode45=12 % -
  \catcode46=12 % .
  \catcode47=12 % /
  \catcode58=12 % :
  \catcode64=11 % @
  \catcode91=12 % [
  \catcode93=12 % ]
  \catcode123=1 % {
  \catcode125=2 % }
  \expandafter\ifx\csname ProvidesPackage\endcsname\relax
    \def\x#1#2#3[#4]{\endgroup
      \immediate\write-1{Package: #3 #4}%
      \xdef#1{#4}%
    }%
  \else
    \def\x#1#2[#3]{\endgroup
      #2[{#3}]%
      \ifx#1\@undefined
        \xdef#1{#3}%
      \fi
      \ifx#1\relax
        \xdef#1{#3}%
      \fi
    }%
  \fi
\expandafter\x\csname ver@infwarerr.sty\endcsname
\ProvidesPackage{infwarerr}%
  [2010/04/08 v1.3 Providing info/warning/message (HO)]%
%    \end{macrocode}
%
% \subsection{Catcodes}
%
%    \begin{macrocode}
\begingroup\catcode61\catcode48\catcode32=10\relax%
  \catcode13=5 % ^^M
  \endlinechar=13 %
  \catcode123=1 % {
  \catcode125=2 % }
  \catcode64=11 % @
  \def\x{\endgroup
    \expandafter\edef\csname InfWarErr@AtEnd\endcsname{%
      \endlinechar=\the\endlinechar\relax
      \catcode13=\the\catcode13\relax
      \catcode32=\the\catcode32\relax
      \catcode35=\the\catcode35\relax
      \catcode61=\the\catcode61\relax
      \catcode64=\the\catcode64\relax
      \catcode123=\the\catcode123\relax
      \catcode125=\the\catcode125\relax
    }%
  }%
\x\catcode61\catcode48\catcode32=10\relax%
\catcode13=5 % ^^M
\endlinechar=13 %
\catcode35=6 % #
\catcode64=11 % @
\catcode123=1 % {
\catcode125=2 % }
\def\TMP@EnsureCode#1#2{%
  \edef\InfWarErr@AtEnd{%
    \InfWarErr@AtEnd
    \catcode#1=\the\catcode#1\relax
  }%
  \catcode#1=#2\relax
}
\TMP@EnsureCode{10}{12}% ^^J
\TMP@EnsureCode{39}{12}% '
\TMP@EnsureCode{40}{12}% (
\TMP@EnsureCode{41}{12}% )
\TMP@EnsureCode{44}{12}% ,
\TMP@EnsureCode{45}{12}% -
\TMP@EnsureCode{46}{12}% .
\TMP@EnsureCode{58}{12}% :
\TMP@EnsureCode{60}{12}% <
\TMP@EnsureCode{62}{12}% >
\TMP@EnsureCode{94}{7}% ^
\edef\InfWarErr@AtEnd{\InfWarErr@AtEnd\noexpand\endinput}
%    \end{macrocode}
%
% \subsection{\LaTeX\ detection}
%
%    \begin{macrocode}
\begingroup
  \let\x=Y%
  \def\check#1{%
    \@check{Generic}{#1}%
    \@check{Package}{#1}%
    \@check{Class}{#1}%
  }%
  \def\@check#1#2{%
    \expandafter\ifx\csname #1#2\endcsname\relax
      \let\x=N%
    \fi
  }%
  \check{Info}%
  \check{Warning}%
  \check{Error}%
  \@check{Package}{WarningNoLine}%
  \@check{Class}{WarningNoLine}%
  \@check{}{@ehc}%
  \@check{}{@ehd}%
\expandafter\endgroup
\ifx\x Y%
%    \end{macrocode}
%
% \subsection{Version for \LaTeX}
%
%    \begin{macro}{\@PackageInfo}
%    \begin{macrocode}
  \let\@PackageInfo\PackageInfo
%    \end{macrocode}
%    \end{macro}
%    \begin{macro}{\@PackageInfoNoLine}
%    \begin{macrocode}
  \def\@PackageInfoNoLine#1#2{%
    \PackageInfo{#1}{#2\@gobble}%
  }%
%    \end{macrocode}
%    \end{macro}
%    \begin{macro}{\@ClassInfo}
%    \begin{macrocode}
  \let\@ClassInfo\ClassInfo
%    \end{macrocode}
%    \end{macro}
%    \begin{macro}{\@ClassInfoNoLine}
%    \begin{macrocode}
  \def\@ClassInfoNoLine#1#2{%
    \ClassInfo{#1}{#2\@gobble}%
  }%
%    \end{macrocode}
%    \end{macro}
%    \begin{macro}{\@PackageWarning}
%    \begin{macrocode}
  \let\@PackageWarning\PackageWarning
%    \end{macrocode}
%    \end{macro}
%    \begin{macro}{\@PackageWarningNoLine}
%    \begin{macrocode}
  \let\@PackageWarningNoLine\PackageWarningNoLine
%    \end{macrocode}
%    \end{macro}
%    \begin{macro}{\@ClassWarning}
%    \begin{macrocode}
  \let\@ClassWarning\ClassWarning
%    \end{macrocode}
%    \end{macro}
%    \begin{macro}{\@ClassWarningNoLine}
%    \begin{macrocode}
  \let\@ClassWarningNoLine\ClassWarningNoLine
%    \end{macrocode}
%    \end{macro}
%    \begin{macro}{\@PackageError}
%    \begin{macrocode}
  \let\@PackageError\PackageError
%    \end{macrocode}
%    \end{macro}
%    \begin{macro}{\@ClassError}
%    \begin{macrocode}
  \let\@ClassError\ClassError
%    \end{macrocode}
%    \end{macro}
%    \begin{macrocode}
  \expandafter\InfWarErr@AtEnd
\fi%
%    \end{macrocode}
%
% \subsection{Version for \plainTeX}
%
%    \begin{macro}{\InfWarErr@online}
%    \begin{macrocode}
\begingroup\expandafter\endgroup
\ifcase\expandafter\ifx\csname inputlineno\endcsname\relax
         0 %
       \else
         \ifnum\inputlineno<0 %
           0 %
         \else
           1 %
         \fi
       \fi
  \def\InfWarErr@online{}%
\else
  \def\InfWarErr@online{ on input line \the\inputlineno}%
\fi
%    \end{macrocode}
%    \end{macro}
%
%    \begin{macro}{\InfWarErr@protected}
%    \begin{macrocode}
\begingroup\expandafter\expandafter\expandafter\endgroup
\expandafter\ifx\csname protected\endcsname\relax
  \def\InfWarErr@protected{}%
\else
  \let\InfWarErr@protected\protected
\fi
%    \end{macrocode}
%    \end{macro}
%
%    \begin{macro}{\InfWarErr@unused}
%    \begin{macrocode}
\chardef\InfWarErr@unused=16 %
%    \end{macrocode}
%    \end{macro}
%
%    \begin{macro}{\InfWarErr@ehc}
%    \begin{macrocode}
\def\InfWarErr@ehc{%
  Try typing \space <return> \space to proceed.\MessageBreak
  If that doesn't work, type \space X <return> \space to quit.%
}
%    \end{macrocode}
%    \end{macro}
%    \begin{macro}{\InfWarErr@ehd}
%    \begin{macrocode}
\def\InfWarErr@ehd{%
  You're in trouble here. \space\@ehc
}
%    \end{macrocode}
%    \end{macro}
%
%    \begin{macro}{\InfWarErr@GenericInfo}
%    \begin{macrocode}
\InfWarErr@protected\def\InfWarErr@GenericInfo#1#2{%
  \begingroup
    \let\on@line\InfWarErr@online
    \long\def\@gobble##1{}%
    \def\space{ }%
    \def\@spaces{\space\space\space\space}%
    \def\MessageBreak{^^J#1}%
    \newlinechar=10 %
    \escapechar=92 %
    \immediate\write-1{#2\on@line.}%
  \endgroup
}
%    \end{macrocode}
%    \end{macro}
%    \begin{macro}{\InfWarErr@GenericWarning}
%    \begin{macrocode}
\InfWarErr@protected\def\InfWarErr@GenericWarning#1#2{%
  \begingroup
    \let\on@line\InfWarErr@online
    \long\def\@gobble##1{}%
    \def\space{ }%
    \def\@spaces{\space\space\space\space}%
    \def\MessageBreak{^^J#1}%
    \newlinechar=10 %
    \escapechar=92 %
    \immediate\write\InfWarErr@unused{^^J#2\on@line.^^J}%
  \endgroup
}
%    \end{macrocode}
%    \end{macro}
%    \begin{macro}{\InfWarErr@GenericError}
%    Old \TeX\ versions below 3.141 where |^^J| is not working
%    as newline in error message and help texts are not supported.
%    \begin{macrocode}
\InfWarErr@protected\def\InfWarErr@GenericError#1#2#3#4{%
  \begingroup
    \immediate\write\InfWarErr@unused{}%
    \def\space{ }%
    \def\@spaces{\space\space\space\space}%
    \let\@ehc\InfWarErr@ehc
    \let\@ehd\InfWarErr@ehd
    \let\on@line\InfWarErr@online
    \def\MessageBreak{^^J}%
    \newlinechar=10 %
    \escapechar=92 %
    \edef\InfWarErr@help{\noexpand\errhelp{#4}}%
    \InfWarErr@help
    \def\MessageBreak{^^J#1}%
    \errmessage{%
      #2.^^J^^J%
      #3^^J%
      Type \space H <return> \space for immediate help.^^J ...^^J%
    }%
  \endgroup
}
%    \end{macrocode}
%    \end{macro}
%
%    \begin{macro}{\@PackageInfo}
%    \begin{macrocode}
\def\@PackageInfo#1#2{%
  \InfWarErr@GenericInfo{%
    (#1) \@spaces\@spaces\@spaces
  }{%
    Package #1 Info: #2%
  }%
}
%    \end{macrocode}
%    \end{macro}
%    \begin{macro}{\@PackageInfoNoLine}
%    \begin{macrocode}
\def\@PackageInfoNoLine#1#2{%
  \@PackageInfo{#1}{#2\@gobble}%
}
%    \end{macrocode}
%    \end{macro}
%    \begin{macro}{\@ClassInfo}
%    \begin{macrocode}
\def\@ClassInfo#1#2{%
  \InfWarErr@GenericInfo{%
    (#1) \space\space\@spaces\@spaces
  }{%
    Class #1 Info: #2%
  }%
}
%    \end{macrocode}
%    \end{macro}
%    \begin{macro}{\@ClassInfoNoLine}
%    \begin{macrocode}
\def\@ClassInfoNoLine#1#2{%
  \@ClassInfo{#1}{#2\@gobble}%
}
%    \end{macrocode}
%    \end{macro}
%    \begin{macro}{\@PackageWarning}
%    \begin{macrocode}
\def\@PackageWarning#1#2{%
  \InfWarErr@GenericWarning{%
    (#1)\@spaces\@spaces\@spaces\@spaces
  }{%
    Package #1 Warning: #2%
  }%
}
%    \end{macrocode}
%    \end{macro}
%    \begin{macro}{\@PackageWarningNoLine}
%    \begin{macrocode}
\def\@PackageWarningNoLine#1#2{%
  \@PackageWarning{#1}{#2\@gobble}%
}
%    \end{macrocode}
%    \end{macro}
%    \begin{macro}{\@ClassWarning}
%    \begin{macrocode}
\def\@ClassWarning#1#2{%
  \InfWarErr@GenericWarning{%
    (#1) \space\@spaces\@spaces\@spaces
  }{%
    Class #1 Warning: #2%
  }%
}
%    \end{macrocode}
%    \end{macro}
%    \begin{macro}{\@ClassWarningNoLine}
%    \begin{macrocode}
\def\@ClassWarningNoLine#1#2{%
  \@ClassWarning{#1}{#2\@gobble}%
}
%    \end{macrocode}
%    \end{macro}
%    \begin{macro}{\@PackageError}
%    \begin{macrocode}
\def\@PackageError#1#2#3{%
  \InfWarErr@GenericError{%
    (#1)\@spaces\@spaces\@spaces\@spaces
  }{%
    Package #1 Error: #2%
  }{%
    See the #1 package documentation for explanation.%
  }{#3}%
}
%    \end{macrocode}
%    \end{macro}
%    \begin{macro}{\@ClassError}
%    \begin{macrocode}
\def\@ClassError#1#2#3{%
  \InfWarErr@GenericError{%
    (#1) \space\@spaces\@spaces\@spaces
  }{%
    Class #1 Error: #2%
  }{%
    See the #1 class documentation for explanation.%
  }{#3}%
}
%    \end{macrocode}
%    \end{macro}
%
%    \begin{macrocode}
\InfWarErr@AtEnd%
%</package>
%    \end{macrocode}
%
% \section{Test}
%
% \subsection{Catcode checks for loading}
%
%    \begin{macrocode}
%<*test1>
%    \end{macrocode}
%    \begin{macrocode}
\catcode`\{=1 %
\catcode`\}=2 %
\catcode`\#=6 %
\catcode`\@=11 %
\expandafter\ifx\csname count@\endcsname\relax
  \countdef\count@=255 %
\fi
\expandafter\ifx\csname @gobble\endcsname\relax
  \long\def\@gobble#1{}%
\fi
\expandafter\ifx\csname @firstofone\endcsname\relax
  \long\def\@firstofone#1{#1}%
\fi
\expandafter\ifx\csname loop\endcsname\relax
  \expandafter\@firstofone
\else
  \expandafter\@gobble
\fi
{%
  \def\loop#1\repeat{%
    \def\body{#1}%
    \iterate
  }%
  \def\iterate{%
    \body
      \let\next\iterate
    \else
      \let\next\relax
    \fi
    \next
  }%
  \let\repeat=\fi
}%
\def\RestoreCatcodes{}
\count@=0 %
\loop
  \edef\RestoreCatcodes{%
    \RestoreCatcodes
    \catcode\the\count@=\the\catcode\count@\relax
  }%
\ifnum\count@<255 %
  \advance\count@ 1 %
\repeat

\def\RangeCatcodeInvalid#1#2{%
  \count@=#1\relax
  \loop
    \catcode\count@=15 %
  \ifnum\count@<#2\relax
    \advance\count@ 1 %
  \repeat
}
\def\RangeCatcodeCheck#1#2#3{%
  \count@=#1\relax
  \loop
    \ifnum#3=\catcode\count@
    \else
      \errmessage{%
        Character \the\count@\space
        with wrong catcode \the\catcode\count@\space
        instead of \number#3%
      }%
    \fi
  \ifnum\count@<#2\relax
    \advance\count@ 1 %
  \repeat
}
\def\space{ }
\expandafter\ifx\csname LoadCommand\endcsname\relax
  \def\LoadCommand{\input infwarerr.sty\relax}%
\fi
\def\Test{%
  \RangeCatcodeInvalid{0}{47}%
  \RangeCatcodeInvalid{58}{64}%
  \RangeCatcodeInvalid{91}{96}%
  \RangeCatcodeInvalid{123}{255}%
  \catcode`\@=12 %
  \catcode`\\=0 %
  \catcode`\%=14 %
  \LoadCommand
  \RangeCatcodeCheck{0}{36}{15}%
  \RangeCatcodeCheck{37}{37}{14}%
  \RangeCatcodeCheck{38}{47}{15}%
  \RangeCatcodeCheck{48}{57}{12}%
  \RangeCatcodeCheck{58}{63}{15}%
  \RangeCatcodeCheck{64}{64}{12}%
  \RangeCatcodeCheck{65}{90}{11}%
  \RangeCatcodeCheck{91}{91}{15}%
  \RangeCatcodeCheck{92}{92}{0}%
  \RangeCatcodeCheck{93}{96}{15}%
  \RangeCatcodeCheck{97}{122}{11}%
  \RangeCatcodeCheck{123}{255}{15}%
  \RestoreCatcodes
}
\Test
\csname @@end\endcsname
\end
%    \end{macrocode}
%    \begin{macrocode}
%</test1>
%    \end{macrocode}
%
% \subsection{Macro tests}
%
%    \begin{macrocode}
%<*test2>
\let\PackageInfo\relax
\input infwarerr.sty\relax
\let\OrgWrite\write
\chardef\WriteUnused=16 %
\def\msg#{\immediate\OrgWrite\WriteUnused}
\msg{File: infwarerr-test2.tex 2010/04/08 v1.3 Test file for plain-TeX}
\catcode64=11 %
%</test2>
%    \end{macrocode}
%
%    \begin{macrocode}
%<*test3>
\NeedsTeXFormat{LaTeX2e}
\ProvidesFile{infwarerr-test3.tex}[2010/04/08 v1.3 Test file for LaTeX]
\RequirePackage{infwarerr}[2010/04/08]
\makeatletter
\let\OrgWrite\write
\chardef\WriteUnused=\@unused
\def\msg#{\immediate\OrgWrite\WriteUnused}
%</test3>
%    \end{macrocode}
%
%    \begin{macrocode}
%<*test2|test3>
\def\pkgname{PACKAGE}
\def\clsname{CLASS}
\def\msgtext{MESSAGE\MessageBreak SECOND LINE}
\def\hlptext{HELP\MessageBreak SECOND LINE}

\let\OrgWrite\write
\let\OrgErrMessage\errmessage

\def\write#1#{%
  \@write{#1}%
}
\def\@write#1#2{%
  \OrgWrite#1{#2}%
  \xdef\WriteNum{\number#1}%
  \xdef\WriteText{#2}%
}
\def\TestFailed#1#2{%
  \OrgErrMessage{Test (\string#1) failed: #2}%
}
\def\TestWrite#1#2#3{%
  \ifnum\WriteNum=#2\relax
  \else
    \TestFailed#1{write stream number: \WriteNum <> #2}%
  \fi
  \begingroup
    \let~\space
    \edef\TestString{#3}%
    \ifx\WriteText\TestString
    \else
      \msg{}%
      \msg{[\WriteText]}%
      \msg{<>}%
      \msg{[#3]}% hash-ok
      \TestFailed#1{write text}%
    \fi
  \endgroup
}
\newtoks\ErrHelpToks
\def\errmessage#1{%
  \global\ErrHelpToks=\expandafter{\the\errhelp}%
  \msg{**************** error message ****************}%
  \msg{! #1}%
  \msg{**************** help message *****************}%
  \msg{\the\errhelp}%
  \msg{***********************************************}%
  \xdef\ErrMessage{! #1}%
}
\def\TestInfo{%
  \@PackageInfo\pkgname\msgtext
  \TestWrite\@PackageInfo{-1}{%
    Package PACKAGE Info: MESSAGE^^J%
    (PACKAGE)~~~~~~~~~~~~~SECOND LINE on input line \the\inputlineno.%
  }%
  \@PackageInfoNoLine\pkgname\msgtext
  \TestWrite\@PackageInfoNoLine{-1}{%
    Package PACKAGE Info: MESSAGE^^J%
    (PACKAGE)~~~~~~~~~~~~~SECOND LINE.%
  }%
  \@ClassInfo\clsname\msgtext
  \TestWrite\@ClassInfo{-1}{%
    Class CLASS Info: MESSAGE^^J%
    (CLASS)~~~~~~~~~~~SECOND LINE on input line \the\inputlineno.%
  }%
  \@ClassInfoNoLine\clsname\msgtext
  \TestWrite\@ClassInfoNoLine{-1}{%
    Class CLASS Info: MESSAGE^^J%
    (CLASS)~~~~~~~~~~~SECOND LINE.%
  }%
}
\def\TestWarning{%
  \@PackageWarning\pkgname\msgtext
  \TestWrite\@PackageWarning\WriteUnused{%
    ^^J%
    Package PACKAGE Warning: MESSAGE^^J%
    (PACKAGE)~~~~~~~~~~~~~~~~SECOND LINE on input line \the\inputlineno.%
    ^^J%
  }%
  \@PackageWarningNoLine\pkgname\msgtext
  \TestWrite\@PackageWarningNoLine\WriteUnused{%
    ^^J%
    Package PACKAGE Warning: MESSAGE^^J%
    (PACKAGE)~~~~~~~~~~~~~~~~SECOND LINE.%
    ^^J%
  }%
  \@ClassWarning\clsname\msgtext
  \TestWrite\@ClassWarning\WriteUnused{%
    ^^J%
    Class CLASS Warning: MESSAGE^^J%
    (CLASS)~~~~~~~~~~~~~~SECOND LINE on input line \the\inputlineno.%
    ^^J%
  }%
  \@ClassWarningNoLine\clsname\msgtext
  \TestWrite\@ClassWarningNoLine\WriteUnused{%
    ^^J%
    Class CLASS Warning: MESSAGE^^J%
    (CLASS)~~~~~~~~~~~~~~SECOND LINE.%
    ^^J%
  }%
}
\def\TestError{%
%<test3>  \let\write\OrgWrite
%<test3>  \def\plaindots{}
%<test2>  \def\plaindots{.^^J ...^^J}
  \@PackageError\pkgname\msgtext\hlptext
  \@TestError\@PackageError{%
    ! Package PACKAGE Error: MESSAGE^^J%
    (PACKAGE)~~~~~~~~~~~~~~~~SECOND LINE.^^J%
    ^^J%
    See the PACKAGE package documentation for explanation.^^J%
    Type~~H <return>~~for immediate help\plaindots
  }{%
    HELP^^J%
    SECOND LINE%
  }%
  \@ClassError\clsname\msgtext\hlptext
  \@TestError\@ClassError{%
    ! Class CLASS Error: MESSAGE^^J%
    (CLASS)~~~~~~~~~~~~~~SECOND LINE.^^J%
    ^^J%
    See the CLASS class documentation for explanation.^^J%
    Type~~H <return>~~for immediate help\plaindots
  }{%
    HELP^^J%
    SECOND LINE%
  }%
  \@PackageError\pkgname\msgtext\@ehc
  \@TestError\@PackageError{%
    ! Package PACKAGE Error: MESSAGE^^J%
    (PACKAGE)~~~~~~~~~~~~~~~~SECOND LINE.^^J%
    ^^J%
    See the PACKAGE package documentation for explanation.^^J%
    Type~~H <return>~~for immediate help\plaindots
  }{%
    Try typing~~<return>~~to proceed.^^J%
    If that doesn't work, type~~X <return>~~to quit.%
  }%
  \@PackageError\pkgname\msgtext\@ehd
  \@TestError\@PackageError{%
    ! Package PACKAGE Error: MESSAGE^^J%
    (PACKAGE)~~~~~~~~~~~~~~~~SECOND LINE.^^J%
    ^^J%
    See the PACKAGE package documentation for explanation.^^J%
    Type~~H <return>~~for immediate help\plaindots
  }{%
    You're in trouble here.~~Try typing~~<return>~~to proceed.^^J%
    If that doesn't work, type~~X <return>~~to quit.%
  }%
}
\def\@TestError#1#2#3{%
  \begingroup
    \let~\space
    \edef\x{#2}%
%<test3>  \@onelevel@sanitize\x
%<test3>  \@onelevel@sanitize\ErrMessage
    \ifx\x\ErrMessage
    \else
      \msg{}%
      \msg{[\ErrMessage]}%
      \msg{<>}%
      \msg{[\x]}%
      \TestFailed#1{errmessage text}%
    \fi
    \edef\x{\the\ErrHelpToks}%
    \edef\y{#3}%
    \ifx\x\y
    \else
      \msg{}%
      \msg{[\the\ErrHelpToks]}%
      \msg{<>}%
      \msg{[\y]}%
      \TestFailed#1{errhelp text}%
    \fi
  \endgroup
}
\TestInfo
\TestWarning
\TestError
%</test2|test3>
%<test2>\end
%<test3>\@@end
%    \end{macrocode}
%
%
% \section{Installation}
%
% \subsection{Download}
%
% \paragraph{Package.} This package is available on
% CTAN\footnote{\url{ftp://ftp.ctan.org/tex-archive/}}:
% \begin{description}
% \item[\CTAN{macros/latex/contrib/oberdiek/infwarerr.dtx}] The source file.
% \item[\CTAN{macros/latex/contrib/oberdiek/infwarerr.pdf}] Documentation.
% \end{description}
%
%
% \paragraph{Bundle.} All the packages of the bundle `oberdiek'
% are also available in a TDS compliant ZIP archive. There
% the packages are already unpacked and the documentation files
% are generated. The files and directories obey the TDS standard.
% \begin{description}
% \item[\CTAN{install/macros/latex/contrib/oberdiek.tds.zip}]
% \end{description}
% \emph{TDS} refers to the standard ``A Directory Structure
% for \TeX\ Files'' (\CTAN{tds/tds.pdf}). Directories
% with \xfile{texmf} in their name are usually organized this way.
%
% \subsection{Bundle installation}
%
% \paragraph{Unpacking.} Unpack the \xfile{oberdiek.tds.zip} in the
% TDS tree (also known as \xfile{texmf} tree) of your choice.
% Example (linux):
% \begin{quote}
%   |unzip oberdiek.tds.zip -d ~/texmf|
% \end{quote}
%
% \paragraph{Script installation.}
% Check the directory \xfile{TDS:scripts/oberdiek/} for
% scripts that need further installation steps.
% Package \xpackage{attachfile2} comes with the Perl script
% \xfile{pdfatfi.pl} that should be installed in such a way
% that it can be called as \texttt{pdfatfi}.
% Example (linux):
% \begin{quote}
%   |chmod +x scripts/oberdiek/pdfatfi.pl|\\
%   |cp scripts/oberdiek/pdfatfi.pl /usr/local/bin/|
% \end{quote}
%
% \subsection{Package installation}
%
% \paragraph{Unpacking.} The \xfile{.dtx} file is a self-extracting
% \docstrip\ archive. The files are extracted by running the
% \xfile{.dtx} through \plainTeX:
% \begin{quote}
%   \verb|tex infwarerr.dtx|
% \end{quote}
%
% \paragraph{TDS.} Now the different files must be moved into
% the different directories in your installation TDS tree
% (also known as \xfile{texmf} tree):
% \begin{quote}
% \def\t{^^A
% \begin{tabular}{@{}>{\ttfamily}l@{ $\rightarrow$ }>{\ttfamily}l@{}}
%   infwarerr.sty & tex/generic/oberdiek/infwarerr.sty\\
%   infwarerr.pdf & doc/latex/oberdiek/infwarerr.pdf\\
%   test/infwarerr-test1.tex & doc/latex/oberdiek/test/infwarerr-test1.tex\\
%   test/infwarerr-test2.tex & doc/latex/oberdiek/test/infwarerr-test2.tex\\
%   test/infwarerr-test3.tex & doc/latex/oberdiek/test/infwarerr-test3.tex\\
%   infwarerr.dtx & source/latex/oberdiek/infwarerr.dtx\\
% \end{tabular}^^A
% }^^A
% \sbox0{\t}^^A
% \ifdim\wd0>\linewidth
%   \begingroup
%     \advance\linewidth by\leftmargin
%     \advance\linewidth by\rightmargin
%   \edef\x{\endgroup
%     \def\noexpand\lw{\the\linewidth}^^A
%   }\x
%   \def\lwbox{^^A
%     \leavevmode
%     \hbox to \linewidth{^^A
%       \kern-\leftmargin\relax
%       \hss
%       \usebox0
%       \hss
%       \kern-\rightmargin\relax
%     }^^A
%   }^^A
%   \ifdim\wd0>\lw
%     \sbox0{\small\t}^^A
%     \ifdim\wd0>\linewidth
%       \ifdim\wd0>\lw
%         \sbox0{\footnotesize\t}^^A
%         \ifdim\wd0>\linewidth
%           \ifdim\wd0>\lw
%             \sbox0{\scriptsize\t}^^A
%             \ifdim\wd0>\linewidth
%               \ifdim\wd0>\lw
%                 \sbox0{\tiny\t}^^A
%                 \ifdim\wd0>\linewidth
%                   \lwbox
%                 \else
%                   \usebox0
%                 \fi
%               \else
%                 \lwbox
%               \fi
%             \else
%               \usebox0
%             \fi
%           \else
%             \lwbox
%           \fi
%         \else
%           \usebox0
%         \fi
%       \else
%         \lwbox
%       \fi
%     \else
%       \usebox0
%     \fi
%   \else
%     \lwbox
%   \fi
% \else
%   \usebox0
% \fi
% \end{quote}
% If you have a \xfile{docstrip.cfg} that configures and enables \docstrip's
% TDS installing feature, then some files can already be in the right
% place, see the documentation of \docstrip.
%
% \subsection{Refresh file name databases}
%
% If your \TeX~distribution
% (\teTeX, \mikTeX, \dots) relies on file name databases, you must refresh
% these. For example, \teTeX\ users run \verb|texhash| or
% \verb|mktexlsr|.
%
% \subsection{Some details for the interested}
%
% \paragraph{Attached source.}
%
% The PDF documentation on CTAN also includes the
% \xfile{.dtx} source file. It can be extracted by
% AcrobatReader 6 or higher. Another option is \textsf{pdftk},
% e.g. unpack the file into the current directory:
% \begin{quote}
%   \verb|pdftk infwarerr.pdf unpack_files output .|
% \end{quote}
%
% \paragraph{Unpacking with \LaTeX.}
% The \xfile{.dtx} chooses its action depending on the format:
% \begin{description}
% \item[\plainTeX:] Run \docstrip\ and extract the files.
% \item[\LaTeX:] Generate the documentation.
% \end{description}
% If you insist on using \LaTeX\ for \docstrip\ (really,
% \docstrip\ does not need \LaTeX), then inform the autodetect routine
% about your intention:
% \begin{quote}
%   \verb|latex \let\install=y\input{infwarerr.dtx}|
% \end{quote}
% Do not forget to quote the argument according to the demands
% of your shell.
%
% \paragraph{Generating the documentation.}
% You can use both the \xfile{.dtx} or the \xfile{.drv} to generate
% the documentation. The process can be configured by the
% configuration file \xfile{ltxdoc.cfg}. For instance, put this
% line into this file, if you want to have A4 as paper format:
% \begin{quote}
%   \verb|\PassOptionsToClass{a4paper}{article}|
% \end{quote}
% An example follows how to generate the
% documentation with pdf\LaTeX:
% \begin{quote}
%\begin{verbatim}
%pdflatex infwarerr.dtx
%makeindex -s gind.ist infwarerr.idx
%pdflatex infwarerr.dtx
%makeindex -s gind.ist infwarerr.idx
%pdflatex infwarerr.dtx
%\end{verbatim}
% \end{quote}
%
% \begin{History}
%   \begin{Version}{2007/04/26 v1.0}
%   \item
%     First version.
%   \end{Version}
%   \begin{Version}{2007/06/14 v1.1}
%   \item
%     Small fixes.
%   \end{Version}
%   \begin{Version}{2007/09/09 v1.2}
%   \item
%     Catcode section extended.
%   \end{Version}
%   \begin{Version}{2010/04/08 v1.3}
%   \item
%     Ensure \cs{escapechar}=92 (backslash) for infos/warnings/errors.
%   \end{Version}
% \end{History}
%
% \PrintIndex
%
% \Finale
\endinput

%        (quote the arguments according to the demands of your shell)
% 
% Documentation:
%    (a) If infwarerr.drv is present:
%           latex infwarerr.drv
%    (b) Without infwarerr.drv:
%           latex infwarerr.dtx; ...
%    The class ltxdoc loads the configuration file ltxdoc.cfg
%    if available. Here you can specify further options, e.g.
%    use A4 as paper format:
%       \PassOptionsToClass{a4paper}{article}
% 
%    Programm calls to get the documentation (example):
%       pdflatex infwarerr.dtx
%       makeindex -s gind.ist infwarerr.idx
%       pdflatex infwarerr.dtx
%       makeindex -s gind.ist infwarerr.idx
%       pdflatex infwarerr.dtx
% 
% Installation:
%    TDS:tex/generic/oberdiek/infwarerr.sty
%    TDS:doc/latex/oberdiek/infwarerr.pdf
%    TDS:doc/latex/oberdiek/test/infwarerr-test1.tex
%    TDS:doc/latex/oberdiek/test/infwarerr-test2.tex
%    TDS:doc/latex/oberdiek/test/infwarerr-test3.tex
%    TDS:source/latex/oberdiek/infwarerr.dtx
%
%<*ignore>
\begingroup
  \catcode123=1 %
  \catcode125=2 %
  \def\x{LaTeX2e}%
\expandafter\endgroup
\ifcase 0\ifx\install y1\fi\expandafter
         \ifx\csname processbatchFile\endcsname\relax\else1\fi
         \ifx\fmtname\x\else 1\fi\relax
\else\csname fi\endcsname
%</ignore>
%<*install>
\input docstrip.tex
\Msg{************************************************************************}
\Msg{* Installation}
\Msg{* Package: infwarerr 2010/04/08 v1.3 Providing info/warning/message (HO)}
\Msg{************************************************************************}

\keepsilent
\askforoverwritefalse

\let\MetaPrefix\relax
\preamble

This is a generated file.

Project: infwarerr
Version: 2010/04/08 v1.3

Copyright (C) 2007, 2010 by
   Heiko Oberdiek <heiko.oberdiek at googlemail.com>

This work may be distributed and/or modified under the
conditions of the LaTeX Project Public License, either
version 1.3c of this license or (at your option) any later
version. This version of this license is in
   http://www.latex-project.org/lppl/lppl-1-3c.txt
and the latest version of this license is in
   http://www.latex-project.org/lppl.txt
and version 1.3 or later is part of all distributions of
LaTeX version 2005/12/01 or later.

This work has the LPPL maintenance status "maintained".

This Current Maintainer of this work is Heiko Oberdiek.

The Base Interpreter refers to any `TeX-Format',
because some files are installed in TDS:tex/generic//.

This work consists of the main source file infwarerr.dtx
and the derived files
   infwarerr.sty, infwarerr.pdf, infwarerr.ins, infwarerr.drv,
   infwarerr-test1.tex, infwarerr-test2.tex,
   infwarerr-test3.tex.

\endpreamble
\let\MetaPrefix\DoubleperCent

\generate{%
  \file{infwarerr.ins}{\from{infwarerr.dtx}{install}}%
  \file{infwarerr.drv}{\from{infwarerr.dtx}{driver}}%
  \usedir{tex/generic/oberdiek}%
  \file{infwarerr.sty}{\from{infwarerr.dtx}{package}}%
  \usedir{doc/latex/oberdiek/test}%
  \file{infwarerr-test1.tex}{\from{infwarerr.dtx}{test1}}%
  \file{infwarerr-test2.tex}{\from{infwarerr.dtx}{test2}}%
  \file{infwarerr-test3.tex}{\from{infwarerr.dtx}{test3}}%
}

\catcode32=13\relax% active space
\let =\space%
\Msg{************************************************************************}
\Msg{*}
\Msg{* To finish the installation you have to move the following}
\Msg{* file into a directory searched by TeX:}
\Msg{*}
\Msg{*     infwarerr.sty}
\Msg{*}
\Msg{* To produce the documentation run the file `infwarerr.drv'}
\Msg{* through LaTeX.}
\Msg{*}
\Msg{* Happy TeXing!}
\Msg{*}
\Msg{************************************************************************}

\endbatchfile
%</install>
%<*ignore>
\fi
%</ignore>
%<*driver>
\NeedsTeXFormat{LaTeX2e}
\ProvidesFile{infwarerr.drv}%
  [2010/04/08 v1.3 Providing info/warning/message (HO)]%
\documentclass{ltxdoc}
\usepackage{holtxdoc}[2011/02/04]
\begin{document}
  \DocInput{infwarerr.dtx}%
\end{document}
%</driver>
% \fi
%
% \CheckSum{871}
%
% \CharacterTable
%  {Upper-case    \A\B\C\D\E\F\G\H\I\J\K\L\M\N\O\P\Q\R\S\T\U\V\W\X\Y\Z
%   Lower-case    \a\b\c\d\e\f\g\h\i\j\k\l\m\n\o\p\q\r\s\t\u\v\w\x\y\z
%   Digits        \0\1\2\3\4\5\6\7\8\9
%   Exclamation   \!     Double quote  \"     Hash (number) \#
%   Dollar        \$     Percent       \%     Ampersand     \&
%   Acute accent  \'     Left paren    \(     Right paren   \)
%   Asterisk      \*     Plus          \+     Comma         \,
%   Minus         \-     Point         \.     Solidus       \/
%   Colon         \:     Semicolon     \;     Less than     \<
%   Equals        \=     Greater than  \>     Question mark \?
%   Commercial at \@     Left bracket  \[     Backslash     \\
%   Right bracket \]     Circumflex    \^     Underscore    \_
%   Grave accent  \`     Left brace    \{     Vertical bar  \|
%   Right brace   \}     Tilde         \~}
%
% \GetFileInfo{infwarerr.drv}
%
% \title{The \xpackage{infwarerr} package}
% \date{2010/04/08 v1.3}
% \author{Heiko Oberdiek\\\xemail{heiko.oberdiek at googlemail.com}}
%
% \maketitle
%
% \begin{abstract}
% This package provides a complete set of macros for informations,
% warnings and error messages with support for \plainTeX.
% \end{abstract}
%
% \tableofcontents
%
% \section{Documentation}
%
% Package \xpackage{infwarerr} defines the following set of commands,
% given as pseudo regular expression:
%   \begin{quote}
%     \verb+\@(Package|Class)(Info|InfoNoLine|Warning|WarningNoLine|Error)+^^A
%     \kern-\rightmargin
%   \end{quote}
% The first argument is the package or class in question. The second
% argument contains the message text. \cs{MessageBreak} is supported.
% In case of error messages, the third argument contains the help
% text. Supported are \LaTeX's message texts \cs{@ehc}:
%   \begin{quote}
%     |Try typing  <return>  to proceed|\\
%     |If that doesn't work, type  X <return>  to quit.|
%   \end{quote}
% and \cs{@ehd}:
%   \begin{quote}
%     |You're in trouble here.  Try typing  <return>  to proceed|\\
%     |If that doesn't work, type  X <return>  to quit.|
%   \end{quote}
%
% The package is intended for macro writers that want to write
% code that works for both \LaTeX\ and \plainTeX. If \LaTeX\ is
% detected, then the macros of this package will only be wrappers
% for \LaTeX's macros. Otherwise \LaTeX's macros will be mimicked
% for use with \plainTeX.
%
% \StopEventually{
% }
%
% \section{Implementation}
%
%    \begin{macrocode}
%<*package>
%    \end{macrocode}
%
% \subsection{Reload check and package identification}
%    Reload check, especially if the package is not used with \LaTeX.
%    \begin{macrocode}
\begingroup\catcode61\catcode48\catcode32=10\relax%
  \catcode13=5 % ^^M
  \endlinechar=13 %
  \catcode35=6 % #
  \catcode39=12 % '
  \catcode44=12 % ,
  \catcode45=12 % -
  \catcode46=12 % .
  \catcode58=12 % :
  \catcode64=11 % @
  \catcode123=1 % {
  \catcode125=2 % }
  \expandafter\let\expandafter\x\csname ver@infwarerr.sty\endcsname
  \ifx\x\relax % plain-TeX, first loading
  \else
    \def\empty{}%
    \ifx\x\empty % LaTeX, first loading,
      % variable is initialized, but \ProvidesPackage not yet seen
    \else
      \expandafter\ifx\csname PackageInfo\endcsname\relax
        \def\x#1#2{%
          \immediate\write-1{Package #1 Info: #2.}%
        }%
      \else
        \def\x#1#2{\PackageInfo{#1}{#2, stopped}}%
      \fi
      \x{infwarerr}{The package is already loaded}%
      \aftergroup\endinput
    \fi
  \fi
\endgroup%
%    \end{macrocode}
%    Package identification:
%    \begin{macrocode}
\begingroup\catcode61\catcode48\catcode32=10\relax%
  \catcode13=5 % ^^M
  \endlinechar=13 %
  \catcode35=6 % #
  \catcode39=12 % '
  \catcode40=12 % (
  \catcode41=12 % )
  \catcode44=12 % ,
  \catcode45=12 % -
  \catcode46=12 % .
  \catcode47=12 % /
  \catcode58=12 % :
  \catcode64=11 % @
  \catcode91=12 % [
  \catcode93=12 % ]
  \catcode123=1 % {
  \catcode125=2 % }
  \expandafter\ifx\csname ProvidesPackage\endcsname\relax
    \def\x#1#2#3[#4]{\endgroup
      \immediate\write-1{Package: #3 #4}%
      \xdef#1{#4}%
    }%
  \else
    \def\x#1#2[#3]{\endgroup
      #2[{#3}]%
      \ifx#1\@undefined
        \xdef#1{#3}%
      \fi
      \ifx#1\relax
        \xdef#1{#3}%
      \fi
    }%
  \fi
\expandafter\x\csname ver@infwarerr.sty\endcsname
\ProvidesPackage{infwarerr}%
  [2010/04/08 v1.3 Providing info/warning/message (HO)]%
%    \end{macrocode}
%
% \subsection{Catcodes}
%
%    \begin{macrocode}
\begingroup\catcode61\catcode48\catcode32=10\relax%
  \catcode13=5 % ^^M
  \endlinechar=13 %
  \catcode123=1 % {
  \catcode125=2 % }
  \catcode64=11 % @
  \def\x{\endgroup
    \expandafter\edef\csname InfWarErr@AtEnd\endcsname{%
      \endlinechar=\the\endlinechar\relax
      \catcode13=\the\catcode13\relax
      \catcode32=\the\catcode32\relax
      \catcode35=\the\catcode35\relax
      \catcode61=\the\catcode61\relax
      \catcode64=\the\catcode64\relax
      \catcode123=\the\catcode123\relax
      \catcode125=\the\catcode125\relax
    }%
  }%
\x\catcode61\catcode48\catcode32=10\relax%
\catcode13=5 % ^^M
\endlinechar=13 %
\catcode35=6 % #
\catcode64=11 % @
\catcode123=1 % {
\catcode125=2 % }
\def\TMP@EnsureCode#1#2{%
  \edef\InfWarErr@AtEnd{%
    \InfWarErr@AtEnd
    \catcode#1=\the\catcode#1\relax
  }%
  \catcode#1=#2\relax
}
\TMP@EnsureCode{10}{12}% ^^J
\TMP@EnsureCode{39}{12}% '
\TMP@EnsureCode{40}{12}% (
\TMP@EnsureCode{41}{12}% )
\TMP@EnsureCode{44}{12}% ,
\TMP@EnsureCode{45}{12}% -
\TMP@EnsureCode{46}{12}% .
\TMP@EnsureCode{58}{12}% :
\TMP@EnsureCode{60}{12}% <
\TMP@EnsureCode{62}{12}% >
\TMP@EnsureCode{94}{7}% ^
\edef\InfWarErr@AtEnd{\InfWarErr@AtEnd\noexpand\endinput}
%    \end{macrocode}
%
% \subsection{\LaTeX\ detection}
%
%    \begin{macrocode}
\begingroup
  \let\x=Y%
  \def\check#1{%
    \@check{Generic}{#1}%
    \@check{Package}{#1}%
    \@check{Class}{#1}%
  }%
  \def\@check#1#2{%
    \expandafter\ifx\csname #1#2\endcsname\relax
      \let\x=N%
    \fi
  }%
  \check{Info}%
  \check{Warning}%
  \check{Error}%
  \@check{Package}{WarningNoLine}%
  \@check{Class}{WarningNoLine}%
  \@check{}{@ehc}%
  \@check{}{@ehd}%
\expandafter\endgroup
\ifx\x Y%
%    \end{macrocode}
%
% \subsection{Version for \LaTeX}
%
%    \begin{macro}{\@PackageInfo}
%    \begin{macrocode}
  \let\@PackageInfo\PackageInfo
%    \end{macrocode}
%    \end{macro}
%    \begin{macro}{\@PackageInfoNoLine}
%    \begin{macrocode}
  \def\@PackageInfoNoLine#1#2{%
    \PackageInfo{#1}{#2\@gobble}%
  }%
%    \end{macrocode}
%    \end{macro}
%    \begin{macro}{\@ClassInfo}
%    \begin{macrocode}
  \let\@ClassInfo\ClassInfo
%    \end{macrocode}
%    \end{macro}
%    \begin{macro}{\@ClassInfoNoLine}
%    \begin{macrocode}
  \def\@ClassInfoNoLine#1#2{%
    \ClassInfo{#1}{#2\@gobble}%
  }%
%    \end{macrocode}
%    \end{macro}
%    \begin{macro}{\@PackageWarning}
%    \begin{macrocode}
  \let\@PackageWarning\PackageWarning
%    \end{macrocode}
%    \end{macro}
%    \begin{macro}{\@PackageWarningNoLine}
%    \begin{macrocode}
  \let\@PackageWarningNoLine\PackageWarningNoLine
%    \end{macrocode}
%    \end{macro}
%    \begin{macro}{\@ClassWarning}
%    \begin{macrocode}
  \let\@ClassWarning\ClassWarning
%    \end{macrocode}
%    \end{macro}
%    \begin{macro}{\@ClassWarningNoLine}
%    \begin{macrocode}
  \let\@ClassWarningNoLine\ClassWarningNoLine
%    \end{macrocode}
%    \end{macro}
%    \begin{macro}{\@PackageError}
%    \begin{macrocode}
  \let\@PackageError\PackageError
%    \end{macrocode}
%    \end{macro}
%    \begin{macro}{\@ClassError}
%    \begin{macrocode}
  \let\@ClassError\ClassError
%    \end{macrocode}
%    \end{macro}
%    \begin{macrocode}
  \expandafter\InfWarErr@AtEnd
\fi%
%    \end{macrocode}
%
% \subsection{Version for \plainTeX}
%
%    \begin{macro}{\InfWarErr@online}
%    \begin{macrocode}
\begingroup\expandafter\endgroup
\ifcase\expandafter\ifx\csname inputlineno\endcsname\relax
         0 %
       \else
         \ifnum\inputlineno<0 %
           0 %
         \else
           1 %
         \fi
       \fi
  \def\InfWarErr@online{}%
\else
  \def\InfWarErr@online{ on input line \the\inputlineno}%
\fi
%    \end{macrocode}
%    \end{macro}
%
%    \begin{macro}{\InfWarErr@protected}
%    \begin{macrocode}
\begingroup\expandafter\expandafter\expandafter\endgroup
\expandafter\ifx\csname protected\endcsname\relax
  \def\InfWarErr@protected{}%
\else
  \let\InfWarErr@protected\protected
\fi
%    \end{macrocode}
%    \end{macro}
%
%    \begin{macro}{\InfWarErr@unused}
%    \begin{macrocode}
\chardef\InfWarErr@unused=16 %
%    \end{macrocode}
%    \end{macro}
%
%    \begin{macro}{\InfWarErr@ehc}
%    \begin{macrocode}
\def\InfWarErr@ehc{%
  Try typing \space <return> \space to proceed.\MessageBreak
  If that doesn't work, type \space X <return> \space to quit.%
}
%    \end{macrocode}
%    \end{macro}
%    \begin{macro}{\InfWarErr@ehd}
%    \begin{macrocode}
\def\InfWarErr@ehd{%
  You're in trouble here. \space\@ehc
}
%    \end{macrocode}
%    \end{macro}
%
%    \begin{macro}{\InfWarErr@GenericInfo}
%    \begin{macrocode}
\InfWarErr@protected\def\InfWarErr@GenericInfo#1#2{%
  \begingroup
    \let\on@line\InfWarErr@online
    \long\def\@gobble##1{}%
    \def\space{ }%
    \def\@spaces{\space\space\space\space}%
    \def\MessageBreak{^^J#1}%
    \newlinechar=10 %
    \escapechar=92 %
    \immediate\write-1{#2\on@line.}%
  \endgroup
}
%    \end{macrocode}
%    \end{macro}
%    \begin{macro}{\InfWarErr@GenericWarning}
%    \begin{macrocode}
\InfWarErr@protected\def\InfWarErr@GenericWarning#1#2{%
  \begingroup
    \let\on@line\InfWarErr@online
    \long\def\@gobble##1{}%
    \def\space{ }%
    \def\@spaces{\space\space\space\space}%
    \def\MessageBreak{^^J#1}%
    \newlinechar=10 %
    \escapechar=92 %
    \immediate\write\InfWarErr@unused{^^J#2\on@line.^^J}%
  \endgroup
}
%    \end{macrocode}
%    \end{macro}
%    \begin{macro}{\InfWarErr@GenericError}
%    Old \TeX\ versions below 3.141 where |^^J| is not working
%    as newline in error message and help texts are not supported.
%    \begin{macrocode}
\InfWarErr@protected\def\InfWarErr@GenericError#1#2#3#4{%
  \begingroup
    \immediate\write\InfWarErr@unused{}%
    \def\space{ }%
    \def\@spaces{\space\space\space\space}%
    \let\@ehc\InfWarErr@ehc
    \let\@ehd\InfWarErr@ehd
    \let\on@line\InfWarErr@online
    \def\MessageBreak{^^J}%
    \newlinechar=10 %
    \escapechar=92 %
    \edef\InfWarErr@help{\noexpand\errhelp{#4}}%
    \InfWarErr@help
    \def\MessageBreak{^^J#1}%
    \errmessage{%
      #2.^^J^^J%
      #3^^J%
      Type \space H <return> \space for immediate help.^^J ...^^J%
    }%
  \endgroup
}
%    \end{macrocode}
%    \end{macro}
%
%    \begin{macro}{\@PackageInfo}
%    \begin{macrocode}
\def\@PackageInfo#1#2{%
  \InfWarErr@GenericInfo{%
    (#1) \@spaces\@spaces\@spaces
  }{%
    Package #1 Info: #2%
  }%
}
%    \end{macrocode}
%    \end{macro}
%    \begin{macro}{\@PackageInfoNoLine}
%    \begin{macrocode}
\def\@PackageInfoNoLine#1#2{%
  \@PackageInfo{#1}{#2\@gobble}%
}
%    \end{macrocode}
%    \end{macro}
%    \begin{macro}{\@ClassInfo}
%    \begin{macrocode}
\def\@ClassInfo#1#2{%
  \InfWarErr@GenericInfo{%
    (#1) \space\space\@spaces\@spaces
  }{%
    Class #1 Info: #2%
  }%
}
%    \end{macrocode}
%    \end{macro}
%    \begin{macro}{\@ClassInfoNoLine}
%    \begin{macrocode}
\def\@ClassInfoNoLine#1#2{%
  \@ClassInfo{#1}{#2\@gobble}%
}
%    \end{macrocode}
%    \end{macro}
%    \begin{macro}{\@PackageWarning}
%    \begin{macrocode}
\def\@PackageWarning#1#2{%
  \InfWarErr@GenericWarning{%
    (#1)\@spaces\@spaces\@spaces\@spaces
  }{%
    Package #1 Warning: #2%
  }%
}
%    \end{macrocode}
%    \end{macro}
%    \begin{macro}{\@PackageWarningNoLine}
%    \begin{macrocode}
\def\@PackageWarningNoLine#1#2{%
  \@PackageWarning{#1}{#2\@gobble}%
}
%    \end{macrocode}
%    \end{macro}
%    \begin{macro}{\@ClassWarning}
%    \begin{macrocode}
\def\@ClassWarning#1#2{%
  \InfWarErr@GenericWarning{%
    (#1) \space\@spaces\@spaces\@spaces
  }{%
    Class #1 Warning: #2%
  }%
}
%    \end{macrocode}
%    \end{macro}
%    \begin{macro}{\@ClassWarningNoLine}
%    \begin{macrocode}
\def\@ClassWarningNoLine#1#2{%
  \@ClassWarning{#1}{#2\@gobble}%
}
%    \end{macrocode}
%    \end{macro}
%    \begin{macro}{\@PackageError}
%    \begin{macrocode}
\def\@PackageError#1#2#3{%
  \InfWarErr@GenericError{%
    (#1)\@spaces\@spaces\@spaces\@spaces
  }{%
    Package #1 Error: #2%
  }{%
    See the #1 package documentation for explanation.%
  }{#3}%
}
%    \end{macrocode}
%    \end{macro}
%    \begin{macro}{\@ClassError}
%    \begin{macrocode}
\def\@ClassError#1#2#3{%
  \InfWarErr@GenericError{%
    (#1) \space\@spaces\@spaces\@spaces
  }{%
    Class #1 Error: #2%
  }{%
    See the #1 class documentation for explanation.%
  }{#3}%
}
%    \end{macrocode}
%    \end{macro}
%
%    \begin{macrocode}
\InfWarErr@AtEnd%
%</package>
%    \end{macrocode}
%
% \section{Test}
%
% \subsection{Catcode checks for loading}
%
%    \begin{macrocode}
%<*test1>
%    \end{macrocode}
%    \begin{macrocode}
\catcode`\{=1 %
\catcode`\}=2 %
\catcode`\#=6 %
\catcode`\@=11 %
\expandafter\ifx\csname count@\endcsname\relax
  \countdef\count@=255 %
\fi
\expandafter\ifx\csname @gobble\endcsname\relax
  \long\def\@gobble#1{}%
\fi
\expandafter\ifx\csname @firstofone\endcsname\relax
  \long\def\@firstofone#1{#1}%
\fi
\expandafter\ifx\csname loop\endcsname\relax
  \expandafter\@firstofone
\else
  \expandafter\@gobble
\fi
{%
  \def\loop#1\repeat{%
    \def\body{#1}%
    \iterate
  }%
  \def\iterate{%
    \body
      \let\next\iterate
    \else
      \let\next\relax
    \fi
    \next
  }%
  \let\repeat=\fi
}%
\def\RestoreCatcodes{}
\count@=0 %
\loop
  \edef\RestoreCatcodes{%
    \RestoreCatcodes
    \catcode\the\count@=\the\catcode\count@\relax
  }%
\ifnum\count@<255 %
  \advance\count@ 1 %
\repeat

\def\RangeCatcodeInvalid#1#2{%
  \count@=#1\relax
  \loop
    \catcode\count@=15 %
  \ifnum\count@<#2\relax
    \advance\count@ 1 %
  \repeat
}
\def\RangeCatcodeCheck#1#2#3{%
  \count@=#1\relax
  \loop
    \ifnum#3=\catcode\count@
    \else
      \errmessage{%
        Character \the\count@\space
        with wrong catcode \the\catcode\count@\space
        instead of \number#3%
      }%
    \fi
  \ifnum\count@<#2\relax
    \advance\count@ 1 %
  \repeat
}
\def\space{ }
\expandafter\ifx\csname LoadCommand\endcsname\relax
  \def\LoadCommand{\input infwarerr.sty\relax}%
\fi
\def\Test{%
  \RangeCatcodeInvalid{0}{47}%
  \RangeCatcodeInvalid{58}{64}%
  \RangeCatcodeInvalid{91}{96}%
  \RangeCatcodeInvalid{123}{255}%
  \catcode`\@=12 %
  \catcode`\\=0 %
  \catcode`\%=14 %
  \LoadCommand
  \RangeCatcodeCheck{0}{36}{15}%
  \RangeCatcodeCheck{37}{37}{14}%
  \RangeCatcodeCheck{38}{47}{15}%
  \RangeCatcodeCheck{48}{57}{12}%
  \RangeCatcodeCheck{58}{63}{15}%
  \RangeCatcodeCheck{64}{64}{12}%
  \RangeCatcodeCheck{65}{90}{11}%
  \RangeCatcodeCheck{91}{91}{15}%
  \RangeCatcodeCheck{92}{92}{0}%
  \RangeCatcodeCheck{93}{96}{15}%
  \RangeCatcodeCheck{97}{122}{11}%
  \RangeCatcodeCheck{123}{255}{15}%
  \RestoreCatcodes
}
\Test
\csname @@end\endcsname
\end
%    \end{macrocode}
%    \begin{macrocode}
%</test1>
%    \end{macrocode}
%
% \subsection{Macro tests}
%
%    \begin{macrocode}
%<*test2>
\let\PackageInfo\relax
\input infwarerr.sty\relax
\let\OrgWrite\write
\chardef\WriteUnused=16 %
\def\msg#{\immediate\OrgWrite\WriteUnused}
\msg{File: infwarerr-test2.tex 2010/04/08 v1.3 Test file for plain-TeX}
\catcode64=11 %
%</test2>
%    \end{macrocode}
%
%    \begin{macrocode}
%<*test3>
\NeedsTeXFormat{LaTeX2e}
\ProvidesFile{infwarerr-test3.tex}[2010/04/08 v1.3 Test file for LaTeX]
\RequirePackage{infwarerr}[2010/04/08]
\makeatletter
\let\OrgWrite\write
\chardef\WriteUnused=\@unused
\def\msg#{\immediate\OrgWrite\WriteUnused}
%</test3>
%    \end{macrocode}
%
%    \begin{macrocode}
%<*test2|test3>
\def\pkgname{PACKAGE}
\def\clsname{CLASS}
\def\msgtext{MESSAGE\MessageBreak SECOND LINE}
\def\hlptext{HELP\MessageBreak SECOND LINE}

\let\OrgWrite\write
\let\OrgErrMessage\errmessage

\def\write#1#{%
  \@write{#1}%
}
\def\@write#1#2{%
  \OrgWrite#1{#2}%
  \xdef\WriteNum{\number#1}%
  \xdef\WriteText{#2}%
}
\def\TestFailed#1#2{%
  \OrgErrMessage{Test (\string#1) failed: #2}%
}
\def\TestWrite#1#2#3{%
  \ifnum\WriteNum=#2\relax
  \else
    \TestFailed#1{write stream number: \WriteNum <> #2}%
  \fi
  \begingroup
    \let~\space
    \edef\TestString{#3}%
    \ifx\WriteText\TestString
    \else
      \msg{}%
      \msg{[\WriteText]}%
      \msg{<>}%
      \msg{[#3]}% hash-ok
      \TestFailed#1{write text}%
    \fi
  \endgroup
}
\newtoks\ErrHelpToks
\def\errmessage#1{%
  \global\ErrHelpToks=\expandafter{\the\errhelp}%
  \msg{**************** error message ****************}%
  \msg{! #1}%
  \msg{**************** help message *****************}%
  \msg{\the\errhelp}%
  \msg{***********************************************}%
  \xdef\ErrMessage{! #1}%
}
\def\TestInfo{%
  \@PackageInfo\pkgname\msgtext
  \TestWrite\@PackageInfo{-1}{%
    Package PACKAGE Info: MESSAGE^^J%
    (PACKAGE)~~~~~~~~~~~~~SECOND LINE on input line \the\inputlineno.%
  }%
  \@PackageInfoNoLine\pkgname\msgtext
  \TestWrite\@PackageInfoNoLine{-1}{%
    Package PACKAGE Info: MESSAGE^^J%
    (PACKAGE)~~~~~~~~~~~~~SECOND LINE.%
  }%
  \@ClassInfo\clsname\msgtext
  \TestWrite\@ClassInfo{-1}{%
    Class CLASS Info: MESSAGE^^J%
    (CLASS)~~~~~~~~~~~SECOND LINE on input line \the\inputlineno.%
  }%
  \@ClassInfoNoLine\clsname\msgtext
  \TestWrite\@ClassInfoNoLine{-1}{%
    Class CLASS Info: MESSAGE^^J%
    (CLASS)~~~~~~~~~~~SECOND LINE.%
  }%
}
\def\TestWarning{%
  \@PackageWarning\pkgname\msgtext
  \TestWrite\@PackageWarning\WriteUnused{%
    ^^J%
    Package PACKAGE Warning: MESSAGE^^J%
    (PACKAGE)~~~~~~~~~~~~~~~~SECOND LINE on input line \the\inputlineno.%
    ^^J%
  }%
  \@PackageWarningNoLine\pkgname\msgtext
  \TestWrite\@PackageWarningNoLine\WriteUnused{%
    ^^J%
    Package PACKAGE Warning: MESSAGE^^J%
    (PACKAGE)~~~~~~~~~~~~~~~~SECOND LINE.%
    ^^J%
  }%
  \@ClassWarning\clsname\msgtext
  \TestWrite\@ClassWarning\WriteUnused{%
    ^^J%
    Class CLASS Warning: MESSAGE^^J%
    (CLASS)~~~~~~~~~~~~~~SECOND LINE on input line \the\inputlineno.%
    ^^J%
  }%
  \@ClassWarningNoLine\clsname\msgtext
  \TestWrite\@ClassWarningNoLine\WriteUnused{%
    ^^J%
    Class CLASS Warning: MESSAGE^^J%
    (CLASS)~~~~~~~~~~~~~~SECOND LINE.%
    ^^J%
  }%
}
\def\TestError{%
%<test3>  \let\write\OrgWrite
%<test3>  \def\plaindots{}
%<test2>  \def\plaindots{.^^J ...^^J}
  \@PackageError\pkgname\msgtext\hlptext
  \@TestError\@PackageError{%
    ! Package PACKAGE Error: MESSAGE^^J%
    (PACKAGE)~~~~~~~~~~~~~~~~SECOND LINE.^^J%
    ^^J%
    See the PACKAGE package documentation for explanation.^^J%
    Type~~H <return>~~for immediate help\plaindots
  }{%
    HELP^^J%
    SECOND LINE%
  }%
  \@ClassError\clsname\msgtext\hlptext
  \@TestError\@ClassError{%
    ! Class CLASS Error: MESSAGE^^J%
    (CLASS)~~~~~~~~~~~~~~SECOND LINE.^^J%
    ^^J%
    See the CLASS class documentation for explanation.^^J%
    Type~~H <return>~~for immediate help\plaindots
  }{%
    HELP^^J%
    SECOND LINE%
  }%
  \@PackageError\pkgname\msgtext\@ehc
  \@TestError\@PackageError{%
    ! Package PACKAGE Error: MESSAGE^^J%
    (PACKAGE)~~~~~~~~~~~~~~~~SECOND LINE.^^J%
    ^^J%
    See the PACKAGE package documentation for explanation.^^J%
    Type~~H <return>~~for immediate help\plaindots
  }{%
    Try typing~~<return>~~to proceed.^^J%
    If that doesn't work, type~~X <return>~~to quit.%
  }%
  \@PackageError\pkgname\msgtext\@ehd
  \@TestError\@PackageError{%
    ! Package PACKAGE Error: MESSAGE^^J%
    (PACKAGE)~~~~~~~~~~~~~~~~SECOND LINE.^^J%
    ^^J%
    See the PACKAGE package documentation for explanation.^^J%
    Type~~H <return>~~for immediate help\plaindots
  }{%
    You're in trouble here.~~Try typing~~<return>~~to proceed.^^J%
    If that doesn't work, type~~X <return>~~to quit.%
  }%
}
\def\@TestError#1#2#3{%
  \begingroup
    \let~\space
    \edef\x{#2}%
%<test3>  \@onelevel@sanitize\x
%<test3>  \@onelevel@sanitize\ErrMessage
    \ifx\x\ErrMessage
    \else
      \msg{}%
      \msg{[\ErrMessage]}%
      \msg{<>}%
      \msg{[\x]}%
      \TestFailed#1{errmessage text}%
    \fi
    \edef\x{\the\ErrHelpToks}%
    \edef\y{#3}%
    \ifx\x\y
    \else
      \msg{}%
      \msg{[\the\ErrHelpToks]}%
      \msg{<>}%
      \msg{[\y]}%
      \TestFailed#1{errhelp text}%
    \fi
  \endgroup
}
\TestInfo
\TestWarning
\TestError
%</test2|test3>
%<test2>\end
%<test3>\@@end
%    \end{macrocode}
%
%
% \section{Installation}
%
% \subsection{Download}
%
% \paragraph{Package.} This package is available on
% CTAN\footnote{\url{ftp://ftp.ctan.org/tex-archive/}}:
% \begin{description}
% \item[\CTAN{macros/latex/contrib/oberdiek/infwarerr.dtx}] The source file.
% \item[\CTAN{macros/latex/contrib/oberdiek/infwarerr.pdf}] Documentation.
% \end{description}
%
%
% \paragraph{Bundle.} All the packages of the bundle `oberdiek'
% are also available in a TDS compliant ZIP archive. There
% the packages are already unpacked and the documentation files
% are generated. The files and directories obey the TDS standard.
% \begin{description}
% \item[\CTAN{install/macros/latex/contrib/oberdiek.tds.zip}]
% \end{description}
% \emph{TDS} refers to the standard ``A Directory Structure
% for \TeX\ Files'' (\CTAN{tds/tds.pdf}). Directories
% with \xfile{texmf} in their name are usually organized this way.
%
% \subsection{Bundle installation}
%
% \paragraph{Unpacking.} Unpack the \xfile{oberdiek.tds.zip} in the
% TDS tree (also known as \xfile{texmf} tree) of your choice.
% Example (linux):
% \begin{quote}
%   |unzip oberdiek.tds.zip -d ~/texmf|
% \end{quote}
%
% \paragraph{Script installation.}
% Check the directory \xfile{TDS:scripts/oberdiek/} for
% scripts that need further installation steps.
% Package \xpackage{attachfile2} comes with the Perl script
% \xfile{pdfatfi.pl} that should be installed in such a way
% that it can be called as \texttt{pdfatfi}.
% Example (linux):
% \begin{quote}
%   |chmod +x scripts/oberdiek/pdfatfi.pl|\\
%   |cp scripts/oberdiek/pdfatfi.pl /usr/local/bin/|
% \end{quote}
%
% \subsection{Package installation}
%
% \paragraph{Unpacking.} The \xfile{.dtx} file is a self-extracting
% \docstrip\ archive. The files are extracted by running the
% \xfile{.dtx} through \plainTeX:
% \begin{quote}
%   \verb|tex infwarerr.dtx|
% \end{quote}
%
% \paragraph{TDS.} Now the different files must be moved into
% the different directories in your installation TDS tree
% (also known as \xfile{texmf} tree):
% \begin{quote}
% \def\t{^^A
% \begin{tabular}{@{}>{\ttfamily}l@{ $\rightarrow$ }>{\ttfamily}l@{}}
%   infwarerr.sty & tex/generic/oberdiek/infwarerr.sty\\
%   infwarerr.pdf & doc/latex/oberdiek/infwarerr.pdf\\
%   test/infwarerr-test1.tex & doc/latex/oberdiek/test/infwarerr-test1.tex\\
%   test/infwarerr-test2.tex & doc/latex/oberdiek/test/infwarerr-test2.tex\\
%   test/infwarerr-test3.tex & doc/latex/oberdiek/test/infwarerr-test3.tex\\
%   infwarerr.dtx & source/latex/oberdiek/infwarerr.dtx\\
% \end{tabular}^^A
% }^^A
% \sbox0{\t}^^A
% \ifdim\wd0>\linewidth
%   \begingroup
%     \advance\linewidth by\leftmargin
%     \advance\linewidth by\rightmargin
%   \edef\x{\endgroup
%     \def\noexpand\lw{\the\linewidth}^^A
%   }\x
%   \def\lwbox{^^A
%     \leavevmode
%     \hbox to \linewidth{^^A
%       \kern-\leftmargin\relax
%       \hss
%       \usebox0
%       \hss
%       \kern-\rightmargin\relax
%     }^^A
%   }^^A
%   \ifdim\wd0>\lw
%     \sbox0{\small\t}^^A
%     \ifdim\wd0>\linewidth
%       \ifdim\wd0>\lw
%         \sbox0{\footnotesize\t}^^A
%         \ifdim\wd0>\linewidth
%           \ifdim\wd0>\lw
%             \sbox0{\scriptsize\t}^^A
%             \ifdim\wd0>\linewidth
%               \ifdim\wd0>\lw
%                 \sbox0{\tiny\t}^^A
%                 \ifdim\wd0>\linewidth
%                   \lwbox
%                 \else
%                   \usebox0
%                 \fi
%               \else
%                 \lwbox
%               \fi
%             \else
%               \usebox0
%             \fi
%           \else
%             \lwbox
%           \fi
%         \else
%           \usebox0
%         \fi
%       \else
%         \lwbox
%       \fi
%     \else
%       \usebox0
%     \fi
%   \else
%     \lwbox
%   \fi
% \else
%   \usebox0
% \fi
% \end{quote}
% If you have a \xfile{docstrip.cfg} that configures and enables \docstrip's
% TDS installing feature, then some files can already be in the right
% place, see the documentation of \docstrip.
%
% \subsection{Refresh file name databases}
%
% If your \TeX~distribution
% (\teTeX, \mikTeX, \dots) relies on file name databases, you must refresh
% these. For example, \teTeX\ users run \verb|texhash| or
% \verb|mktexlsr|.
%
% \subsection{Some details for the interested}
%
% \paragraph{Attached source.}
%
% The PDF documentation on CTAN also includes the
% \xfile{.dtx} source file. It can be extracted by
% AcrobatReader 6 or higher. Another option is \textsf{pdftk},
% e.g. unpack the file into the current directory:
% \begin{quote}
%   \verb|pdftk infwarerr.pdf unpack_files output .|
% \end{quote}
%
% \paragraph{Unpacking with \LaTeX.}
% The \xfile{.dtx} chooses its action depending on the format:
% \begin{description}
% \item[\plainTeX:] Run \docstrip\ and extract the files.
% \item[\LaTeX:] Generate the documentation.
% \end{description}
% If you insist on using \LaTeX\ for \docstrip\ (really,
% \docstrip\ does not need \LaTeX), then inform the autodetect routine
% about your intention:
% \begin{quote}
%   \verb|latex \let\install=y% \iffalse meta-comment
% 
% File: infwarerr.dtx
% Version: 2010/04/08 v1.3
% 
% Copyright (C) 2007, 2010 by
%    Heiko Oberdiek <heiko.oberdiek at googlemail.com>
% 
% This work may be distributed and/or modified under the
% conditions of the LaTeX Project Public License, either
% version 1.3c of this license or (at your option) any later
% version. This version of this license is in
%    http://www.latex-project.org/lppl/lppl-1-3c.txt
% and the latest version of this license is in
%    http://www.latex-project.org/lppl.txt
% and version 1.3 or later is part of all distributions of
% LaTeX version 2005/12/01 or later.
% 
% This work has the LPPL maintenance status "maintained".
% 
% This Current Maintainer of this work is Heiko Oberdiek.
% 
% The Base Interpreter refers to any `TeX-Format',
% because some files are installed in TDS:tex/generic//.
% 
% This work consists of the main source file infwarerr.dtx
% and the derived files
%    infwarerr.sty, infwarerr.pdf, infwarerr.ins, infwarerr.drv,
%    infwarerr-test1.tex, infwarerr-test2.tex,
%    infwarerr-test3.tex.
% 
% Distribution:
%    CTAN:macros/latex/contrib/oberdiek/infwarerr.dtx
%    CTAN:macros/latex/contrib/oberdiek/infwarerr.pdf
% 
% Unpacking:
%    (a) If infwarerr.ins is present:
%           tex infwarerr.ins
%    (b) Without infwarerr.ins:
%           tex infwarerr.dtx
%    (c) If you insist on using LaTeX
%           latex \let\install=y\input{infwarerr.dtx}
%        (quote the arguments according to the demands of your shell)
% 
% Documentation:
%    (a) If infwarerr.drv is present:
%           latex infwarerr.drv
%    (b) Without infwarerr.drv:
%           latex infwarerr.dtx; ...
%    The class ltxdoc loads the configuration file ltxdoc.cfg
%    if available. Here you can specify further options, e.g.
%    use A4 as paper format:
%       \PassOptionsToClass{a4paper}{article}
% 
%    Programm calls to get the documentation (example):
%       pdflatex infwarerr.dtx
%       makeindex -s gind.ist infwarerr.idx
%       pdflatex infwarerr.dtx
%       makeindex -s gind.ist infwarerr.idx
%       pdflatex infwarerr.dtx
% 
% Installation:
%    TDS:tex/generic/oberdiek/infwarerr.sty
%    TDS:doc/latex/oberdiek/infwarerr.pdf
%    TDS:doc/latex/oberdiek/test/infwarerr-test1.tex
%    TDS:doc/latex/oberdiek/test/infwarerr-test2.tex
%    TDS:doc/latex/oberdiek/test/infwarerr-test3.tex
%    TDS:source/latex/oberdiek/infwarerr.dtx
%
%<*ignore>
\begingroup
  \catcode123=1 %
  \catcode125=2 %
  \def\x{LaTeX2e}%
\expandafter\endgroup
\ifcase 0\ifx\install y1\fi\expandafter
         \ifx\csname processbatchFile\endcsname\relax\else1\fi
         \ifx\fmtname\x\else 1\fi\relax
\else\csname fi\endcsname
%</ignore>
%<*install>
\input docstrip.tex
\Msg{************************************************************************}
\Msg{* Installation}
\Msg{* Package: infwarerr 2010/04/08 v1.3 Providing info/warning/message (HO)}
\Msg{************************************************************************}

\keepsilent
\askforoverwritefalse

\let\MetaPrefix\relax
\preamble

This is a generated file.

Project: infwarerr
Version: 2010/04/08 v1.3

Copyright (C) 2007, 2010 by
   Heiko Oberdiek <heiko.oberdiek at googlemail.com>

This work may be distributed and/or modified under the
conditions of the LaTeX Project Public License, either
version 1.3c of this license or (at your option) any later
version. This version of this license is in
   http://www.latex-project.org/lppl/lppl-1-3c.txt
and the latest version of this license is in
   http://www.latex-project.org/lppl.txt
and version 1.3 or later is part of all distributions of
LaTeX version 2005/12/01 or later.

This work has the LPPL maintenance status "maintained".

This Current Maintainer of this work is Heiko Oberdiek.

The Base Interpreter refers to any `TeX-Format',
because some files are installed in TDS:tex/generic//.

This work consists of the main source file infwarerr.dtx
and the derived files
   infwarerr.sty, infwarerr.pdf, infwarerr.ins, infwarerr.drv,
   infwarerr-test1.tex, infwarerr-test2.tex,
   infwarerr-test3.tex.

\endpreamble
\let\MetaPrefix\DoubleperCent

\generate{%
  \file{infwarerr.ins}{\from{infwarerr.dtx}{install}}%
  \file{infwarerr.drv}{\from{infwarerr.dtx}{driver}}%
  \usedir{tex/generic/oberdiek}%
  \file{infwarerr.sty}{\from{infwarerr.dtx}{package}}%
  \usedir{doc/latex/oberdiek/test}%
  \file{infwarerr-test1.tex}{\from{infwarerr.dtx}{test1}}%
  \file{infwarerr-test2.tex}{\from{infwarerr.dtx}{test2}}%
  \file{infwarerr-test3.tex}{\from{infwarerr.dtx}{test3}}%
}

\catcode32=13\relax% active space
\let =\space%
\Msg{************************************************************************}
\Msg{*}
\Msg{* To finish the installation you have to move the following}
\Msg{* file into a directory searched by TeX:}
\Msg{*}
\Msg{*     infwarerr.sty}
\Msg{*}
\Msg{* To produce the documentation run the file `infwarerr.drv'}
\Msg{* through LaTeX.}
\Msg{*}
\Msg{* Happy TeXing!}
\Msg{*}
\Msg{************************************************************************}

\endbatchfile
%</install>
%<*ignore>
\fi
%</ignore>
%<*driver>
\NeedsTeXFormat{LaTeX2e}
\ProvidesFile{infwarerr.drv}%
  [2010/04/08 v1.3 Providing info/warning/message (HO)]%
\documentclass{ltxdoc}
\usepackage{holtxdoc}[2011/02/04]
\begin{document}
  \DocInput{infwarerr.dtx}%
\end{document}
%</driver>
% \fi
%
% \CheckSum{871}
%
% \CharacterTable
%  {Upper-case    \A\B\C\D\E\F\G\H\I\J\K\L\M\N\O\P\Q\R\S\T\U\V\W\X\Y\Z
%   Lower-case    \a\b\c\d\e\f\g\h\i\j\k\l\m\n\o\p\q\r\s\t\u\v\w\x\y\z
%   Digits        \0\1\2\3\4\5\6\7\8\9
%   Exclamation   \!     Double quote  \"     Hash (number) \#
%   Dollar        \$     Percent       \%     Ampersand     \&
%   Acute accent  \'     Left paren    \(     Right paren   \)
%   Asterisk      \*     Plus          \+     Comma         \,
%   Minus         \-     Point         \.     Solidus       \/
%   Colon         \:     Semicolon     \;     Less than     \<
%   Equals        \=     Greater than  \>     Question mark \?
%   Commercial at \@     Left bracket  \[     Backslash     \\
%   Right bracket \]     Circumflex    \^     Underscore    \_
%   Grave accent  \`     Left brace    \{     Vertical bar  \|
%   Right brace   \}     Tilde         \~}
%
% \GetFileInfo{infwarerr.drv}
%
% \title{The \xpackage{infwarerr} package}
% \date{2010/04/08 v1.3}
% \author{Heiko Oberdiek\\\xemail{heiko.oberdiek at googlemail.com}}
%
% \maketitle
%
% \begin{abstract}
% This package provides a complete set of macros for informations,
% warnings and error messages with support for \plainTeX.
% \end{abstract}
%
% \tableofcontents
%
% \section{Documentation}
%
% Package \xpackage{infwarerr} defines the following set of commands,
% given as pseudo regular expression:
%   \begin{quote}
%     \verb+\@(Package|Class)(Info|InfoNoLine|Warning|WarningNoLine|Error)+^^A
%     \kern-\rightmargin
%   \end{quote}
% The first argument is the package or class in question. The second
% argument contains the message text. \cs{MessageBreak} is supported.
% In case of error messages, the third argument contains the help
% text. Supported are \LaTeX's message texts \cs{@ehc}:
%   \begin{quote}
%     |Try typing  <return>  to proceed|\\
%     |If that doesn't work, type  X <return>  to quit.|
%   \end{quote}
% and \cs{@ehd}:
%   \begin{quote}
%     |You're in trouble here.  Try typing  <return>  to proceed|\\
%     |If that doesn't work, type  X <return>  to quit.|
%   \end{quote}
%
% The package is intended for macro writers that want to write
% code that works for both \LaTeX\ and \plainTeX. If \LaTeX\ is
% detected, then the macros of this package will only be wrappers
% for \LaTeX's macros. Otherwise \LaTeX's macros will be mimicked
% for use with \plainTeX.
%
% \StopEventually{
% }
%
% \section{Implementation}
%
%    \begin{macrocode}
%<*package>
%    \end{macrocode}
%
% \subsection{Reload check and package identification}
%    Reload check, especially if the package is not used with \LaTeX.
%    \begin{macrocode}
\begingroup\catcode61\catcode48\catcode32=10\relax%
  \catcode13=5 % ^^M
  \endlinechar=13 %
  \catcode35=6 % #
  \catcode39=12 % '
  \catcode44=12 % ,
  \catcode45=12 % -
  \catcode46=12 % .
  \catcode58=12 % :
  \catcode64=11 % @
  \catcode123=1 % {
  \catcode125=2 % }
  \expandafter\let\expandafter\x\csname ver@infwarerr.sty\endcsname
  \ifx\x\relax % plain-TeX, first loading
  \else
    \def\empty{}%
    \ifx\x\empty % LaTeX, first loading,
      % variable is initialized, but \ProvidesPackage not yet seen
    \else
      \expandafter\ifx\csname PackageInfo\endcsname\relax
        \def\x#1#2{%
          \immediate\write-1{Package #1 Info: #2.}%
        }%
      \else
        \def\x#1#2{\PackageInfo{#1}{#2, stopped}}%
      \fi
      \x{infwarerr}{The package is already loaded}%
      \aftergroup\endinput
    \fi
  \fi
\endgroup%
%    \end{macrocode}
%    Package identification:
%    \begin{macrocode}
\begingroup\catcode61\catcode48\catcode32=10\relax%
  \catcode13=5 % ^^M
  \endlinechar=13 %
  \catcode35=6 % #
  \catcode39=12 % '
  \catcode40=12 % (
  \catcode41=12 % )
  \catcode44=12 % ,
  \catcode45=12 % -
  \catcode46=12 % .
  \catcode47=12 % /
  \catcode58=12 % :
  \catcode64=11 % @
  \catcode91=12 % [
  \catcode93=12 % ]
  \catcode123=1 % {
  \catcode125=2 % }
  \expandafter\ifx\csname ProvidesPackage\endcsname\relax
    \def\x#1#2#3[#4]{\endgroup
      \immediate\write-1{Package: #3 #4}%
      \xdef#1{#4}%
    }%
  \else
    \def\x#1#2[#3]{\endgroup
      #2[{#3}]%
      \ifx#1\@undefined
        \xdef#1{#3}%
      \fi
      \ifx#1\relax
        \xdef#1{#3}%
      \fi
    }%
  \fi
\expandafter\x\csname ver@infwarerr.sty\endcsname
\ProvidesPackage{infwarerr}%
  [2010/04/08 v1.3 Providing info/warning/message (HO)]%
%    \end{macrocode}
%
% \subsection{Catcodes}
%
%    \begin{macrocode}
\begingroup\catcode61\catcode48\catcode32=10\relax%
  \catcode13=5 % ^^M
  \endlinechar=13 %
  \catcode123=1 % {
  \catcode125=2 % }
  \catcode64=11 % @
  \def\x{\endgroup
    \expandafter\edef\csname InfWarErr@AtEnd\endcsname{%
      \endlinechar=\the\endlinechar\relax
      \catcode13=\the\catcode13\relax
      \catcode32=\the\catcode32\relax
      \catcode35=\the\catcode35\relax
      \catcode61=\the\catcode61\relax
      \catcode64=\the\catcode64\relax
      \catcode123=\the\catcode123\relax
      \catcode125=\the\catcode125\relax
    }%
  }%
\x\catcode61\catcode48\catcode32=10\relax%
\catcode13=5 % ^^M
\endlinechar=13 %
\catcode35=6 % #
\catcode64=11 % @
\catcode123=1 % {
\catcode125=2 % }
\def\TMP@EnsureCode#1#2{%
  \edef\InfWarErr@AtEnd{%
    \InfWarErr@AtEnd
    \catcode#1=\the\catcode#1\relax
  }%
  \catcode#1=#2\relax
}
\TMP@EnsureCode{10}{12}% ^^J
\TMP@EnsureCode{39}{12}% '
\TMP@EnsureCode{40}{12}% (
\TMP@EnsureCode{41}{12}% )
\TMP@EnsureCode{44}{12}% ,
\TMP@EnsureCode{45}{12}% -
\TMP@EnsureCode{46}{12}% .
\TMP@EnsureCode{58}{12}% :
\TMP@EnsureCode{60}{12}% <
\TMP@EnsureCode{62}{12}% >
\TMP@EnsureCode{94}{7}% ^
\edef\InfWarErr@AtEnd{\InfWarErr@AtEnd\noexpand\endinput}
%    \end{macrocode}
%
% \subsection{\LaTeX\ detection}
%
%    \begin{macrocode}
\begingroup
  \let\x=Y%
  \def\check#1{%
    \@check{Generic}{#1}%
    \@check{Package}{#1}%
    \@check{Class}{#1}%
  }%
  \def\@check#1#2{%
    \expandafter\ifx\csname #1#2\endcsname\relax
      \let\x=N%
    \fi
  }%
  \check{Info}%
  \check{Warning}%
  \check{Error}%
  \@check{Package}{WarningNoLine}%
  \@check{Class}{WarningNoLine}%
  \@check{}{@ehc}%
  \@check{}{@ehd}%
\expandafter\endgroup
\ifx\x Y%
%    \end{macrocode}
%
% \subsection{Version for \LaTeX}
%
%    \begin{macro}{\@PackageInfo}
%    \begin{macrocode}
  \let\@PackageInfo\PackageInfo
%    \end{macrocode}
%    \end{macro}
%    \begin{macro}{\@PackageInfoNoLine}
%    \begin{macrocode}
  \def\@PackageInfoNoLine#1#2{%
    \PackageInfo{#1}{#2\@gobble}%
  }%
%    \end{macrocode}
%    \end{macro}
%    \begin{macro}{\@ClassInfo}
%    \begin{macrocode}
  \let\@ClassInfo\ClassInfo
%    \end{macrocode}
%    \end{macro}
%    \begin{macro}{\@ClassInfoNoLine}
%    \begin{macrocode}
  \def\@ClassInfoNoLine#1#2{%
    \ClassInfo{#1}{#2\@gobble}%
  }%
%    \end{macrocode}
%    \end{macro}
%    \begin{macro}{\@PackageWarning}
%    \begin{macrocode}
  \let\@PackageWarning\PackageWarning
%    \end{macrocode}
%    \end{macro}
%    \begin{macro}{\@PackageWarningNoLine}
%    \begin{macrocode}
  \let\@PackageWarningNoLine\PackageWarningNoLine
%    \end{macrocode}
%    \end{macro}
%    \begin{macro}{\@ClassWarning}
%    \begin{macrocode}
  \let\@ClassWarning\ClassWarning
%    \end{macrocode}
%    \end{macro}
%    \begin{macro}{\@ClassWarningNoLine}
%    \begin{macrocode}
  \let\@ClassWarningNoLine\ClassWarningNoLine
%    \end{macrocode}
%    \end{macro}
%    \begin{macro}{\@PackageError}
%    \begin{macrocode}
  \let\@PackageError\PackageError
%    \end{macrocode}
%    \end{macro}
%    \begin{macro}{\@ClassError}
%    \begin{macrocode}
  \let\@ClassError\ClassError
%    \end{macrocode}
%    \end{macro}
%    \begin{macrocode}
  \expandafter\InfWarErr@AtEnd
\fi%
%    \end{macrocode}
%
% \subsection{Version for \plainTeX}
%
%    \begin{macro}{\InfWarErr@online}
%    \begin{macrocode}
\begingroup\expandafter\endgroup
\ifcase\expandafter\ifx\csname inputlineno\endcsname\relax
         0 %
       \else
         \ifnum\inputlineno<0 %
           0 %
         \else
           1 %
         \fi
       \fi
  \def\InfWarErr@online{}%
\else
  \def\InfWarErr@online{ on input line \the\inputlineno}%
\fi
%    \end{macrocode}
%    \end{macro}
%
%    \begin{macro}{\InfWarErr@protected}
%    \begin{macrocode}
\begingroup\expandafter\expandafter\expandafter\endgroup
\expandafter\ifx\csname protected\endcsname\relax
  \def\InfWarErr@protected{}%
\else
  \let\InfWarErr@protected\protected
\fi
%    \end{macrocode}
%    \end{macro}
%
%    \begin{macro}{\InfWarErr@unused}
%    \begin{macrocode}
\chardef\InfWarErr@unused=16 %
%    \end{macrocode}
%    \end{macro}
%
%    \begin{macro}{\InfWarErr@ehc}
%    \begin{macrocode}
\def\InfWarErr@ehc{%
  Try typing \space <return> \space to proceed.\MessageBreak
  If that doesn't work, type \space X <return> \space to quit.%
}
%    \end{macrocode}
%    \end{macro}
%    \begin{macro}{\InfWarErr@ehd}
%    \begin{macrocode}
\def\InfWarErr@ehd{%
  You're in trouble here. \space\@ehc
}
%    \end{macrocode}
%    \end{macro}
%
%    \begin{macro}{\InfWarErr@GenericInfo}
%    \begin{macrocode}
\InfWarErr@protected\def\InfWarErr@GenericInfo#1#2{%
  \begingroup
    \let\on@line\InfWarErr@online
    \long\def\@gobble##1{}%
    \def\space{ }%
    \def\@spaces{\space\space\space\space}%
    \def\MessageBreak{^^J#1}%
    \newlinechar=10 %
    \escapechar=92 %
    \immediate\write-1{#2\on@line.}%
  \endgroup
}
%    \end{macrocode}
%    \end{macro}
%    \begin{macro}{\InfWarErr@GenericWarning}
%    \begin{macrocode}
\InfWarErr@protected\def\InfWarErr@GenericWarning#1#2{%
  \begingroup
    \let\on@line\InfWarErr@online
    \long\def\@gobble##1{}%
    \def\space{ }%
    \def\@spaces{\space\space\space\space}%
    \def\MessageBreak{^^J#1}%
    \newlinechar=10 %
    \escapechar=92 %
    \immediate\write\InfWarErr@unused{^^J#2\on@line.^^J}%
  \endgroup
}
%    \end{macrocode}
%    \end{macro}
%    \begin{macro}{\InfWarErr@GenericError}
%    Old \TeX\ versions below 3.141 where |^^J| is not working
%    as newline in error message and help texts are not supported.
%    \begin{macrocode}
\InfWarErr@protected\def\InfWarErr@GenericError#1#2#3#4{%
  \begingroup
    \immediate\write\InfWarErr@unused{}%
    \def\space{ }%
    \def\@spaces{\space\space\space\space}%
    \let\@ehc\InfWarErr@ehc
    \let\@ehd\InfWarErr@ehd
    \let\on@line\InfWarErr@online
    \def\MessageBreak{^^J}%
    \newlinechar=10 %
    \escapechar=92 %
    \edef\InfWarErr@help{\noexpand\errhelp{#4}}%
    \InfWarErr@help
    \def\MessageBreak{^^J#1}%
    \errmessage{%
      #2.^^J^^J%
      #3^^J%
      Type \space H <return> \space for immediate help.^^J ...^^J%
    }%
  \endgroup
}
%    \end{macrocode}
%    \end{macro}
%
%    \begin{macro}{\@PackageInfo}
%    \begin{macrocode}
\def\@PackageInfo#1#2{%
  \InfWarErr@GenericInfo{%
    (#1) \@spaces\@spaces\@spaces
  }{%
    Package #1 Info: #2%
  }%
}
%    \end{macrocode}
%    \end{macro}
%    \begin{macro}{\@PackageInfoNoLine}
%    \begin{macrocode}
\def\@PackageInfoNoLine#1#2{%
  \@PackageInfo{#1}{#2\@gobble}%
}
%    \end{macrocode}
%    \end{macro}
%    \begin{macro}{\@ClassInfo}
%    \begin{macrocode}
\def\@ClassInfo#1#2{%
  \InfWarErr@GenericInfo{%
    (#1) \space\space\@spaces\@spaces
  }{%
    Class #1 Info: #2%
  }%
}
%    \end{macrocode}
%    \end{macro}
%    \begin{macro}{\@ClassInfoNoLine}
%    \begin{macrocode}
\def\@ClassInfoNoLine#1#2{%
  \@ClassInfo{#1}{#2\@gobble}%
}
%    \end{macrocode}
%    \end{macro}
%    \begin{macro}{\@PackageWarning}
%    \begin{macrocode}
\def\@PackageWarning#1#2{%
  \InfWarErr@GenericWarning{%
    (#1)\@spaces\@spaces\@spaces\@spaces
  }{%
    Package #1 Warning: #2%
  }%
}
%    \end{macrocode}
%    \end{macro}
%    \begin{macro}{\@PackageWarningNoLine}
%    \begin{macrocode}
\def\@PackageWarningNoLine#1#2{%
  \@PackageWarning{#1}{#2\@gobble}%
}
%    \end{macrocode}
%    \end{macro}
%    \begin{macro}{\@ClassWarning}
%    \begin{macrocode}
\def\@ClassWarning#1#2{%
  \InfWarErr@GenericWarning{%
    (#1) \space\@spaces\@spaces\@spaces
  }{%
    Class #1 Warning: #2%
  }%
}
%    \end{macrocode}
%    \end{macro}
%    \begin{macro}{\@ClassWarningNoLine}
%    \begin{macrocode}
\def\@ClassWarningNoLine#1#2{%
  \@ClassWarning{#1}{#2\@gobble}%
}
%    \end{macrocode}
%    \end{macro}
%    \begin{macro}{\@PackageError}
%    \begin{macrocode}
\def\@PackageError#1#2#3{%
  \InfWarErr@GenericError{%
    (#1)\@spaces\@spaces\@spaces\@spaces
  }{%
    Package #1 Error: #2%
  }{%
    See the #1 package documentation for explanation.%
  }{#3}%
}
%    \end{macrocode}
%    \end{macro}
%    \begin{macro}{\@ClassError}
%    \begin{macrocode}
\def\@ClassError#1#2#3{%
  \InfWarErr@GenericError{%
    (#1) \space\@spaces\@spaces\@spaces
  }{%
    Class #1 Error: #2%
  }{%
    See the #1 class documentation for explanation.%
  }{#3}%
}
%    \end{macrocode}
%    \end{macro}
%
%    \begin{macrocode}
\InfWarErr@AtEnd%
%</package>
%    \end{macrocode}
%
% \section{Test}
%
% \subsection{Catcode checks for loading}
%
%    \begin{macrocode}
%<*test1>
%    \end{macrocode}
%    \begin{macrocode}
\catcode`\{=1 %
\catcode`\}=2 %
\catcode`\#=6 %
\catcode`\@=11 %
\expandafter\ifx\csname count@\endcsname\relax
  \countdef\count@=255 %
\fi
\expandafter\ifx\csname @gobble\endcsname\relax
  \long\def\@gobble#1{}%
\fi
\expandafter\ifx\csname @firstofone\endcsname\relax
  \long\def\@firstofone#1{#1}%
\fi
\expandafter\ifx\csname loop\endcsname\relax
  \expandafter\@firstofone
\else
  \expandafter\@gobble
\fi
{%
  \def\loop#1\repeat{%
    \def\body{#1}%
    \iterate
  }%
  \def\iterate{%
    \body
      \let\next\iterate
    \else
      \let\next\relax
    \fi
    \next
  }%
  \let\repeat=\fi
}%
\def\RestoreCatcodes{}
\count@=0 %
\loop
  \edef\RestoreCatcodes{%
    \RestoreCatcodes
    \catcode\the\count@=\the\catcode\count@\relax
  }%
\ifnum\count@<255 %
  \advance\count@ 1 %
\repeat

\def\RangeCatcodeInvalid#1#2{%
  \count@=#1\relax
  \loop
    \catcode\count@=15 %
  \ifnum\count@<#2\relax
    \advance\count@ 1 %
  \repeat
}
\def\RangeCatcodeCheck#1#2#3{%
  \count@=#1\relax
  \loop
    \ifnum#3=\catcode\count@
    \else
      \errmessage{%
        Character \the\count@\space
        with wrong catcode \the\catcode\count@\space
        instead of \number#3%
      }%
    \fi
  \ifnum\count@<#2\relax
    \advance\count@ 1 %
  \repeat
}
\def\space{ }
\expandafter\ifx\csname LoadCommand\endcsname\relax
  \def\LoadCommand{\input infwarerr.sty\relax}%
\fi
\def\Test{%
  \RangeCatcodeInvalid{0}{47}%
  \RangeCatcodeInvalid{58}{64}%
  \RangeCatcodeInvalid{91}{96}%
  \RangeCatcodeInvalid{123}{255}%
  \catcode`\@=12 %
  \catcode`\\=0 %
  \catcode`\%=14 %
  \LoadCommand
  \RangeCatcodeCheck{0}{36}{15}%
  \RangeCatcodeCheck{37}{37}{14}%
  \RangeCatcodeCheck{38}{47}{15}%
  \RangeCatcodeCheck{48}{57}{12}%
  \RangeCatcodeCheck{58}{63}{15}%
  \RangeCatcodeCheck{64}{64}{12}%
  \RangeCatcodeCheck{65}{90}{11}%
  \RangeCatcodeCheck{91}{91}{15}%
  \RangeCatcodeCheck{92}{92}{0}%
  \RangeCatcodeCheck{93}{96}{15}%
  \RangeCatcodeCheck{97}{122}{11}%
  \RangeCatcodeCheck{123}{255}{15}%
  \RestoreCatcodes
}
\Test
\csname @@end\endcsname
\end
%    \end{macrocode}
%    \begin{macrocode}
%</test1>
%    \end{macrocode}
%
% \subsection{Macro tests}
%
%    \begin{macrocode}
%<*test2>
\let\PackageInfo\relax
\input infwarerr.sty\relax
\let\OrgWrite\write
\chardef\WriteUnused=16 %
\def\msg#{\immediate\OrgWrite\WriteUnused}
\msg{File: infwarerr-test2.tex 2010/04/08 v1.3 Test file for plain-TeX}
\catcode64=11 %
%</test2>
%    \end{macrocode}
%
%    \begin{macrocode}
%<*test3>
\NeedsTeXFormat{LaTeX2e}
\ProvidesFile{infwarerr-test3.tex}[2010/04/08 v1.3 Test file for LaTeX]
\RequirePackage{infwarerr}[2010/04/08]
\makeatletter
\let\OrgWrite\write
\chardef\WriteUnused=\@unused
\def\msg#{\immediate\OrgWrite\WriteUnused}
%</test3>
%    \end{macrocode}
%
%    \begin{macrocode}
%<*test2|test3>
\def\pkgname{PACKAGE}
\def\clsname{CLASS}
\def\msgtext{MESSAGE\MessageBreak SECOND LINE}
\def\hlptext{HELP\MessageBreak SECOND LINE}

\let\OrgWrite\write
\let\OrgErrMessage\errmessage

\def\write#1#{%
  \@write{#1}%
}
\def\@write#1#2{%
  \OrgWrite#1{#2}%
  \xdef\WriteNum{\number#1}%
  \xdef\WriteText{#2}%
}
\def\TestFailed#1#2{%
  \OrgErrMessage{Test (\string#1) failed: #2}%
}
\def\TestWrite#1#2#3{%
  \ifnum\WriteNum=#2\relax
  \else
    \TestFailed#1{write stream number: \WriteNum <> #2}%
  \fi
  \begingroup
    \let~\space
    \edef\TestString{#3}%
    \ifx\WriteText\TestString
    \else
      \msg{}%
      \msg{[\WriteText]}%
      \msg{<>}%
      \msg{[#3]}% hash-ok
      \TestFailed#1{write text}%
    \fi
  \endgroup
}
\newtoks\ErrHelpToks
\def\errmessage#1{%
  \global\ErrHelpToks=\expandafter{\the\errhelp}%
  \msg{**************** error message ****************}%
  \msg{! #1}%
  \msg{**************** help message *****************}%
  \msg{\the\errhelp}%
  \msg{***********************************************}%
  \xdef\ErrMessage{! #1}%
}
\def\TestInfo{%
  \@PackageInfo\pkgname\msgtext
  \TestWrite\@PackageInfo{-1}{%
    Package PACKAGE Info: MESSAGE^^J%
    (PACKAGE)~~~~~~~~~~~~~SECOND LINE on input line \the\inputlineno.%
  }%
  \@PackageInfoNoLine\pkgname\msgtext
  \TestWrite\@PackageInfoNoLine{-1}{%
    Package PACKAGE Info: MESSAGE^^J%
    (PACKAGE)~~~~~~~~~~~~~SECOND LINE.%
  }%
  \@ClassInfo\clsname\msgtext
  \TestWrite\@ClassInfo{-1}{%
    Class CLASS Info: MESSAGE^^J%
    (CLASS)~~~~~~~~~~~SECOND LINE on input line \the\inputlineno.%
  }%
  \@ClassInfoNoLine\clsname\msgtext
  \TestWrite\@ClassInfoNoLine{-1}{%
    Class CLASS Info: MESSAGE^^J%
    (CLASS)~~~~~~~~~~~SECOND LINE.%
  }%
}
\def\TestWarning{%
  \@PackageWarning\pkgname\msgtext
  \TestWrite\@PackageWarning\WriteUnused{%
    ^^J%
    Package PACKAGE Warning: MESSAGE^^J%
    (PACKAGE)~~~~~~~~~~~~~~~~SECOND LINE on input line \the\inputlineno.%
    ^^J%
  }%
  \@PackageWarningNoLine\pkgname\msgtext
  \TestWrite\@PackageWarningNoLine\WriteUnused{%
    ^^J%
    Package PACKAGE Warning: MESSAGE^^J%
    (PACKAGE)~~~~~~~~~~~~~~~~SECOND LINE.%
    ^^J%
  }%
  \@ClassWarning\clsname\msgtext
  \TestWrite\@ClassWarning\WriteUnused{%
    ^^J%
    Class CLASS Warning: MESSAGE^^J%
    (CLASS)~~~~~~~~~~~~~~SECOND LINE on input line \the\inputlineno.%
    ^^J%
  }%
  \@ClassWarningNoLine\clsname\msgtext
  \TestWrite\@ClassWarningNoLine\WriteUnused{%
    ^^J%
    Class CLASS Warning: MESSAGE^^J%
    (CLASS)~~~~~~~~~~~~~~SECOND LINE.%
    ^^J%
  }%
}
\def\TestError{%
%<test3>  \let\write\OrgWrite
%<test3>  \def\plaindots{}
%<test2>  \def\plaindots{.^^J ...^^J}
  \@PackageError\pkgname\msgtext\hlptext
  \@TestError\@PackageError{%
    ! Package PACKAGE Error: MESSAGE^^J%
    (PACKAGE)~~~~~~~~~~~~~~~~SECOND LINE.^^J%
    ^^J%
    See the PACKAGE package documentation for explanation.^^J%
    Type~~H <return>~~for immediate help\plaindots
  }{%
    HELP^^J%
    SECOND LINE%
  }%
  \@ClassError\clsname\msgtext\hlptext
  \@TestError\@ClassError{%
    ! Class CLASS Error: MESSAGE^^J%
    (CLASS)~~~~~~~~~~~~~~SECOND LINE.^^J%
    ^^J%
    See the CLASS class documentation for explanation.^^J%
    Type~~H <return>~~for immediate help\plaindots
  }{%
    HELP^^J%
    SECOND LINE%
  }%
  \@PackageError\pkgname\msgtext\@ehc
  \@TestError\@PackageError{%
    ! Package PACKAGE Error: MESSAGE^^J%
    (PACKAGE)~~~~~~~~~~~~~~~~SECOND LINE.^^J%
    ^^J%
    See the PACKAGE package documentation for explanation.^^J%
    Type~~H <return>~~for immediate help\plaindots
  }{%
    Try typing~~<return>~~to proceed.^^J%
    If that doesn't work, type~~X <return>~~to quit.%
  }%
  \@PackageError\pkgname\msgtext\@ehd
  \@TestError\@PackageError{%
    ! Package PACKAGE Error: MESSAGE^^J%
    (PACKAGE)~~~~~~~~~~~~~~~~SECOND LINE.^^J%
    ^^J%
    See the PACKAGE package documentation for explanation.^^J%
    Type~~H <return>~~for immediate help\plaindots
  }{%
    You're in trouble here.~~Try typing~~<return>~~to proceed.^^J%
    If that doesn't work, type~~X <return>~~to quit.%
  }%
}
\def\@TestError#1#2#3{%
  \begingroup
    \let~\space
    \edef\x{#2}%
%<test3>  \@onelevel@sanitize\x
%<test3>  \@onelevel@sanitize\ErrMessage
    \ifx\x\ErrMessage
    \else
      \msg{}%
      \msg{[\ErrMessage]}%
      \msg{<>}%
      \msg{[\x]}%
      \TestFailed#1{errmessage text}%
    \fi
    \edef\x{\the\ErrHelpToks}%
    \edef\y{#3}%
    \ifx\x\y
    \else
      \msg{}%
      \msg{[\the\ErrHelpToks]}%
      \msg{<>}%
      \msg{[\y]}%
      \TestFailed#1{errhelp text}%
    \fi
  \endgroup
}
\TestInfo
\TestWarning
\TestError
%</test2|test3>
%<test2>\end
%<test3>\@@end
%    \end{macrocode}
%
%
% \section{Installation}
%
% \subsection{Download}
%
% \paragraph{Package.} This package is available on
% CTAN\footnote{\url{ftp://ftp.ctan.org/tex-archive/}}:
% \begin{description}
% \item[\CTAN{macros/latex/contrib/oberdiek/infwarerr.dtx}] The source file.
% \item[\CTAN{macros/latex/contrib/oberdiek/infwarerr.pdf}] Documentation.
% \end{description}
%
%
% \paragraph{Bundle.} All the packages of the bundle `oberdiek'
% are also available in a TDS compliant ZIP archive. There
% the packages are already unpacked and the documentation files
% are generated. The files and directories obey the TDS standard.
% \begin{description}
% \item[\CTAN{install/macros/latex/contrib/oberdiek.tds.zip}]
% \end{description}
% \emph{TDS} refers to the standard ``A Directory Structure
% for \TeX\ Files'' (\CTAN{tds/tds.pdf}). Directories
% with \xfile{texmf} in their name are usually organized this way.
%
% \subsection{Bundle installation}
%
% \paragraph{Unpacking.} Unpack the \xfile{oberdiek.tds.zip} in the
% TDS tree (also known as \xfile{texmf} tree) of your choice.
% Example (linux):
% \begin{quote}
%   |unzip oberdiek.tds.zip -d ~/texmf|
% \end{quote}
%
% \paragraph{Script installation.}
% Check the directory \xfile{TDS:scripts/oberdiek/} for
% scripts that need further installation steps.
% Package \xpackage{attachfile2} comes with the Perl script
% \xfile{pdfatfi.pl} that should be installed in such a way
% that it can be called as \texttt{pdfatfi}.
% Example (linux):
% \begin{quote}
%   |chmod +x scripts/oberdiek/pdfatfi.pl|\\
%   |cp scripts/oberdiek/pdfatfi.pl /usr/local/bin/|
% \end{quote}
%
% \subsection{Package installation}
%
% \paragraph{Unpacking.} The \xfile{.dtx} file is a self-extracting
% \docstrip\ archive. The files are extracted by running the
% \xfile{.dtx} through \plainTeX:
% \begin{quote}
%   \verb|tex infwarerr.dtx|
% \end{quote}
%
% \paragraph{TDS.} Now the different files must be moved into
% the different directories in your installation TDS tree
% (also known as \xfile{texmf} tree):
% \begin{quote}
% \def\t{^^A
% \begin{tabular}{@{}>{\ttfamily}l@{ $\rightarrow$ }>{\ttfamily}l@{}}
%   infwarerr.sty & tex/generic/oberdiek/infwarerr.sty\\
%   infwarerr.pdf & doc/latex/oberdiek/infwarerr.pdf\\
%   test/infwarerr-test1.tex & doc/latex/oberdiek/test/infwarerr-test1.tex\\
%   test/infwarerr-test2.tex & doc/latex/oberdiek/test/infwarerr-test2.tex\\
%   test/infwarerr-test3.tex & doc/latex/oberdiek/test/infwarerr-test3.tex\\
%   infwarerr.dtx & source/latex/oberdiek/infwarerr.dtx\\
% \end{tabular}^^A
% }^^A
% \sbox0{\t}^^A
% \ifdim\wd0>\linewidth
%   \begingroup
%     \advance\linewidth by\leftmargin
%     \advance\linewidth by\rightmargin
%   \edef\x{\endgroup
%     \def\noexpand\lw{\the\linewidth}^^A
%   }\x
%   \def\lwbox{^^A
%     \leavevmode
%     \hbox to \linewidth{^^A
%       \kern-\leftmargin\relax
%       \hss
%       \usebox0
%       \hss
%       \kern-\rightmargin\relax
%     }^^A
%   }^^A
%   \ifdim\wd0>\lw
%     \sbox0{\small\t}^^A
%     \ifdim\wd0>\linewidth
%       \ifdim\wd0>\lw
%         \sbox0{\footnotesize\t}^^A
%         \ifdim\wd0>\linewidth
%           \ifdim\wd0>\lw
%             \sbox0{\scriptsize\t}^^A
%             \ifdim\wd0>\linewidth
%               \ifdim\wd0>\lw
%                 \sbox0{\tiny\t}^^A
%                 \ifdim\wd0>\linewidth
%                   \lwbox
%                 \else
%                   \usebox0
%                 \fi
%               \else
%                 \lwbox
%               \fi
%             \else
%               \usebox0
%             \fi
%           \else
%             \lwbox
%           \fi
%         \else
%           \usebox0
%         \fi
%       \else
%         \lwbox
%       \fi
%     \else
%       \usebox0
%     \fi
%   \else
%     \lwbox
%   \fi
% \else
%   \usebox0
% \fi
% \end{quote}
% If you have a \xfile{docstrip.cfg} that configures and enables \docstrip's
% TDS installing feature, then some files can already be in the right
% place, see the documentation of \docstrip.
%
% \subsection{Refresh file name databases}
%
% If your \TeX~distribution
% (\teTeX, \mikTeX, \dots) relies on file name databases, you must refresh
% these. For example, \teTeX\ users run \verb|texhash| or
% \verb|mktexlsr|.
%
% \subsection{Some details for the interested}
%
% \paragraph{Attached source.}
%
% The PDF documentation on CTAN also includes the
% \xfile{.dtx} source file. It can be extracted by
% AcrobatReader 6 or higher. Another option is \textsf{pdftk},
% e.g. unpack the file into the current directory:
% \begin{quote}
%   \verb|pdftk infwarerr.pdf unpack_files output .|
% \end{quote}
%
% \paragraph{Unpacking with \LaTeX.}
% The \xfile{.dtx} chooses its action depending on the format:
% \begin{description}
% \item[\plainTeX:] Run \docstrip\ and extract the files.
% \item[\LaTeX:] Generate the documentation.
% \end{description}
% If you insist on using \LaTeX\ for \docstrip\ (really,
% \docstrip\ does not need \LaTeX), then inform the autodetect routine
% about your intention:
% \begin{quote}
%   \verb|latex \let\install=y\input{infwarerr.dtx}|
% \end{quote}
% Do not forget to quote the argument according to the demands
% of your shell.
%
% \paragraph{Generating the documentation.}
% You can use both the \xfile{.dtx} or the \xfile{.drv} to generate
% the documentation. The process can be configured by the
% configuration file \xfile{ltxdoc.cfg}. For instance, put this
% line into this file, if you want to have A4 as paper format:
% \begin{quote}
%   \verb|\PassOptionsToClass{a4paper}{article}|
% \end{quote}
% An example follows how to generate the
% documentation with pdf\LaTeX:
% \begin{quote}
%\begin{verbatim}
%pdflatex infwarerr.dtx
%makeindex -s gind.ist infwarerr.idx
%pdflatex infwarerr.dtx
%makeindex -s gind.ist infwarerr.idx
%pdflatex infwarerr.dtx
%\end{verbatim}
% \end{quote}
%
% \begin{History}
%   \begin{Version}{2007/04/26 v1.0}
%   \item
%     First version.
%   \end{Version}
%   \begin{Version}{2007/06/14 v1.1}
%   \item
%     Small fixes.
%   \end{Version}
%   \begin{Version}{2007/09/09 v1.2}
%   \item
%     Catcode section extended.
%   \end{Version}
%   \begin{Version}{2010/04/08 v1.3}
%   \item
%     Ensure \cs{escapechar}=92 (backslash) for infos/warnings/errors.
%   \end{Version}
% \end{History}
%
% \PrintIndex
%
% \Finale
\endinput
|
% \end{quote}
% Do not forget to quote the argument according to the demands
% of your shell.
%
% \paragraph{Generating the documentation.}
% You can use both the \xfile{.dtx} or the \xfile{.drv} to generate
% the documentation. The process can be configured by the
% configuration file \xfile{ltxdoc.cfg}. For instance, put this
% line into this file, if you want to have A4 as paper format:
% \begin{quote}
%   \verb|\PassOptionsToClass{a4paper}{article}|
% \end{quote}
% An example follows how to generate the
% documentation with pdf\LaTeX:
% \begin{quote}
%\begin{verbatim}
%pdflatex infwarerr.dtx
%makeindex -s gind.ist infwarerr.idx
%pdflatex infwarerr.dtx
%makeindex -s gind.ist infwarerr.idx
%pdflatex infwarerr.dtx
%\end{verbatim}
% \end{quote}
%
% \begin{History}
%   \begin{Version}{2007/04/26 v1.0}
%   \item
%     First version.
%   \end{Version}
%   \begin{Version}{2007/06/14 v1.1}
%   \item
%     Small fixes.
%   \end{Version}
%   \begin{Version}{2007/09/09 v1.2}
%   \item
%     Catcode section extended.
%   \end{Version}
%   \begin{Version}{2010/04/08 v1.3}
%   \item
%     Ensure \cs{escapechar}=92 (backslash) for infos/warnings/errors.
%   \end{Version}
% \end{History}
%
% \PrintIndex
%
% \Finale
\endinput
|
% \end{quote}
% Do not forget to quote the argument according to the demands
% of your shell.
%
% \paragraph{Generating the documentation.}
% You can use both the \xfile{.dtx} or the \xfile{.drv} to generate
% the documentation. The process can be configured by the
% configuration file \xfile{ltxdoc.cfg}. For instance, put this
% line into this file, if you want to have A4 as paper format:
% \begin{quote}
%   \verb|\PassOptionsToClass{a4paper}{article}|
% \end{quote}
% An example follows how to generate the
% documentation with pdf\LaTeX:
% \begin{quote}
%\begin{verbatim}
%pdflatex infwarerr.dtx
%makeindex -s gind.ist infwarerr.idx
%pdflatex infwarerr.dtx
%makeindex -s gind.ist infwarerr.idx
%pdflatex infwarerr.dtx
%\end{verbatim}
% \end{quote}
%
% \begin{History}
%   \begin{Version}{2007/04/26 v1.0}
%   \item
%     First version.
%   \end{Version}
%   \begin{Version}{2007/06/14 v1.1}
%   \item
%     Small fixes.
%   \end{Version}
%   \begin{Version}{2007/09/09 v1.2}
%   \item
%     Catcode section extended.
%   \end{Version}
%   \begin{Version}{2010/04/08 v1.3}
%   \item
%     Ensure \cs{escapechar}=92 (backslash) for infos/warnings/errors.
%   \end{Version}
% \end{History}
%
% \PrintIndex
%
% \Finale
\endinput

%        (quote the arguments according to the demands of your shell)
% 
% Documentation:
%    (a) If infwarerr.drv is present:
%           latex infwarerr.drv
%    (b) Without infwarerr.drv:
%           latex infwarerr.dtx; ...
%    The class ltxdoc loads the configuration file ltxdoc.cfg
%    if available. Here you can specify further options, e.g.
%    use A4 as paper format:
%       \PassOptionsToClass{a4paper}{article}
% 
%    Programm calls to get the documentation (example):
%       pdflatex infwarerr.dtx
%       makeindex -s gind.ist infwarerr.idx
%       pdflatex infwarerr.dtx
%       makeindex -s gind.ist infwarerr.idx
%       pdflatex infwarerr.dtx
% 
% Installation:
%    TDS:tex/generic/oberdiek/infwarerr.sty
%    TDS:doc/latex/oberdiek/infwarerr.pdf
%    TDS:doc/latex/oberdiek/test/infwarerr-test1.tex
%    TDS:doc/latex/oberdiek/test/infwarerr-test2.tex
%    TDS:doc/latex/oberdiek/test/infwarerr-test3.tex
%    TDS:source/latex/oberdiek/infwarerr.dtx
%
%<*ignore>
\begingroup
  \catcode123=1 %
  \catcode125=2 %
  \def\x{LaTeX2e}%
\expandafter\endgroup
\ifcase 0\ifx\install y1\fi\expandafter
         \ifx\csname processbatchFile\endcsname\relax\else1\fi
         \ifx\fmtname\x\else 1\fi\relax
\else\csname fi\endcsname
%</ignore>
%<*install>
\input docstrip.tex
\Msg{************************************************************************}
\Msg{* Installation}
\Msg{* Package: infwarerr 2010/04/08 v1.3 Providing info/warning/message (HO)}
\Msg{************************************************************************}

\keepsilent
\askforoverwritefalse

\let\MetaPrefix\relax
\preamble

This is a generated file.

Project: infwarerr
Version: 2010/04/08 v1.3

Copyright (C) 2007, 2010 by
   Heiko Oberdiek <heiko.oberdiek at googlemail.com>

This work may be distributed and/or modified under the
conditions of the LaTeX Project Public License, either
version 1.3c of this license or (at your option) any later
version. This version of this license is in
   http://www.latex-project.org/lppl/lppl-1-3c.txt
and the latest version of this license is in
   http://www.latex-project.org/lppl.txt
and version 1.3 or later is part of all distributions of
LaTeX version 2005/12/01 or later.

This work has the LPPL maintenance status "maintained".

This Current Maintainer of this work is Heiko Oberdiek.

The Base Interpreter refers to any `TeX-Format',
because some files are installed in TDS:tex/generic//.

This work consists of the main source file infwarerr.dtx
and the derived files
   infwarerr.sty, infwarerr.pdf, infwarerr.ins, infwarerr.drv,
   infwarerr-test1.tex, infwarerr-test2.tex,
   infwarerr-test3.tex.

\endpreamble
\let\MetaPrefix\DoubleperCent

\generate{%
  \file{infwarerr.ins}{\from{infwarerr.dtx}{install}}%
  \file{infwarerr.drv}{\from{infwarerr.dtx}{driver}}%
  \usedir{tex/generic/oberdiek}%
  \file{infwarerr.sty}{\from{infwarerr.dtx}{package}}%
  \usedir{doc/latex/oberdiek/test}%
  \file{infwarerr-test1.tex}{\from{infwarerr.dtx}{test1}}%
  \file{infwarerr-test2.tex}{\from{infwarerr.dtx}{test2}}%
  \file{infwarerr-test3.tex}{\from{infwarerr.dtx}{test3}}%
}

\catcode32=13\relax% active space
\let =\space%
\Msg{************************************************************************}
\Msg{*}
\Msg{* To finish the installation you have to move the following}
\Msg{* file into a directory searched by TeX:}
\Msg{*}
\Msg{*     infwarerr.sty}
\Msg{*}
\Msg{* To produce the documentation run the file `infwarerr.drv'}
\Msg{* through LaTeX.}
\Msg{*}
\Msg{* Happy TeXing!}
\Msg{*}
\Msg{************************************************************************}

\endbatchfile
%</install>
%<*ignore>
\fi
%</ignore>
%<*driver>
\NeedsTeXFormat{LaTeX2e}
\ProvidesFile{infwarerr.drv}%
  [2010/04/08 v1.3 Providing info/warning/message (HO)]%
\documentclass{ltxdoc}
\usepackage{holtxdoc}[2011/02/04]
\begin{document}
  \DocInput{infwarerr.dtx}%
\end{document}
%</driver>
% \fi
%
% \CheckSum{871}
%
% \CharacterTable
%  {Upper-case    \A\B\C\D\E\F\G\H\I\J\K\L\M\N\O\P\Q\R\S\T\U\V\W\X\Y\Z
%   Lower-case    \a\b\c\d\e\f\g\h\i\j\k\l\m\n\o\p\q\r\s\t\u\v\w\x\y\z
%   Digits        \0\1\2\3\4\5\6\7\8\9
%   Exclamation   \!     Double quote  \"     Hash (number) \#
%   Dollar        \$     Percent       \%     Ampersand     \&
%   Acute accent  \'     Left paren    \(     Right paren   \)
%   Asterisk      \*     Plus          \+     Comma         \,
%   Minus         \-     Point         \.     Solidus       \/
%   Colon         \:     Semicolon     \;     Less than     \<
%   Equals        \=     Greater than  \>     Question mark \?
%   Commercial at \@     Left bracket  \[     Backslash     \\
%   Right bracket \]     Circumflex    \^     Underscore    \_
%   Grave accent  \`     Left brace    \{     Vertical bar  \|
%   Right brace   \}     Tilde         \~}
%
% \GetFileInfo{infwarerr.drv}
%
% \title{The \xpackage{infwarerr} package}
% \date{2010/04/08 v1.3}
% \author{Heiko Oberdiek\\\xemail{heiko.oberdiek at googlemail.com}}
%
% \maketitle
%
% \begin{abstract}
% This package provides a complete set of macros for informations,
% warnings and error messages with support for \plainTeX.
% \end{abstract}
%
% \tableofcontents
%
% \section{Documentation}
%
% Package \xpackage{infwarerr} defines the following set of commands,
% given as pseudo regular expression:
%   \begin{quote}
%     \verb+\@(Package|Class)(Info|InfoNoLine|Warning|WarningNoLine|Error)+^^A
%     \kern-\rightmargin
%   \end{quote}
% The first argument is the package or class in question. The second
% argument contains the message text. \cs{MessageBreak} is supported.
% In case of error messages, the third argument contains the help
% text. Supported are \LaTeX's message texts \cs{@ehc}:
%   \begin{quote}
%     |Try typing  <return>  to proceed|\\
%     |If that doesn't work, type  X <return>  to quit.|
%   \end{quote}
% and \cs{@ehd}:
%   \begin{quote}
%     |You're in trouble here.  Try typing  <return>  to proceed|\\
%     |If that doesn't work, type  X <return>  to quit.|
%   \end{quote}
%
% The package is intended for macro writers that want to write
% code that works for both \LaTeX\ and \plainTeX. If \LaTeX\ is
% detected, then the macros of this package will only be wrappers
% for \LaTeX's macros. Otherwise \LaTeX's macros will be mimicked
% for use with \plainTeX.
%
% \StopEventually{
% }
%
% \section{Implementation}
%
%    \begin{macrocode}
%<*package>
%    \end{macrocode}
%
% \subsection{Reload check and package identification}
%    Reload check, especially if the package is not used with \LaTeX.
%    \begin{macrocode}
\begingroup\catcode61\catcode48\catcode32=10\relax%
  \catcode13=5 % ^^M
  \endlinechar=13 %
  \catcode35=6 % #
  \catcode39=12 % '
  \catcode44=12 % ,
  \catcode45=12 % -
  \catcode46=12 % .
  \catcode58=12 % :
  \catcode64=11 % @
  \catcode123=1 % {
  \catcode125=2 % }
  \expandafter\let\expandafter\x\csname ver@infwarerr.sty\endcsname
  \ifx\x\relax % plain-TeX, first loading
  \else
    \def\empty{}%
    \ifx\x\empty % LaTeX, first loading,
      % variable is initialized, but \ProvidesPackage not yet seen
    \else
      \expandafter\ifx\csname PackageInfo\endcsname\relax
        \def\x#1#2{%
          \immediate\write-1{Package #1 Info: #2.}%
        }%
      \else
        \def\x#1#2{\PackageInfo{#1}{#2, stopped}}%
      \fi
      \x{infwarerr}{The package is already loaded}%
      \aftergroup\endinput
    \fi
  \fi
\endgroup%
%    \end{macrocode}
%    Package identification:
%    \begin{macrocode}
\begingroup\catcode61\catcode48\catcode32=10\relax%
  \catcode13=5 % ^^M
  \endlinechar=13 %
  \catcode35=6 % #
  \catcode39=12 % '
  \catcode40=12 % (
  \catcode41=12 % )
  \catcode44=12 % ,
  \catcode45=12 % -
  \catcode46=12 % .
  \catcode47=12 % /
  \catcode58=12 % :
  \catcode64=11 % @
  \catcode91=12 % [
  \catcode93=12 % ]
  \catcode123=1 % {
  \catcode125=2 % }
  \expandafter\ifx\csname ProvidesPackage\endcsname\relax
    \def\x#1#2#3[#4]{\endgroup
      \immediate\write-1{Package: #3 #4}%
      \xdef#1{#4}%
    }%
  \else
    \def\x#1#2[#3]{\endgroup
      #2[{#3}]%
      \ifx#1\@undefined
        \xdef#1{#3}%
      \fi
      \ifx#1\relax
        \xdef#1{#3}%
      \fi
    }%
  \fi
\expandafter\x\csname ver@infwarerr.sty\endcsname
\ProvidesPackage{infwarerr}%
  [2010/04/08 v1.3 Providing info/warning/message (HO)]%
%    \end{macrocode}
%
% \subsection{Catcodes}
%
%    \begin{macrocode}
\begingroup\catcode61\catcode48\catcode32=10\relax%
  \catcode13=5 % ^^M
  \endlinechar=13 %
  \catcode123=1 % {
  \catcode125=2 % }
  \catcode64=11 % @
  \def\x{\endgroup
    \expandafter\edef\csname InfWarErr@AtEnd\endcsname{%
      \endlinechar=\the\endlinechar\relax
      \catcode13=\the\catcode13\relax
      \catcode32=\the\catcode32\relax
      \catcode35=\the\catcode35\relax
      \catcode61=\the\catcode61\relax
      \catcode64=\the\catcode64\relax
      \catcode123=\the\catcode123\relax
      \catcode125=\the\catcode125\relax
    }%
  }%
\x\catcode61\catcode48\catcode32=10\relax%
\catcode13=5 % ^^M
\endlinechar=13 %
\catcode35=6 % #
\catcode64=11 % @
\catcode123=1 % {
\catcode125=2 % }
\def\TMP@EnsureCode#1#2{%
  \edef\InfWarErr@AtEnd{%
    \InfWarErr@AtEnd
    \catcode#1=\the\catcode#1\relax
  }%
  \catcode#1=#2\relax
}
\TMP@EnsureCode{10}{12}% ^^J
\TMP@EnsureCode{39}{12}% '
\TMP@EnsureCode{40}{12}% (
\TMP@EnsureCode{41}{12}% )
\TMP@EnsureCode{44}{12}% ,
\TMP@EnsureCode{45}{12}% -
\TMP@EnsureCode{46}{12}% .
\TMP@EnsureCode{58}{12}% :
\TMP@EnsureCode{60}{12}% <
\TMP@EnsureCode{62}{12}% >
\TMP@EnsureCode{94}{7}% ^
\edef\InfWarErr@AtEnd{\InfWarErr@AtEnd\noexpand\endinput}
%    \end{macrocode}
%
% \subsection{\LaTeX\ detection}
%
%    \begin{macrocode}
\begingroup
  \let\x=Y%
  \def\check#1{%
    \@check{Generic}{#1}%
    \@check{Package}{#1}%
    \@check{Class}{#1}%
  }%
  \def\@check#1#2{%
    \expandafter\ifx\csname #1#2\endcsname\relax
      \let\x=N%
    \fi
  }%
  \check{Info}%
  \check{Warning}%
  \check{Error}%
  \@check{Package}{WarningNoLine}%
  \@check{Class}{WarningNoLine}%
  \@check{}{@ehc}%
  \@check{}{@ehd}%
\expandafter\endgroup
\ifx\x Y%
%    \end{macrocode}
%
% \subsection{Version for \LaTeX}
%
%    \begin{macro}{\@PackageInfo}
%    \begin{macrocode}
  \let\@PackageInfo\PackageInfo
%    \end{macrocode}
%    \end{macro}
%    \begin{macro}{\@PackageInfoNoLine}
%    \begin{macrocode}
  \def\@PackageInfoNoLine#1#2{%
    \PackageInfo{#1}{#2\@gobble}%
  }%
%    \end{macrocode}
%    \end{macro}
%    \begin{macro}{\@ClassInfo}
%    \begin{macrocode}
  \let\@ClassInfo\ClassInfo
%    \end{macrocode}
%    \end{macro}
%    \begin{macro}{\@ClassInfoNoLine}
%    \begin{macrocode}
  \def\@ClassInfoNoLine#1#2{%
    \ClassInfo{#1}{#2\@gobble}%
  }%
%    \end{macrocode}
%    \end{macro}
%    \begin{macro}{\@PackageWarning}
%    \begin{macrocode}
  \let\@PackageWarning\PackageWarning
%    \end{macrocode}
%    \end{macro}
%    \begin{macro}{\@PackageWarningNoLine}
%    \begin{macrocode}
  \let\@PackageWarningNoLine\PackageWarningNoLine
%    \end{macrocode}
%    \end{macro}
%    \begin{macro}{\@ClassWarning}
%    \begin{macrocode}
  \let\@ClassWarning\ClassWarning
%    \end{macrocode}
%    \end{macro}
%    \begin{macro}{\@ClassWarningNoLine}
%    \begin{macrocode}
  \let\@ClassWarningNoLine\ClassWarningNoLine
%    \end{macrocode}
%    \end{macro}
%    \begin{macro}{\@PackageError}
%    \begin{macrocode}
  \let\@PackageError\PackageError
%    \end{macrocode}
%    \end{macro}
%    \begin{macro}{\@ClassError}
%    \begin{macrocode}
  \let\@ClassError\ClassError
%    \end{macrocode}
%    \end{macro}
%    \begin{macrocode}
  \expandafter\InfWarErr@AtEnd
\fi%
%    \end{macrocode}
%
% \subsection{Version for \plainTeX}
%
%    \begin{macro}{\InfWarErr@online}
%    \begin{macrocode}
\begingroup\expandafter\endgroup
\ifcase\expandafter\ifx\csname inputlineno\endcsname\relax
         0 %
       \else
         \ifnum\inputlineno<0 %
           0 %
         \else
           1 %
         \fi
       \fi
  \def\InfWarErr@online{}%
\else
  \def\InfWarErr@online{ on input line \the\inputlineno}%
\fi
%    \end{macrocode}
%    \end{macro}
%
%    \begin{macro}{\InfWarErr@protected}
%    \begin{macrocode}
\begingroup\expandafter\expandafter\expandafter\endgroup
\expandafter\ifx\csname protected\endcsname\relax
  \def\InfWarErr@protected{}%
\else
  \let\InfWarErr@protected\protected
\fi
%    \end{macrocode}
%    \end{macro}
%
%    \begin{macro}{\InfWarErr@unused}
%    \begin{macrocode}
\chardef\InfWarErr@unused=16 %
%    \end{macrocode}
%    \end{macro}
%
%    \begin{macro}{\InfWarErr@ehc}
%    \begin{macrocode}
\def\InfWarErr@ehc{%
  Try typing \space <return> \space to proceed.\MessageBreak
  If that doesn't work, type \space X <return> \space to quit.%
}
%    \end{macrocode}
%    \end{macro}
%    \begin{macro}{\InfWarErr@ehd}
%    \begin{macrocode}
\def\InfWarErr@ehd{%
  You're in trouble here. \space\@ehc
}
%    \end{macrocode}
%    \end{macro}
%
%    \begin{macro}{\InfWarErr@GenericInfo}
%    \begin{macrocode}
\InfWarErr@protected\def\InfWarErr@GenericInfo#1#2{%
  \begingroup
    \let\on@line\InfWarErr@online
    \long\def\@gobble##1{}%
    \def\space{ }%
    \def\@spaces{\space\space\space\space}%
    \def\MessageBreak{^^J#1}%
    \newlinechar=10 %
    \escapechar=92 %
    \immediate\write-1{#2\on@line.}%
  \endgroup
}
%    \end{macrocode}
%    \end{macro}
%    \begin{macro}{\InfWarErr@GenericWarning}
%    \begin{macrocode}
\InfWarErr@protected\def\InfWarErr@GenericWarning#1#2{%
  \begingroup
    \let\on@line\InfWarErr@online
    \long\def\@gobble##1{}%
    \def\space{ }%
    \def\@spaces{\space\space\space\space}%
    \def\MessageBreak{^^J#1}%
    \newlinechar=10 %
    \escapechar=92 %
    \immediate\write\InfWarErr@unused{^^J#2\on@line.^^J}%
  \endgroup
}
%    \end{macrocode}
%    \end{macro}
%    \begin{macro}{\InfWarErr@GenericError}
%    Old \TeX\ versions below 3.141 where |^^J| is not working
%    as newline in error message and help texts are not supported.
%    \begin{macrocode}
\InfWarErr@protected\def\InfWarErr@GenericError#1#2#3#4{%
  \begingroup
    \immediate\write\InfWarErr@unused{}%
    \def\space{ }%
    \def\@spaces{\space\space\space\space}%
    \let\@ehc\InfWarErr@ehc
    \let\@ehd\InfWarErr@ehd
    \let\on@line\InfWarErr@online
    \def\MessageBreak{^^J}%
    \newlinechar=10 %
    \escapechar=92 %
    \edef\InfWarErr@help{\noexpand\errhelp{#4}}%
    \InfWarErr@help
    \def\MessageBreak{^^J#1}%
    \errmessage{%
      #2.^^J^^J%
      #3^^J%
      Type \space H <return> \space for immediate help.^^J ...^^J%
    }%
  \endgroup
}
%    \end{macrocode}
%    \end{macro}
%
%    \begin{macro}{\@PackageInfo}
%    \begin{macrocode}
\def\@PackageInfo#1#2{%
  \InfWarErr@GenericInfo{%
    (#1) \@spaces\@spaces\@spaces
  }{%
    Package #1 Info: #2%
  }%
}
%    \end{macrocode}
%    \end{macro}
%    \begin{macro}{\@PackageInfoNoLine}
%    \begin{macrocode}
\def\@PackageInfoNoLine#1#2{%
  \@PackageInfo{#1}{#2\@gobble}%
}
%    \end{macrocode}
%    \end{macro}
%    \begin{macro}{\@ClassInfo}
%    \begin{macrocode}
\def\@ClassInfo#1#2{%
  \InfWarErr@GenericInfo{%
    (#1) \space\space\@spaces\@spaces
  }{%
    Class #1 Info: #2%
  }%
}
%    \end{macrocode}
%    \end{macro}
%    \begin{macro}{\@ClassInfoNoLine}
%    \begin{macrocode}
\def\@ClassInfoNoLine#1#2{%
  \@ClassInfo{#1}{#2\@gobble}%
}
%    \end{macrocode}
%    \end{macro}
%    \begin{macro}{\@PackageWarning}
%    \begin{macrocode}
\def\@PackageWarning#1#2{%
  \InfWarErr@GenericWarning{%
    (#1)\@spaces\@spaces\@spaces\@spaces
  }{%
    Package #1 Warning: #2%
  }%
}
%    \end{macrocode}
%    \end{macro}
%    \begin{macro}{\@PackageWarningNoLine}
%    \begin{macrocode}
\def\@PackageWarningNoLine#1#2{%
  \@PackageWarning{#1}{#2\@gobble}%
}
%    \end{macrocode}
%    \end{macro}
%    \begin{macro}{\@ClassWarning}
%    \begin{macrocode}
\def\@ClassWarning#1#2{%
  \InfWarErr@GenericWarning{%
    (#1) \space\@spaces\@spaces\@spaces
  }{%
    Class #1 Warning: #2%
  }%
}
%    \end{macrocode}
%    \end{macro}
%    \begin{macro}{\@ClassWarningNoLine}
%    \begin{macrocode}
\def\@ClassWarningNoLine#1#2{%
  \@ClassWarning{#1}{#2\@gobble}%
}
%    \end{macrocode}
%    \end{macro}
%    \begin{macro}{\@PackageError}
%    \begin{macrocode}
\def\@PackageError#1#2#3{%
  \InfWarErr@GenericError{%
    (#1)\@spaces\@spaces\@spaces\@spaces
  }{%
    Package #1 Error: #2%
  }{%
    See the #1 package documentation for explanation.%
  }{#3}%
}
%    \end{macrocode}
%    \end{macro}
%    \begin{macro}{\@ClassError}
%    \begin{macrocode}
\def\@ClassError#1#2#3{%
  \InfWarErr@GenericError{%
    (#1) \space\@spaces\@spaces\@spaces
  }{%
    Class #1 Error: #2%
  }{%
    See the #1 class documentation for explanation.%
  }{#3}%
}
%    \end{macrocode}
%    \end{macro}
%
%    \begin{macrocode}
\InfWarErr@AtEnd%
%</package>
%    \end{macrocode}
%
% \section{Test}
%
% \subsection{Catcode checks for loading}
%
%    \begin{macrocode}
%<*test1>
%    \end{macrocode}
%    \begin{macrocode}
\catcode`\{=1 %
\catcode`\}=2 %
\catcode`\#=6 %
\catcode`\@=11 %
\expandafter\ifx\csname count@\endcsname\relax
  \countdef\count@=255 %
\fi
\expandafter\ifx\csname @gobble\endcsname\relax
  \long\def\@gobble#1{}%
\fi
\expandafter\ifx\csname @firstofone\endcsname\relax
  \long\def\@firstofone#1{#1}%
\fi
\expandafter\ifx\csname loop\endcsname\relax
  \expandafter\@firstofone
\else
  \expandafter\@gobble
\fi
{%
  \def\loop#1\repeat{%
    \def\body{#1}%
    \iterate
  }%
  \def\iterate{%
    \body
      \let\next\iterate
    \else
      \let\next\relax
    \fi
    \next
  }%
  \let\repeat=\fi
}%
\def\RestoreCatcodes{}
\count@=0 %
\loop
  \edef\RestoreCatcodes{%
    \RestoreCatcodes
    \catcode\the\count@=\the\catcode\count@\relax
  }%
\ifnum\count@<255 %
  \advance\count@ 1 %
\repeat

\def\RangeCatcodeInvalid#1#2{%
  \count@=#1\relax
  \loop
    \catcode\count@=15 %
  \ifnum\count@<#2\relax
    \advance\count@ 1 %
  \repeat
}
\def\RangeCatcodeCheck#1#2#3{%
  \count@=#1\relax
  \loop
    \ifnum#3=\catcode\count@
    \else
      \errmessage{%
        Character \the\count@\space
        with wrong catcode \the\catcode\count@\space
        instead of \number#3%
      }%
    \fi
  \ifnum\count@<#2\relax
    \advance\count@ 1 %
  \repeat
}
\def\space{ }
\expandafter\ifx\csname LoadCommand\endcsname\relax
  \def\LoadCommand{\input infwarerr.sty\relax}%
\fi
\def\Test{%
  \RangeCatcodeInvalid{0}{47}%
  \RangeCatcodeInvalid{58}{64}%
  \RangeCatcodeInvalid{91}{96}%
  \RangeCatcodeInvalid{123}{255}%
  \catcode`\@=12 %
  \catcode`\\=0 %
  \catcode`\%=14 %
  \LoadCommand
  \RangeCatcodeCheck{0}{36}{15}%
  \RangeCatcodeCheck{37}{37}{14}%
  \RangeCatcodeCheck{38}{47}{15}%
  \RangeCatcodeCheck{48}{57}{12}%
  \RangeCatcodeCheck{58}{63}{15}%
  \RangeCatcodeCheck{64}{64}{12}%
  \RangeCatcodeCheck{65}{90}{11}%
  \RangeCatcodeCheck{91}{91}{15}%
  \RangeCatcodeCheck{92}{92}{0}%
  \RangeCatcodeCheck{93}{96}{15}%
  \RangeCatcodeCheck{97}{122}{11}%
  \RangeCatcodeCheck{123}{255}{15}%
  \RestoreCatcodes
}
\Test
\csname @@end\endcsname
\end
%    \end{macrocode}
%    \begin{macrocode}
%</test1>
%    \end{macrocode}
%
% \subsection{Macro tests}
%
%    \begin{macrocode}
%<*test2>
\let\PackageInfo\relax
\input infwarerr.sty\relax
\let\OrgWrite\write
\chardef\WriteUnused=16 %
\def\msg#{\immediate\OrgWrite\WriteUnused}
\msg{File: infwarerr-test2.tex 2010/04/08 v1.3 Test file for plain-TeX}
\catcode64=11 %
%</test2>
%    \end{macrocode}
%
%    \begin{macrocode}
%<*test3>
\NeedsTeXFormat{LaTeX2e}
\ProvidesFile{infwarerr-test3.tex}[2010/04/08 v1.3 Test file for LaTeX]
\RequirePackage{infwarerr}[2010/04/08]
\makeatletter
\let\OrgWrite\write
\chardef\WriteUnused=\@unused
\def\msg#{\immediate\OrgWrite\WriteUnused}
%</test3>
%    \end{macrocode}
%
%    \begin{macrocode}
%<*test2|test3>
\def\pkgname{PACKAGE}
\def\clsname{CLASS}
\def\msgtext{MESSAGE\MessageBreak SECOND LINE}
\def\hlptext{HELP\MessageBreak SECOND LINE}

\let\OrgWrite\write
\let\OrgErrMessage\errmessage

\def\write#1#{%
  \@write{#1}%
}
\def\@write#1#2{%
  \OrgWrite#1{#2}%
  \xdef\WriteNum{\number#1}%
  \xdef\WriteText{#2}%
}
\def\TestFailed#1#2{%
  \OrgErrMessage{Test (\string#1) failed: #2}%
}
\def\TestWrite#1#2#3{%
  \ifnum\WriteNum=#2\relax
  \else
    \TestFailed#1{write stream number: \WriteNum <> #2}%
  \fi
  \begingroup
    \let~\space
    \edef\TestString{#3}%
    \ifx\WriteText\TestString
    \else
      \msg{}%
      \msg{[\WriteText]}%
      \msg{<>}%
      \msg{[#3]}% hash-ok
      \TestFailed#1{write text}%
    \fi
  \endgroup
}
\newtoks\ErrHelpToks
\def\errmessage#1{%
  \global\ErrHelpToks=\expandafter{\the\errhelp}%
  \msg{**************** error message ****************}%
  \msg{! #1}%
  \msg{**************** help message *****************}%
  \msg{\the\errhelp}%
  \msg{***********************************************}%
  \xdef\ErrMessage{! #1}%
}
\def\TestInfo{%
  \@PackageInfo\pkgname\msgtext
  \TestWrite\@PackageInfo{-1}{%
    Package PACKAGE Info: MESSAGE^^J%
    (PACKAGE)~~~~~~~~~~~~~SECOND LINE on input line \the\inputlineno.%
  }%
  \@PackageInfoNoLine\pkgname\msgtext
  \TestWrite\@PackageInfoNoLine{-1}{%
    Package PACKAGE Info: MESSAGE^^J%
    (PACKAGE)~~~~~~~~~~~~~SECOND LINE.%
  }%
  \@ClassInfo\clsname\msgtext
  \TestWrite\@ClassInfo{-1}{%
    Class CLASS Info: MESSAGE^^J%
    (CLASS)~~~~~~~~~~~SECOND LINE on input line \the\inputlineno.%
  }%
  \@ClassInfoNoLine\clsname\msgtext
  \TestWrite\@ClassInfoNoLine{-1}{%
    Class CLASS Info: MESSAGE^^J%
    (CLASS)~~~~~~~~~~~SECOND LINE.%
  }%
}
\def\TestWarning{%
  \@PackageWarning\pkgname\msgtext
  \TestWrite\@PackageWarning\WriteUnused{%
    ^^J%
    Package PACKAGE Warning: MESSAGE^^J%
    (PACKAGE)~~~~~~~~~~~~~~~~SECOND LINE on input line \the\inputlineno.%
    ^^J%
  }%
  \@PackageWarningNoLine\pkgname\msgtext
  \TestWrite\@PackageWarningNoLine\WriteUnused{%
    ^^J%
    Package PACKAGE Warning: MESSAGE^^J%
    (PACKAGE)~~~~~~~~~~~~~~~~SECOND LINE.%
    ^^J%
  }%
  \@ClassWarning\clsname\msgtext
  \TestWrite\@ClassWarning\WriteUnused{%
    ^^J%
    Class CLASS Warning: MESSAGE^^J%
    (CLASS)~~~~~~~~~~~~~~SECOND LINE on input line \the\inputlineno.%
    ^^J%
  }%
  \@ClassWarningNoLine\clsname\msgtext
  \TestWrite\@ClassWarningNoLine\WriteUnused{%
    ^^J%
    Class CLASS Warning: MESSAGE^^J%
    (CLASS)~~~~~~~~~~~~~~SECOND LINE.%
    ^^J%
  }%
}
\def\TestError{%
%<test3>  \let\write\OrgWrite
%<test3>  \def\plaindots{}
%<test2>  \def\plaindots{.^^J ...^^J}
  \@PackageError\pkgname\msgtext\hlptext
  \@TestError\@PackageError{%
    ! Package PACKAGE Error: MESSAGE^^J%
    (PACKAGE)~~~~~~~~~~~~~~~~SECOND LINE.^^J%
    ^^J%
    See the PACKAGE package documentation for explanation.^^J%
    Type~~H <return>~~for immediate help\plaindots
  }{%
    HELP^^J%
    SECOND LINE%
  }%
  \@ClassError\clsname\msgtext\hlptext
  \@TestError\@ClassError{%
    ! Class CLASS Error: MESSAGE^^J%
    (CLASS)~~~~~~~~~~~~~~SECOND LINE.^^J%
    ^^J%
    See the CLASS class documentation for explanation.^^J%
    Type~~H <return>~~for immediate help\plaindots
  }{%
    HELP^^J%
    SECOND LINE%
  }%
  \@PackageError\pkgname\msgtext\@ehc
  \@TestError\@PackageError{%
    ! Package PACKAGE Error: MESSAGE^^J%
    (PACKAGE)~~~~~~~~~~~~~~~~SECOND LINE.^^J%
    ^^J%
    See the PACKAGE package documentation for explanation.^^J%
    Type~~H <return>~~for immediate help\plaindots
  }{%
    Try typing~~<return>~~to proceed.^^J%
    If that doesn't work, type~~X <return>~~to quit.%
  }%
  \@PackageError\pkgname\msgtext\@ehd
  \@TestError\@PackageError{%
    ! Package PACKAGE Error: MESSAGE^^J%
    (PACKAGE)~~~~~~~~~~~~~~~~SECOND LINE.^^J%
    ^^J%
    See the PACKAGE package documentation for explanation.^^J%
    Type~~H <return>~~for immediate help\plaindots
  }{%
    You're in trouble here.~~Try typing~~<return>~~to proceed.^^J%
    If that doesn't work, type~~X <return>~~to quit.%
  }%
}
\def\@TestError#1#2#3{%
  \begingroup
    \let~\space
    \edef\x{#2}%
%<test3>  \@onelevel@sanitize\x
%<test3>  \@onelevel@sanitize\ErrMessage
    \ifx\x\ErrMessage
    \else
      \msg{}%
      \msg{[\ErrMessage]}%
      \msg{<>}%
      \msg{[\x]}%
      \TestFailed#1{errmessage text}%
    \fi
    \edef\x{\the\ErrHelpToks}%
    \edef\y{#3}%
    \ifx\x\y
    \else
      \msg{}%
      \msg{[\the\ErrHelpToks]}%
      \msg{<>}%
      \msg{[\y]}%
      \TestFailed#1{errhelp text}%
    \fi
  \endgroup
}
\TestInfo
\TestWarning
\TestError
%</test2|test3>
%<test2>\end
%<test3>\@@end
%    \end{macrocode}
%
%
% \section{Installation}
%
% \subsection{Download}
%
% \paragraph{Package.} This package is available on
% CTAN\footnote{\url{ftp://ftp.ctan.org/tex-archive/}}:
% \begin{description}
% \item[\CTAN{macros/latex/contrib/oberdiek/infwarerr.dtx}] The source file.
% \item[\CTAN{macros/latex/contrib/oberdiek/infwarerr.pdf}] Documentation.
% \end{description}
%
%
% \paragraph{Bundle.} All the packages of the bundle `oberdiek'
% are also available in a TDS compliant ZIP archive. There
% the packages are already unpacked and the documentation files
% are generated. The files and directories obey the TDS standard.
% \begin{description}
% \item[\CTAN{install/macros/latex/contrib/oberdiek.tds.zip}]
% \end{description}
% \emph{TDS} refers to the standard ``A Directory Structure
% for \TeX\ Files'' (\CTAN{tds/tds.pdf}). Directories
% with \xfile{texmf} in their name are usually organized this way.
%
% \subsection{Bundle installation}
%
% \paragraph{Unpacking.} Unpack the \xfile{oberdiek.tds.zip} in the
% TDS tree (also known as \xfile{texmf} tree) of your choice.
% Example (linux):
% \begin{quote}
%   |unzip oberdiek.tds.zip -d ~/texmf|
% \end{quote}
%
% \paragraph{Script installation.}
% Check the directory \xfile{TDS:scripts/oberdiek/} for
% scripts that need further installation steps.
% Package \xpackage{attachfile2} comes with the Perl script
% \xfile{pdfatfi.pl} that should be installed in such a way
% that it can be called as \texttt{pdfatfi}.
% Example (linux):
% \begin{quote}
%   |chmod +x scripts/oberdiek/pdfatfi.pl|\\
%   |cp scripts/oberdiek/pdfatfi.pl /usr/local/bin/|
% \end{quote}
%
% \subsection{Package installation}
%
% \paragraph{Unpacking.} The \xfile{.dtx} file is a self-extracting
% \docstrip\ archive. The files are extracted by running the
% \xfile{.dtx} through \plainTeX:
% \begin{quote}
%   \verb|tex infwarerr.dtx|
% \end{quote}
%
% \paragraph{TDS.} Now the different files must be moved into
% the different directories in your installation TDS tree
% (also known as \xfile{texmf} tree):
% \begin{quote}
% \def\t{^^A
% \begin{tabular}{@{}>{\ttfamily}l@{ $\rightarrow$ }>{\ttfamily}l@{}}
%   infwarerr.sty & tex/generic/oberdiek/infwarerr.sty\\
%   infwarerr.pdf & doc/latex/oberdiek/infwarerr.pdf\\
%   test/infwarerr-test1.tex & doc/latex/oberdiek/test/infwarerr-test1.tex\\
%   test/infwarerr-test2.tex & doc/latex/oberdiek/test/infwarerr-test2.tex\\
%   test/infwarerr-test3.tex & doc/latex/oberdiek/test/infwarerr-test3.tex\\
%   infwarerr.dtx & source/latex/oberdiek/infwarerr.dtx\\
% \end{tabular}^^A
% }^^A
% \sbox0{\t}^^A
% \ifdim\wd0>\linewidth
%   \begingroup
%     \advance\linewidth by\leftmargin
%     \advance\linewidth by\rightmargin
%   \edef\x{\endgroup
%     \def\noexpand\lw{\the\linewidth}^^A
%   }\x
%   \def\lwbox{^^A
%     \leavevmode
%     \hbox to \linewidth{^^A
%       \kern-\leftmargin\relax
%       \hss
%       \usebox0
%       \hss
%       \kern-\rightmargin\relax
%     }^^A
%   }^^A
%   \ifdim\wd0>\lw
%     \sbox0{\small\t}^^A
%     \ifdim\wd0>\linewidth
%       \ifdim\wd0>\lw
%         \sbox0{\footnotesize\t}^^A
%         \ifdim\wd0>\linewidth
%           \ifdim\wd0>\lw
%             \sbox0{\scriptsize\t}^^A
%             \ifdim\wd0>\linewidth
%               \ifdim\wd0>\lw
%                 \sbox0{\tiny\t}^^A
%                 \ifdim\wd0>\linewidth
%                   \lwbox
%                 \else
%                   \usebox0
%                 \fi
%               \else
%                 \lwbox
%               \fi
%             \else
%               \usebox0
%             \fi
%           \else
%             \lwbox
%           \fi
%         \else
%           \usebox0
%         \fi
%       \else
%         \lwbox
%       \fi
%     \else
%       \usebox0
%     \fi
%   \else
%     \lwbox
%   \fi
% \else
%   \usebox0
% \fi
% \end{quote}
% If you have a \xfile{docstrip.cfg} that configures and enables \docstrip's
% TDS installing feature, then some files can already be in the right
% place, see the documentation of \docstrip.
%
% \subsection{Refresh file name databases}
%
% If your \TeX~distribution
% (\teTeX, \mikTeX, \dots) relies on file name databases, you must refresh
% these. For example, \teTeX\ users run \verb|texhash| or
% \verb|mktexlsr|.
%
% \subsection{Some details for the interested}
%
% \paragraph{Attached source.}
%
% The PDF documentation on CTAN also includes the
% \xfile{.dtx} source file. It can be extracted by
% AcrobatReader 6 or higher. Another option is \textsf{pdftk},
% e.g. unpack the file into the current directory:
% \begin{quote}
%   \verb|pdftk infwarerr.pdf unpack_files output .|
% \end{quote}
%
% \paragraph{Unpacking with \LaTeX.}
% The \xfile{.dtx} chooses its action depending on the format:
% \begin{description}
% \item[\plainTeX:] Run \docstrip\ and extract the files.
% \item[\LaTeX:] Generate the documentation.
% \end{description}
% If you insist on using \LaTeX\ for \docstrip\ (really,
% \docstrip\ does not need \LaTeX), then inform the autodetect routine
% about your intention:
% \begin{quote}
%   \verb|latex \let\install=y% \iffalse meta-comment
% 
% File: infwarerr.dtx
% Version: 2010/04/08 v1.3
% 
% Copyright (C) 2007, 2010 by
%    Heiko Oberdiek <heiko.oberdiek at googlemail.com>
% 
% This work may be distributed and/or modified under the
% conditions of the LaTeX Project Public License, either
% version 1.3c of this license or (at your option) any later
% version. This version of this license is in
%    http://www.latex-project.org/lppl/lppl-1-3c.txt
% and the latest version of this license is in
%    http://www.latex-project.org/lppl.txt
% and version 1.3 or later is part of all distributions of
% LaTeX version 2005/12/01 or later.
% 
% This work has the LPPL maintenance status "maintained".
% 
% This Current Maintainer of this work is Heiko Oberdiek.
% 
% The Base Interpreter refers to any `TeX-Format',
% because some files are installed in TDS:tex/generic//.
% 
% This work consists of the main source file infwarerr.dtx
% and the derived files
%    infwarerr.sty, infwarerr.pdf, infwarerr.ins, infwarerr.drv,
%    infwarerr-test1.tex, infwarerr-test2.tex,
%    infwarerr-test3.tex.
% 
% Distribution:
%    CTAN:macros/latex/contrib/oberdiek/infwarerr.dtx
%    CTAN:macros/latex/contrib/oberdiek/infwarerr.pdf
% 
% Unpacking:
%    (a) If infwarerr.ins is present:
%           tex infwarerr.ins
%    (b) Without infwarerr.ins:
%           tex infwarerr.dtx
%    (c) If you insist on using LaTeX
%           latex \let\install=y% \iffalse meta-comment
% 
% File: infwarerr.dtx
% Version: 2010/04/08 v1.3
% 
% Copyright (C) 2007, 2010 by
%    Heiko Oberdiek <heiko.oberdiek at googlemail.com>
% 
% This work may be distributed and/or modified under the
% conditions of the LaTeX Project Public License, either
% version 1.3c of this license or (at your option) any later
% version. This version of this license is in
%    http://www.latex-project.org/lppl/lppl-1-3c.txt
% and the latest version of this license is in
%    http://www.latex-project.org/lppl.txt
% and version 1.3 or later is part of all distributions of
% LaTeX version 2005/12/01 or later.
% 
% This work has the LPPL maintenance status "maintained".
% 
% This Current Maintainer of this work is Heiko Oberdiek.
% 
% The Base Interpreter refers to any `TeX-Format',
% because some files are installed in TDS:tex/generic//.
% 
% This work consists of the main source file infwarerr.dtx
% and the derived files
%    infwarerr.sty, infwarerr.pdf, infwarerr.ins, infwarerr.drv,
%    infwarerr-test1.tex, infwarerr-test2.tex,
%    infwarerr-test3.tex.
% 
% Distribution:
%    CTAN:macros/latex/contrib/oberdiek/infwarerr.dtx
%    CTAN:macros/latex/contrib/oberdiek/infwarerr.pdf
% 
% Unpacking:
%    (a) If infwarerr.ins is present:
%           tex infwarerr.ins
%    (b) Without infwarerr.ins:
%           tex infwarerr.dtx
%    (c) If you insist on using LaTeX
%           latex \let\install=y% \iffalse meta-comment
% 
% File: infwarerr.dtx
% Version: 2010/04/08 v1.3
% 
% Copyright (C) 2007, 2010 by
%    Heiko Oberdiek <heiko.oberdiek at googlemail.com>
% 
% This work may be distributed and/or modified under the
% conditions of the LaTeX Project Public License, either
% version 1.3c of this license or (at your option) any later
% version. This version of this license is in
%    http://www.latex-project.org/lppl/lppl-1-3c.txt
% and the latest version of this license is in
%    http://www.latex-project.org/lppl.txt
% and version 1.3 or later is part of all distributions of
% LaTeX version 2005/12/01 or later.
% 
% This work has the LPPL maintenance status "maintained".
% 
% This Current Maintainer of this work is Heiko Oberdiek.
% 
% The Base Interpreter refers to any `TeX-Format',
% because some files are installed in TDS:tex/generic//.
% 
% This work consists of the main source file infwarerr.dtx
% and the derived files
%    infwarerr.sty, infwarerr.pdf, infwarerr.ins, infwarerr.drv,
%    infwarerr-test1.tex, infwarerr-test2.tex,
%    infwarerr-test3.tex.
% 
% Distribution:
%    CTAN:macros/latex/contrib/oberdiek/infwarerr.dtx
%    CTAN:macros/latex/contrib/oberdiek/infwarerr.pdf
% 
% Unpacking:
%    (a) If infwarerr.ins is present:
%           tex infwarerr.ins
%    (b) Without infwarerr.ins:
%           tex infwarerr.dtx
%    (c) If you insist on using LaTeX
%           latex \let\install=y\input{infwarerr.dtx}
%        (quote the arguments according to the demands of your shell)
% 
% Documentation:
%    (a) If infwarerr.drv is present:
%           latex infwarerr.drv
%    (b) Without infwarerr.drv:
%           latex infwarerr.dtx; ...
%    The class ltxdoc loads the configuration file ltxdoc.cfg
%    if available. Here you can specify further options, e.g.
%    use A4 as paper format:
%       \PassOptionsToClass{a4paper}{article}
% 
%    Programm calls to get the documentation (example):
%       pdflatex infwarerr.dtx
%       makeindex -s gind.ist infwarerr.idx
%       pdflatex infwarerr.dtx
%       makeindex -s gind.ist infwarerr.idx
%       pdflatex infwarerr.dtx
% 
% Installation:
%    TDS:tex/generic/oberdiek/infwarerr.sty
%    TDS:doc/latex/oberdiek/infwarerr.pdf
%    TDS:doc/latex/oberdiek/test/infwarerr-test1.tex
%    TDS:doc/latex/oberdiek/test/infwarerr-test2.tex
%    TDS:doc/latex/oberdiek/test/infwarerr-test3.tex
%    TDS:source/latex/oberdiek/infwarerr.dtx
%
%<*ignore>
\begingroup
  \catcode123=1 %
  \catcode125=2 %
  \def\x{LaTeX2e}%
\expandafter\endgroup
\ifcase 0\ifx\install y1\fi\expandafter
         \ifx\csname processbatchFile\endcsname\relax\else1\fi
         \ifx\fmtname\x\else 1\fi\relax
\else\csname fi\endcsname
%</ignore>
%<*install>
\input docstrip.tex
\Msg{************************************************************************}
\Msg{* Installation}
\Msg{* Package: infwarerr 2010/04/08 v1.3 Providing info/warning/message (HO)}
\Msg{************************************************************************}

\keepsilent
\askforoverwritefalse

\let\MetaPrefix\relax
\preamble

This is a generated file.

Project: infwarerr
Version: 2010/04/08 v1.3

Copyright (C) 2007, 2010 by
   Heiko Oberdiek <heiko.oberdiek at googlemail.com>

This work may be distributed and/or modified under the
conditions of the LaTeX Project Public License, either
version 1.3c of this license or (at your option) any later
version. This version of this license is in
   http://www.latex-project.org/lppl/lppl-1-3c.txt
and the latest version of this license is in
   http://www.latex-project.org/lppl.txt
and version 1.3 or later is part of all distributions of
LaTeX version 2005/12/01 or later.

This work has the LPPL maintenance status "maintained".

This Current Maintainer of this work is Heiko Oberdiek.

The Base Interpreter refers to any `TeX-Format',
because some files are installed in TDS:tex/generic//.

This work consists of the main source file infwarerr.dtx
and the derived files
   infwarerr.sty, infwarerr.pdf, infwarerr.ins, infwarerr.drv,
   infwarerr-test1.tex, infwarerr-test2.tex,
   infwarerr-test3.tex.

\endpreamble
\let\MetaPrefix\DoubleperCent

\generate{%
  \file{infwarerr.ins}{\from{infwarerr.dtx}{install}}%
  \file{infwarerr.drv}{\from{infwarerr.dtx}{driver}}%
  \usedir{tex/generic/oberdiek}%
  \file{infwarerr.sty}{\from{infwarerr.dtx}{package}}%
  \usedir{doc/latex/oberdiek/test}%
  \file{infwarerr-test1.tex}{\from{infwarerr.dtx}{test1}}%
  \file{infwarerr-test2.tex}{\from{infwarerr.dtx}{test2}}%
  \file{infwarerr-test3.tex}{\from{infwarerr.dtx}{test3}}%
}

\catcode32=13\relax% active space
\let =\space%
\Msg{************************************************************************}
\Msg{*}
\Msg{* To finish the installation you have to move the following}
\Msg{* file into a directory searched by TeX:}
\Msg{*}
\Msg{*     infwarerr.sty}
\Msg{*}
\Msg{* To produce the documentation run the file `infwarerr.drv'}
\Msg{* through LaTeX.}
\Msg{*}
\Msg{* Happy TeXing!}
\Msg{*}
\Msg{************************************************************************}

\endbatchfile
%</install>
%<*ignore>
\fi
%</ignore>
%<*driver>
\NeedsTeXFormat{LaTeX2e}
\ProvidesFile{infwarerr.drv}%
  [2010/04/08 v1.3 Providing info/warning/message (HO)]%
\documentclass{ltxdoc}
\usepackage{holtxdoc}[2011/02/04]
\begin{document}
  \DocInput{infwarerr.dtx}%
\end{document}
%</driver>
% \fi
%
% \CheckSum{871}
%
% \CharacterTable
%  {Upper-case    \A\B\C\D\E\F\G\H\I\J\K\L\M\N\O\P\Q\R\S\T\U\V\W\X\Y\Z
%   Lower-case    \a\b\c\d\e\f\g\h\i\j\k\l\m\n\o\p\q\r\s\t\u\v\w\x\y\z
%   Digits        \0\1\2\3\4\5\6\7\8\9
%   Exclamation   \!     Double quote  \"     Hash (number) \#
%   Dollar        \$     Percent       \%     Ampersand     \&
%   Acute accent  \'     Left paren    \(     Right paren   \)
%   Asterisk      \*     Plus          \+     Comma         \,
%   Minus         \-     Point         \.     Solidus       \/
%   Colon         \:     Semicolon     \;     Less than     \<
%   Equals        \=     Greater than  \>     Question mark \?
%   Commercial at \@     Left bracket  \[     Backslash     \\
%   Right bracket \]     Circumflex    \^     Underscore    \_
%   Grave accent  \`     Left brace    \{     Vertical bar  \|
%   Right brace   \}     Tilde         \~}
%
% \GetFileInfo{infwarerr.drv}
%
% \title{The \xpackage{infwarerr} package}
% \date{2010/04/08 v1.3}
% \author{Heiko Oberdiek\\\xemail{heiko.oberdiek at googlemail.com}}
%
% \maketitle
%
% \begin{abstract}
% This package provides a complete set of macros for informations,
% warnings and error messages with support for \plainTeX.
% \end{abstract}
%
% \tableofcontents
%
% \section{Documentation}
%
% Package \xpackage{infwarerr} defines the following set of commands,
% given as pseudo regular expression:
%   \begin{quote}
%     \verb+\@(Package|Class)(Info|InfoNoLine|Warning|WarningNoLine|Error)+^^A
%     \kern-\rightmargin
%   \end{quote}
% The first argument is the package or class in question. The second
% argument contains the message text. \cs{MessageBreak} is supported.
% In case of error messages, the third argument contains the help
% text. Supported are \LaTeX's message texts \cs{@ehc}:
%   \begin{quote}
%     |Try typing  <return>  to proceed|\\
%     |If that doesn't work, type  X <return>  to quit.|
%   \end{quote}
% and \cs{@ehd}:
%   \begin{quote}
%     |You're in trouble here.  Try typing  <return>  to proceed|\\
%     |If that doesn't work, type  X <return>  to quit.|
%   \end{quote}
%
% The package is intended for macro writers that want to write
% code that works for both \LaTeX\ and \plainTeX. If \LaTeX\ is
% detected, then the macros of this package will only be wrappers
% for \LaTeX's macros. Otherwise \LaTeX's macros will be mimicked
% for use with \plainTeX.
%
% \StopEventually{
% }
%
% \section{Implementation}
%
%    \begin{macrocode}
%<*package>
%    \end{macrocode}
%
% \subsection{Reload check and package identification}
%    Reload check, especially if the package is not used with \LaTeX.
%    \begin{macrocode}
\begingroup\catcode61\catcode48\catcode32=10\relax%
  \catcode13=5 % ^^M
  \endlinechar=13 %
  \catcode35=6 % #
  \catcode39=12 % '
  \catcode44=12 % ,
  \catcode45=12 % -
  \catcode46=12 % .
  \catcode58=12 % :
  \catcode64=11 % @
  \catcode123=1 % {
  \catcode125=2 % }
  \expandafter\let\expandafter\x\csname ver@infwarerr.sty\endcsname
  \ifx\x\relax % plain-TeX, first loading
  \else
    \def\empty{}%
    \ifx\x\empty % LaTeX, first loading,
      % variable is initialized, but \ProvidesPackage not yet seen
    \else
      \expandafter\ifx\csname PackageInfo\endcsname\relax
        \def\x#1#2{%
          \immediate\write-1{Package #1 Info: #2.}%
        }%
      \else
        \def\x#1#2{\PackageInfo{#1}{#2, stopped}}%
      \fi
      \x{infwarerr}{The package is already loaded}%
      \aftergroup\endinput
    \fi
  \fi
\endgroup%
%    \end{macrocode}
%    Package identification:
%    \begin{macrocode}
\begingroup\catcode61\catcode48\catcode32=10\relax%
  \catcode13=5 % ^^M
  \endlinechar=13 %
  \catcode35=6 % #
  \catcode39=12 % '
  \catcode40=12 % (
  \catcode41=12 % )
  \catcode44=12 % ,
  \catcode45=12 % -
  \catcode46=12 % .
  \catcode47=12 % /
  \catcode58=12 % :
  \catcode64=11 % @
  \catcode91=12 % [
  \catcode93=12 % ]
  \catcode123=1 % {
  \catcode125=2 % }
  \expandafter\ifx\csname ProvidesPackage\endcsname\relax
    \def\x#1#2#3[#4]{\endgroup
      \immediate\write-1{Package: #3 #4}%
      \xdef#1{#4}%
    }%
  \else
    \def\x#1#2[#3]{\endgroup
      #2[{#3}]%
      \ifx#1\@undefined
        \xdef#1{#3}%
      \fi
      \ifx#1\relax
        \xdef#1{#3}%
      \fi
    }%
  \fi
\expandafter\x\csname ver@infwarerr.sty\endcsname
\ProvidesPackage{infwarerr}%
  [2010/04/08 v1.3 Providing info/warning/message (HO)]%
%    \end{macrocode}
%
% \subsection{Catcodes}
%
%    \begin{macrocode}
\begingroup\catcode61\catcode48\catcode32=10\relax%
  \catcode13=5 % ^^M
  \endlinechar=13 %
  \catcode123=1 % {
  \catcode125=2 % }
  \catcode64=11 % @
  \def\x{\endgroup
    \expandafter\edef\csname InfWarErr@AtEnd\endcsname{%
      \endlinechar=\the\endlinechar\relax
      \catcode13=\the\catcode13\relax
      \catcode32=\the\catcode32\relax
      \catcode35=\the\catcode35\relax
      \catcode61=\the\catcode61\relax
      \catcode64=\the\catcode64\relax
      \catcode123=\the\catcode123\relax
      \catcode125=\the\catcode125\relax
    }%
  }%
\x\catcode61\catcode48\catcode32=10\relax%
\catcode13=5 % ^^M
\endlinechar=13 %
\catcode35=6 % #
\catcode64=11 % @
\catcode123=1 % {
\catcode125=2 % }
\def\TMP@EnsureCode#1#2{%
  \edef\InfWarErr@AtEnd{%
    \InfWarErr@AtEnd
    \catcode#1=\the\catcode#1\relax
  }%
  \catcode#1=#2\relax
}
\TMP@EnsureCode{10}{12}% ^^J
\TMP@EnsureCode{39}{12}% '
\TMP@EnsureCode{40}{12}% (
\TMP@EnsureCode{41}{12}% )
\TMP@EnsureCode{44}{12}% ,
\TMP@EnsureCode{45}{12}% -
\TMP@EnsureCode{46}{12}% .
\TMP@EnsureCode{58}{12}% :
\TMP@EnsureCode{60}{12}% <
\TMP@EnsureCode{62}{12}% >
\TMP@EnsureCode{94}{7}% ^
\edef\InfWarErr@AtEnd{\InfWarErr@AtEnd\noexpand\endinput}
%    \end{macrocode}
%
% \subsection{\LaTeX\ detection}
%
%    \begin{macrocode}
\begingroup
  \let\x=Y%
  \def\check#1{%
    \@check{Generic}{#1}%
    \@check{Package}{#1}%
    \@check{Class}{#1}%
  }%
  \def\@check#1#2{%
    \expandafter\ifx\csname #1#2\endcsname\relax
      \let\x=N%
    \fi
  }%
  \check{Info}%
  \check{Warning}%
  \check{Error}%
  \@check{Package}{WarningNoLine}%
  \@check{Class}{WarningNoLine}%
  \@check{}{@ehc}%
  \@check{}{@ehd}%
\expandafter\endgroup
\ifx\x Y%
%    \end{macrocode}
%
% \subsection{Version for \LaTeX}
%
%    \begin{macro}{\@PackageInfo}
%    \begin{macrocode}
  \let\@PackageInfo\PackageInfo
%    \end{macrocode}
%    \end{macro}
%    \begin{macro}{\@PackageInfoNoLine}
%    \begin{macrocode}
  \def\@PackageInfoNoLine#1#2{%
    \PackageInfo{#1}{#2\@gobble}%
  }%
%    \end{macrocode}
%    \end{macro}
%    \begin{macro}{\@ClassInfo}
%    \begin{macrocode}
  \let\@ClassInfo\ClassInfo
%    \end{macrocode}
%    \end{macro}
%    \begin{macro}{\@ClassInfoNoLine}
%    \begin{macrocode}
  \def\@ClassInfoNoLine#1#2{%
    \ClassInfo{#1}{#2\@gobble}%
  }%
%    \end{macrocode}
%    \end{macro}
%    \begin{macro}{\@PackageWarning}
%    \begin{macrocode}
  \let\@PackageWarning\PackageWarning
%    \end{macrocode}
%    \end{macro}
%    \begin{macro}{\@PackageWarningNoLine}
%    \begin{macrocode}
  \let\@PackageWarningNoLine\PackageWarningNoLine
%    \end{macrocode}
%    \end{macro}
%    \begin{macro}{\@ClassWarning}
%    \begin{macrocode}
  \let\@ClassWarning\ClassWarning
%    \end{macrocode}
%    \end{macro}
%    \begin{macro}{\@ClassWarningNoLine}
%    \begin{macrocode}
  \let\@ClassWarningNoLine\ClassWarningNoLine
%    \end{macrocode}
%    \end{macro}
%    \begin{macro}{\@PackageError}
%    \begin{macrocode}
  \let\@PackageError\PackageError
%    \end{macrocode}
%    \end{macro}
%    \begin{macro}{\@ClassError}
%    \begin{macrocode}
  \let\@ClassError\ClassError
%    \end{macrocode}
%    \end{macro}
%    \begin{macrocode}
  \expandafter\InfWarErr@AtEnd
\fi%
%    \end{macrocode}
%
% \subsection{Version for \plainTeX}
%
%    \begin{macro}{\InfWarErr@online}
%    \begin{macrocode}
\begingroup\expandafter\endgroup
\ifcase\expandafter\ifx\csname inputlineno\endcsname\relax
         0 %
       \else
         \ifnum\inputlineno<0 %
           0 %
         \else
           1 %
         \fi
       \fi
  \def\InfWarErr@online{}%
\else
  \def\InfWarErr@online{ on input line \the\inputlineno}%
\fi
%    \end{macrocode}
%    \end{macro}
%
%    \begin{macro}{\InfWarErr@protected}
%    \begin{macrocode}
\begingroup\expandafter\expandafter\expandafter\endgroup
\expandafter\ifx\csname protected\endcsname\relax
  \def\InfWarErr@protected{}%
\else
  \let\InfWarErr@protected\protected
\fi
%    \end{macrocode}
%    \end{macro}
%
%    \begin{macro}{\InfWarErr@unused}
%    \begin{macrocode}
\chardef\InfWarErr@unused=16 %
%    \end{macrocode}
%    \end{macro}
%
%    \begin{macro}{\InfWarErr@ehc}
%    \begin{macrocode}
\def\InfWarErr@ehc{%
  Try typing \space <return> \space to proceed.\MessageBreak
  If that doesn't work, type \space X <return> \space to quit.%
}
%    \end{macrocode}
%    \end{macro}
%    \begin{macro}{\InfWarErr@ehd}
%    \begin{macrocode}
\def\InfWarErr@ehd{%
  You're in trouble here. \space\@ehc
}
%    \end{macrocode}
%    \end{macro}
%
%    \begin{macro}{\InfWarErr@GenericInfo}
%    \begin{macrocode}
\InfWarErr@protected\def\InfWarErr@GenericInfo#1#2{%
  \begingroup
    \let\on@line\InfWarErr@online
    \long\def\@gobble##1{}%
    \def\space{ }%
    \def\@spaces{\space\space\space\space}%
    \def\MessageBreak{^^J#1}%
    \newlinechar=10 %
    \escapechar=92 %
    \immediate\write-1{#2\on@line.}%
  \endgroup
}
%    \end{macrocode}
%    \end{macro}
%    \begin{macro}{\InfWarErr@GenericWarning}
%    \begin{macrocode}
\InfWarErr@protected\def\InfWarErr@GenericWarning#1#2{%
  \begingroup
    \let\on@line\InfWarErr@online
    \long\def\@gobble##1{}%
    \def\space{ }%
    \def\@spaces{\space\space\space\space}%
    \def\MessageBreak{^^J#1}%
    \newlinechar=10 %
    \escapechar=92 %
    \immediate\write\InfWarErr@unused{^^J#2\on@line.^^J}%
  \endgroup
}
%    \end{macrocode}
%    \end{macro}
%    \begin{macro}{\InfWarErr@GenericError}
%    Old \TeX\ versions below 3.141 where |^^J| is not working
%    as newline in error message and help texts are not supported.
%    \begin{macrocode}
\InfWarErr@protected\def\InfWarErr@GenericError#1#2#3#4{%
  \begingroup
    \immediate\write\InfWarErr@unused{}%
    \def\space{ }%
    \def\@spaces{\space\space\space\space}%
    \let\@ehc\InfWarErr@ehc
    \let\@ehd\InfWarErr@ehd
    \let\on@line\InfWarErr@online
    \def\MessageBreak{^^J}%
    \newlinechar=10 %
    \escapechar=92 %
    \edef\InfWarErr@help{\noexpand\errhelp{#4}}%
    \InfWarErr@help
    \def\MessageBreak{^^J#1}%
    \errmessage{%
      #2.^^J^^J%
      #3^^J%
      Type \space H <return> \space for immediate help.^^J ...^^J%
    }%
  \endgroup
}
%    \end{macrocode}
%    \end{macro}
%
%    \begin{macro}{\@PackageInfo}
%    \begin{macrocode}
\def\@PackageInfo#1#2{%
  \InfWarErr@GenericInfo{%
    (#1) \@spaces\@spaces\@spaces
  }{%
    Package #1 Info: #2%
  }%
}
%    \end{macrocode}
%    \end{macro}
%    \begin{macro}{\@PackageInfoNoLine}
%    \begin{macrocode}
\def\@PackageInfoNoLine#1#2{%
  \@PackageInfo{#1}{#2\@gobble}%
}
%    \end{macrocode}
%    \end{macro}
%    \begin{macro}{\@ClassInfo}
%    \begin{macrocode}
\def\@ClassInfo#1#2{%
  \InfWarErr@GenericInfo{%
    (#1) \space\space\@spaces\@spaces
  }{%
    Class #1 Info: #2%
  }%
}
%    \end{macrocode}
%    \end{macro}
%    \begin{macro}{\@ClassInfoNoLine}
%    \begin{macrocode}
\def\@ClassInfoNoLine#1#2{%
  \@ClassInfo{#1}{#2\@gobble}%
}
%    \end{macrocode}
%    \end{macro}
%    \begin{macro}{\@PackageWarning}
%    \begin{macrocode}
\def\@PackageWarning#1#2{%
  \InfWarErr@GenericWarning{%
    (#1)\@spaces\@spaces\@spaces\@spaces
  }{%
    Package #1 Warning: #2%
  }%
}
%    \end{macrocode}
%    \end{macro}
%    \begin{macro}{\@PackageWarningNoLine}
%    \begin{macrocode}
\def\@PackageWarningNoLine#1#2{%
  \@PackageWarning{#1}{#2\@gobble}%
}
%    \end{macrocode}
%    \end{macro}
%    \begin{macro}{\@ClassWarning}
%    \begin{macrocode}
\def\@ClassWarning#1#2{%
  \InfWarErr@GenericWarning{%
    (#1) \space\@spaces\@spaces\@spaces
  }{%
    Class #1 Warning: #2%
  }%
}
%    \end{macrocode}
%    \end{macro}
%    \begin{macro}{\@ClassWarningNoLine}
%    \begin{macrocode}
\def\@ClassWarningNoLine#1#2{%
  \@ClassWarning{#1}{#2\@gobble}%
}
%    \end{macrocode}
%    \end{macro}
%    \begin{macro}{\@PackageError}
%    \begin{macrocode}
\def\@PackageError#1#2#3{%
  \InfWarErr@GenericError{%
    (#1)\@spaces\@spaces\@spaces\@spaces
  }{%
    Package #1 Error: #2%
  }{%
    See the #1 package documentation for explanation.%
  }{#3}%
}
%    \end{macrocode}
%    \end{macro}
%    \begin{macro}{\@ClassError}
%    \begin{macrocode}
\def\@ClassError#1#2#3{%
  \InfWarErr@GenericError{%
    (#1) \space\@spaces\@spaces\@spaces
  }{%
    Class #1 Error: #2%
  }{%
    See the #1 class documentation for explanation.%
  }{#3}%
}
%    \end{macrocode}
%    \end{macro}
%
%    \begin{macrocode}
\InfWarErr@AtEnd%
%</package>
%    \end{macrocode}
%
% \section{Test}
%
% \subsection{Catcode checks for loading}
%
%    \begin{macrocode}
%<*test1>
%    \end{macrocode}
%    \begin{macrocode}
\catcode`\{=1 %
\catcode`\}=2 %
\catcode`\#=6 %
\catcode`\@=11 %
\expandafter\ifx\csname count@\endcsname\relax
  \countdef\count@=255 %
\fi
\expandafter\ifx\csname @gobble\endcsname\relax
  \long\def\@gobble#1{}%
\fi
\expandafter\ifx\csname @firstofone\endcsname\relax
  \long\def\@firstofone#1{#1}%
\fi
\expandafter\ifx\csname loop\endcsname\relax
  \expandafter\@firstofone
\else
  \expandafter\@gobble
\fi
{%
  \def\loop#1\repeat{%
    \def\body{#1}%
    \iterate
  }%
  \def\iterate{%
    \body
      \let\next\iterate
    \else
      \let\next\relax
    \fi
    \next
  }%
  \let\repeat=\fi
}%
\def\RestoreCatcodes{}
\count@=0 %
\loop
  \edef\RestoreCatcodes{%
    \RestoreCatcodes
    \catcode\the\count@=\the\catcode\count@\relax
  }%
\ifnum\count@<255 %
  \advance\count@ 1 %
\repeat

\def\RangeCatcodeInvalid#1#2{%
  \count@=#1\relax
  \loop
    \catcode\count@=15 %
  \ifnum\count@<#2\relax
    \advance\count@ 1 %
  \repeat
}
\def\RangeCatcodeCheck#1#2#3{%
  \count@=#1\relax
  \loop
    \ifnum#3=\catcode\count@
    \else
      \errmessage{%
        Character \the\count@\space
        with wrong catcode \the\catcode\count@\space
        instead of \number#3%
      }%
    \fi
  \ifnum\count@<#2\relax
    \advance\count@ 1 %
  \repeat
}
\def\space{ }
\expandafter\ifx\csname LoadCommand\endcsname\relax
  \def\LoadCommand{\input infwarerr.sty\relax}%
\fi
\def\Test{%
  \RangeCatcodeInvalid{0}{47}%
  \RangeCatcodeInvalid{58}{64}%
  \RangeCatcodeInvalid{91}{96}%
  \RangeCatcodeInvalid{123}{255}%
  \catcode`\@=12 %
  \catcode`\\=0 %
  \catcode`\%=14 %
  \LoadCommand
  \RangeCatcodeCheck{0}{36}{15}%
  \RangeCatcodeCheck{37}{37}{14}%
  \RangeCatcodeCheck{38}{47}{15}%
  \RangeCatcodeCheck{48}{57}{12}%
  \RangeCatcodeCheck{58}{63}{15}%
  \RangeCatcodeCheck{64}{64}{12}%
  \RangeCatcodeCheck{65}{90}{11}%
  \RangeCatcodeCheck{91}{91}{15}%
  \RangeCatcodeCheck{92}{92}{0}%
  \RangeCatcodeCheck{93}{96}{15}%
  \RangeCatcodeCheck{97}{122}{11}%
  \RangeCatcodeCheck{123}{255}{15}%
  \RestoreCatcodes
}
\Test
\csname @@end\endcsname
\end
%    \end{macrocode}
%    \begin{macrocode}
%</test1>
%    \end{macrocode}
%
% \subsection{Macro tests}
%
%    \begin{macrocode}
%<*test2>
\let\PackageInfo\relax
\input infwarerr.sty\relax
\let\OrgWrite\write
\chardef\WriteUnused=16 %
\def\msg#{\immediate\OrgWrite\WriteUnused}
\msg{File: infwarerr-test2.tex 2010/04/08 v1.3 Test file for plain-TeX}
\catcode64=11 %
%</test2>
%    \end{macrocode}
%
%    \begin{macrocode}
%<*test3>
\NeedsTeXFormat{LaTeX2e}
\ProvidesFile{infwarerr-test3.tex}[2010/04/08 v1.3 Test file for LaTeX]
\RequirePackage{infwarerr}[2010/04/08]
\makeatletter
\let\OrgWrite\write
\chardef\WriteUnused=\@unused
\def\msg#{\immediate\OrgWrite\WriteUnused}
%</test3>
%    \end{macrocode}
%
%    \begin{macrocode}
%<*test2|test3>
\def\pkgname{PACKAGE}
\def\clsname{CLASS}
\def\msgtext{MESSAGE\MessageBreak SECOND LINE}
\def\hlptext{HELP\MessageBreak SECOND LINE}

\let\OrgWrite\write
\let\OrgErrMessage\errmessage

\def\write#1#{%
  \@write{#1}%
}
\def\@write#1#2{%
  \OrgWrite#1{#2}%
  \xdef\WriteNum{\number#1}%
  \xdef\WriteText{#2}%
}
\def\TestFailed#1#2{%
  \OrgErrMessage{Test (\string#1) failed: #2}%
}
\def\TestWrite#1#2#3{%
  \ifnum\WriteNum=#2\relax
  \else
    \TestFailed#1{write stream number: \WriteNum <> #2}%
  \fi
  \begingroup
    \let~\space
    \edef\TestString{#3}%
    \ifx\WriteText\TestString
    \else
      \msg{}%
      \msg{[\WriteText]}%
      \msg{<>}%
      \msg{[#3]}% hash-ok
      \TestFailed#1{write text}%
    \fi
  \endgroup
}
\newtoks\ErrHelpToks
\def\errmessage#1{%
  \global\ErrHelpToks=\expandafter{\the\errhelp}%
  \msg{**************** error message ****************}%
  \msg{! #1}%
  \msg{**************** help message *****************}%
  \msg{\the\errhelp}%
  \msg{***********************************************}%
  \xdef\ErrMessage{! #1}%
}
\def\TestInfo{%
  \@PackageInfo\pkgname\msgtext
  \TestWrite\@PackageInfo{-1}{%
    Package PACKAGE Info: MESSAGE^^J%
    (PACKAGE)~~~~~~~~~~~~~SECOND LINE on input line \the\inputlineno.%
  }%
  \@PackageInfoNoLine\pkgname\msgtext
  \TestWrite\@PackageInfoNoLine{-1}{%
    Package PACKAGE Info: MESSAGE^^J%
    (PACKAGE)~~~~~~~~~~~~~SECOND LINE.%
  }%
  \@ClassInfo\clsname\msgtext
  \TestWrite\@ClassInfo{-1}{%
    Class CLASS Info: MESSAGE^^J%
    (CLASS)~~~~~~~~~~~SECOND LINE on input line \the\inputlineno.%
  }%
  \@ClassInfoNoLine\clsname\msgtext
  \TestWrite\@ClassInfoNoLine{-1}{%
    Class CLASS Info: MESSAGE^^J%
    (CLASS)~~~~~~~~~~~SECOND LINE.%
  }%
}
\def\TestWarning{%
  \@PackageWarning\pkgname\msgtext
  \TestWrite\@PackageWarning\WriteUnused{%
    ^^J%
    Package PACKAGE Warning: MESSAGE^^J%
    (PACKAGE)~~~~~~~~~~~~~~~~SECOND LINE on input line \the\inputlineno.%
    ^^J%
  }%
  \@PackageWarningNoLine\pkgname\msgtext
  \TestWrite\@PackageWarningNoLine\WriteUnused{%
    ^^J%
    Package PACKAGE Warning: MESSAGE^^J%
    (PACKAGE)~~~~~~~~~~~~~~~~SECOND LINE.%
    ^^J%
  }%
  \@ClassWarning\clsname\msgtext
  \TestWrite\@ClassWarning\WriteUnused{%
    ^^J%
    Class CLASS Warning: MESSAGE^^J%
    (CLASS)~~~~~~~~~~~~~~SECOND LINE on input line \the\inputlineno.%
    ^^J%
  }%
  \@ClassWarningNoLine\clsname\msgtext
  \TestWrite\@ClassWarningNoLine\WriteUnused{%
    ^^J%
    Class CLASS Warning: MESSAGE^^J%
    (CLASS)~~~~~~~~~~~~~~SECOND LINE.%
    ^^J%
  }%
}
\def\TestError{%
%<test3>  \let\write\OrgWrite
%<test3>  \def\plaindots{}
%<test2>  \def\plaindots{.^^J ...^^J}
  \@PackageError\pkgname\msgtext\hlptext
  \@TestError\@PackageError{%
    ! Package PACKAGE Error: MESSAGE^^J%
    (PACKAGE)~~~~~~~~~~~~~~~~SECOND LINE.^^J%
    ^^J%
    See the PACKAGE package documentation for explanation.^^J%
    Type~~H <return>~~for immediate help\plaindots
  }{%
    HELP^^J%
    SECOND LINE%
  }%
  \@ClassError\clsname\msgtext\hlptext
  \@TestError\@ClassError{%
    ! Class CLASS Error: MESSAGE^^J%
    (CLASS)~~~~~~~~~~~~~~SECOND LINE.^^J%
    ^^J%
    See the CLASS class documentation for explanation.^^J%
    Type~~H <return>~~for immediate help\plaindots
  }{%
    HELP^^J%
    SECOND LINE%
  }%
  \@PackageError\pkgname\msgtext\@ehc
  \@TestError\@PackageError{%
    ! Package PACKAGE Error: MESSAGE^^J%
    (PACKAGE)~~~~~~~~~~~~~~~~SECOND LINE.^^J%
    ^^J%
    See the PACKAGE package documentation for explanation.^^J%
    Type~~H <return>~~for immediate help\plaindots
  }{%
    Try typing~~<return>~~to proceed.^^J%
    If that doesn't work, type~~X <return>~~to quit.%
  }%
  \@PackageError\pkgname\msgtext\@ehd
  \@TestError\@PackageError{%
    ! Package PACKAGE Error: MESSAGE^^J%
    (PACKAGE)~~~~~~~~~~~~~~~~SECOND LINE.^^J%
    ^^J%
    See the PACKAGE package documentation for explanation.^^J%
    Type~~H <return>~~for immediate help\plaindots
  }{%
    You're in trouble here.~~Try typing~~<return>~~to proceed.^^J%
    If that doesn't work, type~~X <return>~~to quit.%
  }%
}
\def\@TestError#1#2#3{%
  \begingroup
    \let~\space
    \edef\x{#2}%
%<test3>  \@onelevel@sanitize\x
%<test3>  \@onelevel@sanitize\ErrMessage
    \ifx\x\ErrMessage
    \else
      \msg{}%
      \msg{[\ErrMessage]}%
      \msg{<>}%
      \msg{[\x]}%
      \TestFailed#1{errmessage text}%
    \fi
    \edef\x{\the\ErrHelpToks}%
    \edef\y{#3}%
    \ifx\x\y
    \else
      \msg{}%
      \msg{[\the\ErrHelpToks]}%
      \msg{<>}%
      \msg{[\y]}%
      \TestFailed#1{errhelp text}%
    \fi
  \endgroup
}
\TestInfo
\TestWarning
\TestError
%</test2|test3>
%<test2>\end
%<test3>\@@end
%    \end{macrocode}
%
%
% \section{Installation}
%
% \subsection{Download}
%
% \paragraph{Package.} This package is available on
% CTAN\footnote{\url{ftp://ftp.ctan.org/tex-archive/}}:
% \begin{description}
% \item[\CTAN{macros/latex/contrib/oberdiek/infwarerr.dtx}] The source file.
% \item[\CTAN{macros/latex/contrib/oberdiek/infwarerr.pdf}] Documentation.
% \end{description}
%
%
% \paragraph{Bundle.} All the packages of the bundle `oberdiek'
% are also available in a TDS compliant ZIP archive. There
% the packages are already unpacked and the documentation files
% are generated. The files and directories obey the TDS standard.
% \begin{description}
% \item[\CTAN{install/macros/latex/contrib/oberdiek.tds.zip}]
% \end{description}
% \emph{TDS} refers to the standard ``A Directory Structure
% for \TeX\ Files'' (\CTAN{tds/tds.pdf}). Directories
% with \xfile{texmf} in their name are usually organized this way.
%
% \subsection{Bundle installation}
%
% \paragraph{Unpacking.} Unpack the \xfile{oberdiek.tds.zip} in the
% TDS tree (also known as \xfile{texmf} tree) of your choice.
% Example (linux):
% \begin{quote}
%   |unzip oberdiek.tds.zip -d ~/texmf|
% \end{quote}
%
% \paragraph{Script installation.}
% Check the directory \xfile{TDS:scripts/oberdiek/} for
% scripts that need further installation steps.
% Package \xpackage{attachfile2} comes with the Perl script
% \xfile{pdfatfi.pl} that should be installed in such a way
% that it can be called as \texttt{pdfatfi}.
% Example (linux):
% \begin{quote}
%   |chmod +x scripts/oberdiek/pdfatfi.pl|\\
%   |cp scripts/oberdiek/pdfatfi.pl /usr/local/bin/|
% \end{quote}
%
% \subsection{Package installation}
%
% \paragraph{Unpacking.} The \xfile{.dtx} file is a self-extracting
% \docstrip\ archive. The files are extracted by running the
% \xfile{.dtx} through \plainTeX:
% \begin{quote}
%   \verb|tex infwarerr.dtx|
% \end{quote}
%
% \paragraph{TDS.} Now the different files must be moved into
% the different directories in your installation TDS tree
% (also known as \xfile{texmf} tree):
% \begin{quote}
% \def\t{^^A
% \begin{tabular}{@{}>{\ttfamily}l@{ $\rightarrow$ }>{\ttfamily}l@{}}
%   infwarerr.sty & tex/generic/oberdiek/infwarerr.sty\\
%   infwarerr.pdf & doc/latex/oberdiek/infwarerr.pdf\\
%   test/infwarerr-test1.tex & doc/latex/oberdiek/test/infwarerr-test1.tex\\
%   test/infwarerr-test2.tex & doc/latex/oberdiek/test/infwarerr-test2.tex\\
%   test/infwarerr-test3.tex & doc/latex/oberdiek/test/infwarerr-test3.tex\\
%   infwarerr.dtx & source/latex/oberdiek/infwarerr.dtx\\
% \end{tabular}^^A
% }^^A
% \sbox0{\t}^^A
% \ifdim\wd0>\linewidth
%   \begingroup
%     \advance\linewidth by\leftmargin
%     \advance\linewidth by\rightmargin
%   \edef\x{\endgroup
%     \def\noexpand\lw{\the\linewidth}^^A
%   }\x
%   \def\lwbox{^^A
%     \leavevmode
%     \hbox to \linewidth{^^A
%       \kern-\leftmargin\relax
%       \hss
%       \usebox0
%       \hss
%       \kern-\rightmargin\relax
%     }^^A
%   }^^A
%   \ifdim\wd0>\lw
%     \sbox0{\small\t}^^A
%     \ifdim\wd0>\linewidth
%       \ifdim\wd0>\lw
%         \sbox0{\footnotesize\t}^^A
%         \ifdim\wd0>\linewidth
%           \ifdim\wd0>\lw
%             \sbox0{\scriptsize\t}^^A
%             \ifdim\wd0>\linewidth
%               \ifdim\wd0>\lw
%                 \sbox0{\tiny\t}^^A
%                 \ifdim\wd0>\linewidth
%                   \lwbox
%                 \else
%                   \usebox0
%                 \fi
%               \else
%                 \lwbox
%               \fi
%             \else
%               \usebox0
%             \fi
%           \else
%             \lwbox
%           \fi
%         \else
%           \usebox0
%         \fi
%       \else
%         \lwbox
%       \fi
%     \else
%       \usebox0
%     \fi
%   \else
%     \lwbox
%   \fi
% \else
%   \usebox0
% \fi
% \end{quote}
% If you have a \xfile{docstrip.cfg} that configures and enables \docstrip's
% TDS installing feature, then some files can already be in the right
% place, see the documentation of \docstrip.
%
% \subsection{Refresh file name databases}
%
% If your \TeX~distribution
% (\teTeX, \mikTeX, \dots) relies on file name databases, you must refresh
% these. For example, \teTeX\ users run \verb|texhash| or
% \verb|mktexlsr|.
%
% \subsection{Some details for the interested}
%
% \paragraph{Attached source.}
%
% The PDF documentation on CTAN also includes the
% \xfile{.dtx} source file. It can be extracted by
% AcrobatReader 6 or higher. Another option is \textsf{pdftk},
% e.g. unpack the file into the current directory:
% \begin{quote}
%   \verb|pdftk infwarerr.pdf unpack_files output .|
% \end{quote}
%
% \paragraph{Unpacking with \LaTeX.}
% The \xfile{.dtx} chooses its action depending on the format:
% \begin{description}
% \item[\plainTeX:] Run \docstrip\ and extract the files.
% \item[\LaTeX:] Generate the documentation.
% \end{description}
% If you insist on using \LaTeX\ for \docstrip\ (really,
% \docstrip\ does not need \LaTeX), then inform the autodetect routine
% about your intention:
% \begin{quote}
%   \verb|latex \let\install=y\input{infwarerr.dtx}|
% \end{quote}
% Do not forget to quote the argument according to the demands
% of your shell.
%
% \paragraph{Generating the documentation.}
% You can use both the \xfile{.dtx} or the \xfile{.drv} to generate
% the documentation. The process can be configured by the
% configuration file \xfile{ltxdoc.cfg}. For instance, put this
% line into this file, if you want to have A4 as paper format:
% \begin{quote}
%   \verb|\PassOptionsToClass{a4paper}{article}|
% \end{quote}
% An example follows how to generate the
% documentation with pdf\LaTeX:
% \begin{quote}
%\begin{verbatim}
%pdflatex infwarerr.dtx
%makeindex -s gind.ist infwarerr.idx
%pdflatex infwarerr.dtx
%makeindex -s gind.ist infwarerr.idx
%pdflatex infwarerr.dtx
%\end{verbatim}
% \end{quote}
%
% \begin{History}
%   \begin{Version}{2007/04/26 v1.0}
%   \item
%     First version.
%   \end{Version}
%   \begin{Version}{2007/06/14 v1.1}
%   \item
%     Small fixes.
%   \end{Version}
%   \begin{Version}{2007/09/09 v1.2}
%   \item
%     Catcode section extended.
%   \end{Version}
%   \begin{Version}{2010/04/08 v1.3}
%   \item
%     Ensure \cs{escapechar}=92 (backslash) for infos/warnings/errors.
%   \end{Version}
% \end{History}
%
% \PrintIndex
%
% \Finale
\endinput

%        (quote the arguments according to the demands of your shell)
% 
% Documentation:
%    (a) If infwarerr.drv is present:
%           latex infwarerr.drv
%    (b) Without infwarerr.drv:
%           latex infwarerr.dtx; ...
%    The class ltxdoc loads the configuration file ltxdoc.cfg
%    if available. Here you can specify further options, e.g.
%    use A4 as paper format:
%       \PassOptionsToClass{a4paper}{article}
% 
%    Programm calls to get the documentation (example):
%       pdflatex infwarerr.dtx
%       makeindex -s gind.ist infwarerr.idx
%       pdflatex infwarerr.dtx
%       makeindex -s gind.ist infwarerr.idx
%       pdflatex infwarerr.dtx
% 
% Installation:
%    TDS:tex/generic/oberdiek/infwarerr.sty
%    TDS:doc/latex/oberdiek/infwarerr.pdf
%    TDS:doc/latex/oberdiek/test/infwarerr-test1.tex
%    TDS:doc/latex/oberdiek/test/infwarerr-test2.tex
%    TDS:doc/latex/oberdiek/test/infwarerr-test3.tex
%    TDS:source/latex/oberdiek/infwarerr.dtx
%
%<*ignore>
\begingroup
  \catcode123=1 %
  \catcode125=2 %
  \def\x{LaTeX2e}%
\expandafter\endgroup
\ifcase 0\ifx\install y1\fi\expandafter
         \ifx\csname processbatchFile\endcsname\relax\else1\fi
         \ifx\fmtname\x\else 1\fi\relax
\else\csname fi\endcsname
%</ignore>
%<*install>
\input docstrip.tex
\Msg{************************************************************************}
\Msg{* Installation}
\Msg{* Package: infwarerr 2010/04/08 v1.3 Providing info/warning/message (HO)}
\Msg{************************************************************************}

\keepsilent
\askforoverwritefalse

\let\MetaPrefix\relax
\preamble

This is a generated file.

Project: infwarerr
Version: 2010/04/08 v1.3

Copyright (C) 2007, 2010 by
   Heiko Oberdiek <heiko.oberdiek at googlemail.com>

This work may be distributed and/or modified under the
conditions of the LaTeX Project Public License, either
version 1.3c of this license or (at your option) any later
version. This version of this license is in
   http://www.latex-project.org/lppl/lppl-1-3c.txt
and the latest version of this license is in
   http://www.latex-project.org/lppl.txt
and version 1.3 or later is part of all distributions of
LaTeX version 2005/12/01 or later.

This work has the LPPL maintenance status "maintained".

This Current Maintainer of this work is Heiko Oberdiek.

The Base Interpreter refers to any `TeX-Format',
because some files are installed in TDS:tex/generic//.

This work consists of the main source file infwarerr.dtx
and the derived files
   infwarerr.sty, infwarerr.pdf, infwarerr.ins, infwarerr.drv,
   infwarerr-test1.tex, infwarerr-test2.tex,
   infwarerr-test3.tex.

\endpreamble
\let\MetaPrefix\DoubleperCent

\generate{%
  \file{infwarerr.ins}{\from{infwarerr.dtx}{install}}%
  \file{infwarerr.drv}{\from{infwarerr.dtx}{driver}}%
  \usedir{tex/generic/oberdiek}%
  \file{infwarerr.sty}{\from{infwarerr.dtx}{package}}%
  \usedir{doc/latex/oberdiek/test}%
  \file{infwarerr-test1.tex}{\from{infwarerr.dtx}{test1}}%
  \file{infwarerr-test2.tex}{\from{infwarerr.dtx}{test2}}%
  \file{infwarerr-test3.tex}{\from{infwarerr.dtx}{test3}}%
}

\catcode32=13\relax% active space
\let =\space%
\Msg{************************************************************************}
\Msg{*}
\Msg{* To finish the installation you have to move the following}
\Msg{* file into a directory searched by TeX:}
\Msg{*}
\Msg{*     infwarerr.sty}
\Msg{*}
\Msg{* To produce the documentation run the file `infwarerr.drv'}
\Msg{* through LaTeX.}
\Msg{*}
\Msg{* Happy TeXing!}
\Msg{*}
\Msg{************************************************************************}

\endbatchfile
%</install>
%<*ignore>
\fi
%</ignore>
%<*driver>
\NeedsTeXFormat{LaTeX2e}
\ProvidesFile{infwarerr.drv}%
  [2010/04/08 v1.3 Providing info/warning/message (HO)]%
\documentclass{ltxdoc}
\usepackage{holtxdoc}[2011/02/04]
\begin{document}
  \DocInput{infwarerr.dtx}%
\end{document}
%</driver>
% \fi
%
% \CheckSum{871}
%
% \CharacterTable
%  {Upper-case    \A\B\C\D\E\F\G\H\I\J\K\L\M\N\O\P\Q\R\S\T\U\V\W\X\Y\Z
%   Lower-case    \a\b\c\d\e\f\g\h\i\j\k\l\m\n\o\p\q\r\s\t\u\v\w\x\y\z
%   Digits        \0\1\2\3\4\5\6\7\8\9
%   Exclamation   \!     Double quote  \"     Hash (number) \#
%   Dollar        \$     Percent       \%     Ampersand     \&
%   Acute accent  \'     Left paren    \(     Right paren   \)
%   Asterisk      \*     Plus          \+     Comma         \,
%   Minus         \-     Point         \.     Solidus       \/
%   Colon         \:     Semicolon     \;     Less than     \<
%   Equals        \=     Greater than  \>     Question mark \?
%   Commercial at \@     Left bracket  \[     Backslash     \\
%   Right bracket \]     Circumflex    \^     Underscore    \_
%   Grave accent  \`     Left brace    \{     Vertical bar  \|
%   Right brace   \}     Tilde         \~}
%
% \GetFileInfo{infwarerr.drv}
%
% \title{The \xpackage{infwarerr} package}
% \date{2010/04/08 v1.3}
% \author{Heiko Oberdiek\\\xemail{heiko.oberdiek at googlemail.com}}
%
% \maketitle
%
% \begin{abstract}
% This package provides a complete set of macros for informations,
% warnings and error messages with support for \plainTeX.
% \end{abstract}
%
% \tableofcontents
%
% \section{Documentation}
%
% Package \xpackage{infwarerr} defines the following set of commands,
% given as pseudo regular expression:
%   \begin{quote}
%     \verb+\@(Package|Class)(Info|InfoNoLine|Warning|WarningNoLine|Error)+^^A
%     \kern-\rightmargin
%   \end{quote}
% The first argument is the package or class in question. The second
% argument contains the message text. \cs{MessageBreak} is supported.
% In case of error messages, the third argument contains the help
% text. Supported are \LaTeX's message texts \cs{@ehc}:
%   \begin{quote}
%     |Try typing  <return>  to proceed|\\
%     |If that doesn't work, type  X <return>  to quit.|
%   \end{quote}
% and \cs{@ehd}:
%   \begin{quote}
%     |You're in trouble here.  Try typing  <return>  to proceed|\\
%     |If that doesn't work, type  X <return>  to quit.|
%   \end{quote}
%
% The package is intended for macro writers that want to write
% code that works for both \LaTeX\ and \plainTeX. If \LaTeX\ is
% detected, then the macros of this package will only be wrappers
% for \LaTeX's macros. Otherwise \LaTeX's macros will be mimicked
% for use with \plainTeX.
%
% \StopEventually{
% }
%
% \section{Implementation}
%
%    \begin{macrocode}
%<*package>
%    \end{macrocode}
%
% \subsection{Reload check and package identification}
%    Reload check, especially if the package is not used with \LaTeX.
%    \begin{macrocode}
\begingroup\catcode61\catcode48\catcode32=10\relax%
  \catcode13=5 % ^^M
  \endlinechar=13 %
  \catcode35=6 % #
  \catcode39=12 % '
  \catcode44=12 % ,
  \catcode45=12 % -
  \catcode46=12 % .
  \catcode58=12 % :
  \catcode64=11 % @
  \catcode123=1 % {
  \catcode125=2 % }
  \expandafter\let\expandafter\x\csname ver@infwarerr.sty\endcsname
  \ifx\x\relax % plain-TeX, first loading
  \else
    \def\empty{}%
    \ifx\x\empty % LaTeX, first loading,
      % variable is initialized, but \ProvidesPackage not yet seen
    \else
      \expandafter\ifx\csname PackageInfo\endcsname\relax
        \def\x#1#2{%
          \immediate\write-1{Package #1 Info: #2.}%
        }%
      \else
        \def\x#1#2{\PackageInfo{#1}{#2, stopped}}%
      \fi
      \x{infwarerr}{The package is already loaded}%
      \aftergroup\endinput
    \fi
  \fi
\endgroup%
%    \end{macrocode}
%    Package identification:
%    \begin{macrocode}
\begingroup\catcode61\catcode48\catcode32=10\relax%
  \catcode13=5 % ^^M
  \endlinechar=13 %
  \catcode35=6 % #
  \catcode39=12 % '
  \catcode40=12 % (
  \catcode41=12 % )
  \catcode44=12 % ,
  \catcode45=12 % -
  \catcode46=12 % .
  \catcode47=12 % /
  \catcode58=12 % :
  \catcode64=11 % @
  \catcode91=12 % [
  \catcode93=12 % ]
  \catcode123=1 % {
  \catcode125=2 % }
  \expandafter\ifx\csname ProvidesPackage\endcsname\relax
    \def\x#1#2#3[#4]{\endgroup
      \immediate\write-1{Package: #3 #4}%
      \xdef#1{#4}%
    }%
  \else
    \def\x#1#2[#3]{\endgroup
      #2[{#3}]%
      \ifx#1\@undefined
        \xdef#1{#3}%
      \fi
      \ifx#1\relax
        \xdef#1{#3}%
      \fi
    }%
  \fi
\expandafter\x\csname ver@infwarerr.sty\endcsname
\ProvidesPackage{infwarerr}%
  [2010/04/08 v1.3 Providing info/warning/message (HO)]%
%    \end{macrocode}
%
% \subsection{Catcodes}
%
%    \begin{macrocode}
\begingroup\catcode61\catcode48\catcode32=10\relax%
  \catcode13=5 % ^^M
  \endlinechar=13 %
  \catcode123=1 % {
  \catcode125=2 % }
  \catcode64=11 % @
  \def\x{\endgroup
    \expandafter\edef\csname InfWarErr@AtEnd\endcsname{%
      \endlinechar=\the\endlinechar\relax
      \catcode13=\the\catcode13\relax
      \catcode32=\the\catcode32\relax
      \catcode35=\the\catcode35\relax
      \catcode61=\the\catcode61\relax
      \catcode64=\the\catcode64\relax
      \catcode123=\the\catcode123\relax
      \catcode125=\the\catcode125\relax
    }%
  }%
\x\catcode61\catcode48\catcode32=10\relax%
\catcode13=5 % ^^M
\endlinechar=13 %
\catcode35=6 % #
\catcode64=11 % @
\catcode123=1 % {
\catcode125=2 % }
\def\TMP@EnsureCode#1#2{%
  \edef\InfWarErr@AtEnd{%
    \InfWarErr@AtEnd
    \catcode#1=\the\catcode#1\relax
  }%
  \catcode#1=#2\relax
}
\TMP@EnsureCode{10}{12}% ^^J
\TMP@EnsureCode{39}{12}% '
\TMP@EnsureCode{40}{12}% (
\TMP@EnsureCode{41}{12}% )
\TMP@EnsureCode{44}{12}% ,
\TMP@EnsureCode{45}{12}% -
\TMP@EnsureCode{46}{12}% .
\TMP@EnsureCode{58}{12}% :
\TMP@EnsureCode{60}{12}% <
\TMP@EnsureCode{62}{12}% >
\TMP@EnsureCode{94}{7}% ^
\edef\InfWarErr@AtEnd{\InfWarErr@AtEnd\noexpand\endinput}
%    \end{macrocode}
%
% \subsection{\LaTeX\ detection}
%
%    \begin{macrocode}
\begingroup
  \let\x=Y%
  \def\check#1{%
    \@check{Generic}{#1}%
    \@check{Package}{#1}%
    \@check{Class}{#1}%
  }%
  \def\@check#1#2{%
    \expandafter\ifx\csname #1#2\endcsname\relax
      \let\x=N%
    \fi
  }%
  \check{Info}%
  \check{Warning}%
  \check{Error}%
  \@check{Package}{WarningNoLine}%
  \@check{Class}{WarningNoLine}%
  \@check{}{@ehc}%
  \@check{}{@ehd}%
\expandafter\endgroup
\ifx\x Y%
%    \end{macrocode}
%
% \subsection{Version for \LaTeX}
%
%    \begin{macro}{\@PackageInfo}
%    \begin{macrocode}
  \let\@PackageInfo\PackageInfo
%    \end{macrocode}
%    \end{macro}
%    \begin{macro}{\@PackageInfoNoLine}
%    \begin{macrocode}
  \def\@PackageInfoNoLine#1#2{%
    \PackageInfo{#1}{#2\@gobble}%
  }%
%    \end{macrocode}
%    \end{macro}
%    \begin{macro}{\@ClassInfo}
%    \begin{macrocode}
  \let\@ClassInfo\ClassInfo
%    \end{macrocode}
%    \end{macro}
%    \begin{macro}{\@ClassInfoNoLine}
%    \begin{macrocode}
  \def\@ClassInfoNoLine#1#2{%
    \ClassInfo{#1}{#2\@gobble}%
  }%
%    \end{macrocode}
%    \end{macro}
%    \begin{macro}{\@PackageWarning}
%    \begin{macrocode}
  \let\@PackageWarning\PackageWarning
%    \end{macrocode}
%    \end{macro}
%    \begin{macro}{\@PackageWarningNoLine}
%    \begin{macrocode}
  \let\@PackageWarningNoLine\PackageWarningNoLine
%    \end{macrocode}
%    \end{macro}
%    \begin{macro}{\@ClassWarning}
%    \begin{macrocode}
  \let\@ClassWarning\ClassWarning
%    \end{macrocode}
%    \end{macro}
%    \begin{macro}{\@ClassWarningNoLine}
%    \begin{macrocode}
  \let\@ClassWarningNoLine\ClassWarningNoLine
%    \end{macrocode}
%    \end{macro}
%    \begin{macro}{\@PackageError}
%    \begin{macrocode}
  \let\@PackageError\PackageError
%    \end{macrocode}
%    \end{macro}
%    \begin{macro}{\@ClassError}
%    \begin{macrocode}
  \let\@ClassError\ClassError
%    \end{macrocode}
%    \end{macro}
%    \begin{macrocode}
  \expandafter\InfWarErr@AtEnd
\fi%
%    \end{macrocode}
%
% \subsection{Version for \plainTeX}
%
%    \begin{macro}{\InfWarErr@online}
%    \begin{macrocode}
\begingroup\expandafter\endgroup
\ifcase\expandafter\ifx\csname inputlineno\endcsname\relax
         0 %
       \else
         \ifnum\inputlineno<0 %
           0 %
         \else
           1 %
         \fi
       \fi
  \def\InfWarErr@online{}%
\else
  \def\InfWarErr@online{ on input line \the\inputlineno}%
\fi
%    \end{macrocode}
%    \end{macro}
%
%    \begin{macro}{\InfWarErr@protected}
%    \begin{macrocode}
\begingroup\expandafter\expandafter\expandafter\endgroup
\expandafter\ifx\csname protected\endcsname\relax
  \def\InfWarErr@protected{}%
\else
  \let\InfWarErr@protected\protected
\fi
%    \end{macrocode}
%    \end{macro}
%
%    \begin{macro}{\InfWarErr@unused}
%    \begin{macrocode}
\chardef\InfWarErr@unused=16 %
%    \end{macrocode}
%    \end{macro}
%
%    \begin{macro}{\InfWarErr@ehc}
%    \begin{macrocode}
\def\InfWarErr@ehc{%
  Try typing \space <return> \space to proceed.\MessageBreak
  If that doesn't work, type \space X <return> \space to quit.%
}
%    \end{macrocode}
%    \end{macro}
%    \begin{macro}{\InfWarErr@ehd}
%    \begin{macrocode}
\def\InfWarErr@ehd{%
  You're in trouble here. \space\@ehc
}
%    \end{macrocode}
%    \end{macro}
%
%    \begin{macro}{\InfWarErr@GenericInfo}
%    \begin{macrocode}
\InfWarErr@protected\def\InfWarErr@GenericInfo#1#2{%
  \begingroup
    \let\on@line\InfWarErr@online
    \long\def\@gobble##1{}%
    \def\space{ }%
    \def\@spaces{\space\space\space\space}%
    \def\MessageBreak{^^J#1}%
    \newlinechar=10 %
    \escapechar=92 %
    \immediate\write-1{#2\on@line.}%
  \endgroup
}
%    \end{macrocode}
%    \end{macro}
%    \begin{macro}{\InfWarErr@GenericWarning}
%    \begin{macrocode}
\InfWarErr@protected\def\InfWarErr@GenericWarning#1#2{%
  \begingroup
    \let\on@line\InfWarErr@online
    \long\def\@gobble##1{}%
    \def\space{ }%
    \def\@spaces{\space\space\space\space}%
    \def\MessageBreak{^^J#1}%
    \newlinechar=10 %
    \escapechar=92 %
    \immediate\write\InfWarErr@unused{^^J#2\on@line.^^J}%
  \endgroup
}
%    \end{macrocode}
%    \end{macro}
%    \begin{macro}{\InfWarErr@GenericError}
%    Old \TeX\ versions below 3.141 where |^^J| is not working
%    as newline in error message and help texts are not supported.
%    \begin{macrocode}
\InfWarErr@protected\def\InfWarErr@GenericError#1#2#3#4{%
  \begingroup
    \immediate\write\InfWarErr@unused{}%
    \def\space{ }%
    \def\@spaces{\space\space\space\space}%
    \let\@ehc\InfWarErr@ehc
    \let\@ehd\InfWarErr@ehd
    \let\on@line\InfWarErr@online
    \def\MessageBreak{^^J}%
    \newlinechar=10 %
    \escapechar=92 %
    \edef\InfWarErr@help{\noexpand\errhelp{#4}}%
    \InfWarErr@help
    \def\MessageBreak{^^J#1}%
    \errmessage{%
      #2.^^J^^J%
      #3^^J%
      Type \space H <return> \space for immediate help.^^J ...^^J%
    }%
  \endgroup
}
%    \end{macrocode}
%    \end{macro}
%
%    \begin{macro}{\@PackageInfo}
%    \begin{macrocode}
\def\@PackageInfo#1#2{%
  \InfWarErr@GenericInfo{%
    (#1) \@spaces\@spaces\@spaces
  }{%
    Package #1 Info: #2%
  }%
}
%    \end{macrocode}
%    \end{macro}
%    \begin{macro}{\@PackageInfoNoLine}
%    \begin{macrocode}
\def\@PackageInfoNoLine#1#2{%
  \@PackageInfo{#1}{#2\@gobble}%
}
%    \end{macrocode}
%    \end{macro}
%    \begin{macro}{\@ClassInfo}
%    \begin{macrocode}
\def\@ClassInfo#1#2{%
  \InfWarErr@GenericInfo{%
    (#1) \space\space\@spaces\@spaces
  }{%
    Class #1 Info: #2%
  }%
}
%    \end{macrocode}
%    \end{macro}
%    \begin{macro}{\@ClassInfoNoLine}
%    \begin{macrocode}
\def\@ClassInfoNoLine#1#2{%
  \@ClassInfo{#1}{#2\@gobble}%
}
%    \end{macrocode}
%    \end{macro}
%    \begin{macro}{\@PackageWarning}
%    \begin{macrocode}
\def\@PackageWarning#1#2{%
  \InfWarErr@GenericWarning{%
    (#1)\@spaces\@spaces\@spaces\@spaces
  }{%
    Package #1 Warning: #2%
  }%
}
%    \end{macrocode}
%    \end{macro}
%    \begin{macro}{\@PackageWarningNoLine}
%    \begin{macrocode}
\def\@PackageWarningNoLine#1#2{%
  \@PackageWarning{#1}{#2\@gobble}%
}
%    \end{macrocode}
%    \end{macro}
%    \begin{macro}{\@ClassWarning}
%    \begin{macrocode}
\def\@ClassWarning#1#2{%
  \InfWarErr@GenericWarning{%
    (#1) \space\@spaces\@spaces\@spaces
  }{%
    Class #1 Warning: #2%
  }%
}
%    \end{macrocode}
%    \end{macro}
%    \begin{macro}{\@ClassWarningNoLine}
%    \begin{macrocode}
\def\@ClassWarningNoLine#1#2{%
  \@ClassWarning{#1}{#2\@gobble}%
}
%    \end{macrocode}
%    \end{macro}
%    \begin{macro}{\@PackageError}
%    \begin{macrocode}
\def\@PackageError#1#2#3{%
  \InfWarErr@GenericError{%
    (#1)\@spaces\@spaces\@spaces\@spaces
  }{%
    Package #1 Error: #2%
  }{%
    See the #1 package documentation for explanation.%
  }{#3}%
}
%    \end{macrocode}
%    \end{macro}
%    \begin{macro}{\@ClassError}
%    \begin{macrocode}
\def\@ClassError#1#2#3{%
  \InfWarErr@GenericError{%
    (#1) \space\@spaces\@spaces\@spaces
  }{%
    Class #1 Error: #2%
  }{%
    See the #1 class documentation for explanation.%
  }{#3}%
}
%    \end{macrocode}
%    \end{macro}
%
%    \begin{macrocode}
\InfWarErr@AtEnd%
%</package>
%    \end{macrocode}
%
% \section{Test}
%
% \subsection{Catcode checks for loading}
%
%    \begin{macrocode}
%<*test1>
%    \end{macrocode}
%    \begin{macrocode}
\catcode`\{=1 %
\catcode`\}=2 %
\catcode`\#=6 %
\catcode`\@=11 %
\expandafter\ifx\csname count@\endcsname\relax
  \countdef\count@=255 %
\fi
\expandafter\ifx\csname @gobble\endcsname\relax
  \long\def\@gobble#1{}%
\fi
\expandafter\ifx\csname @firstofone\endcsname\relax
  \long\def\@firstofone#1{#1}%
\fi
\expandafter\ifx\csname loop\endcsname\relax
  \expandafter\@firstofone
\else
  \expandafter\@gobble
\fi
{%
  \def\loop#1\repeat{%
    \def\body{#1}%
    \iterate
  }%
  \def\iterate{%
    \body
      \let\next\iterate
    \else
      \let\next\relax
    \fi
    \next
  }%
  \let\repeat=\fi
}%
\def\RestoreCatcodes{}
\count@=0 %
\loop
  \edef\RestoreCatcodes{%
    \RestoreCatcodes
    \catcode\the\count@=\the\catcode\count@\relax
  }%
\ifnum\count@<255 %
  \advance\count@ 1 %
\repeat

\def\RangeCatcodeInvalid#1#2{%
  \count@=#1\relax
  \loop
    \catcode\count@=15 %
  \ifnum\count@<#2\relax
    \advance\count@ 1 %
  \repeat
}
\def\RangeCatcodeCheck#1#2#3{%
  \count@=#1\relax
  \loop
    \ifnum#3=\catcode\count@
    \else
      \errmessage{%
        Character \the\count@\space
        with wrong catcode \the\catcode\count@\space
        instead of \number#3%
      }%
    \fi
  \ifnum\count@<#2\relax
    \advance\count@ 1 %
  \repeat
}
\def\space{ }
\expandafter\ifx\csname LoadCommand\endcsname\relax
  \def\LoadCommand{\input infwarerr.sty\relax}%
\fi
\def\Test{%
  \RangeCatcodeInvalid{0}{47}%
  \RangeCatcodeInvalid{58}{64}%
  \RangeCatcodeInvalid{91}{96}%
  \RangeCatcodeInvalid{123}{255}%
  \catcode`\@=12 %
  \catcode`\\=0 %
  \catcode`\%=14 %
  \LoadCommand
  \RangeCatcodeCheck{0}{36}{15}%
  \RangeCatcodeCheck{37}{37}{14}%
  \RangeCatcodeCheck{38}{47}{15}%
  \RangeCatcodeCheck{48}{57}{12}%
  \RangeCatcodeCheck{58}{63}{15}%
  \RangeCatcodeCheck{64}{64}{12}%
  \RangeCatcodeCheck{65}{90}{11}%
  \RangeCatcodeCheck{91}{91}{15}%
  \RangeCatcodeCheck{92}{92}{0}%
  \RangeCatcodeCheck{93}{96}{15}%
  \RangeCatcodeCheck{97}{122}{11}%
  \RangeCatcodeCheck{123}{255}{15}%
  \RestoreCatcodes
}
\Test
\csname @@end\endcsname
\end
%    \end{macrocode}
%    \begin{macrocode}
%</test1>
%    \end{macrocode}
%
% \subsection{Macro tests}
%
%    \begin{macrocode}
%<*test2>
\let\PackageInfo\relax
\input infwarerr.sty\relax
\let\OrgWrite\write
\chardef\WriteUnused=16 %
\def\msg#{\immediate\OrgWrite\WriteUnused}
\msg{File: infwarerr-test2.tex 2010/04/08 v1.3 Test file for plain-TeX}
\catcode64=11 %
%</test2>
%    \end{macrocode}
%
%    \begin{macrocode}
%<*test3>
\NeedsTeXFormat{LaTeX2e}
\ProvidesFile{infwarerr-test3.tex}[2010/04/08 v1.3 Test file for LaTeX]
\RequirePackage{infwarerr}[2010/04/08]
\makeatletter
\let\OrgWrite\write
\chardef\WriteUnused=\@unused
\def\msg#{\immediate\OrgWrite\WriteUnused}
%</test3>
%    \end{macrocode}
%
%    \begin{macrocode}
%<*test2|test3>
\def\pkgname{PACKAGE}
\def\clsname{CLASS}
\def\msgtext{MESSAGE\MessageBreak SECOND LINE}
\def\hlptext{HELP\MessageBreak SECOND LINE}

\let\OrgWrite\write
\let\OrgErrMessage\errmessage

\def\write#1#{%
  \@write{#1}%
}
\def\@write#1#2{%
  \OrgWrite#1{#2}%
  \xdef\WriteNum{\number#1}%
  \xdef\WriteText{#2}%
}
\def\TestFailed#1#2{%
  \OrgErrMessage{Test (\string#1) failed: #2}%
}
\def\TestWrite#1#2#3{%
  \ifnum\WriteNum=#2\relax
  \else
    \TestFailed#1{write stream number: \WriteNum <> #2}%
  \fi
  \begingroup
    \let~\space
    \edef\TestString{#3}%
    \ifx\WriteText\TestString
    \else
      \msg{}%
      \msg{[\WriteText]}%
      \msg{<>}%
      \msg{[#3]}% hash-ok
      \TestFailed#1{write text}%
    \fi
  \endgroup
}
\newtoks\ErrHelpToks
\def\errmessage#1{%
  \global\ErrHelpToks=\expandafter{\the\errhelp}%
  \msg{**************** error message ****************}%
  \msg{! #1}%
  \msg{**************** help message *****************}%
  \msg{\the\errhelp}%
  \msg{***********************************************}%
  \xdef\ErrMessage{! #1}%
}
\def\TestInfo{%
  \@PackageInfo\pkgname\msgtext
  \TestWrite\@PackageInfo{-1}{%
    Package PACKAGE Info: MESSAGE^^J%
    (PACKAGE)~~~~~~~~~~~~~SECOND LINE on input line \the\inputlineno.%
  }%
  \@PackageInfoNoLine\pkgname\msgtext
  \TestWrite\@PackageInfoNoLine{-1}{%
    Package PACKAGE Info: MESSAGE^^J%
    (PACKAGE)~~~~~~~~~~~~~SECOND LINE.%
  }%
  \@ClassInfo\clsname\msgtext
  \TestWrite\@ClassInfo{-1}{%
    Class CLASS Info: MESSAGE^^J%
    (CLASS)~~~~~~~~~~~SECOND LINE on input line \the\inputlineno.%
  }%
  \@ClassInfoNoLine\clsname\msgtext
  \TestWrite\@ClassInfoNoLine{-1}{%
    Class CLASS Info: MESSAGE^^J%
    (CLASS)~~~~~~~~~~~SECOND LINE.%
  }%
}
\def\TestWarning{%
  \@PackageWarning\pkgname\msgtext
  \TestWrite\@PackageWarning\WriteUnused{%
    ^^J%
    Package PACKAGE Warning: MESSAGE^^J%
    (PACKAGE)~~~~~~~~~~~~~~~~SECOND LINE on input line \the\inputlineno.%
    ^^J%
  }%
  \@PackageWarningNoLine\pkgname\msgtext
  \TestWrite\@PackageWarningNoLine\WriteUnused{%
    ^^J%
    Package PACKAGE Warning: MESSAGE^^J%
    (PACKAGE)~~~~~~~~~~~~~~~~SECOND LINE.%
    ^^J%
  }%
  \@ClassWarning\clsname\msgtext
  \TestWrite\@ClassWarning\WriteUnused{%
    ^^J%
    Class CLASS Warning: MESSAGE^^J%
    (CLASS)~~~~~~~~~~~~~~SECOND LINE on input line \the\inputlineno.%
    ^^J%
  }%
  \@ClassWarningNoLine\clsname\msgtext
  \TestWrite\@ClassWarningNoLine\WriteUnused{%
    ^^J%
    Class CLASS Warning: MESSAGE^^J%
    (CLASS)~~~~~~~~~~~~~~SECOND LINE.%
    ^^J%
  }%
}
\def\TestError{%
%<test3>  \let\write\OrgWrite
%<test3>  \def\plaindots{}
%<test2>  \def\plaindots{.^^J ...^^J}
  \@PackageError\pkgname\msgtext\hlptext
  \@TestError\@PackageError{%
    ! Package PACKAGE Error: MESSAGE^^J%
    (PACKAGE)~~~~~~~~~~~~~~~~SECOND LINE.^^J%
    ^^J%
    See the PACKAGE package documentation for explanation.^^J%
    Type~~H <return>~~for immediate help\plaindots
  }{%
    HELP^^J%
    SECOND LINE%
  }%
  \@ClassError\clsname\msgtext\hlptext
  \@TestError\@ClassError{%
    ! Class CLASS Error: MESSAGE^^J%
    (CLASS)~~~~~~~~~~~~~~SECOND LINE.^^J%
    ^^J%
    See the CLASS class documentation for explanation.^^J%
    Type~~H <return>~~for immediate help\plaindots
  }{%
    HELP^^J%
    SECOND LINE%
  }%
  \@PackageError\pkgname\msgtext\@ehc
  \@TestError\@PackageError{%
    ! Package PACKAGE Error: MESSAGE^^J%
    (PACKAGE)~~~~~~~~~~~~~~~~SECOND LINE.^^J%
    ^^J%
    See the PACKAGE package documentation for explanation.^^J%
    Type~~H <return>~~for immediate help\plaindots
  }{%
    Try typing~~<return>~~to proceed.^^J%
    If that doesn't work, type~~X <return>~~to quit.%
  }%
  \@PackageError\pkgname\msgtext\@ehd
  \@TestError\@PackageError{%
    ! Package PACKAGE Error: MESSAGE^^J%
    (PACKAGE)~~~~~~~~~~~~~~~~SECOND LINE.^^J%
    ^^J%
    See the PACKAGE package documentation for explanation.^^J%
    Type~~H <return>~~for immediate help\plaindots
  }{%
    You're in trouble here.~~Try typing~~<return>~~to proceed.^^J%
    If that doesn't work, type~~X <return>~~to quit.%
  }%
}
\def\@TestError#1#2#3{%
  \begingroup
    \let~\space
    \edef\x{#2}%
%<test3>  \@onelevel@sanitize\x
%<test3>  \@onelevel@sanitize\ErrMessage
    \ifx\x\ErrMessage
    \else
      \msg{}%
      \msg{[\ErrMessage]}%
      \msg{<>}%
      \msg{[\x]}%
      \TestFailed#1{errmessage text}%
    \fi
    \edef\x{\the\ErrHelpToks}%
    \edef\y{#3}%
    \ifx\x\y
    \else
      \msg{}%
      \msg{[\the\ErrHelpToks]}%
      \msg{<>}%
      \msg{[\y]}%
      \TestFailed#1{errhelp text}%
    \fi
  \endgroup
}
\TestInfo
\TestWarning
\TestError
%</test2|test3>
%<test2>\end
%<test3>\@@end
%    \end{macrocode}
%
%
% \section{Installation}
%
% \subsection{Download}
%
% \paragraph{Package.} This package is available on
% CTAN\footnote{\url{ftp://ftp.ctan.org/tex-archive/}}:
% \begin{description}
% \item[\CTAN{macros/latex/contrib/oberdiek/infwarerr.dtx}] The source file.
% \item[\CTAN{macros/latex/contrib/oberdiek/infwarerr.pdf}] Documentation.
% \end{description}
%
%
% \paragraph{Bundle.} All the packages of the bundle `oberdiek'
% are also available in a TDS compliant ZIP archive. There
% the packages are already unpacked and the documentation files
% are generated. The files and directories obey the TDS standard.
% \begin{description}
% \item[\CTAN{install/macros/latex/contrib/oberdiek.tds.zip}]
% \end{description}
% \emph{TDS} refers to the standard ``A Directory Structure
% for \TeX\ Files'' (\CTAN{tds/tds.pdf}). Directories
% with \xfile{texmf} in their name are usually organized this way.
%
% \subsection{Bundle installation}
%
% \paragraph{Unpacking.} Unpack the \xfile{oberdiek.tds.zip} in the
% TDS tree (also known as \xfile{texmf} tree) of your choice.
% Example (linux):
% \begin{quote}
%   |unzip oberdiek.tds.zip -d ~/texmf|
% \end{quote}
%
% \paragraph{Script installation.}
% Check the directory \xfile{TDS:scripts/oberdiek/} for
% scripts that need further installation steps.
% Package \xpackage{attachfile2} comes with the Perl script
% \xfile{pdfatfi.pl} that should be installed in such a way
% that it can be called as \texttt{pdfatfi}.
% Example (linux):
% \begin{quote}
%   |chmod +x scripts/oberdiek/pdfatfi.pl|\\
%   |cp scripts/oberdiek/pdfatfi.pl /usr/local/bin/|
% \end{quote}
%
% \subsection{Package installation}
%
% \paragraph{Unpacking.} The \xfile{.dtx} file is a self-extracting
% \docstrip\ archive. The files are extracted by running the
% \xfile{.dtx} through \plainTeX:
% \begin{quote}
%   \verb|tex infwarerr.dtx|
% \end{quote}
%
% \paragraph{TDS.} Now the different files must be moved into
% the different directories in your installation TDS tree
% (also known as \xfile{texmf} tree):
% \begin{quote}
% \def\t{^^A
% \begin{tabular}{@{}>{\ttfamily}l@{ $\rightarrow$ }>{\ttfamily}l@{}}
%   infwarerr.sty & tex/generic/oberdiek/infwarerr.sty\\
%   infwarerr.pdf & doc/latex/oberdiek/infwarerr.pdf\\
%   test/infwarerr-test1.tex & doc/latex/oberdiek/test/infwarerr-test1.tex\\
%   test/infwarerr-test2.tex & doc/latex/oberdiek/test/infwarerr-test2.tex\\
%   test/infwarerr-test3.tex & doc/latex/oberdiek/test/infwarerr-test3.tex\\
%   infwarerr.dtx & source/latex/oberdiek/infwarerr.dtx\\
% \end{tabular}^^A
% }^^A
% \sbox0{\t}^^A
% \ifdim\wd0>\linewidth
%   \begingroup
%     \advance\linewidth by\leftmargin
%     \advance\linewidth by\rightmargin
%   \edef\x{\endgroup
%     \def\noexpand\lw{\the\linewidth}^^A
%   }\x
%   \def\lwbox{^^A
%     \leavevmode
%     \hbox to \linewidth{^^A
%       \kern-\leftmargin\relax
%       \hss
%       \usebox0
%       \hss
%       \kern-\rightmargin\relax
%     }^^A
%   }^^A
%   \ifdim\wd0>\lw
%     \sbox0{\small\t}^^A
%     \ifdim\wd0>\linewidth
%       \ifdim\wd0>\lw
%         \sbox0{\footnotesize\t}^^A
%         \ifdim\wd0>\linewidth
%           \ifdim\wd0>\lw
%             \sbox0{\scriptsize\t}^^A
%             \ifdim\wd0>\linewidth
%               \ifdim\wd0>\lw
%                 \sbox0{\tiny\t}^^A
%                 \ifdim\wd0>\linewidth
%                   \lwbox
%                 \else
%                   \usebox0
%                 \fi
%               \else
%                 \lwbox
%               \fi
%             \else
%               \usebox0
%             \fi
%           \else
%             \lwbox
%           \fi
%         \else
%           \usebox0
%         \fi
%       \else
%         \lwbox
%       \fi
%     \else
%       \usebox0
%     \fi
%   \else
%     \lwbox
%   \fi
% \else
%   \usebox0
% \fi
% \end{quote}
% If you have a \xfile{docstrip.cfg} that configures and enables \docstrip's
% TDS installing feature, then some files can already be in the right
% place, see the documentation of \docstrip.
%
% \subsection{Refresh file name databases}
%
% If your \TeX~distribution
% (\teTeX, \mikTeX, \dots) relies on file name databases, you must refresh
% these. For example, \teTeX\ users run \verb|texhash| or
% \verb|mktexlsr|.
%
% \subsection{Some details for the interested}
%
% \paragraph{Attached source.}
%
% The PDF documentation on CTAN also includes the
% \xfile{.dtx} source file. It can be extracted by
% AcrobatReader 6 or higher. Another option is \textsf{pdftk},
% e.g. unpack the file into the current directory:
% \begin{quote}
%   \verb|pdftk infwarerr.pdf unpack_files output .|
% \end{quote}
%
% \paragraph{Unpacking with \LaTeX.}
% The \xfile{.dtx} chooses its action depending on the format:
% \begin{description}
% \item[\plainTeX:] Run \docstrip\ and extract the files.
% \item[\LaTeX:] Generate the documentation.
% \end{description}
% If you insist on using \LaTeX\ for \docstrip\ (really,
% \docstrip\ does not need \LaTeX), then inform the autodetect routine
% about your intention:
% \begin{quote}
%   \verb|latex \let\install=y% \iffalse meta-comment
% 
% File: infwarerr.dtx
% Version: 2010/04/08 v1.3
% 
% Copyright (C) 2007, 2010 by
%    Heiko Oberdiek <heiko.oberdiek at googlemail.com>
% 
% This work may be distributed and/or modified under the
% conditions of the LaTeX Project Public License, either
% version 1.3c of this license or (at your option) any later
% version. This version of this license is in
%    http://www.latex-project.org/lppl/lppl-1-3c.txt
% and the latest version of this license is in
%    http://www.latex-project.org/lppl.txt
% and version 1.3 or later is part of all distributions of
% LaTeX version 2005/12/01 or later.
% 
% This work has the LPPL maintenance status "maintained".
% 
% This Current Maintainer of this work is Heiko Oberdiek.
% 
% The Base Interpreter refers to any `TeX-Format',
% because some files are installed in TDS:tex/generic//.
% 
% This work consists of the main source file infwarerr.dtx
% and the derived files
%    infwarerr.sty, infwarerr.pdf, infwarerr.ins, infwarerr.drv,
%    infwarerr-test1.tex, infwarerr-test2.tex,
%    infwarerr-test3.tex.
% 
% Distribution:
%    CTAN:macros/latex/contrib/oberdiek/infwarerr.dtx
%    CTAN:macros/latex/contrib/oberdiek/infwarerr.pdf
% 
% Unpacking:
%    (a) If infwarerr.ins is present:
%           tex infwarerr.ins
%    (b) Without infwarerr.ins:
%           tex infwarerr.dtx
%    (c) If you insist on using LaTeX
%           latex \let\install=y\input{infwarerr.dtx}
%        (quote the arguments according to the demands of your shell)
% 
% Documentation:
%    (a) If infwarerr.drv is present:
%           latex infwarerr.drv
%    (b) Without infwarerr.drv:
%           latex infwarerr.dtx; ...
%    The class ltxdoc loads the configuration file ltxdoc.cfg
%    if available. Here you can specify further options, e.g.
%    use A4 as paper format:
%       \PassOptionsToClass{a4paper}{article}
% 
%    Programm calls to get the documentation (example):
%       pdflatex infwarerr.dtx
%       makeindex -s gind.ist infwarerr.idx
%       pdflatex infwarerr.dtx
%       makeindex -s gind.ist infwarerr.idx
%       pdflatex infwarerr.dtx
% 
% Installation:
%    TDS:tex/generic/oberdiek/infwarerr.sty
%    TDS:doc/latex/oberdiek/infwarerr.pdf
%    TDS:doc/latex/oberdiek/test/infwarerr-test1.tex
%    TDS:doc/latex/oberdiek/test/infwarerr-test2.tex
%    TDS:doc/latex/oberdiek/test/infwarerr-test3.tex
%    TDS:source/latex/oberdiek/infwarerr.dtx
%
%<*ignore>
\begingroup
  \catcode123=1 %
  \catcode125=2 %
  \def\x{LaTeX2e}%
\expandafter\endgroup
\ifcase 0\ifx\install y1\fi\expandafter
         \ifx\csname processbatchFile\endcsname\relax\else1\fi
         \ifx\fmtname\x\else 1\fi\relax
\else\csname fi\endcsname
%</ignore>
%<*install>
\input docstrip.tex
\Msg{************************************************************************}
\Msg{* Installation}
\Msg{* Package: infwarerr 2010/04/08 v1.3 Providing info/warning/message (HO)}
\Msg{************************************************************************}

\keepsilent
\askforoverwritefalse

\let\MetaPrefix\relax
\preamble

This is a generated file.

Project: infwarerr
Version: 2010/04/08 v1.3

Copyright (C) 2007, 2010 by
   Heiko Oberdiek <heiko.oberdiek at googlemail.com>

This work may be distributed and/or modified under the
conditions of the LaTeX Project Public License, either
version 1.3c of this license or (at your option) any later
version. This version of this license is in
   http://www.latex-project.org/lppl/lppl-1-3c.txt
and the latest version of this license is in
   http://www.latex-project.org/lppl.txt
and version 1.3 or later is part of all distributions of
LaTeX version 2005/12/01 or later.

This work has the LPPL maintenance status "maintained".

This Current Maintainer of this work is Heiko Oberdiek.

The Base Interpreter refers to any `TeX-Format',
because some files are installed in TDS:tex/generic//.

This work consists of the main source file infwarerr.dtx
and the derived files
   infwarerr.sty, infwarerr.pdf, infwarerr.ins, infwarerr.drv,
   infwarerr-test1.tex, infwarerr-test2.tex,
   infwarerr-test3.tex.

\endpreamble
\let\MetaPrefix\DoubleperCent

\generate{%
  \file{infwarerr.ins}{\from{infwarerr.dtx}{install}}%
  \file{infwarerr.drv}{\from{infwarerr.dtx}{driver}}%
  \usedir{tex/generic/oberdiek}%
  \file{infwarerr.sty}{\from{infwarerr.dtx}{package}}%
  \usedir{doc/latex/oberdiek/test}%
  \file{infwarerr-test1.tex}{\from{infwarerr.dtx}{test1}}%
  \file{infwarerr-test2.tex}{\from{infwarerr.dtx}{test2}}%
  \file{infwarerr-test3.tex}{\from{infwarerr.dtx}{test3}}%
}

\catcode32=13\relax% active space
\let =\space%
\Msg{************************************************************************}
\Msg{*}
\Msg{* To finish the installation you have to move the following}
\Msg{* file into a directory searched by TeX:}
\Msg{*}
\Msg{*     infwarerr.sty}
\Msg{*}
\Msg{* To produce the documentation run the file `infwarerr.drv'}
\Msg{* through LaTeX.}
\Msg{*}
\Msg{* Happy TeXing!}
\Msg{*}
\Msg{************************************************************************}

\endbatchfile
%</install>
%<*ignore>
\fi
%</ignore>
%<*driver>
\NeedsTeXFormat{LaTeX2e}
\ProvidesFile{infwarerr.drv}%
  [2010/04/08 v1.3 Providing info/warning/message (HO)]%
\documentclass{ltxdoc}
\usepackage{holtxdoc}[2011/02/04]
\begin{document}
  \DocInput{infwarerr.dtx}%
\end{document}
%</driver>
% \fi
%
% \CheckSum{871}
%
% \CharacterTable
%  {Upper-case    \A\B\C\D\E\F\G\H\I\J\K\L\M\N\O\P\Q\R\S\T\U\V\W\X\Y\Z
%   Lower-case    \a\b\c\d\e\f\g\h\i\j\k\l\m\n\o\p\q\r\s\t\u\v\w\x\y\z
%   Digits        \0\1\2\3\4\5\6\7\8\9
%   Exclamation   \!     Double quote  \"     Hash (number) \#
%   Dollar        \$     Percent       \%     Ampersand     \&
%   Acute accent  \'     Left paren    \(     Right paren   \)
%   Asterisk      \*     Plus          \+     Comma         \,
%   Minus         \-     Point         \.     Solidus       \/
%   Colon         \:     Semicolon     \;     Less than     \<
%   Equals        \=     Greater than  \>     Question mark \?
%   Commercial at \@     Left bracket  \[     Backslash     \\
%   Right bracket \]     Circumflex    \^     Underscore    \_
%   Grave accent  \`     Left brace    \{     Vertical bar  \|
%   Right brace   \}     Tilde         \~}
%
% \GetFileInfo{infwarerr.drv}
%
% \title{The \xpackage{infwarerr} package}
% \date{2010/04/08 v1.3}
% \author{Heiko Oberdiek\\\xemail{heiko.oberdiek at googlemail.com}}
%
% \maketitle
%
% \begin{abstract}
% This package provides a complete set of macros for informations,
% warnings and error messages with support for \plainTeX.
% \end{abstract}
%
% \tableofcontents
%
% \section{Documentation}
%
% Package \xpackage{infwarerr} defines the following set of commands,
% given as pseudo regular expression:
%   \begin{quote}
%     \verb+\@(Package|Class)(Info|InfoNoLine|Warning|WarningNoLine|Error)+^^A
%     \kern-\rightmargin
%   \end{quote}
% The first argument is the package or class in question. The second
% argument contains the message text. \cs{MessageBreak} is supported.
% In case of error messages, the third argument contains the help
% text. Supported are \LaTeX's message texts \cs{@ehc}:
%   \begin{quote}
%     |Try typing  <return>  to proceed|\\
%     |If that doesn't work, type  X <return>  to quit.|
%   \end{quote}
% and \cs{@ehd}:
%   \begin{quote}
%     |You're in trouble here.  Try typing  <return>  to proceed|\\
%     |If that doesn't work, type  X <return>  to quit.|
%   \end{quote}
%
% The package is intended for macro writers that want to write
% code that works for both \LaTeX\ and \plainTeX. If \LaTeX\ is
% detected, then the macros of this package will only be wrappers
% for \LaTeX's macros. Otherwise \LaTeX's macros will be mimicked
% for use with \plainTeX.
%
% \StopEventually{
% }
%
% \section{Implementation}
%
%    \begin{macrocode}
%<*package>
%    \end{macrocode}
%
% \subsection{Reload check and package identification}
%    Reload check, especially if the package is not used with \LaTeX.
%    \begin{macrocode}
\begingroup\catcode61\catcode48\catcode32=10\relax%
  \catcode13=5 % ^^M
  \endlinechar=13 %
  \catcode35=6 % #
  \catcode39=12 % '
  \catcode44=12 % ,
  \catcode45=12 % -
  \catcode46=12 % .
  \catcode58=12 % :
  \catcode64=11 % @
  \catcode123=1 % {
  \catcode125=2 % }
  \expandafter\let\expandafter\x\csname ver@infwarerr.sty\endcsname
  \ifx\x\relax % plain-TeX, first loading
  \else
    \def\empty{}%
    \ifx\x\empty % LaTeX, first loading,
      % variable is initialized, but \ProvidesPackage not yet seen
    \else
      \expandafter\ifx\csname PackageInfo\endcsname\relax
        \def\x#1#2{%
          \immediate\write-1{Package #1 Info: #2.}%
        }%
      \else
        \def\x#1#2{\PackageInfo{#1}{#2, stopped}}%
      \fi
      \x{infwarerr}{The package is already loaded}%
      \aftergroup\endinput
    \fi
  \fi
\endgroup%
%    \end{macrocode}
%    Package identification:
%    \begin{macrocode}
\begingroup\catcode61\catcode48\catcode32=10\relax%
  \catcode13=5 % ^^M
  \endlinechar=13 %
  \catcode35=6 % #
  \catcode39=12 % '
  \catcode40=12 % (
  \catcode41=12 % )
  \catcode44=12 % ,
  \catcode45=12 % -
  \catcode46=12 % .
  \catcode47=12 % /
  \catcode58=12 % :
  \catcode64=11 % @
  \catcode91=12 % [
  \catcode93=12 % ]
  \catcode123=1 % {
  \catcode125=2 % }
  \expandafter\ifx\csname ProvidesPackage\endcsname\relax
    \def\x#1#2#3[#4]{\endgroup
      \immediate\write-1{Package: #3 #4}%
      \xdef#1{#4}%
    }%
  \else
    \def\x#1#2[#3]{\endgroup
      #2[{#3}]%
      \ifx#1\@undefined
        \xdef#1{#3}%
      \fi
      \ifx#1\relax
        \xdef#1{#3}%
      \fi
    }%
  \fi
\expandafter\x\csname ver@infwarerr.sty\endcsname
\ProvidesPackage{infwarerr}%
  [2010/04/08 v1.3 Providing info/warning/message (HO)]%
%    \end{macrocode}
%
% \subsection{Catcodes}
%
%    \begin{macrocode}
\begingroup\catcode61\catcode48\catcode32=10\relax%
  \catcode13=5 % ^^M
  \endlinechar=13 %
  \catcode123=1 % {
  \catcode125=2 % }
  \catcode64=11 % @
  \def\x{\endgroup
    \expandafter\edef\csname InfWarErr@AtEnd\endcsname{%
      \endlinechar=\the\endlinechar\relax
      \catcode13=\the\catcode13\relax
      \catcode32=\the\catcode32\relax
      \catcode35=\the\catcode35\relax
      \catcode61=\the\catcode61\relax
      \catcode64=\the\catcode64\relax
      \catcode123=\the\catcode123\relax
      \catcode125=\the\catcode125\relax
    }%
  }%
\x\catcode61\catcode48\catcode32=10\relax%
\catcode13=5 % ^^M
\endlinechar=13 %
\catcode35=6 % #
\catcode64=11 % @
\catcode123=1 % {
\catcode125=2 % }
\def\TMP@EnsureCode#1#2{%
  \edef\InfWarErr@AtEnd{%
    \InfWarErr@AtEnd
    \catcode#1=\the\catcode#1\relax
  }%
  \catcode#1=#2\relax
}
\TMP@EnsureCode{10}{12}% ^^J
\TMP@EnsureCode{39}{12}% '
\TMP@EnsureCode{40}{12}% (
\TMP@EnsureCode{41}{12}% )
\TMP@EnsureCode{44}{12}% ,
\TMP@EnsureCode{45}{12}% -
\TMP@EnsureCode{46}{12}% .
\TMP@EnsureCode{58}{12}% :
\TMP@EnsureCode{60}{12}% <
\TMP@EnsureCode{62}{12}% >
\TMP@EnsureCode{94}{7}% ^
\edef\InfWarErr@AtEnd{\InfWarErr@AtEnd\noexpand\endinput}
%    \end{macrocode}
%
% \subsection{\LaTeX\ detection}
%
%    \begin{macrocode}
\begingroup
  \let\x=Y%
  \def\check#1{%
    \@check{Generic}{#1}%
    \@check{Package}{#1}%
    \@check{Class}{#1}%
  }%
  \def\@check#1#2{%
    \expandafter\ifx\csname #1#2\endcsname\relax
      \let\x=N%
    \fi
  }%
  \check{Info}%
  \check{Warning}%
  \check{Error}%
  \@check{Package}{WarningNoLine}%
  \@check{Class}{WarningNoLine}%
  \@check{}{@ehc}%
  \@check{}{@ehd}%
\expandafter\endgroup
\ifx\x Y%
%    \end{macrocode}
%
% \subsection{Version for \LaTeX}
%
%    \begin{macro}{\@PackageInfo}
%    \begin{macrocode}
  \let\@PackageInfo\PackageInfo
%    \end{macrocode}
%    \end{macro}
%    \begin{macro}{\@PackageInfoNoLine}
%    \begin{macrocode}
  \def\@PackageInfoNoLine#1#2{%
    \PackageInfo{#1}{#2\@gobble}%
  }%
%    \end{macrocode}
%    \end{macro}
%    \begin{macro}{\@ClassInfo}
%    \begin{macrocode}
  \let\@ClassInfo\ClassInfo
%    \end{macrocode}
%    \end{macro}
%    \begin{macro}{\@ClassInfoNoLine}
%    \begin{macrocode}
  \def\@ClassInfoNoLine#1#2{%
    \ClassInfo{#1}{#2\@gobble}%
  }%
%    \end{macrocode}
%    \end{macro}
%    \begin{macro}{\@PackageWarning}
%    \begin{macrocode}
  \let\@PackageWarning\PackageWarning
%    \end{macrocode}
%    \end{macro}
%    \begin{macro}{\@PackageWarningNoLine}
%    \begin{macrocode}
  \let\@PackageWarningNoLine\PackageWarningNoLine
%    \end{macrocode}
%    \end{macro}
%    \begin{macro}{\@ClassWarning}
%    \begin{macrocode}
  \let\@ClassWarning\ClassWarning
%    \end{macrocode}
%    \end{macro}
%    \begin{macro}{\@ClassWarningNoLine}
%    \begin{macrocode}
  \let\@ClassWarningNoLine\ClassWarningNoLine
%    \end{macrocode}
%    \end{macro}
%    \begin{macro}{\@PackageError}
%    \begin{macrocode}
  \let\@PackageError\PackageError
%    \end{macrocode}
%    \end{macro}
%    \begin{macro}{\@ClassError}
%    \begin{macrocode}
  \let\@ClassError\ClassError
%    \end{macrocode}
%    \end{macro}
%    \begin{macrocode}
  \expandafter\InfWarErr@AtEnd
\fi%
%    \end{macrocode}
%
% \subsection{Version for \plainTeX}
%
%    \begin{macro}{\InfWarErr@online}
%    \begin{macrocode}
\begingroup\expandafter\endgroup
\ifcase\expandafter\ifx\csname inputlineno\endcsname\relax
         0 %
       \else
         \ifnum\inputlineno<0 %
           0 %
         \else
           1 %
         \fi
       \fi
  \def\InfWarErr@online{}%
\else
  \def\InfWarErr@online{ on input line \the\inputlineno}%
\fi
%    \end{macrocode}
%    \end{macro}
%
%    \begin{macro}{\InfWarErr@protected}
%    \begin{macrocode}
\begingroup\expandafter\expandafter\expandafter\endgroup
\expandafter\ifx\csname protected\endcsname\relax
  \def\InfWarErr@protected{}%
\else
  \let\InfWarErr@protected\protected
\fi
%    \end{macrocode}
%    \end{macro}
%
%    \begin{macro}{\InfWarErr@unused}
%    \begin{macrocode}
\chardef\InfWarErr@unused=16 %
%    \end{macrocode}
%    \end{macro}
%
%    \begin{macro}{\InfWarErr@ehc}
%    \begin{macrocode}
\def\InfWarErr@ehc{%
  Try typing \space <return> \space to proceed.\MessageBreak
  If that doesn't work, type \space X <return> \space to quit.%
}
%    \end{macrocode}
%    \end{macro}
%    \begin{macro}{\InfWarErr@ehd}
%    \begin{macrocode}
\def\InfWarErr@ehd{%
  You're in trouble here. \space\@ehc
}
%    \end{macrocode}
%    \end{macro}
%
%    \begin{macro}{\InfWarErr@GenericInfo}
%    \begin{macrocode}
\InfWarErr@protected\def\InfWarErr@GenericInfo#1#2{%
  \begingroup
    \let\on@line\InfWarErr@online
    \long\def\@gobble##1{}%
    \def\space{ }%
    \def\@spaces{\space\space\space\space}%
    \def\MessageBreak{^^J#1}%
    \newlinechar=10 %
    \escapechar=92 %
    \immediate\write-1{#2\on@line.}%
  \endgroup
}
%    \end{macrocode}
%    \end{macro}
%    \begin{macro}{\InfWarErr@GenericWarning}
%    \begin{macrocode}
\InfWarErr@protected\def\InfWarErr@GenericWarning#1#2{%
  \begingroup
    \let\on@line\InfWarErr@online
    \long\def\@gobble##1{}%
    \def\space{ }%
    \def\@spaces{\space\space\space\space}%
    \def\MessageBreak{^^J#1}%
    \newlinechar=10 %
    \escapechar=92 %
    \immediate\write\InfWarErr@unused{^^J#2\on@line.^^J}%
  \endgroup
}
%    \end{macrocode}
%    \end{macro}
%    \begin{macro}{\InfWarErr@GenericError}
%    Old \TeX\ versions below 3.141 where |^^J| is not working
%    as newline in error message and help texts are not supported.
%    \begin{macrocode}
\InfWarErr@protected\def\InfWarErr@GenericError#1#2#3#4{%
  \begingroup
    \immediate\write\InfWarErr@unused{}%
    \def\space{ }%
    \def\@spaces{\space\space\space\space}%
    \let\@ehc\InfWarErr@ehc
    \let\@ehd\InfWarErr@ehd
    \let\on@line\InfWarErr@online
    \def\MessageBreak{^^J}%
    \newlinechar=10 %
    \escapechar=92 %
    \edef\InfWarErr@help{\noexpand\errhelp{#4}}%
    \InfWarErr@help
    \def\MessageBreak{^^J#1}%
    \errmessage{%
      #2.^^J^^J%
      #3^^J%
      Type \space H <return> \space for immediate help.^^J ...^^J%
    }%
  \endgroup
}
%    \end{macrocode}
%    \end{macro}
%
%    \begin{macro}{\@PackageInfo}
%    \begin{macrocode}
\def\@PackageInfo#1#2{%
  \InfWarErr@GenericInfo{%
    (#1) \@spaces\@spaces\@spaces
  }{%
    Package #1 Info: #2%
  }%
}
%    \end{macrocode}
%    \end{macro}
%    \begin{macro}{\@PackageInfoNoLine}
%    \begin{macrocode}
\def\@PackageInfoNoLine#1#2{%
  \@PackageInfo{#1}{#2\@gobble}%
}
%    \end{macrocode}
%    \end{macro}
%    \begin{macro}{\@ClassInfo}
%    \begin{macrocode}
\def\@ClassInfo#1#2{%
  \InfWarErr@GenericInfo{%
    (#1) \space\space\@spaces\@spaces
  }{%
    Class #1 Info: #2%
  }%
}
%    \end{macrocode}
%    \end{macro}
%    \begin{macro}{\@ClassInfoNoLine}
%    \begin{macrocode}
\def\@ClassInfoNoLine#1#2{%
  \@ClassInfo{#1}{#2\@gobble}%
}
%    \end{macrocode}
%    \end{macro}
%    \begin{macro}{\@PackageWarning}
%    \begin{macrocode}
\def\@PackageWarning#1#2{%
  \InfWarErr@GenericWarning{%
    (#1)\@spaces\@spaces\@spaces\@spaces
  }{%
    Package #1 Warning: #2%
  }%
}
%    \end{macrocode}
%    \end{macro}
%    \begin{macro}{\@PackageWarningNoLine}
%    \begin{macrocode}
\def\@PackageWarningNoLine#1#2{%
  \@PackageWarning{#1}{#2\@gobble}%
}
%    \end{macrocode}
%    \end{macro}
%    \begin{macro}{\@ClassWarning}
%    \begin{macrocode}
\def\@ClassWarning#1#2{%
  \InfWarErr@GenericWarning{%
    (#1) \space\@spaces\@spaces\@spaces
  }{%
    Class #1 Warning: #2%
  }%
}
%    \end{macrocode}
%    \end{macro}
%    \begin{macro}{\@ClassWarningNoLine}
%    \begin{macrocode}
\def\@ClassWarningNoLine#1#2{%
  \@ClassWarning{#1}{#2\@gobble}%
}
%    \end{macrocode}
%    \end{macro}
%    \begin{macro}{\@PackageError}
%    \begin{macrocode}
\def\@PackageError#1#2#3{%
  \InfWarErr@GenericError{%
    (#1)\@spaces\@spaces\@spaces\@spaces
  }{%
    Package #1 Error: #2%
  }{%
    See the #1 package documentation for explanation.%
  }{#3}%
}
%    \end{macrocode}
%    \end{macro}
%    \begin{macro}{\@ClassError}
%    \begin{macrocode}
\def\@ClassError#1#2#3{%
  \InfWarErr@GenericError{%
    (#1) \space\@spaces\@spaces\@spaces
  }{%
    Class #1 Error: #2%
  }{%
    See the #1 class documentation for explanation.%
  }{#3}%
}
%    \end{macrocode}
%    \end{macro}
%
%    \begin{macrocode}
\InfWarErr@AtEnd%
%</package>
%    \end{macrocode}
%
% \section{Test}
%
% \subsection{Catcode checks for loading}
%
%    \begin{macrocode}
%<*test1>
%    \end{macrocode}
%    \begin{macrocode}
\catcode`\{=1 %
\catcode`\}=2 %
\catcode`\#=6 %
\catcode`\@=11 %
\expandafter\ifx\csname count@\endcsname\relax
  \countdef\count@=255 %
\fi
\expandafter\ifx\csname @gobble\endcsname\relax
  \long\def\@gobble#1{}%
\fi
\expandafter\ifx\csname @firstofone\endcsname\relax
  \long\def\@firstofone#1{#1}%
\fi
\expandafter\ifx\csname loop\endcsname\relax
  \expandafter\@firstofone
\else
  \expandafter\@gobble
\fi
{%
  \def\loop#1\repeat{%
    \def\body{#1}%
    \iterate
  }%
  \def\iterate{%
    \body
      \let\next\iterate
    \else
      \let\next\relax
    \fi
    \next
  }%
  \let\repeat=\fi
}%
\def\RestoreCatcodes{}
\count@=0 %
\loop
  \edef\RestoreCatcodes{%
    \RestoreCatcodes
    \catcode\the\count@=\the\catcode\count@\relax
  }%
\ifnum\count@<255 %
  \advance\count@ 1 %
\repeat

\def\RangeCatcodeInvalid#1#2{%
  \count@=#1\relax
  \loop
    \catcode\count@=15 %
  \ifnum\count@<#2\relax
    \advance\count@ 1 %
  \repeat
}
\def\RangeCatcodeCheck#1#2#3{%
  \count@=#1\relax
  \loop
    \ifnum#3=\catcode\count@
    \else
      \errmessage{%
        Character \the\count@\space
        with wrong catcode \the\catcode\count@\space
        instead of \number#3%
      }%
    \fi
  \ifnum\count@<#2\relax
    \advance\count@ 1 %
  \repeat
}
\def\space{ }
\expandafter\ifx\csname LoadCommand\endcsname\relax
  \def\LoadCommand{\input infwarerr.sty\relax}%
\fi
\def\Test{%
  \RangeCatcodeInvalid{0}{47}%
  \RangeCatcodeInvalid{58}{64}%
  \RangeCatcodeInvalid{91}{96}%
  \RangeCatcodeInvalid{123}{255}%
  \catcode`\@=12 %
  \catcode`\\=0 %
  \catcode`\%=14 %
  \LoadCommand
  \RangeCatcodeCheck{0}{36}{15}%
  \RangeCatcodeCheck{37}{37}{14}%
  \RangeCatcodeCheck{38}{47}{15}%
  \RangeCatcodeCheck{48}{57}{12}%
  \RangeCatcodeCheck{58}{63}{15}%
  \RangeCatcodeCheck{64}{64}{12}%
  \RangeCatcodeCheck{65}{90}{11}%
  \RangeCatcodeCheck{91}{91}{15}%
  \RangeCatcodeCheck{92}{92}{0}%
  \RangeCatcodeCheck{93}{96}{15}%
  \RangeCatcodeCheck{97}{122}{11}%
  \RangeCatcodeCheck{123}{255}{15}%
  \RestoreCatcodes
}
\Test
\csname @@end\endcsname
\end
%    \end{macrocode}
%    \begin{macrocode}
%</test1>
%    \end{macrocode}
%
% \subsection{Macro tests}
%
%    \begin{macrocode}
%<*test2>
\let\PackageInfo\relax
\input infwarerr.sty\relax
\let\OrgWrite\write
\chardef\WriteUnused=16 %
\def\msg#{\immediate\OrgWrite\WriteUnused}
\msg{File: infwarerr-test2.tex 2010/04/08 v1.3 Test file for plain-TeX}
\catcode64=11 %
%</test2>
%    \end{macrocode}
%
%    \begin{macrocode}
%<*test3>
\NeedsTeXFormat{LaTeX2e}
\ProvidesFile{infwarerr-test3.tex}[2010/04/08 v1.3 Test file for LaTeX]
\RequirePackage{infwarerr}[2010/04/08]
\makeatletter
\let\OrgWrite\write
\chardef\WriteUnused=\@unused
\def\msg#{\immediate\OrgWrite\WriteUnused}
%</test3>
%    \end{macrocode}
%
%    \begin{macrocode}
%<*test2|test3>
\def\pkgname{PACKAGE}
\def\clsname{CLASS}
\def\msgtext{MESSAGE\MessageBreak SECOND LINE}
\def\hlptext{HELP\MessageBreak SECOND LINE}

\let\OrgWrite\write
\let\OrgErrMessage\errmessage

\def\write#1#{%
  \@write{#1}%
}
\def\@write#1#2{%
  \OrgWrite#1{#2}%
  \xdef\WriteNum{\number#1}%
  \xdef\WriteText{#2}%
}
\def\TestFailed#1#2{%
  \OrgErrMessage{Test (\string#1) failed: #2}%
}
\def\TestWrite#1#2#3{%
  \ifnum\WriteNum=#2\relax
  \else
    \TestFailed#1{write stream number: \WriteNum <> #2}%
  \fi
  \begingroup
    \let~\space
    \edef\TestString{#3}%
    \ifx\WriteText\TestString
    \else
      \msg{}%
      \msg{[\WriteText]}%
      \msg{<>}%
      \msg{[#3]}% hash-ok
      \TestFailed#1{write text}%
    \fi
  \endgroup
}
\newtoks\ErrHelpToks
\def\errmessage#1{%
  \global\ErrHelpToks=\expandafter{\the\errhelp}%
  \msg{**************** error message ****************}%
  \msg{! #1}%
  \msg{**************** help message *****************}%
  \msg{\the\errhelp}%
  \msg{***********************************************}%
  \xdef\ErrMessage{! #1}%
}
\def\TestInfo{%
  \@PackageInfo\pkgname\msgtext
  \TestWrite\@PackageInfo{-1}{%
    Package PACKAGE Info: MESSAGE^^J%
    (PACKAGE)~~~~~~~~~~~~~SECOND LINE on input line \the\inputlineno.%
  }%
  \@PackageInfoNoLine\pkgname\msgtext
  \TestWrite\@PackageInfoNoLine{-1}{%
    Package PACKAGE Info: MESSAGE^^J%
    (PACKAGE)~~~~~~~~~~~~~SECOND LINE.%
  }%
  \@ClassInfo\clsname\msgtext
  \TestWrite\@ClassInfo{-1}{%
    Class CLASS Info: MESSAGE^^J%
    (CLASS)~~~~~~~~~~~SECOND LINE on input line \the\inputlineno.%
  }%
  \@ClassInfoNoLine\clsname\msgtext
  \TestWrite\@ClassInfoNoLine{-1}{%
    Class CLASS Info: MESSAGE^^J%
    (CLASS)~~~~~~~~~~~SECOND LINE.%
  }%
}
\def\TestWarning{%
  \@PackageWarning\pkgname\msgtext
  \TestWrite\@PackageWarning\WriteUnused{%
    ^^J%
    Package PACKAGE Warning: MESSAGE^^J%
    (PACKAGE)~~~~~~~~~~~~~~~~SECOND LINE on input line \the\inputlineno.%
    ^^J%
  }%
  \@PackageWarningNoLine\pkgname\msgtext
  \TestWrite\@PackageWarningNoLine\WriteUnused{%
    ^^J%
    Package PACKAGE Warning: MESSAGE^^J%
    (PACKAGE)~~~~~~~~~~~~~~~~SECOND LINE.%
    ^^J%
  }%
  \@ClassWarning\clsname\msgtext
  \TestWrite\@ClassWarning\WriteUnused{%
    ^^J%
    Class CLASS Warning: MESSAGE^^J%
    (CLASS)~~~~~~~~~~~~~~SECOND LINE on input line \the\inputlineno.%
    ^^J%
  }%
  \@ClassWarningNoLine\clsname\msgtext
  \TestWrite\@ClassWarningNoLine\WriteUnused{%
    ^^J%
    Class CLASS Warning: MESSAGE^^J%
    (CLASS)~~~~~~~~~~~~~~SECOND LINE.%
    ^^J%
  }%
}
\def\TestError{%
%<test3>  \let\write\OrgWrite
%<test3>  \def\plaindots{}
%<test2>  \def\plaindots{.^^J ...^^J}
  \@PackageError\pkgname\msgtext\hlptext
  \@TestError\@PackageError{%
    ! Package PACKAGE Error: MESSAGE^^J%
    (PACKAGE)~~~~~~~~~~~~~~~~SECOND LINE.^^J%
    ^^J%
    See the PACKAGE package documentation for explanation.^^J%
    Type~~H <return>~~for immediate help\plaindots
  }{%
    HELP^^J%
    SECOND LINE%
  }%
  \@ClassError\clsname\msgtext\hlptext
  \@TestError\@ClassError{%
    ! Class CLASS Error: MESSAGE^^J%
    (CLASS)~~~~~~~~~~~~~~SECOND LINE.^^J%
    ^^J%
    See the CLASS class documentation for explanation.^^J%
    Type~~H <return>~~for immediate help\plaindots
  }{%
    HELP^^J%
    SECOND LINE%
  }%
  \@PackageError\pkgname\msgtext\@ehc
  \@TestError\@PackageError{%
    ! Package PACKAGE Error: MESSAGE^^J%
    (PACKAGE)~~~~~~~~~~~~~~~~SECOND LINE.^^J%
    ^^J%
    See the PACKAGE package documentation for explanation.^^J%
    Type~~H <return>~~for immediate help\plaindots
  }{%
    Try typing~~<return>~~to proceed.^^J%
    If that doesn't work, type~~X <return>~~to quit.%
  }%
  \@PackageError\pkgname\msgtext\@ehd
  \@TestError\@PackageError{%
    ! Package PACKAGE Error: MESSAGE^^J%
    (PACKAGE)~~~~~~~~~~~~~~~~SECOND LINE.^^J%
    ^^J%
    See the PACKAGE package documentation for explanation.^^J%
    Type~~H <return>~~for immediate help\plaindots
  }{%
    You're in trouble here.~~Try typing~~<return>~~to proceed.^^J%
    If that doesn't work, type~~X <return>~~to quit.%
  }%
}
\def\@TestError#1#2#3{%
  \begingroup
    \let~\space
    \edef\x{#2}%
%<test3>  \@onelevel@sanitize\x
%<test3>  \@onelevel@sanitize\ErrMessage
    \ifx\x\ErrMessage
    \else
      \msg{}%
      \msg{[\ErrMessage]}%
      \msg{<>}%
      \msg{[\x]}%
      \TestFailed#1{errmessage text}%
    \fi
    \edef\x{\the\ErrHelpToks}%
    \edef\y{#3}%
    \ifx\x\y
    \else
      \msg{}%
      \msg{[\the\ErrHelpToks]}%
      \msg{<>}%
      \msg{[\y]}%
      \TestFailed#1{errhelp text}%
    \fi
  \endgroup
}
\TestInfo
\TestWarning
\TestError
%</test2|test3>
%<test2>\end
%<test3>\@@end
%    \end{macrocode}
%
%
% \section{Installation}
%
% \subsection{Download}
%
% \paragraph{Package.} This package is available on
% CTAN\footnote{\url{ftp://ftp.ctan.org/tex-archive/}}:
% \begin{description}
% \item[\CTAN{macros/latex/contrib/oberdiek/infwarerr.dtx}] The source file.
% \item[\CTAN{macros/latex/contrib/oberdiek/infwarerr.pdf}] Documentation.
% \end{description}
%
%
% \paragraph{Bundle.} All the packages of the bundle `oberdiek'
% are also available in a TDS compliant ZIP archive. There
% the packages are already unpacked and the documentation files
% are generated. The files and directories obey the TDS standard.
% \begin{description}
% \item[\CTAN{install/macros/latex/contrib/oberdiek.tds.zip}]
% \end{description}
% \emph{TDS} refers to the standard ``A Directory Structure
% for \TeX\ Files'' (\CTAN{tds/tds.pdf}). Directories
% with \xfile{texmf} in their name are usually organized this way.
%
% \subsection{Bundle installation}
%
% \paragraph{Unpacking.} Unpack the \xfile{oberdiek.tds.zip} in the
% TDS tree (also known as \xfile{texmf} tree) of your choice.
% Example (linux):
% \begin{quote}
%   |unzip oberdiek.tds.zip -d ~/texmf|
% \end{quote}
%
% \paragraph{Script installation.}
% Check the directory \xfile{TDS:scripts/oberdiek/} for
% scripts that need further installation steps.
% Package \xpackage{attachfile2} comes with the Perl script
% \xfile{pdfatfi.pl} that should be installed in such a way
% that it can be called as \texttt{pdfatfi}.
% Example (linux):
% \begin{quote}
%   |chmod +x scripts/oberdiek/pdfatfi.pl|\\
%   |cp scripts/oberdiek/pdfatfi.pl /usr/local/bin/|
% \end{quote}
%
% \subsection{Package installation}
%
% \paragraph{Unpacking.} The \xfile{.dtx} file is a self-extracting
% \docstrip\ archive. The files are extracted by running the
% \xfile{.dtx} through \plainTeX:
% \begin{quote}
%   \verb|tex infwarerr.dtx|
% \end{quote}
%
% \paragraph{TDS.} Now the different files must be moved into
% the different directories in your installation TDS tree
% (also known as \xfile{texmf} tree):
% \begin{quote}
% \def\t{^^A
% \begin{tabular}{@{}>{\ttfamily}l@{ $\rightarrow$ }>{\ttfamily}l@{}}
%   infwarerr.sty & tex/generic/oberdiek/infwarerr.sty\\
%   infwarerr.pdf & doc/latex/oberdiek/infwarerr.pdf\\
%   test/infwarerr-test1.tex & doc/latex/oberdiek/test/infwarerr-test1.tex\\
%   test/infwarerr-test2.tex & doc/latex/oberdiek/test/infwarerr-test2.tex\\
%   test/infwarerr-test3.tex & doc/latex/oberdiek/test/infwarerr-test3.tex\\
%   infwarerr.dtx & source/latex/oberdiek/infwarerr.dtx\\
% \end{tabular}^^A
% }^^A
% \sbox0{\t}^^A
% \ifdim\wd0>\linewidth
%   \begingroup
%     \advance\linewidth by\leftmargin
%     \advance\linewidth by\rightmargin
%   \edef\x{\endgroup
%     \def\noexpand\lw{\the\linewidth}^^A
%   }\x
%   \def\lwbox{^^A
%     \leavevmode
%     \hbox to \linewidth{^^A
%       \kern-\leftmargin\relax
%       \hss
%       \usebox0
%       \hss
%       \kern-\rightmargin\relax
%     }^^A
%   }^^A
%   \ifdim\wd0>\lw
%     \sbox0{\small\t}^^A
%     \ifdim\wd0>\linewidth
%       \ifdim\wd0>\lw
%         \sbox0{\footnotesize\t}^^A
%         \ifdim\wd0>\linewidth
%           \ifdim\wd0>\lw
%             \sbox0{\scriptsize\t}^^A
%             \ifdim\wd0>\linewidth
%               \ifdim\wd0>\lw
%                 \sbox0{\tiny\t}^^A
%                 \ifdim\wd0>\linewidth
%                   \lwbox
%                 \else
%                   \usebox0
%                 \fi
%               \else
%                 \lwbox
%               \fi
%             \else
%               \usebox0
%             \fi
%           \else
%             \lwbox
%           \fi
%         \else
%           \usebox0
%         \fi
%       \else
%         \lwbox
%       \fi
%     \else
%       \usebox0
%     \fi
%   \else
%     \lwbox
%   \fi
% \else
%   \usebox0
% \fi
% \end{quote}
% If you have a \xfile{docstrip.cfg} that configures and enables \docstrip's
% TDS installing feature, then some files can already be in the right
% place, see the documentation of \docstrip.
%
% \subsection{Refresh file name databases}
%
% If your \TeX~distribution
% (\teTeX, \mikTeX, \dots) relies on file name databases, you must refresh
% these. For example, \teTeX\ users run \verb|texhash| or
% \verb|mktexlsr|.
%
% \subsection{Some details for the interested}
%
% \paragraph{Attached source.}
%
% The PDF documentation on CTAN also includes the
% \xfile{.dtx} source file. It can be extracted by
% AcrobatReader 6 or higher. Another option is \textsf{pdftk},
% e.g. unpack the file into the current directory:
% \begin{quote}
%   \verb|pdftk infwarerr.pdf unpack_files output .|
% \end{quote}
%
% \paragraph{Unpacking with \LaTeX.}
% The \xfile{.dtx} chooses its action depending on the format:
% \begin{description}
% \item[\plainTeX:] Run \docstrip\ and extract the files.
% \item[\LaTeX:] Generate the documentation.
% \end{description}
% If you insist on using \LaTeX\ for \docstrip\ (really,
% \docstrip\ does not need \LaTeX), then inform the autodetect routine
% about your intention:
% \begin{quote}
%   \verb|latex \let\install=y\input{infwarerr.dtx}|
% \end{quote}
% Do not forget to quote the argument according to the demands
% of your shell.
%
% \paragraph{Generating the documentation.}
% You can use both the \xfile{.dtx} or the \xfile{.drv} to generate
% the documentation. The process can be configured by the
% configuration file \xfile{ltxdoc.cfg}. For instance, put this
% line into this file, if you want to have A4 as paper format:
% \begin{quote}
%   \verb|\PassOptionsToClass{a4paper}{article}|
% \end{quote}
% An example follows how to generate the
% documentation with pdf\LaTeX:
% \begin{quote}
%\begin{verbatim}
%pdflatex infwarerr.dtx
%makeindex -s gind.ist infwarerr.idx
%pdflatex infwarerr.dtx
%makeindex -s gind.ist infwarerr.idx
%pdflatex infwarerr.dtx
%\end{verbatim}
% \end{quote}
%
% \begin{History}
%   \begin{Version}{2007/04/26 v1.0}
%   \item
%     First version.
%   \end{Version}
%   \begin{Version}{2007/06/14 v1.1}
%   \item
%     Small fixes.
%   \end{Version}
%   \begin{Version}{2007/09/09 v1.2}
%   \item
%     Catcode section extended.
%   \end{Version}
%   \begin{Version}{2010/04/08 v1.3}
%   \item
%     Ensure \cs{escapechar}=92 (backslash) for infos/warnings/errors.
%   \end{Version}
% \end{History}
%
% \PrintIndex
%
% \Finale
\endinput
|
% \end{quote}
% Do not forget to quote the argument according to the demands
% of your shell.
%
% \paragraph{Generating the documentation.}
% You can use both the \xfile{.dtx} or the \xfile{.drv} to generate
% the documentation. The process can be configured by the
% configuration file \xfile{ltxdoc.cfg}. For instance, put this
% line into this file, if you want to have A4 as paper format:
% \begin{quote}
%   \verb|\PassOptionsToClass{a4paper}{article}|
% \end{quote}
% An example follows how to generate the
% documentation with pdf\LaTeX:
% \begin{quote}
%\begin{verbatim}
%pdflatex infwarerr.dtx
%makeindex -s gind.ist infwarerr.idx
%pdflatex infwarerr.dtx
%makeindex -s gind.ist infwarerr.idx
%pdflatex infwarerr.dtx
%\end{verbatim}
% \end{quote}
%
% \begin{History}
%   \begin{Version}{2007/04/26 v1.0}
%   \item
%     First version.
%   \end{Version}
%   \begin{Version}{2007/06/14 v1.1}
%   \item
%     Small fixes.
%   \end{Version}
%   \begin{Version}{2007/09/09 v1.2}
%   \item
%     Catcode section extended.
%   \end{Version}
%   \begin{Version}{2010/04/08 v1.3}
%   \item
%     Ensure \cs{escapechar}=92 (backslash) for infos/warnings/errors.
%   \end{Version}
% \end{History}
%
% \PrintIndex
%
% \Finale
\endinput

%        (quote the arguments according to the demands of your shell)
% 
% Documentation:
%    (a) If infwarerr.drv is present:
%           latex infwarerr.drv
%    (b) Without infwarerr.drv:
%           latex infwarerr.dtx; ...
%    The class ltxdoc loads the configuration file ltxdoc.cfg
%    if available. Here you can specify further options, e.g.
%    use A4 as paper format:
%       \PassOptionsToClass{a4paper}{article}
% 
%    Programm calls to get the documentation (example):
%       pdflatex infwarerr.dtx
%       makeindex -s gind.ist infwarerr.idx
%       pdflatex infwarerr.dtx
%       makeindex -s gind.ist infwarerr.idx
%       pdflatex infwarerr.dtx
% 
% Installation:
%    TDS:tex/generic/oberdiek/infwarerr.sty
%    TDS:doc/latex/oberdiek/infwarerr.pdf
%    TDS:doc/latex/oberdiek/test/infwarerr-test1.tex
%    TDS:doc/latex/oberdiek/test/infwarerr-test2.tex
%    TDS:doc/latex/oberdiek/test/infwarerr-test3.tex
%    TDS:source/latex/oberdiek/infwarerr.dtx
%
%<*ignore>
\begingroup
  \catcode123=1 %
  \catcode125=2 %
  \def\x{LaTeX2e}%
\expandafter\endgroup
\ifcase 0\ifx\install y1\fi\expandafter
         \ifx\csname processbatchFile\endcsname\relax\else1\fi
         \ifx\fmtname\x\else 1\fi\relax
\else\csname fi\endcsname
%</ignore>
%<*install>
\input docstrip.tex
\Msg{************************************************************************}
\Msg{* Installation}
\Msg{* Package: infwarerr 2010/04/08 v1.3 Providing info/warning/message (HO)}
\Msg{************************************************************************}

\keepsilent
\askforoverwritefalse

\let\MetaPrefix\relax
\preamble

This is a generated file.

Project: infwarerr
Version: 2010/04/08 v1.3

Copyright (C) 2007, 2010 by
   Heiko Oberdiek <heiko.oberdiek at googlemail.com>

This work may be distributed and/or modified under the
conditions of the LaTeX Project Public License, either
version 1.3c of this license or (at your option) any later
version. This version of this license is in
   http://www.latex-project.org/lppl/lppl-1-3c.txt
and the latest version of this license is in
   http://www.latex-project.org/lppl.txt
and version 1.3 or later is part of all distributions of
LaTeX version 2005/12/01 or later.

This work has the LPPL maintenance status "maintained".

This Current Maintainer of this work is Heiko Oberdiek.

The Base Interpreter refers to any `TeX-Format',
because some files are installed in TDS:tex/generic//.

This work consists of the main source file infwarerr.dtx
and the derived files
   infwarerr.sty, infwarerr.pdf, infwarerr.ins, infwarerr.drv,
   infwarerr-test1.tex, infwarerr-test2.tex,
   infwarerr-test3.tex.

\endpreamble
\let\MetaPrefix\DoubleperCent

\generate{%
  \file{infwarerr.ins}{\from{infwarerr.dtx}{install}}%
  \file{infwarerr.drv}{\from{infwarerr.dtx}{driver}}%
  \usedir{tex/generic/oberdiek}%
  \file{infwarerr.sty}{\from{infwarerr.dtx}{package}}%
  \usedir{doc/latex/oberdiek/test}%
  \file{infwarerr-test1.tex}{\from{infwarerr.dtx}{test1}}%
  \file{infwarerr-test2.tex}{\from{infwarerr.dtx}{test2}}%
  \file{infwarerr-test3.tex}{\from{infwarerr.dtx}{test3}}%
}

\catcode32=13\relax% active space
\let =\space%
\Msg{************************************************************************}
\Msg{*}
\Msg{* To finish the installation you have to move the following}
\Msg{* file into a directory searched by TeX:}
\Msg{*}
\Msg{*     infwarerr.sty}
\Msg{*}
\Msg{* To produce the documentation run the file `infwarerr.drv'}
\Msg{* through LaTeX.}
\Msg{*}
\Msg{* Happy TeXing!}
\Msg{*}
\Msg{************************************************************************}

\endbatchfile
%</install>
%<*ignore>
\fi
%</ignore>
%<*driver>
\NeedsTeXFormat{LaTeX2e}
\ProvidesFile{infwarerr.drv}%
  [2010/04/08 v1.3 Providing info/warning/message (HO)]%
\documentclass{ltxdoc}
\usepackage{holtxdoc}[2011/02/04]
\begin{document}
  \DocInput{infwarerr.dtx}%
\end{document}
%</driver>
% \fi
%
% \CheckSum{871}
%
% \CharacterTable
%  {Upper-case    \A\B\C\D\E\F\G\H\I\J\K\L\M\N\O\P\Q\R\S\T\U\V\W\X\Y\Z
%   Lower-case    \a\b\c\d\e\f\g\h\i\j\k\l\m\n\o\p\q\r\s\t\u\v\w\x\y\z
%   Digits        \0\1\2\3\4\5\6\7\8\9
%   Exclamation   \!     Double quote  \"     Hash (number) \#
%   Dollar        \$     Percent       \%     Ampersand     \&
%   Acute accent  \'     Left paren    \(     Right paren   \)
%   Asterisk      \*     Plus          \+     Comma         \,
%   Minus         \-     Point         \.     Solidus       \/
%   Colon         \:     Semicolon     \;     Less than     \<
%   Equals        \=     Greater than  \>     Question mark \?
%   Commercial at \@     Left bracket  \[     Backslash     \\
%   Right bracket \]     Circumflex    \^     Underscore    \_
%   Grave accent  \`     Left brace    \{     Vertical bar  \|
%   Right brace   \}     Tilde         \~}
%
% \GetFileInfo{infwarerr.drv}
%
% \title{The \xpackage{infwarerr} package}
% \date{2010/04/08 v1.3}
% \author{Heiko Oberdiek\\\xemail{heiko.oberdiek at googlemail.com}}
%
% \maketitle
%
% \begin{abstract}
% This package provides a complete set of macros for informations,
% warnings and error messages with support for \plainTeX.
% \end{abstract}
%
% \tableofcontents
%
% \section{Documentation}
%
% Package \xpackage{infwarerr} defines the following set of commands,
% given as pseudo regular expression:
%   \begin{quote}
%     \verb+\@(Package|Class)(Info|InfoNoLine|Warning|WarningNoLine|Error)+^^A
%     \kern-\rightmargin
%   \end{quote}
% The first argument is the package or class in question. The second
% argument contains the message text. \cs{MessageBreak} is supported.
% In case of error messages, the third argument contains the help
% text. Supported are \LaTeX's message texts \cs{@ehc}:
%   \begin{quote}
%     |Try typing  <return>  to proceed|\\
%     |If that doesn't work, type  X <return>  to quit.|
%   \end{quote}
% and \cs{@ehd}:
%   \begin{quote}
%     |You're in trouble here.  Try typing  <return>  to proceed|\\
%     |If that doesn't work, type  X <return>  to quit.|
%   \end{quote}
%
% The package is intended for macro writers that want to write
% code that works for both \LaTeX\ and \plainTeX. If \LaTeX\ is
% detected, then the macros of this package will only be wrappers
% for \LaTeX's macros. Otherwise \LaTeX's macros will be mimicked
% for use with \plainTeX.
%
% \StopEventually{
% }
%
% \section{Implementation}
%
%    \begin{macrocode}
%<*package>
%    \end{macrocode}
%
% \subsection{Reload check and package identification}
%    Reload check, especially if the package is not used with \LaTeX.
%    \begin{macrocode}
\begingroup\catcode61\catcode48\catcode32=10\relax%
  \catcode13=5 % ^^M
  \endlinechar=13 %
  \catcode35=6 % #
  \catcode39=12 % '
  \catcode44=12 % ,
  \catcode45=12 % -
  \catcode46=12 % .
  \catcode58=12 % :
  \catcode64=11 % @
  \catcode123=1 % {
  \catcode125=2 % }
  \expandafter\let\expandafter\x\csname ver@infwarerr.sty\endcsname
  \ifx\x\relax % plain-TeX, first loading
  \else
    \def\empty{}%
    \ifx\x\empty % LaTeX, first loading,
      % variable is initialized, but \ProvidesPackage not yet seen
    \else
      \expandafter\ifx\csname PackageInfo\endcsname\relax
        \def\x#1#2{%
          \immediate\write-1{Package #1 Info: #2.}%
        }%
      \else
        \def\x#1#2{\PackageInfo{#1}{#2, stopped}}%
      \fi
      \x{infwarerr}{The package is already loaded}%
      \aftergroup\endinput
    \fi
  \fi
\endgroup%
%    \end{macrocode}
%    Package identification:
%    \begin{macrocode}
\begingroup\catcode61\catcode48\catcode32=10\relax%
  \catcode13=5 % ^^M
  \endlinechar=13 %
  \catcode35=6 % #
  \catcode39=12 % '
  \catcode40=12 % (
  \catcode41=12 % )
  \catcode44=12 % ,
  \catcode45=12 % -
  \catcode46=12 % .
  \catcode47=12 % /
  \catcode58=12 % :
  \catcode64=11 % @
  \catcode91=12 % [
  \catcode93=12 % ]
  \catcode123=1 % {
  \catcode125=2 % }
  \expandafter\ifx\csname ProvidesPackage\endcsname\relax
    \def\x#1#2#3[#4]{\endgroup
      \immediate\write-1{Package: #3 #4}%
      \xdef#1{#4}%
    }%
  \else
    \def\x#1#2[#3]{\endgroup
      #2[{#3}]%
      \ifx#1\@undefined
        \xdef#1{#3}%
      \fi
      \ifx#1\relax
        \xdef#1{#3}%
      \fi
    }%
  \fi
\expandafter\x\csname ver@infwarerr.sty\endcsname
\ProvidesPackage{infwarerr}%
  [2010/04/08 v1.3 Providing info/warning/message (HO)]%
%    \end{macrocode}
%
% \subsection{Catcodes}
%
%    \begin{macrocode}
\begingroup\catcode61\catcode48\catcode32=10\relax%
  \catcode13=5 % ^^M
  \endlinechar=13 %
  \catcode123=1 % {
  \catcode125=2 % }
  \catcode64=11 % @
  \def\x{\endgroup
    \expandafter\edef\csname InfWarErr@AtEnd\endcsname{%
      \endlinechar=\the\endlinechar\relax
      \catcode13=\the\catcode13\relax
      \catcode32=\the\catcode32\relax
      \catcode35=\the\catcode35\relax
      \catcode61=\the\catcode61\relax
      \catcode64=\the\catcode64\relax
      \catcode123=\the\catcode123\relax
      \catcode125=\the\catcode125\relax
    }%
  }%
\x\catcode61\catcode48\catcode32=10\relax%
\catcode13=5 % ^^M
\endlinechar=13 %
\catcode35=6 % #
\catcode64=11 % @
\catcode123=1 % {
\catcode125=2 % }
\def\TMP@EnsureCode#1#2{%
  \edef\InfWarErr@AtEnd{%
    \InfWarErr@AtEnd
    \catcode#1=\the\catcode#1\relax
  }%
  \catcode#1=#2\relax
}
\TMP@EnsureCode{10}{12}% ^^J
\TMP@EnsureCode{39}{12}% '
\TMP@EnsureCode{40}{12}% (
\TMP@EnsureCode{41}{12}% )
\TMP@EnsureCode{44}{12}% ,
\TMP@EnsureCode{45}{12}% -
\TMP@EnsureCode{46}{12}% .
\TMP@EnsureCode{58}{12}% :
\TMP@EnsureCode{60}{12}% <
\TMP@EnsureCode{62}{12}% >
\TMP@EnsureCode{94}{7}% ^
\edef\InfWarErr@AtEnd{\InfWarErr@AtEnd\noexpand\endinput}
%    \end{macrocode}
%
% \subsection{\LaTeX\ detection}
%
%    \begin{macrocode}
\begingroup
  \let\x=Y%
  \def\check#1{%
    \@check{Generic}{#1}%
    \@check{Package}{#1}%
    \@check{Class}{#1}%
  }%
  \def\@check#1#2{%
    \expandafter\ifx\csname #1#2\endcsname\relax
      \let\x=N%
    \fi
  }%
  \check{Info}%
  \check{Warning}%
  \check{Error}%
  \@check{Package}{WarningNoLine}%
  \@check{Class}{WarningNoLine}%
  \@check{}{@ehc}%
  \@check{}{@ehd}%
\expandafter\endgroup
\ifx\x Y%
%    \end{macrocode}
%
% \subsection{Version for \LaTeX}
%
%    \begin{macro}{\@PackageInfo}
%    \begin{macrocode}
  \let\@PackageInfo\PackageInfo
%    \end{macrocode}
%    \end{macro}
%    \begin{macro}{\@PackageInfoNoLine}
%    \begin{macrocode}
  \def\@PackageInfoNoLine#1#2{%
    \PackageInfo{#1}{#2\@gobble}%
  }%
%    \end{macrocode}
%    \end{macro}
%    \begin{macro}{\@ClassInfo}
%    \begin{macrocode}
  \let\@ClassInfo\ClassInfo
%    \end{macrocode}
%    \end{macro}
%    \begin{macro}{\@ClassInfoNoLine}
%    \begin{macrocode}
  \def\@ClassInfoNoLine#1#2{%
    \ClassInfo{#1}{#2\@gobble}%
  }%
%    \end{macrocode}
%    \end{macro}
%    \begin{macro}{\@PackageWarning}
%    \begin{macrocode}
  \let\@PackageWarning\PackageWarning
%    \end{macrocode}
%    \end{macro}
%    \begin{macro}{\@PackageWarningNoLine}
%    \begin{macrocode}
  \let\@PackageWarningNoLine\PackageWarningNoLine
%    \end{macrocode}
%    \end{macro}
%    \begin{macro}{\@ClassWarning}
%    \begin{macrocode}
  \let\@ClassWarning\ClassWarning
%    \end{macrocode}
%    \end{macro}
%    \begin{macro}{\@ClassWarningNoLine}
%    \begin{macrocode}
  \let\@ClassWarningNoLine\ClassWarningNoLine
%    \end{macrocode}
%    \end{macro}
%    \begin{macro}{\@PackageError}
%    \begin{macrocode}
  \let\@PackageError\PackageError
%    \end{macrocode}
%    \end{macro}
%    \begin{macro}{\@ClassError}
%    \begin{macrocode}
  \let\@ClassError\ClassError
%    \end{macrocode}
%    \end{macro}
%    \begin{macrocode}
  \expandafter\InfWarErr@AtEnd
\fi%
%    \end{macrocode}
%
% \subsection{Version for \plainTeX}
%
%    \begin{macro}{\InfWarErr@online}
%    \begin{macrocode}
\begingroup\expandafter\endgroup
\ifcase\expandafter\ifx\csname inputlineno\endcsname\relax
         0 %
       \else
         \ifnum\inputlineno<0 %
           0 %
         \else
           1 %
         \fi
       \fi
  \def\InfWarErr@online{}%
\else
  \def\InfWarErr@online{ on input line \the\inputlineno}%
\fi
%    \end{macrocode}
%    \end{macro}
%
%    \begin{macro}{\InfWarErr@protected}
%    \begin{macrocode}
\begingroup\expandafter\expandafter\expandafter\endgroup
\expandafter\ifx\csname protected\endcsname\relax
  \def\InfWarErr@protected{}%
\else
  \let\InfWarErr@protected\protected
\fi
%    \end{macrocode}
%    \end{macro}
%
%    \begin{macro}{\InfWarErr@unused}
%    \begin{macrocode}
\chardef\InfWarErr@unused=16 %
%    \end{macrocode}
%    \end{macro}
%
%    \begin{macro}{\InfWarErr@ehc}
%    \begin{macrocode}
\def\InfWarErr@ehc{%
  Try typing \space <return> \space to proceed.\MessageBreak
  If that doesn't work, type \space X <return> \space to quit.%
}
%    \end{macrocode}
%    \end{macro}
%    \begin{macro}{\InfWarErr@ehd}
%    \begin{macrocode}
\def\InfWarErr@ehd{%
  You're in trouble here. \space\@ehc
}
%    \end{macrocode}
%    \end{macro}
%
%    \begin{macro}{\InfWarErr@GenericInfo}
%    \begin{macrocode}
\InfWarErr@protected\def\InfWarErr@GenericInfo#1#2{%
  \begingroup
    \let\on@line\InfWarErr@online
    \long\def\@gobble##1{}%
    \def\space{ }%
    \def\@spaces{\space\space\space\space}%
    \def\MessageBreak{^^J#1}%
    \newlinechar=10 %
    \escapechar=92 %
    \immediate\write-1{#2\on@line.}%
  \endgroup
}
%    \end{macrocode}
%    \end{macro}
%    \begin{macro}{\InfWarErr@GenericWarning}
%    \begin{macrocode}
\InfWarErr@protected\def\InfWarErr@GenericWarning#1#2{%
  \begingroup
    \let\on@line\InfWarErr@online
    \long\def\@gobble##1{}%
    \def\space{ }%
    \def\@spaces{\space\space\space\space}%
    \def\MessageBreak{^^J#1}%
    \newlinechar=10 %
    \escapechar=92 %
    \immediate\write\InfWarErr@unused{^^J#2\on@line.^^J}%
  \endgroup
}
%    \end{macrocode}
%    \end{macro}
%    \begin{macro}{\InfWarErr@GenericError}
%    Old \TeX\ versions below 3.141 where |^^J| is not working
%    as newline in error message and help texts are not supported.
%    \begin{macrocode}
\InfWarErr@protected\def\InfWarErr@GenericError#1#2#3#4{%
  \begingroup
    \immediate\write\InfWarErr@unused{}%
    \def\space{ }%
    \def\@spaces{\space\space\space\space}%
    \let\@ehc\InfWarErr@ehc
    \let\@ehd\InfWarErr@ehd
    \let\on@line\InfWarErr@online
    \def\MessageBreak{^^J}%
    \newlinechar=10 %
    \escapechar=92 %
    \edef\InfWarErr@help{\noexpand\errhelp{#4}}%
    \InfWarErr@help
    \def\MessageBreak{^^J#1}%
    \errmessage{%
      #2.^^J^^J%
      #3^^J%
      Type \space H <return> \space for immediate help.^^J ...^^J%
    }%
  \endgroup
}
%    \end{macrocode}
%    \end{macro}
%
%    \begin{macro}{\@PackageInfo}
%    \begin{macrocode}
\def\@PackageInfo#1#2{%
  \InfWarErr@GenericInfo{%
    (#1) \@spaces\@spaces\@spaces
  }{%
    Package #1 Info: #2%
  }%
}
%    \end{macrocode}
%    \end{macro}
%    \begin{macro}{\@PackageInfoNoLine}
%    \begin{macrocode}
\def\@PackageInfoNoLine#1#2{%
  \@PackageInfo{#1}{#2\@gobble}%
}
%    \end{macrocode}
%    \end{macro}
%    \begin{macro}{\@ClassInfo}
%    \begin{macrocode}
\def\@ClassInfo#1#2{%
  \InfWarErr@GenericInfo{%
    (#1) \space\space\@spaces\@spaces
  }{%
    Class #1 Info: #2%
  }%
}
%    \end{macrocode}
%    \end{macro}
%    \begin{macro}{\@ClassInfoNoLine}
%    \begin{macrocode}
\def\@ClassInfoNoLine#1#2{%
  \@ClassInfo{#1}{#2\@gobble}%
}
%    \end{macrocode}
%    \end{macro}
%    \begin{macro}{\@PackageWarning}
%    \begin{macrocode}
\def\@PackageWarning#1#2{%
  \InfWarErr@GenericWarning{%
    (#1)\@spaces\@spaces\@spaces\@spaces
  }{%
    Package #1 Warning: #2%
  }%
}
%    \end{macrocode}
%    \end{macro}
%    \begin{macro}{\@PackageWarningNoLine}
%    \begin{macrocode}
\def\@PackageWarningNoLine#1#2{%
  \@PackageWarning{#1}{#2\@gobble}%
}
%    \end{macrocode}
%    \end{macro}
%    \begin{macro}{\@ClassWarning}
%    \begin{macrocode}
\def\@ClassWarning#1#2{%
  \InfWarErr@GenericWarning{%
    (#1) \space\@spaces\@spaces\@spaces
  }{%
    Class #1 Warning: #2%
  }%
}
%    \end{macrocode}
%    \end{macro}
%    \begin{macro}{\@ClassWarningNoLine}
%    \begin{macrocode}
\def\@ClassWarningNoLine#1#2{%
  \@ClassWarning{#1}{#2\@gobble}%
}
%    \end{macrocode}
%    \end{macro}
%    \begin{macro}{\@PackageError}
%    \begin{macrocode}
\def\@PackageError#1#2#3{%
  \InfWarErr@GenericError{%
    (#1)\@spaces\@spaces\@spaces\@spaces
  }{%
    Package #1 Error: #2%
  }{%
    See the #1 package documentation for explanation.%
  }{#3}%
}
%    \end{macrocode}
%    \end{macro}
%    \begin{macro}{\@ClassError}
%    \begin{macrocode}
\def\@ClassError#1#2#3{%
  \InfWarErr@GenericError{%
    (#1) \space\@spaces\@spaces\@spaces
  }{%
    Class #1 Error: #2%
  }{%
    See the #1 class documentation for explanation.%
  }{#3}%
}
%    \end{macrocode}
%    \end{macro}
%
%    \begin{macrocode}
\InfWarErr@AtEnd%
%</package>
%    \end{macrocode}
%
% \section{Test}
%
% \subsection{Catcode checks for loading}
%
%    \begin{macrocode}
%<*test1>
%    \end{macrocode}
%    \begin{macrocode}
\catcode`\{=1 %
\catcode`\}=2 %
\catcode`\#=6 %
\catcode`\@=11 %
\expandafter\ifx\csname count@\endcsname\relax
  \countdef\count@=255 %
\fi
\expandafter\ifx\csname @gobble\endcsname\relax
  \long\def\@gobble#1{}%
\fi
\expandafter\ifx\csname @firstofone\endcsname\relax
  \long\def\@firstofone#1{#1}%
\fi
\expandafter\ifx\csname loop\endcsname\relax
  \expandafter\@firstofone
\else
  \expandafter\@gobble
\fi
{%
  \def\loop#1\repeat{%
    \def\body{#1}%
    \iterate
  }%
  \def\iterate{%
    \body
      \let\next\iterate
    \else
      \let\next\relax
    \fi
    \next
  }%
  \let\repeat=\fi
}%
\def\RestoreCatcodes{}
\count@=0 %
\loop
  \edef\RestoreCatcodes{%
    \RestoreCatcodes
    \catcode\the\count@=\the\catcode\count@\relax
  }%
\ifnum\count@<255 %
  \advance\count@ 1 %
\repeat

\def\RangeCatcodeInvalid#1#2{%
  \count@=#1\relax
  \loop
    \catcode\count@=15 %
  \ifnum\count@<#2\relax
    \advance\count@ 1 %
  \repeat
}
\def\RangeCatcodeCheck#1#2#3{%
  \count@=#1\relax
  \loop
    \ifnum#3=\catcode\count@
    \else
      \errmessage{%
        Character \the\count@\space
        with wrong catcode \the\catcode\count@\space
        instead of \number#3%
      }%
    \fi
  \ifnum\count@<#2\relax
    \advance\count@ 1 %
  \repeat
}
\def\space{ }
\expandafter\ifx\csname LoadCommand\endcsname\relax
  \def\LoadCommand{\input infwarerr.sty\relax}%
\fi
\def\Test{%
  \RangeCatcodeInvalid{0}{47}%
  \RangeCatcodeInvalid{58}{64}%
  \RangeCatcodeInvalid{91}{96}%
  \RangeCatcodeInvalid{123}{255}%
  \catcode`\@=12 %
  \catcode`\\=0 %
  \catcode`\%=14 %
  \LoadCommand
  \RangeCatcodeCheck{0}{36}{15}%
  \RangeCatcodeCheck{37}{37}{14}%
  \RangeCatcodeCheck{38}{47}{15}%
  \RangeCatcodeCheck{48}{57}{12}%
  \RangeCatcodeCheck{58}{63}{15}%
  \RangeCatcodeCheck{64}{64}{12}%
  \RangeCatcodeCheck{65}{90}{11}%
  \RangeCatcodeCheck{91}{91}{15}%
  \RangeCatcodeCheck{92}{92}{0}%
  \RangeCatcodeCheck{93}{96}{15}%
  \RangeCatcodeCheck{97}{122}{11}%
  \RangeCatcodeCheck{123}{255}{15}%
  \RestoreCatcodes
}
\Test
\csname @@end\endcsname
\end
%    \end{macrocode}
%    \begin{macrocode}
%</test1>
%    \end{macrocode}
%
% \subsection{Macro tests}
%
%    \begin{macrocode}
%<*test2>
\let\PackageInfo\relax
\input infwarerr.sty\relax
\let\OrgWrite\write
\chardef\WriteUnused=16 %
\def\msg#{\immediate\OrgWrite\WriteUnused}
\msg{File: infwarerr-test2.tex 2010/04/08 v1.3 Test file for plain-TeX}
\catcode64=11 %
%</test2>
%    \end{macrocode}
%
%    \begin{macrocode}
%<*test3>
\NeedsTeXFormat{LaTeX2e}
\ProvidesFile{infwarerr-test3.tex}[2010/04/08 v1.3 Test file for LaTeX]
\RequirePackage{infwarerr}[2010/04/08]
\makeatletter
\let\OrgWrite\write
\chardef\WriteUnused=\@unused
\def\msg#{\immediate\OrgWrite\WriteUnused}
%</test3>
%    \end{macrocode}
%
%    \begin{macrocode}
%<*test2|test3>
\def\pkgname{PACKAGE}
\def\clsname{CLASS}
\def\msgtext{MESSAGE\MessageBreak SECOND LINE}
\def\hlptext{HELP\MessageBreak SECOND LINE}

\let\OrgWrite\write
\let\OrgErrMessage\errmessage

\def\write#1#{%
  \@write{#1}%
}
\def\@write#1#2{%
  \OrgWrite#1{#2}%
  \xdef\WriteNum{\number#1}%
  \xdef\WriteText{#2}%
}
\def\TestFailed#1#2{%
  \OrgErrMessage{Test (\string#1) failed: #2}%
}
\def\TestWrite#1#2#3{%
  \ifnum\WriteNum=#2\relax
  \else
    \TestFailed#1{write stream number: \WriteNum <> #2}%
  \fi
  \begingroup
    \let~\space
    \edef\TestString{#3}%
    \ifx\WriteText\TestString
    \else
      \msg{}%
      \msg{[\WriteText]}%
      \msg{<>}%
      \msg{[#3]}% hash-ok
      \TestFailed#1{write text}%
    \fi
  \endgroup
}
\newtoks\ErrHelpToks
\def\errmessage#1{%
  \global\ErrHelpToks=\expandafter{\the\errhelp}%
  \msg{**************** error message ****************}%
  \msg{! #1}%
  \msg{**************** help message *****************}%
  \msg{\the\errhelp}%
  \msg{***********************************************}%
  \xdef\ErrMessage{! #1}%
}
\def\TestInfo{%
  \@PackageInfo\pkgname\msgtext
  \TestWrite\@PackageInfo{-1}{%
    Package PACKAGE Info: MESSAGE^^J%
    (PACKAGE)~~~~~~~~~~~~~SECOND LINE on input line \the\inputlineno.%
  }%
  \@PackageInfoNoLine\pkgname\msgtext
  \TestWrite\@PackageInfoNoLine{-1}{%
    Package PACKAGE Info: MESSAGE^^J%
    (PACKAGE)~~~~~~~~~~~~~SECOND LINE.%
  }%
  \@ClassInfo\clsname\msgtext
  \TestWrite\@ClassInfo{-1}{%
    Class CLASS Info: MESSAGE^^J%
    (CLASS)~~~~~~~~~~~SECOND LINE on input line \the\inputlineno.%
  }%
  \@ClassInfoNoLine\clsname\msgtext
  \TestWrite\@ClassInfoNoLine{-1}{%
    Class CLASS Info: MESSAGE^^J%
    (CLASS)~~~~~~~~~~~SECOND LINE.%
  }%
}
\def\TestWarning{%
  \@PackageWarning\pkgname\msgtext
  \TestWrite\@PackageWarning\WriteUnused{%
    ^^J%
    Package PACKAGE Warning: MESSAGE^^J%
    (PACKAGE)~~~~~~~~~~~~~~~~SECOND LINE on input line \the\inputlineno.%
    ^^J%
  }%
  \@PackageWarningNoLine\pkgname\msgtext
  \TestWrite\@PackageWarningNoLine\WriteUnused{%
    ^^J%
    Package PACKAGE Warning: MESSAGE^^J%
    (PACKAGE)~~~~~~~~~~~~~~~~SECOND LINE.%
    ^^J%
  }%
  \@ClassWarning\clsname\msgtext
  \TestWrite\@ClassWarning\WriteUnused{%
    ^^J%
    Class CLASS Warning: MESSAGE^^J%
    (CLASS)~~~~~~~~~~~~~~SECOND LINE on input line \the\inputlineno.%
    ^^J%
  }%
  \@ClassWarningNoLine\clsname\msgtext
  \TestWrite\@ClassWarningNoLine\WriteUnused{%
    ^^J%
    Class CLASS Warning: MESSAGE^^J%
    (CLASS)~~~~~~~~~~~~~~SECOND LINE.%
    ^^J%
  }%
}
\def\TestError{%
%<test3>  \let\write\OrgWrite
%<test3>  \def\plaindots{}
%<test2>  \def\plaindots{.^^J ...^^J}
  \@PackageError\pkgname\msgtext\hlptext
  \@TestError\@PackageError{%
    ! Package PACKAGE Error: MESSAGE^^J%
    (PACKAGE)~~~~~~~~~~~~~~~~SECOND LINE.^^J%
    ^^J%
    See the PACKAGE package documentation for explanation.^^J%
    Type~~H <return>~~for immediate help\plaindots
  }{%
    HELP^^J%
    SECOND LINE%
  }%
  \@ClassError\clsname\msgtext\hlptext
  \@TestError\@ClassError{%
    ! Class CLASS Error: MESSAGE^^J%
    (CLASS)~~~~~~~~~~~~~~SECOND LINE.^^J%
    ^^J%
    See the CLASS class documentation for explanation.^^J%
    Type~~H <return>~~for immediate help\plaindots
  }{%
    HELP^^J%
    SECOND LINE%
  }%
  \@PackageError\pkgname\msgtext\@ehc
  \@TestError\@PackageError{%
    ! Package PACKAGE Error: MESSAGE^^J%
    (PACKAGE)~~~~~~~~~~~~~~~~SECOND LINE.^^J%
    ^^J%
    See the PACKAGE package documentation for explanation.^^J%
    Type~~H <return>~~for immediate help\plaindots
  }{%
    Try typing~~<return>~~to proceed.^^J%
    If that doesn't work, type~~X <return>~~to quit.%
  }%
  \@PackageError\pkgname\msgtext\@ehd
  \@TestError\@PackageError{%
    ! Package PACKAGE Error: MESSAGE^^J%
    (PACKAGE)~~~~~~~~~~~~~~~~SECOND LINE.^^J%
    ^^J%
    See the PACKAGE package documentation for explanation.^^J%
    Type~~H <return>~~for immediate help\plaindots
  }{%
    You're in trouble here.~~Try typing~~<return>~~to proceed.^^J%
    If that doesn't work, type~~X <return>~~to quit.%
  }%
}
\def\@TestError#1#2#3{%
  \begingroup
    \let~\space
    \edef\x{#2}%
%<test3>  \@onelevel@sanitize\x
%<test3>  \@onelevel@sanitize\ErrMessage
    \ifx\x\ErrMessage
    \else
      \msg{}%
      \msg{[\ErrMessage]}%
      \msg{<>}%
      \msg{[\x]}%
      \TestFailed#1{errmessage text}%
    \fi
    \edef\x{\the\ErrHelpToks}%
    \edef\y{#3}%
    \ifx\x\y
    \else
      \msg{}%
      \msg{[\the\ErrHelpToks]}%
      \msg{<>}%
      \msg{[\y]}%
      \TestFailed#1{errhelp text}%
    \fi
  \endgroup
}
\TestInfo
\TestWarning
\TestError
%</test2|test3>
%<test2>\end
%<test3>\@@end
%    \end{macrocode}
%
%
% \section{Installation}
%
% \subsection{Download}
%
% \paragraph{Package.} This package is available on
% CTAN\footnote{\url{ftp://ftp.ctan.org/tex-archive/}}:
% \begin{description}
% \item[\CTAN{macros/latex/contrib/oberdiek/infwarerr.dtx}] The source file.
% \item[\CTAN{macros/latex/contrib/oberdiek/infwarerr.pdf}] Documentation.
% \end{description}
%
%
% \paragraph{Bundle.} All the packages of the bundle `oberdiek'
% are also available in a TDS compliant ZIP archive. There
% the packages are already unpacked and the documentation files
% are generated. The files and directories obey the TDS standard.
% \begin{description}
% \item[\CTAN{install/macros/latex/contrib/oberdiek.tds.zip}]
% \end{description}
% \emph{TDS} refers to the standard ``A Directory Structure
% for \TeX\ Files'' (\CTAN{tds/tds.pdf}). Directories
% with \xfile{texmf} in their name are usually organized this way.
%
% \subsection{Bundle installation}
%
% \paragraph{Unpacking.} Unpack the \xfile{oberdiek.tds.zip} in the
% TDS tree (also known as \xfile{texmf} tree) of your choice.
% Example (linux):
% \begin{quote}
%   |unzip oberdiek.tds.zip -d ~/texmf|
% \end{quote}
%
% \paragraph{Script installation.}
% Check the directory \xfile{TDS:scripts/oberdiek/} for
% scripts that need further installation steps.
% Package \xpackage{attachfile2} comes with the Perl script
% \xfile{pdfatfi.pl} that should be installed in such a way
% that it can be called as \texttt{pdfatfi}.
% Example (linux):
% \begin{quote}
%   |chmod +x scripts/oberdiek/pdfatfi.pl|\\
%   |cp scripts/oberdiek/pdfatfi.pl /usr/local/bin/|
% \end{quote}
%
% \subsection{Package installation}
%
% \paragraph{Unpacking.} The \xfile{.dtx} file is a self-extracting
% \docstrip\ archive. The files are extracted by running the
% \xfile{.dtx} through \plainTeX:
% \begin{quote}
%   \verb|tex infwarerr.dtx|
% \end{quote}
%
% \paragraph{TDS.} Now the different files must be moved into
% the different directories in your installation TDS tree
% (also known as \xfile{texmf} tree):
% \begin{quote}
% \def\t{^^A
% \begin{tabular}{@{}>{\ttfamily}l@{ $\rightarrow$ }>{\ttfamily}l@{}}
%   infwarerr.sty & tex/generic/oberdiek/infwarerr.sty\\
%   infwarerr.pdf & doc/latex/oberdiek/infwarerr.pdf\\
%   test/infwarerr-test1.tex & doc/latex/oberdiek/test/infwarerr-test1.tex\\
%   test/infwarerr-test2.tex & doc/latex/oberdiek/test/infwarerr-test2.tex\\
%   test/infwarerr-test3.tex & doc/latex/oberdiek/test/infwarerr-test3.tex\\
%   infwarerr.dtx & source/latex/oberdiek/infwarerr.dtx\\
% \end{tabular}^^A
% }^^A
% \sbox0{\t}^^A
% \ifdim\wd0>\linewidth
%   \begingroup
%     \advance\linewidth by\leftmargin
%     \advance\linewidth by\rightmargin
%   \edef\x{\endgroup
%     \def\noexpand\lw{\the\linewidth}^^A
%   }\x
%   \def\lwbox{^^A
%     \leavevmode
%     \hbox to \linewidth{^^A
%       \kern-\leftmargin\relax
%       \hss
%       \usebox0
%       \hss
%       \kern-\rightmargin\relax
%     }^^A
%   }^^A
%   \ifdim\wd0>\lw
%     \sbox0{\small\t}^^A
%     \ifdim\wd0>\linewidth
%       \ifdim\wd0>\lw
%         \sbox0{\footnotesize\t}^^A
%         \ifdim\wd0>\linewidth
%           \ifdim\wd0>\lw
%             \sbox0{\scriptsize\t}^^A
%             \ifdim\wd0>\linewidth
%               \ifdim\wd0>\lw
%                 \sbox0{\tiny\t}^^A
%                 \ifdim\wd0>\linewidth
%                   \lwbox
%                 \else
%                   \usebox0
%                 \fi
%               \else
%                 \lwbox
%               \fi
%             \else
%               \usebox0
%             \fi
%           \else
%             \lwbox
%           \fi
%         \else
%           \usebox0
%         \fi
%       \else
%         \lwbox
%       \fi
%     \else
%       \usebox0
%     \fi
%   \else
%     \lwbox
%   \fi
% \else
%   \usebox0
% \fi
% \end{quote}
% If you have a \xfile{docstrip.cfg} that configures and enables \docstrip's
% TDS installing feature, then some files can already be in the right
% place, see the documentation of \docstrip.
%
% \subsection{Refresh file name databases}
%
% If your \TeX~distribution
% (\teTeX, \mikTeX, \dots) relies on file name databases, you must refresh
% these. For example, \teTeX\ users run \verb|texhash| or
% \verb|mktexlsr|.
%
% \subsection{Some details for the interested}
%
% \paragraph{Attached source.}
%
% The PDF documentation on CTAN also includes the
% \xfile{.dtx} source file. It can be extracted by
% AcrobatReader 6 or higher. Another option is \textsf{pdftk},
% e.g. unpack the file into the current directory:
% \begin{quote}
%   \verb|pdftk infwarerr.pdf unpack_files output .|
% \end{quote}
%
% \paragraph{Unpacking with \LaTeX.}
% The \xfile{.dtx} chooses its action depending on the format:
% \begin{description}
% \item[\plainTeX:] Run \docstrip\ and extract the files.
% \item[\LaTeX:] Generate the documentation.
% \end{description}
% If you insist on using \LaTeX\ for \docstrip\ (really,
% \docstrip\ does not need \LaTeX), then inform the autodetect routine
% about your intention:
% \begin{quote}
%   \verb|latex \let\install=y% \iffalse meta-comment
% 
% File: infwarerr.dtx
% Version: 2010/04/08 v1.3
% 
% Copyright (C) 2007, 2010 by
%    Heiko Oberdiek <heiko.oberdiek at googlemail.com>
% 
% This work may be distributed and/or modified under the
% conditions of the LaTeX Project Public License, either
% version 1.3c of this license or (at your option) any later
% version. This version of this license is in
%    http://www.latex-project.org/lppl/lppl-1-3c.txt
% and the latest version of this license is in
%    http://www.latex-project.org/lppl.txt
% and version 1.3 or later is part of all distributions of
% LaTeX version 2005/12/01 or later.
% 
% This work has the LPPL maintenance status "maintained".
% 
% This Current Maintainer of this work is Heiko Oberdiek.
% 
% The Base Interpreter refers to any `TeX-Format',
% because some files are installed in TDS:tex/generic//.
% 
% This work consists of the main source file infwarerr.dtx
% and the derived files
%    infwarerr.sty, infwarerr.pdf, infwarerr.ins, infwarerr.drv,
%    infwarerr-test1.tex, infwarerr-test2.tex,
%    infwarerr-test3.tex.
% 
% Distribution:
%    CTAN:macros/latex/contrib/oberdiek/infwarerr.dtx
%    CTAN:macros/latex/contrib/oberdiek/infwarerr.pdf
% 
% Unpacking:
%    (a) If infwarerr.ins is present:
%           tex infwarerr.ins
%    (b) Without infwarerr.ins:
%           tex infwarerr.dtx
%    (c) If you insist on using LaTeX
%           latex \let\install=y% \iffalse meta-comment
% 
% File: infwarerr.dtx
% Version: 2010/04/08 v1.3
% 
% Copyright (C) 2007, 2010 by
%    Heiko Oberdiek <heiko.oberdiek at googlemail.com>
% 
% This work may be distributed and/or modified under the
% conditions of the LaTeX Project Public License, either
% version 1.3c of this license or (at your option) any later
% version. This version of this license is in
%    http://www.latex-project.org/lppl/lppl-1-3c.txt
% and the latest version of this license is in
%    http://www.latex-project.org/lppl.txt
% and version 1.3 or later is part of all distributions of
% LaTeX version 2005/12/01 or later.
% 
% This work has the LPPL maintenance status "maintained".
% 
% This Current Maintainer of this work is Heiko Oberdiek.
% 
% The Base Interpreter refers to any `TeX-Format',
% because some files are installed in TDS:tex/generic//.
% 
% This work consists of the main source file infwarerr.dtx
% and the derived files
%    infwarerr.sty, infwarerr.pdf, infwarerr.ins, infwarerr.drv,
%    infwarerr-test1.tex, infwarerr-test2.tex,
%    infwarerr-test3.tex.
% 
% Distribution:
%    CTAN:macros/latex/contrib/oberdiek/infwarerr.dtx
%    CTAN:macros/latex/contrib/oberdiek/infwarerr.pdf
% 
% Unpacking:
%    (a) If infwarerr.ins is present:
%           tex infwarerr.ins
%    (b) Without infwarerr.ins:
%           tex infwarerr.dtx
%    (c) If you insist on using LaTeX
%           latex \let\install=y\input{infwarerr.dtx}
%        (quote the arguments according to the demands of your shell)
% 
% Documentation:
%    (a) If infwarerr.drv is present:
%           latex infwarerr.drv
%    (b) Without infwarerr.drv:
%           latex infwarerr.dtx; ...
%    The class ltxdoc loads the configuration file ltxdoc.cfg
%    if available. Here you can specify further options, e.g.
%    use A4 as paper format:
%       \PassOptionsToClass{a4paper}{article}
% 
%    Programm calls to get the documentation (example):
%       pdflatex infwarerr.dtx
%       makeindex -s gind.ist infwarerr.idx
%       pdflatex infwarerr.dtx
%       makeindex -s gind.ist infwarerr.idx
%       pdflatex infwarerr.dtx
% 
% Installation:
%    TDS:tex/generic/oberdiek/infwarerr.sty
%    TDS:doc/latex/oberdiek/infwarerr.pdf
%    TDS:doc/latex/oberdiek/test/infwarerr-test1.tex
%    TDS:doc/latex/oberdiek/test/infwarerr-test2.tex
%    TDS:doc/latex/oberdiek/test/infwarerr-test3.tex
%    TDS:source/latex/oberdiek/infwarerr.dtx
%
%<*ignore>
\begingroup
  \catcode123=1 %
  \catcode125=2 %
  \def\x{LaTeX2e}%
\expandafter\endgroup
\ifcase 0\ifx\install y1\fi\expandafter
         \ifx\csname processbatchFile\endcsname\relax\else1\fi
         \ifx\fmtname\x\else 1\fi\relax
\else\csname fi\endcsname
%</ignore>
%<*install>
\input docstrip.tex
\Msg{************************************************************************}
\Msg{* Installation}
\Msg{* Package: infwarerr 2010/04/08 v1.3 Providing info/warning/message (HO)}
\Msg{************************************************************************}

\keepsilent
\askforoverwritefalse

\let\MetaPrefix\relax
\preamble

This is a generated file.

Project: infwarerr
Version: 2010/04/08 v1.3

Copyright (C) 2007, 2010 by
   Heiko Oberdiek <heiko.oberdiek at googlemail.com>

This work may be distributed and/or modified under the
conditions of the LaTeX Project Public License, either
version 1.3c of this license or (at your option) any later
version. This version of this license is in
   http://www.latex-project.org/lppl/lppl-1-3c.txt
and the latest version of this license is in
   http://www.latex-project.org/lppl.txt
and version 1.3 or later is part of all distributions of
LaTeX version 2005/12/01 or later.

This work has the LPPL maintenance status "maintained".

This Current Maintainer of this work is Heiko Oberdiek.

The Base Interpreter refers to any `TeX-Format',
because some files are installed in TDS:tex/generic//.

This work consists of the main source file infwarerr.dtx
and the derived files
   infwarerr.sty, infwarerr.pdf, infwarerr.ins, infwarerr.drv,
   infwarerr-test1.tex, infwarerr-test2.tex,
   infwarerr-test3.tex.

\endpreamble
\let\MetaPrefix\DoubleperCent

\generate{%
  \file{infwarerr.ins}{\from{infwarerr.dtx}{install}}%
  \file{infwarerr.drv}{\from{infwarerr.dtx}{driver}}%
  \usedir{tex/generic/oberdiek}%
  \file{infwarerr.sty}{\from{infwarerr.dtx}{package}}%
  \usedir{doc/latex/oberdiek/test}%
  \file{infwarerr-test1.tex}{\from{infwarerr.dtx}{test1}}%
  \file{infwarerr-test2.tex}{\from{infwarerr.dtx}{test2}}%
  \file{infwarerr-test3.tex}{\from{infwarerr.dtx}{test3}}%
}

\catcode32=13\relax% active space
\let =\space%
\Msg{************************************************************************}
\Msg{*}
\Msg{* To finish the installation you have to move the following}
\Msg{* file into a directory searched by TeX:}
\Msg{*}
\Msg{*     infwarerr.sty}
\Msg{*}
\Msg{* To produce the documentation run the file `infwarerr.drv'}
\Msg{* through LaTeX.}
\Msg{*}
\Msg{* Happy TeXing!}
\Msg{*}
\Msg{************************************************************************}

\endbatchfile
%</install>
%<*ignore>
\fi
%</ignore>
%<*driver>
\NeedsTeXFormat{LaTeX2e}
\ProvidesFile{infwarerr.drv}%
  [2010/04/08 v1.3 Providing info/warning/message (HO)]%
\documentclass{ltxdoc}
\usepackage{holtxdoc}[2011/02/04]
\begin{document}
  \DocInput{infwarerr.dtx}%
\end{document}
%</driver>
% \fi
%
% \CheckSum{871}
%
% \CharacterTable
%  {Upper-case    \A\B\C\D\E\F\G\H\I\J\K\L\M\N\O\P\Q\R\S\T\U\V\W\X\Y\Z
%   Lower-case    \a\b\c\d\e\f\g\h\i\j\k\l\m\n\o\p\q\r\s\t\u\v\w\x\y\z
%   Digits        \0\1\2\3\4\5\6\7\8\9
%   Exclamation   \!     Double quote  \"     Hash (number) \#
%   Dollar        \$     Percent       \%     Ampersand     \&
%   Acute accent  \'     Left paren    \(     Right paren   \)
%   Asterisk      \*     Plus          \+     Comma         \,
%   Minus         \-     Point         \.     Solidus       \/
%   Colon         \:     Semicolon     \;     Less than     \<
%   Equals        \=     Greater than  \>     Question mark \?
%   Commercial at \@     Left bracket  \[     Backslash     \\
%   Right bracket \]     Circumflex    \^     Underscore    \_
%   Grave accent  \`     Left brace    \{     Vertical bar  \|
%   Right brace   \}     Tilde         \~}
%
% \GetFileInfo{infwarerr.drv}
%
% \title{The \xpackage{infwarerr} package}
% \date{2010/04/08 v1.3}
% \author{Heiko Oberdiek\\\xemail{heiko.oberdiek at googlemail.com}}
%
% \maketitle
%
% \begin{abstract}
% This package provides a complete set of macros for informations,
% warnings and error messages with support for \plainTeX.
% \end{abstract}
%
% \tableofcontents
%
% \section{Documentation}
%
% Package \xpackage{infwarerr} defines the following set of commands,
% given as pseudo regular expression:
%   \begin{quote}
%     \verb+\@(Package|Class)(Info|InfoNoLine|Warning|WarningNoLine|Error)+^^A
%     \kern-\rightmargin
%   \end{quote}
% The first argument is the package or class in question. The second
% argument contains the message text. \cs{MessageBreak} is supported.
% In case of error messages, the third argument contains the help
% text. Supported are \LaTeX's message texts \cs{@ehc}:
%   \begin{quote}
%     |Try typing  <return>  to proceed|\\
%     |If that doesn't work, type  X <return>  to quit.|
%   \end{quote}
% and \cs{@ehd}:
%   \begin{quote}
%     |You're in trouble here.  Try typing  <return>  to proceed|\\
%     |If that doesn't work, type  X <return>  to quit.|
%   \end{quote}
%
% The package is intended for macro writers that want to write
% code that works for both \LaTeX\ and \plainTeX. If \LaTeX\ is
% detected, then the macros of this package will only be wrappers
% for \LaTeX's macros. Otherwise \LaTeX's macros will be mimicked
% for use with \plainTeX.
%
% \StopEventually{
% }
%
% \section{Implementation}
%
%    \begin{macrocode}
%<*package>
%    \end{macrocode}
%
% \subsection{Reload check and package identification}
%    Reload check, especially if the package is not used with \LaTeX.
%    \begin{macrocode}
\begingroup\catcode61\catcode48\catcode32=10\relax%
  \catcode13=5 % ^^M
  \endlinechar=13 %
  \catcode35=6 % #
  \catcode39=12 % '
  \catcode44=12 % ,
  \catcode45=12 % -
  \catcode46=12 % .
  \catcode58=12 % :
  \catcode64=11 % @
  \catcode123=1 % {
  \catcode125=2 % }
  \expandafter\let\expandafter\x\csname ver@infwarerr.sty\endcsname
  \ifx\x\relax % plain-TeX, first loading
  \else
    \def\empty{}%
    \ifx\x\empty % LaTeX, first loading,
      % variable is initialized, but \ProvidesPackage not yet seen
    \else
      \expandafter\ifx\csname PackageInfo\endcsname\relax
        \def\x#1#2{%
          \immediate\write-1{Package #1 Info: #2.}%
        }%
      \else
        \def\x#1#2{\PackageInfo{#1}{#2, stopped}}%
      \fi
      \x{infwarerr}{The package is already loaded}%
      \aftergroup\endinput
    \fi
  \fi
\endgroup%
%    \end{macrocode}
%    Package identification:
%    \begin{macrocode}
\begingroup\catcode61\catcode48\catcode32=10\relax%
  \catcode13=5 % ^^M
  \endlinechar=13 %
  \catcode35=6 % #
  \catcode39=12 % '
  \catcode40=12 % (
  \catcode41=12 % )
  \catcode44=12 % ,
  \catcode45=12 % -
  \catcode46=12 % .
  \catcode47=12 % /
  \catcode58=12 % :
  \catcode64=11 % @
  \catcode91=12 % [
  \catcode93=12 % ]
  \catcode123=1 % {
  \catcode125=2 % }
  \expandafter\ifx\csname ProvidesPackage\endcsname\relax
    \def\x#1#2#3[#4]{\endgroup
      \immediate\write-1{Package: #3 #4}%
      \xdef#1{#4}%
    }%
  \else
    \def\x#1#2[#3]{\endgroup
      #2[{#3}]%
      \ifx#1\@undefined
        \xdef#1{#3}%
      \fi
      \ifx#1\relax
        \xdef#1{#3}%
      \fi
    }%
  \fi
\expandafter\x\csname ver@infwarerr.sty\endcsname
\ProvidesPackage{infwarerr}%
  [2010/04/08 v1.3 Providing info/warning/message (HO)]%
%    \end{macrocode}
%
% \subsection{Catcodes}
%
%    \begin{macrocode}
\begingroup\catcode61\catcode48\catcode32=10\relax%
  \catcode13=5 % ^^M
  \endlinechar=13 %
  \catcode123=1 % {
  \catcode125=2 % }
  \catcode64=11 % @
  \def\x{\endgroup
    \expandafter\edef\csname InfWarErr@AtEnd\endcsname{%
      \endlinechar=\the\endlinechar\relax
      \catcode13=\the\catcode13\relax
      \catcode32=\the\catcode32\relax
      \catcode35=\the\catcode35\relax
      \catcode61=\the\catcode61\relax
      \catcode64=\the\catcode64\relax
      \catcode123=\the\catcode123\relax
      \catcode125=\the\catcode125\relax
    }%
  }%
\x\catcode61\catcode48\catcode32=10\relax%
\catcode13=5 % ^^M
\endlinechar=13 %
\catcode35=6 % #
\catcode64=11 % @
\catcode123=1 % {
\catcode125=2 % }
\def\TMP@EnsureCode#1#2{%
  \edef\InfWarErr@AtEnd{%
    \InfWarErr@AtEnd
    \catcode#1=\the\catcode#1\relax
  }%
  \catcode#1=#2\relax
}
\TMP@EnsureCode{10}{12}% ^^J
\TMP@EnsureCode{39}{12}% '
\TMP@EnsureCode{40}{12}% (
\TMP@EnsureCode{41}{12}% )
\TMP@EnsureCode{44}{12}% ,
\TMP@EnsureCode{45}{12}% -
\TMP@EnsureCode{46}{12}% .
\TMP@EnsureCode{58}{12}% :
\TMP@EnsureCode{60}{12}% <
\TMP@EnsureCode{62}{12}% >
\TMP@EnsureCode{94}{7}% ^
\edef\InfWarErr@AtEnd{\InfWarErr@AtEnd\noexpand\endinput}
%    \end{macrocode}
%
% \subsection{\LaTeX\ detection}
%
%    \begin{macrocode}
\begingroup
  \let\x=Y%
  \def\check#1{%
    \@check{Generic}{#1}%
    \@check{Package}{#1}%
    \@check{Class}{#1}%
  }%
  \def\@check#1#2{%
    \expandafter\ifx\csname #1#2\endcsname\relax
      \let\x=N%
    \fi
  }%
  \check{Info}%
  \check{Warning}%
  \check{Error}%
  \@check{Package}{WarningNoLine}%
  \@check{Class}{WarningNoLine}%
  \@check{}{@ehc}%
  \@check{}{@ehd}%
\expandafter\endgroup
\ifx\x Y%
%    \end{macrocode}
%
% \subsection{Version for \LaTeX}
%
%    \begin{macro}{\@PackageInfo}
%    \begin{macrocode}
  \let\@PackageInfo\PackageInfo
%    \end{macrocode}
%    \end{macro}
%    \begin{macro}{\@PackageInfoNoLine}
%    \begin{macrocode}
  \def\@PackageInfoNoLine#1#2{%
    \PackageInfo{#1}{#2\@gobble}%
  }%
%    \end{macrocode}
%    \end{macro}
%    \begin{macro}{\@ClassInfo}
%    \begin{macrocode}
  \let\@ClassInfo\ClassInfo
%    \end{macrocode}
%    \end{macro}
%    \begin{macro}{\@ClassInfoNoLine}
%    \begin{macrocode}
  \def\@ClassInfoNoLine#1#2{%
    \ClassInfo{#1}{#2\@gobble}%
  }%
%    \end{macrocode}
%    \end{macro}
%    \begin{macro}{\@PackageWarning}
%    \begin{macrocode}
  \let\@PackageWarning\PackageWarning
%    \end{macrocode}
%    \end{macro}
%    \begin{macro}{\@PackageWarningNoLine}
%    \begin{macrocode}
  \let\@PackageWarningNoLine\PackageWarningNoLine
%    \end{macrocode}
%    \end{macro}
%    \begin{macro}{\@ClassWarning}
%    \begin{macrocode}
  \let\@ClassWarning\ClassWarning
%    \end{macrocode}
%    \end{macro}
%    \begin{macro}{\@ClassWarningNoLine}
%    \begin{macrocode}
  \let\@ClassWarningNoLine\ClassWarningNoLine
%    \end{macrocode}
%    \end{macro}
%    \begin{macro}{\@PackageError}
%    \begin{macrocode}
  \let\@PackageError\PackageError
%    \end{macrocode}
%    \end{macro}
%    \begin{macro}{\@ClassError}
%    \begin{macrocode}
  \let\@ClassError\ClassError
%    \end{macrocode}
%    \end{macro}
%    \begin{macrocode}
  \expandafter\InfWarErr@AtEnd
\fi%
%    \end{macrocode}
%
% \subsection{Version for \plainTeX}
%
%    \begin{macro}{\InfWarErr@online}
%    \begin{macrocode}
\begingroup\expandafter\endgroup
\ifcase\expandafter\ifx\csname inputlineno\endcsname\relax
         0 %
       \else
         \ifnum\inputlineno<0 %
           0 %
         \else
           1 %
         \fi
       \fi
  \def\InfWarErr@online{}%
\else
  \def\InfWarErr@online{ on input line \the\inputlineno}%
\fi
%    \end{macrocode}
%    \end{macro}
%
%    \begin{macro}{\InfWarErr@protected}
%    \begin{macrocode}
\begingroup\expandafter\expandafter\expandafter\endgroup
\expandafter\ifx\csname protected\endcsname\relax
  \def\InfWarErr@protected{}%
\else
  \let\InfWarErr@protected\protected
\fi
%    \end{macrocode}
%    \end{macro}
%
%    \begin{macro}{\InfWarErr@unused}
%    \begin{macrocode}
\chardef\InfWarErr@unused=16 %
%    \end{macrocode}
%    \end{macro}
%
%    \begin{macro}{\InfWarErr@ehc}
%    \begin{macrocode}
\def\InfWarErr@ehc{%
  Try typing \space <return> \space to proceed.\MessageBreak
  If that doesn't work, type \space X <return> \space to quit.%
}
%    \end{macrocode}
%    \end{macro}
%    \begin{macro}{\InfWarErr@ehd}
%    \begin{macrocode}
\def\InfWarErr@ehd{%
  You're in trouble here. \space\@ehc
}
%    \end{macrocode}
%    \end{macro}
%
%    \begin{macro}{\InfWarErr@GenericInfo}
%    \begin{macrocode}
\InfWarErr@protected\def\InfWarErr@GenericInfo#1#2{%
  \begingroup
    \let\on@line\InfWarErr@online
    \long\def\@gobble##1{}%
    \def\space{ }%
    \def\@spaces{\space\space\space\space}%
    \def\MessageBreak{^^J#1}%
    \newlinechar=10 %
    \escapechar=92 %
    \immediate\write-1{#2\on@line.}%
  \endgroup
}
%    \end{macrocode}
%    \end{macro}
%    \begin{macro}{\InfWarErr@GenericWarning}
%    \begin{macrocode}
\InfWarErr@protected\def\InfWarErr@GenericWarning#1#2{%
  \begingroup
    \let\on@line\InfWarErr@online
    \long\def\@gobble##1{}%
    \def\space{ }%
    \def\@spaces{\space\space\space\space}%
    \def\MessageBreak{^^J#1}%
    \newlinechar=10 %
    \escapechar=92 %
    \immediate\write\InfWarErr@unused{^^J#2\on@line.^^J}%
  \endgroup
}
%    \end{macrocode}
%    \end{macro}
%    \begin{macro}{\InfWarErr@GenericError}
%    Old \TeX\ versions below 3.141 where |^^J| is not working
%    as newline in error message and help texts are not supported.
%    \begin{macrocode}
\InfWarErr@protected\def\InfWarErr@GenericError#1#2#3#4{%
  \begingroup
    \immediate\write\InfWarErr@unused{}%
    \def\space{ }%
    \def\@spaces{\space\space\space\space}%
    \let\@ehc\InfWarErr@ehc
    \let\@ehd\InfWarErr@ehd
    \let\on@line\InfWarErr@online
    \def\MessageBreak{^^J}%
    \newlinechar=10 %
    \escapechar=92 %
    \edef\InfWarErr@help{\noexpand\errhelp{#4}}%
    \InfWarErr@help
    \def\MessageBreak{^^J#1}%
    \errmessage{%
      #2.^^J^^J%
      #3^^J%
      Type \space H <return> \space for immediate help.^^J ...^^J%
    }%
  \endgroup
}
%    \end{macrocode}
%    \end{macro}
%
%    \begin{macro}{\@PackageInfo}
%    \begin{macrocode}
\def\@PackageInfo#1#2{%
  \InfWarErr@GenericInfo{%
    (#1) \@spaces\@spaces\@spaces
  }{%
    Package #1 Info: #2%
  }%
}
%    \end{macrocode}
%    \end{macro}
%    \begin{macro}{\@PackageInfoNoLine}
%    \begin{macrocode}
\def\@PackageInfoNoLine#1#2{%
  \@PackageInfo{#1}{#2\@gobble}%
}
%    \end{macrocode}
%    \end{macro}
%    \begin{macro}{\@ClassInfo}
%    \begin{macrocode}
\def\@ClassInfo#1#2{%
  \InfWarErr@GenericInfo{%
    (#1) \space\space\@spaces\@spaces
  }{%
    Class #1 Info: #2%
  }%
}
%    \end{macrocode}
%    \end{macro}
%    \begin{macro}{\@ClassInfoNoLine}
%    \begin{macrocode}
\def\@ClassInfoNoLine#1#2{%
  \@ClassInfo{#1}{#2\@gobble}%
}
%    \end{macrocode}
%    \end{macro}
%    \begin{macro}{\@PackageWarning}
%    \begin{macrocode}
\def\@PackageWarning#1#2{%
  \InfWarErr@GenericWarning{%
    (#1)\@spaces\@spaces\@spaces\@spaces
  }{%
    Package #1 Warning: #2%
  }%
}
%    \end{macrocode}
%    \end{macro}
%    \begin{macro}{\@PackageWarningNoLine}
%    \begin{macrocode}
\def\@PackageWarningNoLine#1#2{%
  \@PackageWarning{#1}{#2\@gobble}%
}
%    \end{macrocode}
%    \end{macro}
%    \begin{macro}{\@ClassWarning}
%    \begin{macrocode}
\def\@ClassWarning#1#2{%
  \InfWarErr@GenericWarning{%
    (#1) \space\@spaces\@spaces\@spaces
  }{%
    Class #1 Warning: #2%
  }%
}
%    \end{macrocode}
%    \end{macro}
%    \begin{macro}{\@ClassWarningNoLine}
%    \begin{macrocode}
\def\@ClassWarningNoLine#1#2{%
  \@ClassWarning{#1}{#2\@gobble}%
}
%    \end{macrocode}
%    \end{macro}
%    \begin{macro}{\@PackageError}
%    \begin{macrocode}
\def\@PackageError#1#2#3{%
  \InfWarErr@GenericError{%
    (#1)\@spaces\@spaces\@spaces\@spaces
  }{%
    Package #1 Error: #2%
  }{%
    See the #1 package documentation for explanation.%
  }{#3}%
}
%    \end{macrocode}
%    \end{macro}
%    \begin{macro}{\@ClassError}
%    \begin{macrocode}
\def\@ClassError#1#2#3{%
  \InfWarErr@GenericError{%
    (#1) \space\@spaces\@spaces\@spaces
  }{%
    Class #1 Error: #2%
  }{%
    See the #1 class documentation for explanation.%
  }{#3}%
}
%    \end{macrocode}
%    \end{macro}
%
%    \begin{macrocode}
\InfWarErr@AtEnd%
%</package>
%    \end{macrocode}
%
% \section{Test}
%
% \subsection{Catcode checks for loading}
%
%    \begin{macrocode}
%<*test1>
%    \end{macrocode}
%    \begin{macrocode}
\catcode`\{=1 %
\catcode`\}=2 %
\catcode`\#=6 %
\catcode`\@=11 %
\expandafter\ifx\csname count@\endcsname\relax
  \countdef\count@=255 %
\fi
\expandafter\ifx\csname @gobble\endcsname\relax
  \long\def\@gobble#1{}%
\fi
\expandafter\ifx\csname @firstofone\endcsname\relax
  \long\def\@firstofone#1{#1}%
\fi
\expandafter\ifx\csname loop\endcsname\relax
  \expandafter\@firstofone
\else
  \expandafter\@gobble
\fi
{%
  \def\loop#1\repeat{%
    \def\body{#1}%
    \iterate
  }%
  \def\iterate{%
    \body
      \let\next\iterate
    \else
      \let\next\relax
    \fi
    \next
  }%
  \let\repeat=\fi
}%
\def\RestoreCatcodes{}
\count@=0 %
\loop
  \edef\RestoreCatcodes{%
    \RestoreCatcodes
    \catcode\the\count@=\the\catcode\count@\relax
  }%
\ifnum\count@<255 %
  \advance\count@ 1 %
\repeat

\def\RangeCatcodeInvalid#1#2{%
  \count@=#1\relax
  \loop
    \catcode\count@=15 %
  \ifnum\count@<#2\relax
    \advance\count@ 1 %
  \repeat
}
\def\RangeCatcodeCheck#1#2#3{%
  \count@=#1\relax
  \loop
    \ifnum#3=\catcode\count@
    \else
      \errmessage{%
        Character \the\count@\space
        with wrong catcode \the\catcode\count@\space
        instead of \number#3%
      }%
    \fi
  \ifnum\count@<#2\relax
    \advance\count@ 1 %
  \repeat
}
\def\space{ }
\expandafter\ifx\csname LoadCommand\endcsname\relax
  \def\LoadCommand{\input infwarerr.sty\relax}%
\fi
\def\Test{%
  \RangeCatcodeInvalid{0}{47}%
  \RangeCatcodeInvalid{58}{64}%
  \RangeCatcodeInvalid{91}{96}%
  \RangeCatcodeInvalid{123}{255}%
  \catcode`\@=12 %
  \catcode`\\=0 %
  \catcode`\%=14 %
  \LoadCommand
  \RangeCatcodeCheck{0}{36}{15}%
  \RangeCatcodeCheck{37}{37}{14}%
  \RangeCatcodeCheck{38}{47}{15}%
  \RangeCatcodeCheck{48}{57}{12}%
  \RangeCatcodeCheck{58}{63}{15}%
  \RangeCatcodeCheck{64}{64}{12}%
  \RangeCatcodeCheck{65}{90}{11}%
  \RangeCatcodeCheck{91}{91}{15}%
  \RangeCatcodeCheck{92}{92}{0}%
  \RangeCatcodeCheck{93}{96}{15}%
  \RangeCatcodeCheck{97}{122}{11}%
  \RangeCatcodeCheck{123}{255}{15}%
  \RestoreCatcodes
}
\Test
\csname @@end\endcsname
\end
%    \end{macrocode}
%    \begin{macrocode}
%</test1>
%    \end{macrocode}
%
% \subsection{Macro tests}
%
%    \begin{macrocode}
%<*test2>
\let\PackageInfo\relax
\input infwarerr.sty\relax
\let\OrgWrite\write
\chardef\WriteUnused=16 %
\def\msg#{\immediate\OrgWrite\WriteUnused}
\msg{File: infwarerr-test2.tex 2010/04/08 v1.3 Test file for plain-TeX}
\catcode64=11 %
%</test2>
%    \end{macrocode}
%
%    \begin{macrocode}
%<*test3>
\NeedsTeXFormat{LaTeX2e}
\ProvidesFile{infwarerr-test3.tex}[2010/04/08 v1.3 Test file for LaTeX]
\RequirePackage{infwarerr}[2010/04/08]
\makeatletter
\let\OrgWrite\write
\chardef\WriteUnused=\@unused
\def\msg#{\immediate\OrgWrite\WriteUnused}
%</test3>
%    \end{macrocode}
%
%    \begin{macrocode}
%<*test2|test3>
\def\pkgname{PACKAGE}
\def\clsname{CLASS}
\def\msgtext{MESSAGE\MessageBreak SECOND LINE}
\def\hlptext{HELP\MessageBreak SECOND LINE}

\let\OrgWrite\write
\let\OrgErrMessage\errmessage

\def\write#1#{%
  \@write{#1}%
}
\def\@write#1#2{%
  \OrgWrite#1{#2}%
  \xdef\WriteNum{\number#1}%
  \xdef\WriteText{#2}%
}
\def\TestFailed#1#2{%
  \OrgErrMessage{Test (\string#1) failed: #2}%
}
\def\TestWrite#1#2#3{%
  \ifnum\WriteNum=#2\relax
  \else
    \TestFailed#1{write stream number: \WriteNum <> #2}%
  \fi
  \begingroup
    \let~\space
    \edef\TestString{#3}%
    \ifx\WriteText\TestString
    \else
      \msg{}%
      \msg{[\WriteText]}%
      \msg{<>}%
      \msg{[#3]}% hash-ok
      \TestFailed#1{write text}%
    \fi
  \endgroup
}
\newtoks\ErrHelpToks
\def\errmessage#1{%
  \global\ErrHelpToks=\expandafter{\the\errhelp}%
  \msg{**************** error message ****************}%
  \msg{! #1}%
  \msg{**************** help message *****************}%
  \msg{\the\errhelp}%
  \msg{***********************************************}%
  \xdef\ErrMessage{! #1}%
}
\def\TestInfo{%
  \@PackageInfo\pkgname\msgtext
  \TestWrite\@PackageInfo{-1}{%
    Package PACKAGE Info: MESSAGE^^J%
    (PACKAGE)~~~~~~~~~~~~~SECOND LINE on input line \the\inputlineno.%
  }%
  \@PackageInfoNoLine\pkgname\msgtext
  \TestWrite\@PackageInfoNoLine{-1}{%
    Package PACKAGE Info: MESSAGE^^J%
    (PACKAGE)~~~~~~~~~~~~~SECOND LINE.%
  }%
  \@ClassInfo\clsname\msgtext
  \TestWrite\@ClassInfo{-1}{%
    Class CLASS Info: MESSAGE^^J%
    (CLASS)~~~~~~~~~~~SECOND LINE on input line \the\inputlineno.%
  }%
  \@ClassInfoNoLine\clsname\msgtext
  \TestWrite\@ClassInfoNoLine{-1}{%
    Class CLASS Info: MESSAGE^^J%
    (CLASS)~~~~~~~~~~~SECOND LINE.%
  }%
}
\def\TestWarning{%
  \@PackageWarning\pkgname\msgtext
  \TestWrite\@PackageWarning\WriteUnused{%
    ^^J%
    Package PACKAGE Warning: MESSAGE^^J%
    (PACKAGE)~~~~~~~~~~~~~~~~SECOND LINE on input line \the\inputlineno.%
    ^^J%
  }%
  \@PackageWarningNoLine\pkgname\msgtext
  \TestWrite\@PackageWarningNoLine\WriteUnused{%
    ^^J%
    Package PACKAGE Warning: MESSAGE^^J%
    (PACKAGE)~~~~~~~~~~~~~~~~SECOND LINE.%
    ^^J%
  }%
  \@ClassWarning\clsname\msgtext
  \TestWrite\@ClassWarning\WriteUnused{%
    ^^J%
    Class CLASS Warning: MESSAGE^^J%
    (CLASS)~~~~~~~~~~~~~~SECOND LINE on input line \the\inputlineno.%
    ^^J%
  }%
  \@ClassWarningNoLine\clsname\msgtext
  \TestWrite\@ClassWarningNoLine\WriteUnused{%
    ^^J%
    Class CLASS Warning: MESSAGE^^J%
    (CLASS)~~~~~~~~~~~~~~SECOND LINE.%
    ^^J%
  }%
}
\def\TestError{%
%<test3>  \let\write\OrgWrite
%<test3>  \def\plaindots{}
%<test2>  \def\plaindots{.^^J ...^^J}
  \@PackageError\pkgname\msgtext\hlptext
  \@TestError\@PackageError{%
    ! Package PACKAGE Error: MESSAGE^^J%
    (PACKAGE)~~~~~~~~~~~~~~~~SECOND LINE.^^J%
    ^^J%
    See the PACKAGE package documentation for explanation.^^J%
    Type~~H <return>~~for immediate help\plaindots
  }{%
    HELP^^J%
    SECOND LINE%
  }%
  \@ClassError\clsname\msgtext\hlptext
  \@TestError\@ClassError{%
    ! Class CLASS Error: MESSAGE^^J%
    (CLASS)~~~~~~~~~~~~~~SECOND LINE.^^J%
    ^^J%
    See the CLASS class documentation for explanation.^^J%
    Type~~H <return>~~for immediate help\plaindots
  }{%
    HELP^^J%
    SECOND LINE%
  }%
  \@PackageError\pkgname\msgtext\@ehc
  \@TestError\@PackageError{%
    ! Package PACKAGE Error: MESSAGE^^J%
    (PACKAGE)~~~~~~~~~~~~~~~~SECOND LINE.^^J%
    ^^J%
    See the PACKAGE package documentation for explanation.^^J%
    Type~~H <return>~~for immediate help\plaindots
  }{%
    Try typing~~<return>~~to proceed.^^J%
    If that doesn't work, type~~X <return>~~to quit.%
  }%
  \@PackageError\pkgname\msgtext\@ehd
  \@TestError\@PackageError{%
    ! Package PACKAGE Error: MESSAGE^^J%
    (PACKAGE)~~~~~~~~~~~~~~~~SECOND LINE.^^J%
    ^^J%
    See the PACKAGE package documentation for explanation.^^J%
    Type~~H <return>~~for immediate help\plaindots
  }{%
    You're in trouble here.~~Try typing~~<return>~~to proceed.^^J%
    If that doesn't work, type~~X <return>~~to quit.%
  }%
}
\def\@TestError#1#2#3{%
  \begingroup
    \let~\space
    \edef\x{#2}%
%<test3>  \@onelevel@sanitize\x
%<test3>  \@onelevel@sanitize\ErrMessage
    \ifx\x\ErrMessage
    \else
      \msg{}%
      \msg{[\ErrMessage]}%
      \msg{<>}%
      \msg{[\x]}%
      \TestFailed#1{errmessage text}%
    \fi
    \edef\x{\the\ErrHelpToks}%
    \edef\y{#3}%
    \ifx\x\y
    \else
      \msg{}%
      \msg{[\the\ErrHelpToks]}%
      \msg{<>}%
      \msg{[\y]}%
      \TestFailed#1{errhelp text}%
    \fi
  \endgroup
}
\TestInfo
\TestWarning
\TestError
%</test2|test3>
%<test2>\end
%<test3>\@@end
%    \end{macrocode}
%
%
% \section{Installation}
%
% \subsection{Download}
%
% \paragraph{Package.} This package is available on
% CTAN\footnote{\url{ftp://ftp.ctan.org/tex-archive/}}:
% \begin{description}
% \item[\CTAN{macros/latex/contrib/oberdiek/infwarerr.dtx}] The source file.
% \item[\CTAN{macros/latex/contrib/oberdiek/infwarerr.pdf}] Documentation.
% \end{description}
%
%
% \paragraph{Bundle.} All the packages of the bundle `oberdiek'
% are also available in a TDS compliant ZIP archive. There
% the packages are already unpacked and the documentation files
% are generated. The files and directories obey the TDS standard.
% \begin{description}
% \item[\CTAN{install/macros/latex/contrib/oberdiek.tds.zip}]
% \end{description}
% \emph{TDS} refers to the standard ``A Directory Structure
% for \TeX\ Files'' (\CTAN{tds/tds.pdf}). Directories
% with \xfile{texmf} in their name are usually organized this way.
%
% \subsection{Bundle installation}
%
% \paragraph{Unpacking.} Unpack the \xfile{oberdiek.tds.zip} in the
% TDS tree (also known as \xfile{texmf} tree) of your choice.
% Example (linux):
% \begin{quote}
%   |unzip oberdiek.tds.zip -d ~/texmf|
% \end{quote}
%
% \paragraph{Script installation.}
% Check the directory \xfile{TDS:scripts/oberdiek/} for
% scripts that need further installation steps.
% Package \xpackage{attachfile2} comes with the Perl script
% \xfile{pdfatfi.pl} that should be installed in such a way
% that it can be called as \texttt{pdfatfi}.
% Example (linux):
% \begin{quote}
%   |chmod +x scripts/oberdiek/pdfatfi.pl|\\
%   |cp scripts/oberdiek/pdfatfi.pl /usr/local/bin/|
% \end{quote}
%
% \subsection{Package installation}
%
% \paragraph{Unpacking.} The \xfile{.dtx} file is a self-extracting
% \docstrip\ archive. The files are extracted by running the
% \xfile{.dtx} through \plainTeX:
% \begin{quote}
%   \verb|tex infwarerr.dtx|
% \end{quote}
%
% \paragraph{TDS.} Now the different files must be moved into
% the different directories in your installation TDS tree
% (also known as \xfile{texmf} tree):
% \begin{quote}
% \def\t{^^A
% \begin{tabular}{@{}>{\ttfamily}l@{ $\rightarrow$ }>{\ttfamily}l@{}}
%   infwarerr.sty & tex/generic/oberdiek/infwarerr.sty\\
%   infwarerr.pdf & doc/latex/oberdiek/infwarerr.pdf\\
%   test/infwarerr-test1.tex & doc/latex/oberdiek/test/infwarerr-test1.tex\\
%   test/infwarerr-test2.tex & doc/latex/oberdiek/test/infwarerr-test2.tex\\
%   test/infwarerr-test3.tex & doc/latex/oberdiek/test/infwarerr-test3.tex\\
%   infwarerr.dtx & source/latex/oberdiek/infwarerr.dtx\\
% \end{tabular}^^A
% }^^A
% \sbox0{\t}^^A
% \ifdim\wd0>\linewidth
%   \begingroup
%     \advance\linewidth by\leftmargin
%     \advance\linewidth by\rightmargin
%   \edef\x{\endgroup
%     \def\noexpand\lw{\the\linewidth}^^A
%   }\x
%   \def\lwbox{^^A
%     \leavevmode
%     \hbox to \linewidth{^^A
%       \kern-\leftmargin\relax
%       \hss
%       \usebox0
%       \hss
%       \kern-\rightmargin\relax
%     }^^A
%   }^^A
%   \ifdim\wd0>\lw
%     \sbox0{\small\t}^^A
%     \ifdim\wd0>\linewidth
%       \ifdim\wd0>\lw
%         \sbox0{\footnotesize\t}^^A
%         \ifdim\wd0>\linewidth
%           \ifdim\wd0>\lw
%             \sbox0{\scriptsize\t}^^A
%             \ifdim\wd0>\linewidth
%               \ifdim\wd0>\lw
%                 \sbox0{\tiny\t}^^A
%                 \ifdim\wd0>\linewidth
%                   \lwbox
%                 \else
%                   \usebox0
%                 \fi
%               \else
%                 \lwbox
%               \fi
%             \else
%               \usebox0
%             \fi
%           \else
%             \lwbox
%           \fi
%         \else
%           \usebox0
%         \fi
%       \else
%         \lwbox
%       \fi
%     \else
%       \usebox0
%     \fi
%   \else
%     \lwbox
%   \fi
% \else
%   \usebox0
% \fi
% \end{quote}
% If you have a \xfile{docstrip.cfg} that configures and enables \docstrip's
% TDS installing feature, then some files can already be in the right
% place, see the documentation of \docstrip.
%
% \subsection{Refresh file name databases}
%
% If your \TeX~distribution
% (\teTeX, \mikTeX, \dots) relies on file name databases, you must refresh
% these. For example, \teTeX\ users run \verb|texhash| or
% \verb|mktexlsr|.
%
% \subsection{Some details for the interested}
%
% \paragraph{Attached source.}
%
% The PDF documentation on CTAN also includes the
% \xfile{.dtx} source file. It can be extracted by
% AcrobatReader 6 or higher. Another option is \textsf{pdftk},
% e.g. unpack the file into the current directory:
% \begin{quote}
%   \verb|pdftk infwarerr.pdf unpack_files output .|
% \end{quote}
%
% \paragraph{Unpacking with \LaTeX.}
% The \xfile{.dtx} chooses its action depending on the format:
% \begin{description}
% \item[\plainTeX:] Run \docstrip\ and extract the files.
% \item[\LaTeX:] Generate the documentation.
% \end{description}
% If you insist on using \LaTeX\ for \docstrip\ (really,
% \docstrip\ does not need \LaTeX), then inform the autodetect routine
% about your intention:
% \begin{quote}
%   \verb|latex \let\install=y\input{infwarerr.dtx}|
% \end{quote}
% Do not forget to quote the argument according to the demands
% of your shell.
%
% \paragraph{Generating the documentation.}
% You can use both the \xfile{.dtx} or the \xfile{.drv} to generate
% the documentation. The process can be configured by the
% configuration file \xfile{ltxdoc.cfg}. For instance, put this
% line into this file, if you want to have A4 as paper format:
% \begin{quote}
%   \verb|\PassOptionsToClass{a4paper}{article}|
% \end{quote}
% An example follows how to generate the
% documentation with pdf\LaTeX:
% \begin{quote}
%\begin{verbatim}
%pdflatex infwarerr.dtx
%makeindex -s gind.ist infwarerr.idx
%pdflatex infwarerr.dtx
%makeindex -s gind.ist infwarerr.idx
%pdflatex infwarerr.dtx
%\end{verbatim}
% \end{quote}
%
% \begin{History}
%   \begin{Version}{2007/04/26 v1.0}
%   \item
%     First version.
%   \end{Version}
%   \begin{Version}{2007/06/14 v1.1}
%   \item
%     Small fixes.
%   \end{Version}
%   \begin{Version}{2007/09/09 v1.2}
%   \item
%     Catcode section extended.
%   \end{Version}
%   \begin{Version}{2010/04/08 v1.3}
%   \item
%     Ensure \cs{escapechar}=92 (backslash) for infos/warnings/errors.
%   \end{Version}
% \end{History}
%
% \PrintIndex
%
% \Finale
\endinput

%        (quote the arguments according to the demands of your shell)
% 
% Documentation:
%    (a) If infwarerr.drv is present:
%           latex infwarerr.drv
%    (b) Without infwarerr.drv:
%           latex infwarerr.dtx; ...
%    The class ltxdoc loads the configuration file ltxdoc.cfg
%    if available. Here you can specify further options, e.g.
%    use A4 as paper format:
%       \PassOptionsToClass{a4paper}{article}
% 
%    Programm calls to get the documentation (example):
%       pdflatex infwarerr.dtx
%       makeindex -s gind.ist infwarerr.idx
%       pdflatex infwarerr.dtx
%       makeindex -s gind.ist infwarerr.idx
%       pdflatex infwarerr.dtx
% 
% Installation:
%    TDS:tex/generic/oberdiek/infwarerr.sty
%    TDS:doc/latex/oberdiek/infwarerr.pdf
%    TDS:doc/latex/oberdiek/test/infwarerr-test1.tex
%    TDS:doc/latex/oberdiek/test/infwarerr-test2.tex
%    TDS:doc/latex/oberdiek/test/infwarerr-test3.tex
%    TDS:source/latex/oberdiek/infwarerr.dtx
%
%<*ignore>
\begingroup
  \catcode123=1 %
  \catcode125=2 %
  \def\x{LaTeX2e}%
\expandafter\endgroup
\ifcase 0\ifx\install y1\fi\expandafter
         \ifx\csname processbatchFile\endcsname\relax\else1\fi
         \ifx\fmtname\x\else 1\fi\relax
\else\csname fi\endcsname
%</ignore>
%<*install>
\input docstrip.tex
\Msg{************************************************************************}
\Msg{* Installation}
\Msg{* Package: infwarerr 2010/04/08 v1.3 Providing info/warning/message (HO)}
\Msg{************************************************************************}

\keepsilent
\askforoverwritefalse

\let\MetaPrefix\relax
\preamble

This is a generated file.

Project: infwarerr
Version: 2010/04/08 v1.3

Copyright (C) 2007, 2010 by
   Heiko Oberdiek <heiko.oberdiek at googlemail.com>

This work may be distributed and/or modified under the
conditions of the LaTeX Project Public License, either
version 1.3c of this license or (at your option) any later
version. This version of this license is in
   http://www.latex-project.org/lppl/lppl-1-3c.txt
and the latest version of this license is in
   http://www.latex-project.org/lppl.txt
and version 1.3 or later is part of all distributions of
LaTeX version 2005/12/01 or later.

This work has the LPPL maintenance status "maintained".

This Current Maintainer of this work is Heiko Oberdiek.

The Base Interpreter refers to any `TeX-Format',
because some files are installed in TDS:tex/generic//.

This work consists of the main source file infwarerr.dtx
and the derived files
   infwarerr.sty, infwarerr.pdf, infwarerr.ins, infwarerr.drv,
   infwarerr-test1.tex, infwarerr-test2.tex,
   infwarerr-test3.tex.

\endpreamble
\let\MetaPrefix\DoubleperCent

\generate{%
  \file{infwarerr.ins}{\from{infwarerr.dtx}{install}}%
  \file{infwarerr.drv}{\from{infwarerr.dtx}{driver}}%
  \usedir{tex/generic/oberdiek}%
  \file{infwarerr.sty}{\from{infwarerr.dtx}{package}}%
  \usedir{doc/latex/oberdiek/test}%
  \file{infwarerr-test1.tex}{\from{infwarerr.dtx}{test1}}%
  \file{infwarerr-test2.tex}{\from{infwarerr.dtx}{test2}}%
  \file{infwarerr-test3.tex}{\from{infwarerr.dtx}{test3}}%
}

\catcode32=13\relax% active space
\let =\space%
\Msg{************************************************************************}
\Msg{*}
\Msg{* To finish the installation you have to move the following}
\Msg{* file into a directory searched by TeX:}
\Msg{*}
\Msg{*     infwarerr.sty}
\Msg{*}
\Msg{* To produce the documentation run the file `infwarerr.drv'}
\Msg{* through LaTeX.}
\Msg{*}
\Msg{* Happy TeXing!}
\Msg{*}
\Msg{************************************************************************}

\endbatchfile
%</install>
%<*ignore>
\fi
%</ignore>
%<*driver>
\NeedsTeXFormat{LaTeX2e}
\ProvidesFile{infwarerr.drv}%
  [2010/04/08 v1.3 Providing info/warning/message (HO)]%
\documentclass{ltxdoc}
\usepackage{holtxdoc}[2011/02/04]
\begin{document}
  \DocInput{infwarerr.dtx}%
\end{document}
%</driver>
% \fi
%
% \CheckSum{871}
%
% \CharacterTable
%  {Upper-case    \A\B\C\D\E\F\G\H\I\J\K\L\M\N\O\P\Q\R\S\T\U\V\W\X\Y\Z
%   Lower-case    \a\b\c\d\e\f\g\h\i\j\k\l\m\n\o\p\q\r\s\t\u\v\w\x\y\z
%   Digits        \0\1\2\3\4\5\6\7\8\9
%   Exclamation   \!     Double quote  \"     Hash (number) \#
%   Dollar        \$     Percent       \%     Ampersand     \&
%   Acute accent  \'     Left paren    \(     Right paren   \)
%   Asterisk      \*     Plus          \+     Comma         \,
%   Minus         \-     Point         \.     Solidus       \/
%   Colon         \:     Semicolon     \;     Less than     \<
%   Equals        \=     Greater than  \>     Question mark \?
%   Commercial at \@     Left bracket  \[     Backslash     \\
%   Right bracket \]     Circumflex    \^     Underscore    \_
%   Grave accent  \`     Left brace    \{     Vertical bar  \|
%   Right brace   \}     Tilde         \~}
%
% \GetFileInfo{infwarerr.drv}
%
% \title{The \xpackage{infwarerr} package}
% \date{2010/04/08 v1.3}
% \author{Heiko Oberdiek\\\xemail{heiko.oberdiek at googlemail.com}}
%
% \maketitle
%
% \begin{abstract}
% This package provides a complete set of macros for informations,
% warnings and error messages with support for \plainTeX.
% \end{abstract}
%
% \tableofcontents
%
% \section{Documentation}
%
% Package \xpackage{infwarerr} defines the following set of commands,
% given as pseudo regular expression:
%   \begin{quote}
%     \verb+\@(Package|Class)(Info|InfoNoLine|Warning|WarningNoLine|Error)+^^A
%     \kern-\rightmargin
%   \end{quote}
% The first argument is the package or class in question. The second
% argument contains the message text. \cs{MessageBreak} is supported.
% In case of error messages, the third argument contains the help
% text. Supported are \LaTeX's message texts \cs{@ehc}:
%   \begin{quote}
%     |Try typing  <return>  to proceed|\\
%     |If that doesn't work, type  X <return>  to quit.|
%   \end{quote}
% and \cs{@ehd}:
%   \begin{quote}
%     |You're in trouble here.  Try typing  <return>  to proceed|\\
%     |If that doesn't work, type  X <return>  to quit.|
%   \end{quote}
%
% The package is intended for macro writers that want to write
% code that works for both \LaTeX\ and \plainTeX. If \LaTeX\ is
% detected, then the macros of this package will only be wrappers
% for \LaTeX's macros. Otherwise \LaTeX's macros will be mimicked
% for use with \plainTeX.
%
% \StopEventually{
% }
%
% \section{Implementation}
%
%    \begin{macrocode}
%<*package>
%    \end{macrocode}
%
% \subsection{Reload check and package identification}
%    Reload check, especially if the package is not used with \LaTeX.
%    \begin{macrocode}
\begingroup\catcode61\catcode48\catcode32=10\relax%
  \catcode13=5 % ^^M
  \endlinechar=13 %
  \catcode35=6 % #
  \catcode39=12 % '
  \catcode44=12 % ,
  \catcode45=12 % -
  \catcode46=12 % .
  \catcode58=12 % :
  \catcode64=11 % @
  \catcode123=1 % {
  \catcode125=2 % }
  \expandafter\let\expandafter\x\csname ver@infwarerr.sty\endcsname
  \ifx\x\relax % plain-TeX, first loading
  \else
    \def\empty{}%
    \ifx\x\empty % LaTeX, first loading,
      % variable is initialized, but \ProvidesPackage not yet seen
    \else
      \expandafter\ifx\csname PackageInfo\endcsname\relax
        \def\x#1#2{%
          \immediate\write-1{Package #1 Info: #2.}%
        }%
      \else
        \def\x#1#2{\PackageInfo{#1}{#2, stopped}}%
      \fi
      \x{infwarerr}{The package is already loaded}%
      \aftergroup\endinput
    \fi
  \fi
\endgroup%
%    \end{macrocode}
%    Package identification:
%    \begin{macrocode}
\begingroup\catcode61\catcode48\catcode32=10\relax%
  \catcode13=5 % ^^M
  \endlinechar=13 %
  \catcode35=6 % #
  \catcode39=12 % '
  \catcode40=12 % (
  \catcode41=12 % )
  \catcode44=12 % ,
  \catcode45=12 % -
  \catcode46=12 % .
  \catcode47=12 % /
  \catcode58=12 % :
  \catcode64=11 % @
  \catcode91=12 % [
  \catcode93=12 % ]
  \catcode123=1 % {
  \catcode125=2 % }
  \expandafter\ifx\csname ProvidesPackage\endcsname\relax
    \def\x#1#2#3[#4]{\endgroup
      \immediate\write-1{Package: #3 #4}%
      \xdef#1{#4}%
    }%
  \else
    \def\x#1#2[#3]{\endgroup
      #2[{#3}]%
      \ifx#1\@undefined
        \xdef#1{#3}%
      \fi
      \ifx#1\relax
        \xdef#1{#3}%
      \fi
    }%
  \fi
\expandafter\x\csname ver@infwarerr.sty\endcsname
\ProvidesPackage{infwarerr}%
  [2010/04/08 v1.3 Providing info/warning/message (HO)]%
%    \end{macrocode}
%
% \subsection{Catcodes}
%
%    \begin{macrocode}
\begingroup\catcode61\catcode48\catcode32=10\relax%
  \catcode13=5 % ^^M
  \endlinechar=13 %
  \catcode123=1 % {
  \catcode125=2 % }
  \catcode64=11 % @
  \def\x{\endgroup
    \expandafter\edef\csname InfWarErr@AtEnd\endcsname{%
      \endlinechar=\the\endlinechar\relax
      \catcode13=\the\catcode13\relax
      \catcode32=\the\catcode32\relax
      \catcode35=\the\catcode35\relax
      \catcode61=\the\catcode61\relax
      \catcode64=\the\catcode64\relax
      \catcode123=\the\catcode123\relax
      \catcode125=\the\catcode125\relax
    }%
  }%
\x\catcode61\catcode48\catcode32=10\relax%
\catcode13=5 % ^^M
\endlinechar=13 %
\catcode35=6 % #
\catcode64=11 % @
\catcode123=1 % {
\catcode125=2 % }
\def\TMP@EnsureCode#1#2{%
  \edef\InfWarErr@AtEnd{%
    \InfWarErr@AtEnd
    \catcode#1=\the\catcode#1\relax
  }%
  \catcode#1=#2\relax
}
\TMP@EnsureCode{10}{12}% ^^J
\TMP@EnsureCode{39}{12}% '
\TMP@EnsureCode{40}{12}% (
\TMP@EnsureCode{41}{12}% )
\TMP@EnsureCode{44}{12}% ,
\TMP@EnsureCode{45}{12}% -
\TMP@EnsureCode{46}{12}% .
\TMP@EnsureCode{58}{12}% :
\TMP@EnsureCode{60}{12}% <
\TMP@EnsureCode{62}{12}% >
\TMP@EnsureCode{94}{7}% ^
\edef\InfWarErr@AtEnd{\InfWarErr@AtEnd\noexpand\endinput}
%    \end{macrocode}
%
% \subsection{\LaTeX\ detection}
%
%    \begin{macrocode}
\begingroup
  \let\x=Y%
  \def\check#1{%
    \@check{Generic}{#1}%
    \@check{Package}{#1}%
    \@check{Class}{#1}%
  }%
  \def\@check#1#2{%
    \expandafter\ifx\csname #1#2\endcsname\relax
      \let\x=N%
    \fi
  }%
  \check{Info}%
  \check{Warning}%
  \check{Error}%
  \@check{Package}{WarningNoLine}%
  \@check{Class}{WarningNoLine}%
  \@check{}{@ehc}%
  \@check{}{@ehd}%
\expandafter\endgroup
\ifx\x Y%
%    \end{macrocode}
%
% \subsection{Version for \LaTeX}
%
%    \begin{macro}{\@PackageInfo}
%    \begin{macrocode}
  \let\@PackageInfo\PackageInfo
%    \end{macrocode}
%    \end{macro}
%    \begin{macro}{\@PackageInfoNoLine}
%    \begin{macrocode}
  \def\@PackageInfoNoLine#1#2{%
    \PackageInfo{#1}{#2\@gobble}%
  }%
%    \end{macrocode}
%    \end{macro}
%    \begin{macro}{\@ClassInfo}
%    \begin{macrocode}
  \let\@ClassInfo\ClassInfo
%    \end{macrocode}
%    \end{macro}
%    \begin{macro}{\@ClassInfoNoLine}
%    \begin{macrocode}
  \def\@ClassInfoNoLine#1#2{%
    \ClassInfo{#1}{#2\@gobble}%
  }%
%    \end{macrocode}
%    \end{macro}
%    \begin{macro}{\@PackageWarning}
%    \begin{macrocode}
  \let\@PackageWarning\PackageWarning
%    \end{macrocode}
%    \end{macro}
%    \begin{macro}{\@PackageWarningNoLine}
%    \begin{macrocode}
  \let\@PackageWarningNoLine\PackageWarningNoLine
%    \end{macrocode}
%    \end{macro}
%    \begin{macro}{\@ClassWarning}
%    \begin{macrocode}
  \let\@ClassWarning\ClassWarning
%    \end{macrocode}
%    \end{macro}
%    \begin{macro}{\@ClassWarningNoLine}
%    \begin{macrocode}
  \let\@ClassWarningNoLine\ClassWarningNoLine
%    \end{macrocode}
%    \end{macro}
%    \begin{macro}{\@PackageError}
%    \begin{macrocode}
  \let\@PackageError\PackageError
%    \end{macrocode}
%    \end{macro}
%    \begin{macro}{\@ClassError}
%    \begin{macrocode}
  \let\@ClassError\ClassError
%    \end{macrocode}
%    \end{macro}
%    \begin{macrocode}
  \expandafter\InfWarErr@AtEnd
\fi%
%    \end{macrocode}
%
% \subsection{Version for \plainTeX}
%
%    \begin{macro}{\InfWarErr@online}
%    \begin{macrocode}
\begingroup\expandafter\endgroup
\ifcase\expandafter\ifx\csname inputlineno\endcsname\relax
         0 %
       \else
         \ifnum\inputlineno<0 %
           0 %
         \else
           1 %
         \fi
       \fi
  \def\InfWarErr@online{}%
\else
  \def\InfWarErr@online{ on input line \the\inputlineno}%
\fi
%    \end{macrocode}
%    \end{macro}
%
%    \begin{macro}{\InfWarErr@protected}
%    \begin{macrocode}
\begingroup\expandafter\expandafter\expandafter\endgroup
\expandafter\ifx\csname protected\endcsname\relax
  \def\InfWarErr@protected{}%
\else
  \let\InfWarErr@protected\protected
\fi
%    \end{macrocode}
%    \end{macro}
%
%    \begin{macro}{\InfWarErr@unused}
%    \begin{macrocode}
\chardef\InfWarErr@unused=16 %
%    \end{macrocode}
%    \end{macro}
%
%    \begin{macro}{\InfWarErr@ehc}
%    \begin{macrocode}
\def\InfWarErr@ehc{%
  Try typing \space <return> \space to proceed.\MessageBreak
  If that doesn't work, type \space X <return> \space to quit.%
}
%    \end{macrocode}
%    \end{macro}
%    \begin{macro}{\InfWarErr@ehd}
%    \begin{macrocode}
\def\InfWarErr@ehd{%
  You're in trouble here. \space\@ehc
}
%    \end{macrocode}
%    \end{macro}
%
%    \begin{macro}{\InfWarErr@GenericInfo}
%    \begin{macrocode}
\InfWarErr@protected\def\InfWarErr@GenericInfo#1#2{%
  \begingroup
    \let\on@line\InfWarErr@online
    \long\def\@gobble##1{}%
    \def\space{ }%
    \def\@spaces{\space\space\space\space}%
    \def\MessageBreak{^^J#1}%
    \newlinechar=10 %
    \escapechar=92 %
    \immediate\write-1{#2\on@line.}%
  \endgroup
}
%    \end{macrocode}
%    \end{macro}
%    \begin{macro}{\InfWarErr@GenericWarning}
%    \begin{macrocode}
\InfWarErr@protected\def\InfWarErr@GenericWarning#1#2{%
  \begingroup
    \let\on@line\InfWarErr@online
    \long\def\@gobble##1{}%
    \def\space{ }%
    \def\@spaces{\space\space\space\space}%
    \def\MessageBreak{^^J#1}%
    \newlinechar=10 %
    \escapechar=92 %
    \immediate\write\InfWarErr@unused{^^J#2\on@line.^^J}%
  \endgroup
}
%    \end{macrocode}
%    \end{macro}
%    \begin{macro}{\InfWarErr@GenericError}
%    Old \TeX\ versions below 3.141 where |^^J| is not working
%    as newline in error message and help texts are not supported.
%    \begin{macrocode}
\InfWarErr@protected\def\InfWarErr@GenericError#1#2#3#4{%
  \begingroup
    \immediate\write\InfWarErr@unused{}%
    \def\space{ }%
    \def\@spaces{\space\space\space\space}%
    \let\@ehc\InfWarErr@ehc
    \let\@ehd\InfWarErr@ehd
    \let\on@line\InfWarErr@online
    \def\MessageBreak{^^J}%
    \newlinechar=10 %
    \escapechar=92 %
    \edef\InfWarErr@help{\noexpand\errhelp{#4}}%
    \InfWarErr@help
    \def\MessageBreak{^^J#1}%
    \errmessage{%
      #2.^^J^^J%
      #3^^J%
      Type \space H <return> \space for immediate help.^^J ...^^J%
    }%
  \endgroup
}
%    \end{macrocode}
%    \end{macro}
%
%    \begin{macro}{\@PackageInfo}
%    \begin{macrocode}
\def\@PackageInfo#1#2{%
  \InfWarErr@GenericInfo{%
    (#1) \@spaces\@spaces\@spaces
  }{%
    Package #1 Info: #2%
  }%
}
%    \end{macrocode}
%    \end{macro}
%    \begin{macro}{\@PackageInfoNoLine}
%    \begin{macrocode}
\def\@PackageInfoNoLine#1#2{%
  \@PackageInfo{#1}{#2\@gobble}%
}
%    \end{macrocode}
%    \end{macro}
%    \begin{macro}{\@ClassInfo}
%    \begin{macrocode}
\def\@ClassInfo#1#2{%
  \InfWarErr@GenericInfo{%
    (#1) \space\space\@spaces\@spaces
  }{%
    Class #1 Info: #2%
  }%
}
%    \end{macrocode}
%    \end{macro}
%    \begin{macro}{\@ClassInfoNoLine}
%    \begin{macrocode}
\def\@ClassInfoNoLine#1#2{%
  \@ClassInfo{#1}{#2\@gobble}%
}
%    \end{macrocode}
%    \end{macro}
%    \begin{macro}{\@PackageWarning}
%    \begin{macrocode}
\def\@PackageWarning#1#2{%
  \InfWarErr@GenericWarning{%
    (#1)\@spaces\@spaces\@spaces\@spaces
  }{%
    Package #1 Warning: #2%
  }%
}
%    \end{macrocode}
%    \end{macro}
%    \begin{macro}{\@PackageWarningNoLine}
%    \begin{macrocode}
\def\@PackageWarningNoLine#1#2{%
  \@PackageWarning{#1}{#2\@gobble}%
}
%    \end{macrocode}
%    \end{macro}
%    \begin{macro}{\@ClassWarning}
%    \begin{macrocode}
\def\@ClassWarning#1#2{%
  \InfWarErr@GenericWarning{%
    (#1) \space\@spaces\@spaces\@spaces
  }{%
    Class #1 Warning: #2%
  }%
}
%    \end{macrocode}
%    \end{macro}
%    \begin{macro}{\@ClassWarningNoLine}
%    \begin{macrocode}
\def\@ClassWarningNoLine#1#2{%
  \@ClassWarning{#1}{#2\@gobble}%
}
%    \end{macrocode}
%    \end{macro}
%    \begin{macro}{\@PackageError}
%    \begin{macrocode}
\def\@PackageError#1#2#3{%
  \InfWarErr@GenericError{%
    (#1)\@spaces\@spaces\@spaces\@spaces
  }{%
    Package #1 Error: #2%
  }{%
    See the #1 package documentation for explanation.%
  }{#3}%
}
%    \end{macrocode}
%    \end{macro}
%    \begin{macro}{\@ClassError}
%    \begin{macrocode}
\def\@ClassError#1#2#3{%
  \InfWarErr@GenericError{%
    (#1) \space\@spaces\@spaces\@spaces
  }{%
    Class #1 Error: #2%
  }{%
    See the #1 class documentation for explanation.%
  }{#3}%
}
%    \end{macrocode}
%    \end{macro}
%
%    \begin{macrocode}
\InfWarErr@AtEnd%
%</package>
%    \end{macrocode}
%
% \section{Test}
%
% \subsection{Catcode checks for loading}
%
%    \begin{macrocode}
%<*test1>
%    \end{macrocode}
%    \begin{macrocode}
\catcode`\{=1 %
\catcode`\}=2 %
\catcode`\#=6 %
\catcode`\@=11 %
\expandafter\ifx\csname count@\endcsname\relax
  \countdef\count@=255 %
\fi
\expandafter\ifx\csname @gobble\endcsname\relax
  \long\def\@gobble#1{}%
\fi
\expandafter\ifx\csname @firstofone\endcsname\relax
  \long\def\@firstofone#1{#1}%
\fi
\expandafter\ifx\csname loop\endcsname\relax
  \expandafter\@firstofone
\else
  \expandafter\@gobble
\fi
{%
  \def\loop#1\repeat{%
    \def\body{#1}%
    \iterate
  }%
  \def\iterate{%
    \body
      \let\next\iterate
    \else
      \let\next\relax
    \fi
    \next
  }%
  \let\repeat=\fi
}%
\def\RestoreCatcodes{}
\count@=0 %
\loop
  \edef\RestoreCatcodes{%
    \RestoreCatcodes
    \catcode\the\count@=\the\catcode\count@\relax
  }%
\ifnum\count@<255 %
  \advance\count@ 1 %
\repeat

\def\RangeCatcodeInvalid#1#2{%
  \count@=#1\relax
  \loop
    \catcode\count@=15 %
  \ifnum\count@<#2\relax
    \advance\count@ 1 %
  \repeat
}
\def\RangeCatcodeCheck#1#2#3{%
  \count@=#1\relax
  \loop
    \ifnum#3=\catcode\count@
    \else
      \errmessage{%
        Character \the\count@\space
        with wrong catcode \the\catcode\count@\space
        instead of \number#3%
      }%
    \fi
  \ifnum\count@<#2\relax
    \advance\count@ 1 %
  \repeat
}
\def\space{ }
\expandafter\ifx\csname LoadCommand\endcsname\relax
  \def\LoadCommand{\input infwarerr.sty\relax}%
\fi
\def\Test{%
  \RangeCatcodeInvalid{0}{47}%
  \RangeCatcodeInvalid{58}{64}%
  \RangeCatcodeInvalid{91}{96}%
  \RangeCatcodeInvalid{123}{255}%
  \catcode`\@=12 %
  \catcode`\\=0 %
  \catcode`\%=14 %
  \LoadCommand
  \RangeCatcodeCheck{0}{36}{15}%
  \RangeCatcodeCheck{37}{37}{14}%
  \RangeCatcodeCheck{38}{47}{15}%
  \RangeCatcodeCheck{48}{57}{12}%
  \RangeCatcodeCheck{58}{63}{15}%
  \RangeCatcodeCheck{64}{64}{12}%
  \RangeCatcodeCheck{65}{90}{11}%
  \RangeCatcodeCheck{91}{91}{15}%
  \RangeCatcodeCheck{92}{92}{0}%
  \RangeCatcodeCheck{93}{96}{15}%
  \RangeCatcodeCheck{97}{122}{11}%
  \RangeCatcodeCheck{123}{255}{15}%
  \RestoreCatcodes
}
\Test
\csname @@end\endcsname
\end
%    \end{macrocode}
%    \begin{macrocode}
%</test1>
%    \end{macrocode}
%
% \subsection{Macro tests}
%
%    \begin{macrocode}
%<*test2>
\let\PackageInfo\relax
\input infwarerr.sty\relax
\let\OrgWrite\write
\chardef\WriteUnused=16 %
\def\msg#{\immediate\OrgWrite\WriteUnused}
\msg{File: infwarerr-test2.tex 2010/04/08 v1.3 Test file for plain-TeX}
\catcode64=11 %
%</test2>
%    \end{macrocode}
%
%    \begin{macrocode}
%<*test3>
\NeedsTeXFormat{LaTeX2e}
\ProvidesFile{infwarerr-test3.tex}[2010/04/08 v1.3 Test file for LaTeX]
\RequirePackage{infwarerr}[2010/04/08]
\makeatletter
\let\OrgWrite\write
\chardef\WriteUnused=\@unused
\def\msg#{\immediate\OrgWrite\WriteUnused}
%</test3>
%    \end{macrocode}
%
%    \begin{macrocode}
%<*test2|test3>
\def\pkgname{PACKAGE}
\def\clsname{CLASS}
\def\msgtext{MESSAGE\MessageBreak SECOND LINE}
\def\hlptext{HELP\MessageBreak SECOND LINE}

\let\OrgWrite\write
\let\OrgErrMessage\errmessage

\def\write#1#{%
  \@write{#1}%
}
\def\@write#1#2{%
  \OrgWrite#1{#2}%
  \xdef\WriteNum{\number#1}%
  \xdef\WriteText{#2}%
}
\def\TestFailed#1#2{%
  \OrgErrMessage{Test (\string#1) failed: #2}%
}
\def\TestWrite#1#2#3{%
  \ifnum\WriteNum=#2\relax
  \else
    \TestFailed#1{write stream number: \WriteNum <> #2}%
  \fi
  \begingroup
    \let~\space
    \edef\TestString{#3}%
    \ifx\WriteText\TestString
    \else
      \msg{}%
      \msg{[\WriteText]}%
      \msg{<>}%
      \msg{[#3]}% hash-ok
      \TestFailed#1{write text}%
    \fi
  \endgroup
}
\newtoks\ErrHelpToks
\def\errmessage#1{%
  \global\ErrHelpToks=\expandafter{\the\errhelp}%
  \msg{**************** error message ****************}%
  \msg{! #1}%
  \msg{**************** help message *****************}%
  \msg{\the\errhelp}%
  \msg{***********************************************}%
  \xdef\ErrMessage{! #1}%
}
\def\TestInfo{%
  \@PackageInfo\pkgname\msgtext
  \TestWrite\@PackageInfo{-1}{%
    Package PACKAGE Info: MESSAGE^^J%
    (PACKAGE)~~~~~~~~~~~~~SECOND LINE on input line \the\inputlineno.%
  }%
  \@PackageInfoNoLine\pkgname\msgtext
  \TestWrite\@PackageInfoNoLine{-1}{%
    Package PACKAGE Info: MESSAGE^^J%
    (PACKAGE)~~~~~~~~~~~~~SECOND LINE.%
  }%
  \@ClassInfo\clsname\msgtext
  \TestWrite\@ClassInfo{-1}{%
    Class CLASS Info: MESSAGE^^J%
    (CLASS)~~~~~~~~~~~SECOND LINE on input line \the\inputlineno.%
  }%
  \@ClassInfoNoLine\clsname\msgtext
  \TestWrite\@ClassInfoNoLine{-1}{%
    Class CLASS Info: MESSAGE^^J%
    (CLASS)~~~~~~~~~~~SECOND LINE.%
  }%
}
\def\TestWarning{%
  \@PackageWarning\pkgname\msgtext
  \TestWrite\@PackageWarning\WriteUnused{%
    ^^J%
    Package PACKAGE Warning: MESSAGE^^J%
    (PACKAGE)~~~~~~~~~~~~~~~~SECOND LINE on input line \the\inputlineno.%
    ^^J%
  }%
  \@PackageWarningNoLine\pkgname\msgtext
  \TestWrite\@PackageWarningNoLine\WriteUnused{%
    ^^J%
    Package PACKAGE Warning: MESSAGE^^J%
    (PACKAGE)~~~~~~~~~~~~~~~~SECOND LINE.%
    ^^J%
  }%
  \@ClassWarning\clsname\msgtext
  \TestWrite\@ClassWarning\WriteUnused{%
    ^^J%
    Class CLASS Warning: MESSAGE^^J%
    (CLASS)~~~~~~~~~~~~~~SECOND LINE on input line \the\inputlineno.%
    ^^J%
  }%
  \@ClassWarningNoLine\clsname\msgtext
  \TestWrite\@ClassWarningNoLine\WriteUnused{%
    ^^J%
    Class CLASS Warning: MESSAGE^^J%
    (CLASS)~~~~~~~~~~~~~~SECOND LINE.%
    ^^J%
  }%
}
\def\TestError{%
%<test3>  \let\write\OrgWrite
%<test3>  \def\plaindots{}
%<test2>  \def\plaindots{.^^J ...^^J}
  \@PackageError\pkgname\msgtext\hlptext
  \@TestError\@PackageError{%
    ! Package PACKAGE Error: MESSAGE^^J%
    (PACKAGE)~~~~~~~~~~~~~~~~SECOND LINE.^^J%
    ^^J%
    See the PACKAGE package documentation for explanation.^^J%
    Type~~H <return>~~for immediate help\plaindots
  }{%
    HELP^^J%
    SECOND LINE%
  }%
  \@ClassError\clsname\msgtext\hlptext
  \@TestError\@ClassError{%
    ! Class CLASS Error: MESSAGE^^J%
    (CLASS)~~~~~~~~~~~~~~SECOND LINE.^^J%
    ^^J%
    See the CLASS class documentation for explanation.^^J%
    Type~~H <return>~~for immediate help\plaindots
  }{%
    HELP^^J%
    SECOND LINE%
  }%
  \@PackageError\pkgname\msgtext\@ehc
  \@TestError\@PackageError{%
    ! Package PACKAGE Error: MESSAGE^^J%
    (PACKAGE)~~~~~~~~~~~~~~~~SECOND LINE.^^J%
    ^^J%
    See the PACKAGE package documentation for explanation.^^J%
    Type~~H <return>~~for immediate help\plaindots
  }{%
    Try typing~~<return>~~to proceed.^^J%
    If that doesn't work, type~~X <return>~~to quit.%
  }%
  \@PackageError\pkgname\msgtext\@ehd
  \@TestError\@PackageError{%
    ! Package PACKAGE Error: MESSAGE^^J%
    (PACKAGE)~~~~~~~~~~~~~~~~SECOND LINE.^^J%
    ^^J%
    See the PACKAGE package documentation for explanation.^^J%
    Type~~H <return>~~for immediate help\plaindots
  }{%
    You're in trouble here.~~Try typing~~<return>~~to proceed.^^J%
    If that doesn't work, type~~X <return>~~to quit.%
  }%
}
\def\@TestError#1#2#3{%
  \begingroup
    \let~\space
    \edef\x{#2}%
%<test3>  \@onelevel@sanitize\x
%<test3>  \@onelevel@sanitize\ErrMessage
    \ifx\x\ErrMessage
    \else
      \msg{}%
      \msg{[\ErrMessage]}%
      \msg{<>}%
      \msg{[\x]}%
      \TestFailed#1{errmessage text}%
    \fi
    \edef\x{\the\ErrHelpToks}%
    \edef\y{#3}%
    \ifx\x\y
    \else
      \msg{}%
      \msg{[\the\ErrHelpToks]}%
      \msg{<>}%
      \msg{[\y]}%
      \TestFailed#1{errhelp text}%
    \fi
  \endgroup
}
\TestInfo
\TestWarning
\TestError
%</test2|test3>
%<test2>\end
%<test3>\@@end
%    \end{macrocode}
%
%
% \section{Installation}
%
% \subsection{Download}
%
% \paragraph{Package.} This package is available on
% CTAN\footnote{\url{ftp://ftp.ctan.org/tex-archive/}}:
% \begin{description}
% \item[\CTAN{macros/latex/contrib/oberdiek/infwarerr.dtx}] The source file.
% \item[\CTAN{macros/latex/contrib/oberdiek/infwarerr.pdf}] Documentation.
% \end{description}
%
%
% \paragraph{Bundle.} All the packages of the bundle `oberdiek'
% are also available in a TDS compliant ZIP archive. There
% the packages are already unpacked and the documentation files
% are generated. The files and directories obey the TDS standard.
% \begin{description}
% \item[\CTAN{install/macros/latex/contrib/oberdiek.tds.zip}]
% \end{description}
% \emph{TDS} refers to the standard ``A Directory Structure
% for \TeX\ Files'' (\CTAN{tds/tds.pdf}). Directories
% with \xfile{texmf} in their name are usually organized this way.
%
% \subsection{Bundle installation}
%
% \paragraph{Unpacking.} Unpack the \xfile{oberdiek.tds.zip} in the
% TDS tree (also known as \xfile{texmf} tree) of your choice.
% Example (linux):
% \begin{quote}
%   |unzip oberdiek.tds.zip -d ~/texmf|
% \end{quote}
%
% \paragraph{Script installation.}
% Check the directory \xfile{TDS:scripts/oberdiek/} for
% scripts that need further installation steps.
% Package \xpackage{attachfile2} comes with the Perl script
% \xfile{pdfatfi.pl} that should be installed in such a way
% that it can be called as \texttt{pdfatfi}.
% Example (linux):
% \begin{quote}
%   |chmod +x scripts/oberdiek/pdfatfi.pl|\\
%   |cp scripts/oberdiek/pdfatfi.pl /usr/local/bin/|
% \end{quote}
%
% \subsection{Package installation}
%
% \paragraph{Unpacking.} The \xfile{.dtx} file is a self-extracting
% \docstrip\ archive. The files are extracted by running the
% \xfile{.dtx} through \plainTeX:
% \begin{quote}
%   \verb|tex infwarerr.dtx|
% \end{quote}
%
% \paragraph{TDS.} Now the different files must be moved into
% the different directories in your installation TDS tree
% (also known as \xfile{texmf} tree):
% \begin{quote}
% \def\t{^^A
% \begin{tabular}{@{}>{\ttfamily}l@{ $\rightarrow$ }>{\ttfamily}l@{}}
%   infwarerr.sty & tex/generic/oberdiek/infwarerr.sty\\
%   infwarerr.pdf & doc/latex/oberdiek/infwarerr.pdf\\
%   test/infwarerr-test1.tex & doc/latex/oberdiek/test/infwarerr-test1.tex\\
%   test/infwarerr-test2.tex & doc/latex/oberdiek/test/infwarerr-test2.tex\\
%   test/infwarerr-test3.tex & doc/latex/oberdiek/test/infwarerr-test3.tex\\
%   infwarerr.dtx & source/latex/oberdiek/infwarerr.dtx\\
% \end{tabular}^^A
% }^^A
% \sbox0{\t}^^A
% \ifdim\wd0>\linewidth
%   \begingroup
%     \advance\linewidth by\leftmargin
%     \advance\linewidth by\rightmargin
%   \edef\x{\endgroup
%     \def\noexpand\lw{\the\linewidth}^^A
%   }\x
%   \def\lwbox{^^A
%     \leavevmode
%     \hbox to \linewidth{^^A
%       \kern-\leftmargin\relax
%       \hss
%       \usebox0
%       \hss
%       \kern-\rightmargin\relax
%     }^^A
%   }^^A
%   \ifdim\wd0>\lw
%     \sbox0{\small\t}^^A
%     \ifdim\wd0>\linewidth
%       \ifdim\wd0>\lw
%         \sbox0{\footnotesize\t}^^A
%         \ifdim\wd0>\linewidth
%           \ifdim\wd0>\lw
%             \sbox0{\scriptsize\t}^^A
%             \ifdim\wd0>\linewidth
%               \ifdim\wd0>\lw
%                 \sbox0{\tiny\t}^^A
%                 \ifdim\wd0>\linewidth
%                   \lwbox
%                 \else
%                   \usebox0
%                 \fi
%               \else
%                 \lwbox
%               \fi
%             \else
%               \usebox0
%             \fi
%           \else
%             \lwbox
%           \fi
%         \else
%           \usebox0
%         \fi
%       \else
%         \lwbox
%       \fi
%     \else
%       \usebox0
%     \fi
%   \else
%     \lwbox
%   \fi
% \else
%   \usebox0
% \fi
% \end{quote}
% If you have a \xfile{docstrip.cfg} that configures and enables \docstrip's
% TDS installing feature, then some files can already be in the right
% place, see the documentation of \docstrip.
%
% \subsection{Refresh file name databases}
%
% If your \TeX~distribution
% (\teTeX, \mikTeX, \dots) relies on file name databases, you must refresh
% these. For example, \teTeX\ users run \verb|texhash| or
% \verb|mktexlsr|.
%
% \subsection{Some details for the interested}
%
% \paragraph{Attached source.}
%
% The PDF documentation on CTAN also includes the
% \xfile{.dtx} source file. It can be extracted by
% AcrobatReader 6 or higher. Another option is \textsf{pdftk},
% e.g. unpack the file into the current directory:
% \begin{quote}
%   \verb|pdftk infwarerr.pdf unpack_files output .|
% \end{quote}
%
% \paragraph{Unpacking with \LaTeX.}
% The \xfile{.dtx} chooses its action depending on the format:
% \begin{description}
% \item[\plainTeX:] Run \docstrip\ and extract the files.
% \item[\LaTeX:] Generate the documentation.
% \end{description}
% If you insist on using \LaTeX\ for \docstrip\ (really,
% \docstrip\ does not need \LaTeX), then inform the autodetect routine
% about your intention:
% \begin{quote}
%   \verb|latex \let\install=y% \iffalse meta-comment
% 
% File: infwarerr.dtx
% Version: 2010/04/08 v1.3
% 
% Copyright (C) 2007, 2010 by
%    Heiko Oberdiek <heiko.oberdiek at googlemail.com>
% 
% This work may be distributed and/or modified under the
% conditions of the LaTeX Project Public License, either
% version 1.3c of this license or (at your option) any later
% version. This version of this license is in
%    http://www.latex-project.org/lppl/lppl-1-3c.txt
% and the latest version of this license is in
%    http://www.latex-project.org/lppl.txt
% and version 1.3 or later is part of all distributions of
% LaTeX version 2005/12/01 or later.
% 
% This work has the LPPL maintenance status "maintained".
% 
% This Current Maintainer of this work is Heiko Oberdiek.
% 
% The Base Interpreter refers to any `TeX-Format',
% because some files are installed in TDS:tex/generic//.
% 
% This work consists of the main source file infwarerr.dtx
% and the derived files
%    infwarerr.sty, infwarerr.pdf, infwarerr.ins, infwarerr.drv,
%    infwarerr-test1.tex, infwarerr-test2.tex,
%    infwarerr-test3.tex.
% 
% Distribution:
%    CTAN:macros/latex/contrib/oberdiek/infwarerr.dtx
%    CTAN:macros/latex/contrib/oberdiek/infwarerr.pdf
% 
% Unpacking:
%    (a) If infwarerr.ins is present:
%           tex infwarerr.ins
%    (b) Without infwarerr.ins:
%           tex infwarerr.dtx
%    (c) If you insist on using LaTeX
%           latex \let\install=y\input{infwarerr.dtx}
%        (quote the arguments according to the demands of your shell)
% 
% Documentation:
%    (a) If infwarerr.drv is present:
%           latex infwarerr.drv
%    (b) Without infwarerr.drv:
%           latex infwarerr.dtx; ...
%    The class ltxdoc loads the configuration file ltxdoc.cfg
%    if available. Here you can specify further options, e.g.
%    use A4 as paper format:
%       \PassOptionsToClass{a4paper}{article}
% 
%    Programm calls to get the documentation (example):
%       pdflatex infwarerr.dtx
%       makeindex -s gind.ist infwarerr.idx
%       pdflatex infwarerr.dtx
%       makeindex -s gind.ist infwarerr.idx
%       pdflatex infwarerr.dtx
% 
% Installation:
%    TDS:tex/generic/oberdiek/infwarerr.sty
%    TDS:doc/latex/oberdiek/infwarerr.pdf
%    TDS:doc/latex/oberdiek/test/infwarerr-test1.tex
%    TDS:doc/latex/oberdiek/test/infwarerr-test2.tex
%    TDS:doc/latex/oberdiek/test/infwarerr-test3.tex
%    TDS:source/latex/oberdiek/infwarerr.dtx
%
%<*ignore>
\begingroup
  \catcode123=1 %
  \catcode125=2 %
  \def\x{LaTeX2e}%
\expandafter\endgroup
\ifcase 0\ifx\install y1\fi\expandafter
         \ifx\csname processbatchFile\endcsname\relax\else1\fi
         \ifx\fmtname\x\else 1\fi\relax
\else\csname fi\endcsname
%</ignore>
%<*install>
\input docstrip.tex
\Msg{************************************************************************}
\Msg{* Installation}
\Msg{* Package: infwarerr 2010/04/08 v1.3 Providing info/warning/message (HO)}
\Msg{************************************************************************}

\keepsilent
\askforoverwritefalse

\let\MetaPrefix\relax
\preamble

This is a generated file.

Project: infwarerr
Version: 2010/04/08 v1.3

Copyright (C) 2007, 2010 by
   Heiko Oberdiek <heiko.oberdiek at googlemail.com>

This work may be distributed and/or modified under the
conditions of the LaTeX Project Public License, either
version 1.3c of this license or (at your option) any later
version. This version of this license is in
   http://www.latex-project.org/lppl/lppl-1-3c.txt
and the latest version of this license is in
   http://www.latex-project.org/lppl.txt
and version 1.3 or later is part of all distributions of
LaTeX version 2005/12/01 or later.

This work has the LPPL maintenance status "maintained".

This Current Maintainer of this work is Heiko Oberdiek.

The Base Interpreter refers to any `TeX-Format',
because some files are installed in TDS:tex/generic//.

This work consists of the main source file infwarerr.dtx
and the derived files
   infwarerr.sty, infwarerr.pdf, infwarerr.ins, infwarerr.drv,
   infwarerr-test1.tex, infwarerr-test2.tex,
   infwarerr-test3.tex.

\endpreamble
\let\MetaPrefix\DoubleperCent

\generate{%
  \file{infwarerr.ins}{\from{infwarerr.dtx}{install}}%
  \file{infwarerr.drv}{\from{infwarerr.dtx}{driver}}%
  \usedir{tex/generic/oberdiek}%
  \file{infwarerr.sty}{\from{infwarerr.dtx}{package}}%
  \usedir{doc/latex/oberdiek/test}%
  \file{infwarerr-test1.tex}{\from{infwarerr.dtx}{test1}}%
  \file{infwarerr-test2.tex}{\from{infwarerr.dtx}{test2}}%
  \file{infwarerr-test3.tex}{\from{infwarerr.dtx}{test3}}%
}

\catcode32=13\relax% active space
\let =\space%
\Msg{************************************************************************}
\Msg{*}
\Msg{* To finish the installation you have to move the following}
\Msg{* file into a directory searched by TeX:}
\Msg{*}
\Msg{*     infwarerr.sty}
\Msg{*}
\Msg{* To produce the documentation run the file `infwarerr.drv'}
\Msg{* through LaTeX.}
\Msg{*}
\Msg{* Happy TeXing!}
\Msg{*}
\Msg{************************************************************************}

\endbatchfile
%</install>
%<*ignore>
\fi
%</ignore>
%<*driver>
\NeedsTeXFormat{LaTeX2e}
\ProvidesFile{infwarerr.drv}%
  [2010/04/08 v1.3 Providing info/warning/message (HO)]%
\documentclass{ltxdoc}
\usepackage{holtxdoc}[2011/02/04]
\begin{document}
  \DocInput{infwarerr.dtx}%
\end{document}
%</driver>
% \fi
%
% \CheckSum{871}
%
% \CharacterTable
%  {Upper-case    \A\B\C\D\E\F\G\H\I\J\K\L\M\N\O\P\Q\R\S\T\U\V\W\X\Y\Z
%   Lower-case    \a\b\c\d\e\f\g\h\i\j\k\l\m\n\o\p\q\r\s\t\u\v\w\x\y\z
%   Digits        \0\1\2\3\4\5\6\7\8\9
%   Exclamation   \!     Double quote  \"     Hash (number) \#
%   Dollar        \$     Percent       \%     Ampersand     \&
%   Acute accent  \'     Left paren    \(     Right paren   \)
%   Asterisk      \*     Plus          \+     Comma         \,
%   Minus         \-     Point         \.     Solidus       \/
%   Colon         \:     Semicolon     \;     Less than     \<
%   Equals        \=     Greater than  \>     Question mark \?
%   Commercial at \@     Left bracket  \[     Backslash     \\
%   Right bracket \]     Circumflex    \^     Underscore    \_
%   Grave accent  \`     Left brace    \{     Vertical bar  \|
%   Right brace   \}     Tilde         \~}
%
% \GetFileInfo{infwarerr.drv}
%
% \title{The \xpackage{infwarerr} package}
% \date{2010/04/08 v1.3}
% \author{Heiko Oberdiek\\\xemail{heiko.oberdiek at googlemail.com}}
%
% \maketitle
%
% \begin{abstract}
% This package provides a complete set of macros for informations,
% warnings and error messages with support for \plainTeX.
% \end{abstract}
%
% \tableofcontents
%
% \section{Documentation}
%
% Package \xpackage{infwarerr} defines the following set of commands,
% given as pseudo regular expression:
%   \begin{quote}
%     \verb+\@(Package|Class)(Info|InfoNoLine|Warning|WarningNoLine|Error)+^^A
%     \kern-\rightmargin
%   \end{quote}
% The first argument is the package or class in question. The second
% argument contains the message text. \cs{MessageBreak} is supported.
% In case of error messages, the third argument contains the help
% text. Supported are \LaTeX's message texts \cs{@ehc}:
%   \begin{quote}
%     |Try typing  <return>  to proceed|\\
%     |If that doesn't work, type  X <return>  to quit.|
%   \end{quote}
% and \cs{@ehd}:
%   \begin{quote}
%     |You're in trouble here.  Try typing  <return>  to proceed|\\
%     |If that doesn't work, type  X <return>  to quit.|
%   \end{quote}
%
% The package is intended for macro writers that want to write
% code that works for both \LaTeX\ and \plainTeX. If \LaTeX\ is
% detected, then the macros of this package will only be wrappers
% for \LaTeX's macros. Otherwise \LaTeX's macros will be mimicked
% for use with \plainTeX.
%
% \StopEventually{
% }
%
% \section{Implementation}
%
%    \begin{macrocode}
%<*package>
%    \end{macrocode}
%
% \subsection{Reload check and package identification}
%    Reload check, especially if the package is not used with \LaTeX.
%    \begin{macrocode}
\begingroup\catcode61\catcode48\catcode32=10\relax%
  \catcode13=5 % ^^M
  \endlinechar=13 %
  \catcode35=6 % #
  \catcode39=12 % '
  \catcode44=12 % ,
  \catcode45=12 % -
  \catcode46=12 % .
  \catcode58=12 % :
  \catcode64=11 % @
  \catcode123=1 % {
  \catcode125=2 % }
  \expandafter\let\expandafter\x\csname ver@infwarerr.sty\endcsname
  \ifx\x\relax % plain-TeX, first loading
  \else
    \def\empty{}%
    \ifx\x\empty % LaTeX, first loading,
      % variable is initialized, but \ProvidesPackage not yet seen
    \else
      \expandafter\ifx\csname PackageInfo\endcsname\relax
        \def\x#1#2{%
          \immediate\write-1{Package #1 Info: #2.}%
        }%
      \else
        \def\x#1#2{\PackageInfo{#1}{#2, stopped}}%
      \fi
      \x{infwarerr}{The package is already loaded}%
      \aftergroup\endinput
    \fi
  \fi
\endgroup%
%    \end{macrocode}
%    Package identification:
%    \begin{macrocode}
\begingroup\catcode61\catcode48\catcode32=10\relax%
  \catcode13=5 % ^^M
  \endlinechar=13 %
  \catcode35=6 % #
  \catcode39=12 % '
  \catcode40=12 % (
  \catcode41=12 % )
  \catcode44=12 % ,
  \catcode45=12 % -
  \catcode46=12 % .
  \catcode47=12 % /
  \catcode58=12 % :
  \catcode64=11 % @
  \catcode91=12 % [
  \catcode93=12 % ]
  \catcode123=1 % {
  \catcode125=2 % }
  \expandafter\ifx\csname ProvidesPackage\endcsname\relax
    \def\x#1#2#3[#4]{\endgroup
      \immediate\write-1{Package: #3 #4}%
      \xdef#1{#4}%
    }%
  \else
    \def\x#1#2[#3]{\endgroup
      #2[{#3}]%
      \ifx#1\@undefined
        \xdef#1{#3}%
      \fi
      \ifx#1\relax
        \xdef#1{#3}%
      \fi
    }%
  \fi
\expandafter\x\csname ver@infwarerr.sty\endcsname
\ProvidesPackage{infwarerr}%
  [2010/04/08 v1.3 Providing info/warning/message (HO)]%
%    \end{macrocode}
%
% \subsection{Catcodes}
%
%    \begin{macrocode}
\begingroup\catcode61\catcode48\catcode32=10\relax%
  \catcode13=5 % ^^M
  \endlinechar=13 %
  \catcode123=1 % {
  \catcode125=2 % }
  \catcode64=11 % @
  \def\x{\endgroup
    \expandafter\edef\csname InfWarErr@AtEnd\endcsname{%
      \endlinechar=\the\endlinechar\relax
      \catcode13=\the\catcode13\relax
      \catcode32=\the\catcode32\relax
      \catcode35=\the\catcode35\relax
      \catcode61=\the\catcode61\relax
      \catcode64=\the\catcode64\relax
      \catcode123=\the\catcode123\relax
      \catcode125=\the\catcode125\relax
    }%
  }%
\x\catcode61\catcode48\catcode32=10\relax%
\catcode13=5 % ^^M
\endlinechar=13 %
\catcode35=6 % #
\catcode64=11 % @
\catcode123=1 % {
\catcode125=2 % }
\def\TMP@EnsureCode#1#2{%
  \edef\InfWarErr@AtEnd{%
    \InfWarErr@AtEnd
    \catcode#1=\the\catcode#1\relax
  }%
  \catcode#1=#2\relax
}
\TMP@EnsureCode{10}{12}% ^^J
\TMP@EnsureCode{39}{12}% '
\TMP@EnsureCode{40}{12}% (
\TMP@EnsureCode{41}{12}% )
\TMP@EnsureCode{44}{12}% ,
\TMP@EnsureCode{45}{12}% -
\TMP@EnsureCode{46}{12}% .
\TMP@EnsureCode{58}{12}% :
\TMP@EnsureCode{60}{12}% <
\TMP@EnsureCode{62}{12}% >
\TMP@EnsureCode{94}{7}% ^
\edef\InfWarErr@AtEnd{\InfWarErr@AtEnd\noexpand\endinput}
%    \end{macrocode}
%
% \subsection{\LaTeX\ detection}
%
%    \begin{macrocode}
\begingroup
  \let\x=Y%
  \def\check#1{%
    \@check{Generic}{#1}%
    \@check{Package}{#1}%
    \@check{Class}{#1}%
  }%
  \def\@check#1#2{%
    \expandafter\ifx\csname #1#2\endcsname\relax
      \let\x=N%
    \fi
  }%
  \check{Info}%
  \check{Warning}%
  \check{Error}%
  \@check{Package}{WarningNoLine}%
  \@check{Class}{WarningNoLine}%
  \@check{}{@ehc}%
  \@check{}{@ehd}%
\expandafter\endgroup
\ifx\x Y%
%    \end{macrocode}
%
% \subsection{Version for \LaTeX}
%
%    \begin{macro}{\@PackageInfo}
%    \begin{macrocode}
  \let\@PackageInfo\PackageInfo
%    \end{macrocode}
%    \end{macro}
%    \begin{macro}{\@PackageInfoNoLine}
%    \begin{macrocode}
  \def\@PackageInfoNoLine#1#2{%
    \PackageInfo{#1}{#2\@gobble}%
  }%
%    \end{macrocode}
%    \end{macro}
%    \begin{macro}{\@ClassInfo}
%    \begin{macrocode}
  \let\@ClassInfo\ClassInfo
%    \end{macrocode}
%    \end{macro}
%    \begin{macro}{\@ClassInfoNoLine}
%    \begin{macrocode}
  \def\@ClassInfoNoLine#1#2{%
    \ClassInfo{#1}{#2\@gobble}%
  }%
%    \end{macrocode}
%    \end{macro}
%    \begin{macro}{\@PackageWarning}
%    \begin{macrocode}
  \let\@PackageWarning\PackageWarning
%    \end{macrocode}
%    \end{macro}
%    \begin{macro}{\@PackageWarningNoLine}
%    \begin{macrocode}
  \let\@PackageWarningNoLine\PackageWarningNoLine
%    \end{macrocode}
%    \end{macro}
%    \begin{macro}{\@ClassWarning}
%    \begin{macrocode}
  \let\@ClassWarning\ClassWarning
%    \end{macrocode}
%    \end{macro}
%    \begin{macro}{\@ClassWarningNoLine}
%    \begin{macrocode}
  \let\@ClassWarningNoLine\ClassWarningNoLine
%    \end{macrocode}
%    \end{macro}
%    \begin{macro}{\@PackageError}
%    \begin{macrocode}
  \let\@PackageError\PackageError
%    \end{macrocode}
%    \end{macro}
%    \begin{macro}{\@ClassError}
%    \begin{macrocode}
  \let\@ClassError\ClassError
%    \end{macrocode}
%    \end{macro}
%    \begin{macrocode}
  \expandafter\InfWarErr@AtEnd
\fi%
%    \end{macrocode}
%
% \subsection{Version for \plainTeX}
%
%    \begin{macro}{\InfWarErr@online}
%    \begin{macrocode}
\begingroup\expandafter\endgroup
\ifcase\expandafter\ifx\csname inputlineno\endcsname\relax
         0 %
       \else
         \ifnum\inputlineno<0 %
           0 %
         \else
           1 %
         \fi
       \fi
  \def\InfWarErr@online{}%
\else
  \def\InfWarErr@online{ on input line \the\inputlineno}%
\fi
%    \end{macrocode}
%    \end{macro}
%
%    \begin{macro}{\InfWarErr@protected}
%    \begin{macrocode}
\begingroup\expandafter\expandafter\expandafter\endgroup
\expandafter\ifx\csname protected\endcsname\relax
  \def\InfWarErr@protected{}%
\else
  \let\InfWarErr@protected\protected
\fi
%    \end{macrocode}
%    \end{macro}
%
%    \begin{macro}{\InfWarErr@unused}
%    \begin{macrocode}
\chardef\InfWarErr@unused=16 %
%    \end{macrocode}
%    \end{macro}
%
%    \begin{macro}{\InfWarErr@ehc}
%    \begin{macrocode}
\def\InfWarErr@ehc{%
  Try typing \space <return> \space to proceed.\MessageBreak
  If that doesn't work, type \space X <return> \space to quit.%
}
%    \end{macrocode}
%    \end{macro}
%    \begin{macro}{\InfWarErr@ehd}
%    \begin{macrocode}
\def\InfWarErr@ehd{%
  You're in trouble here. \space\@ehc
}
%    \end{macrocode}
%    \end{macro}
%
%    \begin{macro}{\InfWarErr@GenericInfo}
%    \begin{macrocode}
\InfWarErr@protected\def\InfWarErr@GenericInfo#1#2{%
  \begingroup
    \let\on@line\InfWarErr@online
    \long\def\@gobble##1{}%
    \def\space{ }%
    \def\@spaces{\space\space\space\space}%
    \def\MessageBreak{^^J#1}%
    \newlinechar=10 %
    \escapechar=92 %
    \immediate\write-1{#2\on@line.}%
  \endgroup
}
%    \end{macrocode}
%    \end{macro}
%    \begin{macro}{\InfWarErr@GenericWarning}
%    \begin{macrocode}
\InfWarErr@protected\def\InfWarErr@GenericWarning#1#2{%
  \begingroup
    \let\on@line\InfWarErr@online
    \long\def\@gobble##1{}%
    \def\space{ }%
    \def\@spaces{\space\space\space\space}%
    \def\MessageBreak{^^J#1}%
    \newlinechar=10 %
    \escapechar=92 %
    \immediate\write\InfWarErr@unused{^^J#2\on@line.^^J}%
  \endgroup
}
%    \end{macrocode}
%    \end{macro}
%    \begin{macro}{\InfWarErr@GenericError}
%    Old \TeX\ versions below 3.141 where |^^J| is not working
%    as newline in error message and help texts are not supported.
%    \begin{macrocode}
\InfWarErr@protected\def\InfWarErr@GenericError#1#2#3#4{%
  \begingroup
    \immediate\write\InfWarErr@unused{}%
    \def\space{ }%
    \def\@spaces{\space\space\space\space}%
    \let\@ehc\InfWarErr@ehc
    \let\@ehd\InfWarErr@ehd
    \let\on@line\InfWarErr@online
    \def\MessageBreak{^^J}%
    \newlinechar=10 %
    \escapechar=92 %
    \edef\InfWarErr@help{\noexpand\errhelp{#4}}%
    \InfWarErr@help
    \def\MessageBreak{^^J#1}%
    \errmessage{%
      #2.^^J^^J%
      #3^^J%
      Type \space H <return> \space for immediate help.^^J ...^^J%
    }%
  \endgroup
}
%    \end{macrocode}
%    \end{macro}
%
%    \begin{macro}{\@PackageInfo}
%    \begin{macrocode}
\def\@PackageInfo#1#2{%
  \InfWarErr@GenericInfo{%
    (#1) \@spaces\@spaces\@spaces
  }{%
    Package #1 Info: #2%
  }%
}
%    \end{macrocode}
%    \end{macro}
%    \begin{macro}{\@PackageInfoNoLine}
%    \begin{macrocode}
\def\@PackageInfoNoLine#1#2{%
  \@PackageInfo{#1}{#2\@gobble}%
}
%    \end{macrocode}
%    \end{macro}
%    \begin{macro}{\@ClassInfo}
%    \begin{macrocode}
\def\@ClassInfo#1#2{%
  \InfWarErr@GenericInfo{%
    (#1) \space\space\@spaces\@spaces
  }{%
    Class #1 Info: #2%
  }%
}
%    \end{macrocode}
%    \end{macro}
%    \begin{macro}{\@ClassInfoNoLine}
%    \begin{macrocode}
\def\@ClassInfoNoLine#1#2{%
  \@ClassInfo{#1}{#2\@gobble}%
}
%    \end{macrocode}
%    \end{macro}
%    \begin{macro}{\@PackageWarning}
%    \begin{macrocode}
\def\@PackageWarning#1#2{%
  \InfWarErr@GenericWarning{%
    (#1)\@spaces\@spaces\@spaces\@spaces
  }{%
    Package #1 Warning: #2%
  }%
}
%    \end{macrocode}
%    \end{macro}
%    \begin{macro}{\@PackageWarningNoLine}
%    \begin{macrocode}
\def\@PackageWarningNoLine#1#2{%
  \@PackageWarning{#1}{#2\@gobble}%
}
%    \end{macrocode}
%    \end{macro}
%    \begin{macro}{\@ClassWarning}
%    \begin{macrocode}
\def\@ClassWarning#1#2{%
  \InfWarErr@GenericWarning{%
    (#1) \space\@spaces\@spaces\@spaces
  }{%
    Class #1 Warning: #2%
  }%
}
%    \end{macrocode}
%    \end{macro}
%    \begin{macro}{\@ClassWarningNoLine}
%    \begin{macrocode}
\def\@ClassWarningNoLine#1#2{%
  \@ClassWarning{#1}{#2\@gobble}%
}
%    \end{macrocode}
%    \end{macro}
%    \begin{macro}{\@PackageError}
%    \begin{macrocode}
\def\@PackageError#1#2#3{%
  \InfWarErr@GenericError{%
    (#1)\@spaces\@spaces\@spaces\@spaces
  }{%
    Package #1 Error: #2%
  }{%
    See the #1 package documentation for explanation.%
  }{#3}%
}
%    \end{macrocode}
%    \end{macro}
%    \begin{macro}{\@ClassError}
%    \begin{macrocode}
\def\@ClassError#1#2#3{%
  \InfWarErr@GenericError{%
    (#1) \space\@spaces\@spaces\@spaces
  }{%
    Class #1 Error: #2%
  }{%
    See the #1 class documentation for explanation.%
  }{#3}%
}
%    \end{macrocode}
%    \end{macro}
%
%    \begin{macrocode}
\InfWarErr@AtEnd%
%</package>
%    \end{macrocode}
%
% \section{Test}
%
% \subsection{Catcode checks for loading}
%
%    \begin{macrocode}
%<*test1>
%    \end{macrocode}
%    \begin{macrocode}
\catcode`\{=1 %
\catcode`\}=2 %
\catcode`\#=6 %
\catcode`\@=11 %
\expandafter\ifx\csname count@\endcsname\relax
  \countdef\count@=255 %
\fi
\expandafter\ifx\csname @gobble\endcsname\relax
  \long\def\@gobble#1{}%
\fi
\expandafter\ifx\csname @firstofone\endcsname\relax
  \long\def\@firstofone#1{#1}%
\fi
\expandafter\ifx\csname loop\endcsname\relax
  \expandafter\@firstofone
\else
  \expandafter\@gobble
\fi
{%
  \def\loop#1\repeat{%
    \def\body{#1}%
    \iterate
  }%
  \def\iterate{%
    \body
      \let\next\iterate
    \else
      \let\next\relax
    \fi
    \next
  }%
  \let\repeat=\fi
}%
\def\RestoreCatcodes{}
\count@=0 %
\loop
  \edef\RestoreCatcodes{%
    \RestoreCatcodes
    \catcode\the\count@=\the\catcode\count@\relax
  }%
\ifnum\count@<255 %
  \advance\count@ 1 %
\repeat

\def\RangeCatcodeInvalid#1#2{%
  \count@=#1\relax
  \loop
    \catcode\count@=15 %
  \ifnum\count@<#2\relax
    \advance\count@ 1 %
  \repeat
}
\def\RangeCatcodeCheck#1#2#3{%
  \count@=#1\relax
  \loop
    \ifnum#3=\catcode\count@
    \else
      \errmessage{%
        Character \the\count@\space
        with wrong catcode \the\catcode\count@\space
        instead of \number#3%
      }%
    \fi
  \ifnum\count@<#2\relax
    \advance\count@ 1 %
  \repeat
}
\def\space{ }
\expandafter\ifx\csname LoadCommand\endcsname\relax
  \def\LoadCommand{\input infwarerr.sty\relax}%
\fi
\def\Test{%
  \RangeCatcodeInvalid{0}{47}%
  \RangeCatcodeInvalid{58}{64}%
  \RangeCatcodeInvalid{91}{96}%
  \RangeCatcodeInvalid{123}{255}%
  \catcode`\@=12 %
  \catcode`\\=0 %
  \catcode`\%=14 %
  \LoadCommand
  \RangeCatcodeCheck{0}{36}{15}%
  \RangeCatcodeCheck{37}{37}{14}%
  \RangeCatcodeCheck{38}{47}{15}%
  \RangeCatcodeCheck{48}{57}{12}%
  \RangeCatcodeCheck{58}{63}{15}%
  \RangeCatcodeCheck{64}{64}{12}%
  \RangeCatcodeCheck{65}{90}{11}%
  \RangeCatcodeCheck{91}{91}{15}%
  \RangeCatcodeCheck{92}{92}{0}%
  \RangeCatcodeCheck{93}{96}{15}%
  \RangeCatcodeCheck{97}{122}{11}%
  \RangeCatcodeCheck{123}{255}{15}%
  \RestoreCatcodes
}
\Test
\csname @@end\endcsname
\end
%    \end{macrocode}
%    \begin{macrocode}
%</test1>
%    \end{macrocode}
%
% \subsection{Macro tests}
%
%    \begin{macrocode}
%<*test2>
\let\PackageInfo\relax
\input infwarerr.sty\relax
\let\OrgWrite\write
\chardef\WriteUnused=16 %
\def\msg#{\immediate\OrgWrite\WriteUnused}
\msg{File: infwarerr-test2.tex 2010/04/08 v1.3 Test file for plain-TeX}
\catcode64=11 %
%</test2>
%    \end{macrocode}
%
%    \begin{macrocode}
%<*test3>
\NeedsTeXFormat{LaTeX2e}
\ProvidesFile{infwarerr-test3.tex}[2010/04/08 v1.3 Test file for LaTeX]
\RequirePackage{infwarerr}[2010/04/08]
\makeatletter
\let\OrgWrite\write
\chardef\WriteUnused=\@unused
\def\msg#{\immediate\OrgWrite\WriteUnused}
%</test3>
%    \end{macrocode}
%
%    \begin{macrocode}
%<*test2|test3>
\def\pkgname{PACKAGE}
\def\clsname{CLASS}
\def\msgtext{MESSAGE\MessageBreak SECOND LINE}
\def\hlptext{HELP\MessageBreak SECOND LINE}

\let\OrgWrite\write
\let\OrgErrMessage\errmessage

\def\write#1#{%
  \@write{#1}%
}
\def\@write#1#2{%
  \OrgWrite#1{#2}%
  \xdef\WriteNum{\number#1}%
  \xdef\WriteText{#2}%
}
\def\TestFailed#1#2{%
  \OrgErrMessage{Test (\string#1) failed: #2}%
}
\def\TestWrite#1#2#3{%
  \ifnum\WriteNum=#2\relax
  \else
    \TestFailed#1{write stream number: \WriteNum <> #2}%
  \fi
  \begingroup
    \let~\space
    \edef\TestString{#3}%
    \ifx\WriteText\TestString
    \else
      \msg{}%
      \msg{[\WriteText]}%
      \msg{<>}%
      \msg{[#3]}% hash-ok
      \TestFailed#1{write text}%
    \fi
  \endgroup
}
\newtoks\ErrHelpToks
\def\errmessage#1{%
  \global\ErrHelpToks=\expandafter{\the\errhelp}%
  \msg{**************** error message ****************}%
  \msg{! #1}%
  \msg{**************** help message *****************}%
  \msg{\the\errhelp}%
  \msg{***********************************************}%
  \xdef\ErrMessage{! #1}%
}
\def\TestInfo{%
  \@PackageInfo\pkgname\msgtext
  \TestWrite\@PackageInfo{-1}{%
    Package PACKAGE Info: MESSAGE^^J%
    (PACKAGE)~~~~~~~~~~~~~SECOND LINE on input line \the\inputlineno.%
  }%
  \@PackageInfoNoLine\pkgname\msgtext
  \TestWrite\@PackageInfoNoLine{-1}{%
    Package PACKAGE Info: MESSAGE^^J%
    (PACKAGE)~~~~~~~~~~~~~SECOND LINE.%
  }%
  \@ClassInfo\clsname\msgtext
  \TestWrite\@ClassInfo{-1}{%
    Class CLASS Info: MESSAGE^^J%
    (CLASS)~~~~~~~~~~~SECOND LINE on input line \the\inputlineno.%
  }%
  \@ClassInfoNoLine\clsname\msgtext
  \TestWrite\@ClassInfoNoLine{-1}{%
    Class CLASS Info: MESSAGE^^J%
    (CLASS)~~~~~~~~~~~SECOND LINE.%
  }%
}
\def\TestWarning{%
  \@PackageWarning\pkgname\msgtext
  \TestWrite\@PackageWarning\WriteUnused{%
    ^^J%
    Package PACKAGE Warning: MESSAGE^^J%
    (PACKAGE)~~~~~~~~~~~~~~~~SECOND LINE on input line \the\inputlineno.%
    ^^J%
  }%
  \@PackageWarningNoLine\pkgname\msgtext
  \TestWrite\@PackageWarningNoLine\WriteUnused{%
    ^^J%
    Package PACKAGE Warning: MESSAGE^^J%
    (PACKAGE)~~~~~~~~~~~~~~~~SECOND LINE.%
    ^^J%
  }%
  \@ClassWarning\clsname\msgtext
  \TestWrite\@ClassWarning\WriteUnused{%
    ^^J%
    Class CLASS Warning: MESSAGE^^J%
    (CLASS)~~~~~~~~~~~~~~SECOND LINE on input line \the\inputlineno.%
    ^^J%
  }%
  \@ClassWarningNoLine\clsname\msgtext
  \TestWrite\@ClassWarningNoLine\WriteUnused{%
    ^^J%
    Class CLASS Warning: MESSAGE^^J%
    (CLASS)~~~~~~~~~~~~~~SECOND LINE.%
    ^^J%
  }%
}
\def\TestError{%
%<test3>  \let\write\OrgWrite
%<test3>  \def\plaindots{}
%<test2>  \def\plaindots{.^^J ...^^J}
  \@PackageError\pkgname\msgtext\hlptext
  \@TestError\@PackageError{%
    ! Package PACKAGE Error: MESSAGE^^J%
    (PACKAGE)~~~~~~~~~~~~~~~~SECOND LINE.^^J%
    ^^J%
    See the PACKAGE package documentation for explanation.^^J%
    Type~~H <return>~~for immediate help\plaindots
  }{%
    HELP^^J%
    SECOND LINE%
  }%
  \@ClassError\clsname\msgtext\hlptext
  \@TestError\@ClassError{%
    ! Class CLASS Error: MESSAGE^^J%
    (CLASS)~~~~~~~~~~~~~~SECOND LINE.^^J%
    ^^J%
    See the CLASS class documentation for explanation.^^J%
    Type~~H <return>~~for immediate help\plaindots
  }{%
    HELP^^J%
    SECOND LINE%
  }%
  \@PackageError\pkgname\msgtext\@ehc
  \@TestError\@PackageError{%
    ! Package PACKAGE Error: MESSAGE^^J%
    (PACKAGE)~~~~~~~~~~~~~~~~SECOND LINE.^^J%
    ^^J%
    See the PACKAGE package documentation for explanation.^^J%
    Type~~H <return>~~for immediate help\plaindots
  }{%
    Try typing~~<return>~~to proceed.^^J%
    If that doesn't work, type~~X <return>~~to quit.%
  }%
  \@PackageError\pkgname\msgtext\@ehd
  \@TestError\@PackageError{%
    ! Package PACKAGE Error: MESSAGE^^J%
    (PACKAGE)~~~~~~~~~~~~~~~~SECOND LINE.^^J%
    ^^J%
    See the PACKAGE package documentation for explanation.^^J%
    Type~~H <return>~~for immediate help\plaindots
  }{%
    You're in trouble here.~~Try typing~~<return>~~to proceed.^^J%
    If that doesn't work, type~~X <return>~~to quit.%
  }%
}
\def\@TestError#1#2#3{%
  \begingroup
    \let~\space
    \edef\x{#2}%
%<test3>  \@onelevel@sanitize\x
%<test3>  \@onelevel@sanitize\ErrMessage
    \ifx\x\ErrMessage
    \else
      \msg{}%
      \msg{[\ErrMessage]}%
      \msg{<>}%
      \msg{[\x]}%
      \TestFailed#1{errmessage text}%
    \fi
    \edef\x{\the\ErrHelpToks}%
    \edef\y{#3}%
    \ifx\x\y
    \else
      \msg{}%
      \msg{[\the\ErrHelpToks]}%
      \msg{<>}%
      \msg{[\y]}%
      \TestFailed#1{errhelp text}%
    \fi
  \endgroup
}
\TestInfo
\TestWarning
\TestError
%</test2|test3>
%<test2>\end
%<test3>\@@end
%    \end{macrocode}
%
%
% \section{Installation}
%
% \subsection{Download}
%
% \paragraph{Package.} This package is available on
% CTAN\footnote{\url{ftp://ftp.ctan.org/tex-archive/}}:
% \begin{description}
% \item[\CTAN{macros/latex/contrib/oberdiek/infwarerr.dtx}] The source file.
% \item[\CTAN{macros/latex/contrib/oberdiek/infwarerr.pdf}] Documentation.
% \end{description}
%
%
% \paragraph{Bundle.} All the packages of the bundle `oberdiek'
% are also available in a TDS compliant ZIP archive. There
% the packages are already unpacked and the documentation files
% are generated. The files and directories obey the TDS standard.
% \begin{description}
% \item[\CTAN{install/macros/latex/contrib/oberdiek.tds.zip}]
% \end{description}
% \emph{TDS} refers to the standard ``A Directory Structure
% for \TeX\ Files'' (\CTAN{tds/tds.pdf}). Directories
% with \xfile{texmf} in their name are usually organized this way.
%
% \subsection{Bundle installation}
%
% \paragraph{Unpacking.} Unpack the \xfile{oberdiek.tds.zip} in the
% TDS tree (also known as \xfile{texmf} tree) of your choice.
% Example (linux):
% \begin{quote}
%   |unzip oberdiek.tds.zip -d ~/texmf|
% \end{quote}
%
% \paragraph{Script installation.}
% Check the directory \xfile{TDS:scripts/oberdiek/} for
% scripts that need further installation steps.
% Package \xpackage{attachfile2} comes with the Perl script
% \xfile{pdfatfi.pl} that should be installed in such a way
% that it can be called as \texttt{pdfatfi}.
% Example (linux):
% \begin{quote}
%   |chmod +x scripts/oberdiek/pdfatfi.pl|\\
%   |cp scripts/oberdiek/pdfatfi.pl /usr/local/bin/|
% \end{quote}
%
% \subsection{Package installation}
%
% \paragraph{Unpacking.} The \xfile{.dtx} file is a self-extracting
% \docstrip\ archive. The files are extracted by running the
% \xfile{.dtx} through \plainTeX:
% \begin{quote}
%   \verb|tex infwarerr.dtx|
% \end{quote}
%
% \paragraph{TDS.} Now the different files must be moved into
% the different directories in your installation TDS tree
% (also known as \xfile{texmf} tree):
% \begin{quote}
% \def\t{^^A
% \begin{tabular}{@{}>{\ttfamily}l@{ $\rightarrow$ }>{\ttfamily}l@{}}
%   infwarerr.sty & tex/generic/oberdiek/infwarerr.sty\\
%   infwarerr.pdf & doc/latex/oberdiek/infwarerr.pdf\\
%   test/infwarerr-test1.tex & doc/latex/oberdiek/test/infwarerr-test1.tex\\
%   test/infwarerr-test2.tex & doc/latex/oberdiek/test/infwarerr-test2.tex\\
%   test/infwarerr-test3.tex & doc/latex/oberdiek/test/infwarerr-test3.tex\\
%   infwarerr.dtx & source/latex/oberdiek/infwarerr.dtx\\
% \end{tabular}^^A
% }^^A
% \sbox0{\t}^^A
% \ifdim\wd0>\linewidth
%   \begingroup
%     \advance\linewidth by\leftmargin
%     \advance\linewidth by\rightmargin
%   \edef\x{\endgroup
%     \def\noexpand\lw{\the\linewidth}^^A
%   }\x
%   \def\lwbox{^^A
%     \leavevmode
%     \hbox to \linewidth{^^A
%       \kern-\leftmargin\relax
%       \hss
%       \usebox0
%       \hss
%       \kern-\rightmargin\relax
%     }^^A
%   }^^A
%   \ifdim\wd0>\lw
%     \sbox0{\small\t}^^A
%     \ifdim\wd0>\linewidth
%       \ifdim\wd0>\lw
%         \sbox0{\footnotesize\t}^^A
%         \ifdim\wd0>\linewidth
%           \ifdim\wd0>\lw
%             \sbox0{\scriptsize\t}^^A
%             \ifdim\wd0>\linewidth
%               \ifdim\wd0>\lw
%                 \sbox0{\tiny\t}^^A
%                 \ifdim\wd0>\linewidth
%                   \lwbox
%                 \else
%                   \usebox0
%                 \fi
%               \else
%                 \lwbox
%               \fi
%             \else
%               \usebox0
%             \fi
%           \else
%             \lwbox
%           \fi
%         \else
%           \usebox0
%         \fi
%       \else
%         \lwbox
%       \fi
%     \else
%       \usebox0
%     \fi
%   \else
%     \lwbox
%   \fi
% \else
%   \usebox0
% \fi
% \end{quote}
% If you have a \xfile{docstrip.cfg} that configures and enables \docstrip's
% TDS installing feature, then some files can already be in the right
% place, see the documentation of \docstrip.
%
% \subsection{Refresh file name databases}
%
% If your \TeX~distribution
% (\teTeX, \mikTeX, \dots) relies on file name databases, you must refresh
% these. For example, \teTeX\ users run \verb|texhash| or
% \verb|mktexlsr|.
%
% \subsection{Some details for the interested}
%
% \paragraph{Attached source.}
%
% The PDF documentation on CTAN also includes the
% \xfile{.dtx} source file. It can be extracted by
% AcrobatReader 6 or higher. Another option is \textsf{pdftk},
% e.g. unpack the file into the current directory:
% \begin{quote}
%   \verb|pdftk infwarerr.pdf unpack_files output .|
% \end{quote}
%
% \paragraph{Unpacking with \LaTeX.}
% The \xfile{.dtx} chooses its action depending on the format:
% \begin{description}
% \item[\plainTeX:] Run \docstrip\ and extract the files.
% \item[\LaTeX:] Generate the documentation.
% \end{description}
% If you insist on using \LaTeX\ for \docstrip\ (really,
% \docstrip\ does not need \LaTeX), then inform the autodetect routine
% about your intention:
% \begin{quote}
%   \verb|latex \let\install=y\input{infwarerr.dtx}|
% \end{quote}
% Do not forget to quote the argument according to the demands
% of your shell.
%
% \paragraph{Generating the documentation.}
% You can use both the \xfile{.dtx} or the \xfile{.drv} to generate
% the documentation. The process can be configured by the
% configuration file \xfile{ltxdoc.cfg}. For instance, put this
% line into this file, if you want to have A4 as paper format:
% \begin{quote}
%   \verb|\PassOptionsToClass{a4paper}{article}|
% \end{quote}
% An example follows how to generate the
% documentation with pdf\LaTeX:
% \begin{quote}
%\begin{verbatim}
%pdflatex infwarerr.dtx
%makeindex -s gind.ist infwarerr.idx
%pdflatex infwarerr.dtx
%makeindex -s gind.ist infwarerr.idx
%pdflatex infwarerr.dtx
%\end{verbatim}
% \end{quote}
%
% \begin{History}
%   \begin{Version}{2007/04/26 v1.0}
%   \item
%     First version.
%   \end{Version}
%   \begin{Version}{2007/06/14 v1.1}
%   \item
%     Small fixes.
%   \end{Version}
%   \begin{Version}{2007/09/09 v1.2}
%   \item
%     Catcode section extended.
%   \end{Version}
%   \begin{Version}{2010/04/08 v1.3}
%   \item
%     Ensure \cs{escapechar}=92 (backslash) for infos/warnings/errors.
%   \end{Version}
% \end{History}
%
% \PrintIndex
%
% \Finale
\endinput
|
% \end{quote}
% Do not forget to quote the argument according to the demands
% of your shell.
%
% \paragraph{Generating the documentation.}
% You can use both the \xfile{.dtx} or the \xfile{.drv} to generate
% the documentation. The process can be configured by the
% configuration file \xfile{ltxdoc.cfg}. For instance, put this
% line into this file, if you want to have A4 as paper format:
% \begin{quote}
%   \verb|\PassOptionsToClass{a4paper}{article}|
% \end{quote}
% An example follows how to generate the
% documentation with pdf\LaTeX:
% \begin{quote}
%\begin{verbatim}
%pdflatex infwarerr.dtx
%makeindex -s gind.ist infwarerr.idx
%pdflatex infwarerr.dtx
%makeindex -s gind.ist infwarerr.idx
%pdflatex infwarerr.dtx
%\end{verbatim}
% \end{quote}
%
% \begin{History}
%   \begin{Version}{2007/04/26 v1.0}
%   \item
%     First version.
%   \end{Version}
%   \begin{Version}{2007/06/14 v1.1}
%   \item
%     Small fixes.
%   \end{Version}
%   \begin{Version}{2007/09/09 v1.2}
%   \item
%     Catcode section extended.
%   \end{Version}
%   \begin{Version}{2010/04/08 v1.3}
%   \item
%     Ensure \cs{escapechar}=92 (backslash) for infos/warnings/errors.
%   \end{Version}
% \end{History}
%
% \PrintIndex
%
% \Finale
\endinput
|
% \end{quote}
% Do not forget to quote the argument according to the demands
% of your shell.
%
% \paragraph{Generating the documentation.}
% You can use both the \xfile{.dtx} or the \xfile{.drv} to generate
% the documentation. The process can be configured by the
% configuration file \xfile{ltxdoc.cfg}. For instance, put this
% line into this file, if you want to have A4 as paper format:
% \begin{quote}
%   \verb|\PassOptionsToClass{a4paper}{article}|
% \end{quote}
% An example follows how to generate the
% documentation with pdf\LaTeX:
% \begin{quote}
%\begin{verbatim}
%pdflatex infwarerr.dtx
%makeindex -s gind.ist infwarerr.idx
%pdflatex infwarerr.dtx
%makeindex -s gind.ist infwarerr.idx
%pdflatex infwarerr.dtx
%\end{verbatim}
% \end{quote}
%
% \begin{History}
%   \begin{Version}{2007/04/26 v1.0}
%   \item
%     First version.
%   \end{Version}
%   \begin{Version}{2007/06/14 v1.1}
%   \item
%     Small fixes.
%   \end{Version}
%   \begin{Version}{2007/09/09 v1.2}
%   \item
%     Catcode section extended.
%   \end{Version}
%   \begin{Version}{2010/04/08 v1.3}
%   \item
%     Ensure \cs{escapechar}=92 (backslash) for infos/warnings/errors.
%   \end{Version}
% \end{History}
%
% \PrintIndex
%
% \Finale
\endinput
|
% \end{quote}
% Do not forget to quote the argument according to the demands
% of your shell.
%
% \paragraph{Generating the documentation.}
% You can use both the \xfile{.dtx} or the \xfile{.drv} to generate
% the documentation. The process can be configured by the
% configuration file \xfile{ltxdoc.cfg}. For instance, put this
% line into this file, if you want to have A4 as paper format:
% \begin{quote}
%   \verb|\PassOptionsToClass{a4paper}{article}|
% \end{quote}
% An example follows how to generate the
% documentation with pdf\LaTeX:
% \begin{quote}
%\begin{verbatim}
%pdflatex infwarerr.dtx
%makeindex -s gind.ist infwarerr.idx
%pdflatex infwarerr.dtx
%makeindex -s gind.ist infwarerr.idx
%pdflatex infwarerr.dtx
%\end{verbatim}
% \end{quote}
%
% \begin{History}
%   \begin{Version}{2007/04/26 v1.0}
%   \item
%     First version.
%   \end{Version}
%   \begin{Version}{2007/06/14 v1.1}
%   \item
%     Small fixes.
%   \end{Version}
%   \begin{Version}{2007/09/09 v1.2}
%   \item
%     Catcode section extended.
%   \end{Version}
%   \begin{Version}{2010/04/08 v1.3}
%   \item
%     Ensure \cs{escapechar}=92 (backslash) for infos/warnings/errors.
%   \end{Version}
% \end{History}
%
% \PrintIndex
%
% \Finale
\endinput
